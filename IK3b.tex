%% LyX 2.0.5.1 created this file.  For more info, see http://www.lyx.org/.
%% Do not edit unless you really know what you are doing.
\documentclass[11pt,english,ngerman]{scrreprt}
\usepackage[T1]{fontenc}
\usepackage[utf8]{inputenc}
\usepackage[a4paper]{geometry}
\geometry{verbose,tmargin=2.6cm,bmargin=3.5cm,lmargin=2.6cm,rmargin=2.6cm}
\usepackage{fancyhdr}
\pagestyle{fancy}
\setlength{\parskip}{\smallskipamount}
\setlength{\parindent}{0pt}
\usepackage{babel}
\usepackage{float}
\usepackage{textcomp}
\usepackage{url}
\usepackage{amsmath}
\usepackage{amssymb}
\usepackage{graphicx}
\usepackage{esint}
\usepackage[unicode=true,
 bookmarks=true,bookmarksnumbered=true,bookmarksopen=false,
 breaklinks=true,pdfborder={0 0 0},backref=page,colorlinks=false]
 {hyperref}
\hypersetup{pdftitle={Quantenfeldtheorie},
 pdfauthor={Andreas Völklein},
 pdfkeywords={Quantenfeldtheorie, Physik}}

\makeatletter

%%%%%%%%%%%%%%%%%%%%%%%%%%%%%% LyX specific LaTeX commands.
\providecommand{\LyX}{\texorpdfstring%
  {L\kern-.1667em\lower.25em\hbox{Y}\kern-.125emX\@}
  {LyX}}
\newcommand{\noun}[1]{\textsc{#1}}

%%%%%%%%%%%%%%%%%%%%%%%%%%%%%% Textclass specific LaTeX commands.
\usepackage{enumitem}		% customizable list environments
\newlength{\lyxlabelwidth}      % auxiliary length 

\@ifundefined{date}{}{\date{}}
%%%%%%%%%%%%%%%%%%%%%%%%%%%%%% User specified LaTeX commands.
\usepackage{tikz,pgfplots}
%\usepackage{tikz-3dplot,cancel,polynom}
\usetikzlibrary{matrix,arrows,calc,decorations.pathmorphing,decorations.markings,intersections,shapes}
\usetikzlibrary{external}
\tikzexternalize
\usepackage{latexsym,stmaryrd,stackrel,braket,bbm,subfig,framed,esvect,scrhack,calc,slashed}
\usepackage [OMLmathrm,OMLmathbf,sfdefault=fav]{isomath}
\usepackage[explicit]{titlesec}
\usepackage[activate]{pdfcprot}

\pgfkeys{/pgf/number format/dec sep={\text{,}}}
\pgfplotsset{compat=newest}

% Inhaltsverzeichnis
\usepackage[subfigure]{tocloft}

\tocloftpagestyle{fancy}

\renewcommand{\cftchapindent}{1 em}
\renewcommand{\cftchapnumwidth}{1.5 em}

\renewcommand{\cftsecindent}{2.7 em}
\renewcommand{\cftsecnumwidth}{2.5em}

\renewcommand{\cftsubsecindent}{5.2 em}
\renewcommand{\cftsubsecnumwidth}{3.8 em}

\renewcommand{\cftsubsubsecindent}{9 em}
\renewcommand{\cftsubsubsecnumwidth}{4.5 em}

% Mathe-Operatoren
\DeclareMathOperator*{\exsop}{\exists}
\DeclareMathOperator*{\exsgop}{\exists!}
\DeclareMathOperator*{\fallop}{\forall}
\DeclareMathOperator*{\bcupdop}{\dot{\bigcup}}
\DeclareMathOperator*{\bcapdop}{\dot{\bigcap}}

%Operatornorm
\newcommand{\opnor}[1]{\abs{\hspace*{-1.1pt}\norm{#1}\hspace*{-1.1pt}}}

% nicht-totales Differential
\newcommand{\dBar}{\mathchar'26\mkern-12mu \textnormal{d}}

% Angström
\newcommand{\ang}{\textup{\AA}}

% schöne Vektorpfeile
\renewcommand{\vec}[1]{\vv{#1}}

% Rotieren
\newcommand{\Rotate}[1]{
\tikzset{external/export next=false}
\begin{tikzpicture}[baseline=(X.base)]
\node[rotate=90] (X){\ensuremath{#1}};
\end{tikzpicture}
}

% Rotieren
\newcommand{\RotateX}[2]{
\tikzset{external/export next=false}
\begin{tikzpicture}[baseline=(X.base)]
\node[rotate=#1] (X) {\ensuremath{#2}};
\end{tikzpicture}
}

%QED-Zeichen (Box)
\newcommand{\qed}{\ensuremath{\Box}}
\newcommand{\qqed}[1][\arabic{chapter}.\arabic{section}\ifnum\arabic{subsection}>0{.\arabic{subsection}}\fi]{\hspace*{1mm}\hfill\qed\ensuremath{_{\text{#1}}}}

% Mengen Modulo
\newcommand{\moduloT}[2]{
\mbox{\raisebox{0.6ex}{\ensuremath{\displaystyle #1}}
{\hspace*{-1.5mm}\Large /}
\raisebox{-0.6ex}{\hspace*{-1.5mm}\ensuremath{\displaystyle #2}}
}}

% Links Modulo
\newcommand{\lmoduloT}[2]{
\mbox{\raisebox{-0.6ex}{\ensuremath{\displaystyle #1}}
{\hspace*{-1.5mm}\Large \ensuremath{\backslash}}
\raisebox{0.6ex}{\hspace*{-1.5mm}\ensuremath{\displaystyle #2}}
}}

% Für Z/2Z, um nicht soviel schreiben zu müssen
\newcommand{\modloT}[2]{\moduloT{ \mathbb{#1}}{#2\mathbb{#1}}}

%Laplace-Beltrami-Operator
\newcommand{\LBO}{
\begin{minipage}{6mm}
 \tikzset{external/export next=false}
 \begin{tikzpicture}
   \node at (0,0){$\Delta$};
   \draw[line width=0.75] (0.25,-0.13) -- (0.1,0.15);
 \end{tikzpicture}
\end{minipage}
}

%Die Modulo-Kommandos in klein, für die Darstellungen unter Quantoren.
\newcommand{\moduloScriptT}[2]{
\mbox{\raisebox{0.4ex}{\scriptsize\ensuremath{\displaystyle #1}}
{\hspace*{-1.5mm}\footnotesize /}
\raisebox{-0.4ex}{\hspace*{-1.5mm}\scriptsize\ensuremath{\displaystyle #2}}
}}

\newcommand{\lmoduloScriptT}[2]{
\mbox{\raisebox{-0.4ex}{\scriptsize\ensuremath{\displaystyle #1}}
{\hspace*{-1.5mm}\footnotesize \ensuremath{\backslash}}
\raisebox{0.4ex}{\hspace*{-1.5mm}\scriptsize\ensuremath{\displaystyle #2}}
}}

\newcommand{\modloScriptT}[2]{\moduloScriptT{ \mathbb{#1}}{#2\mathbb{#1}}}

% stehendes Winkelzeichen
\newcommand{\winkel}{
\tikzset{external/export next=false}
\begin{tikzpicture}[scale=0.25]
\draw ({-2+3^(1/2)},0) -- (0,1) -- ({2-3^(1/2)},0);
\draw ($(0,1) + ({cos(235)*0.7},{sin(315)*0.7})$) arc (235:315:0.7);
\end{tikzpicture}}

% Wurzel mit Häkchen
\newcommand{\hsqrt}[2][{}]{\setbox0=\hbox{$\sqrt[#1]{\phantom{|}\!\! #2\hspace*{1pt}}$}\dimen0=\ht0
  \advance\dimen0-0.2\ht0
  \setbox2=\hbox{\vrule height\ht0 depth -\dimen0}
  {\box0\lower0.4pt\box2}}

% Damit nicht immer "Kapitel 1" etc. über der Kapitelüberschrift steht
\titleformat{\chapter}
  {\huge\bfseries}
  {\textrm{\thechapter} }{0pt}
  {\textrm{#1} \thispagestyle{fancy}
  }

% Neudefinition der Abschnittsmarker für die Kopfzeile
\renewcommand\partmark[1]{\markboth{#1}{}}
\renewcommand\chaptermark[1]{\markright{\arabic{chapter} #1}}
\renewcommand\sectionmark[1]{}
\renewcommand\subsectionmark[1]{}

% Schriften auf Serif umstellen
\addtokomafont{descriptionlabel}{\rmfamily}
\addtokomafont{disposition}{\rmfamily}

% Zeilenumbrüche in Gleichungen
 \allowdisplaybreaks

% Kopf- und Fußzeile
% Höhe der Kopfzeile
\setlength{\headheight}{14pt}
% obere Trennlinie
%\renewcommand{\headrulewidth}{0.4pt}
\fancyhf{} %alle Kopf- und Fußzeilenfelder bereinigen
\fancyhead[L]{\textbf{IK3b - Quantenfeldtheorie}} %Kopfzeile links
%\fancyhead[C]{\leftmark} %zentrierte Kopfzeile
\fancyhead[R]{\rightmark} %Kopfzeile rechts
\fancyfoot[C]{\thepage\quad\!\!\!\slash\quad\!\!\!\pageref{END-front}} %Seitenzahl der Front-Matter

\AtBeginDocument{
  \def\labelitemi{\normalfont\bfseries{--}}
  \def\labelitemii{\(\circ\)}
  \def\labelitemiii{\(\triangleright\)}
}

\makeatother

\begin{document}






\global\long\def\norm#1{\left\lVert #1\right\rVert }


\global\long\def\abs#1{\left\lvert #1\right\rvert }


\global\long\def\opnorm#1{\opnor{#1}}


\global\long\def\BRA#1{\Bra{#1}}


\global\long\def\KET#1{\Ket{#1}}


\global\long\def\BraKet#1{\Braket{#1}}


\global\long\def\mins{\textnormal{-}}


\global\long\def\LB{\LBO}


\global\long\def\exs{\exsop}


\global\long\def\exsg{\exsgop}


\global\long\def\fall{\fallop}


\global\long\def\bcupd{\bcupdop}


\global\long\def\bcapd{\bcapdop}


\global\long\def\sr#1#2#3{\underset{#3}{\overset{#2}{#1}}}


\global\long\def\dd{\textnormal{d}}


\global\long\def\DD{\textnormal{D}}


\global\long\def\dbar{\dBar}


\global\long\def\angs{\ang}


\global\long\def\TT{\textnormal{T}}


\global\long\def\ii{\textbf{i}}


\global\long\def\slashd#1{\slashed{#1}}


\global\long\def\modulo#1#2{\moduloT{#1}{#2}}


\global\long\def\lmodulo#1#2{\lmoduloT{#1}{#2}}


\global\long\def\modlo#1#2{\modloT{#1}{#2}}


\global\long\def\moduloScript#1#2{\moduloScriptT{#1}{#2}}


\global\long\def\lmoduloScript#1#2{\lmoduloScriptT{#1}{#2}}


\global\long\def\modloScript#1#2{\modloScriptT{#1}{#2}}


\global\long\def\vek#1{\vectorsym{#1}}


\global\long\def\mat#1{\matrixsym{#1}}


\global\long\def\ten#1{\tensorsym{#1}}


\global\long\def\msd#1{\mathstrut_{#1}}


\global\long\def\msu#1{\mathstrut^{#1}}


\pagenumbering{roman}


\title{\hspace*{1mm}\vspace*{-20mm}\\
{\Huge Integrierter Kurs IIIb}\\
{\Huge Quantenfeldtheorie}}


\author{\vspace*{-5mm}\\
\textit{\small Vorlesung von}\\
\textit{\noun{\small Prof. Dr. Andreas Schäfer}}\\
\textit{\small im Sommersemester 2013}\\
\textit{\small Überarbeitung und Textsatz in \LyX{} von}\\
\textit{\noun{\small Andreas Völklein}}\\
\vspace*{5mm}\\
\includegraphics[clip,width=15cm]{unir}\\
\vspace*{3mm}\\
{\normalsize Stand: \today}\\
\vspace*{-30mm}}

\maketitle
\fancyhead[R]{Lizenz}


\subsubsection*{ACHTUNG}

Diese Mitschrift ersetzt \emph{nicht} die Vorlesung.

Es wird daher \emph{dringend} empfohlen, die Vorlesung zu besuchen.

\vfill{}


\selectlanguage{english}%

\subsubsection*{Copyright Notice}

Copyright © 2013 \noun{Andreas Völklein}

Permission is granted to copy, distribute and/or modify this document
under the terms of the GNU Free Documentation License, Version 1.3
or any later version published by the Free Software Foundation;

with no Invariant Sections, no Front-Cover Texts, and no Back-Cover
Texts.

A copy of the license is included in the document entitled “GFDL”.


\subsubsection*{Disclaimer of Warranty}

\noun{Unless otherwise mutually agreed to by the parties in writing
and to the extent not prohibited by applicable law, }\textbf{\noun{the
Copyright Holders and any other party, who may distribute the Document
as permitted above,   provide the Document “as is}}\textbf{”,}\textbf{\noun{
without warranty of any kind}}\noun{, expressed, implied, statutory
or otherwise, including, but not limited to, the implied warranties
of merchantability, fitness for a particular purpose, non-infringement,
the absence of latent or other defects, accuracy, or the absence of
errors, whether or not discoverable.}


\subsubsection*{Limitation of Liability}

\textbf{\noun{In no event}}\noun{ unless required by applicable law
or agreed to in writing }\textbf{\noun{will the Copyright Holders,
or any other party, who may distribute the Document as permitted above,
be liable to you for any damages}}\noun{, including, but not limited
to, any general, special, incidental, consequential, punitive or exemplary
damages, however caused, regardless of the theory of liability, arising
out of or related to this license or any use of or inability to use
the Document, even if they have been advised of the possibility of
such damages.}

\textbf{\noun{In no event will the Copyright Holders'/Distributor's
liability to you}}\noun{, whether in contract, tort (including negligence),
or otherwise, }\textbf{\noun{exceed the amount you paid the Copyright
Holders/Distributor}}\noun{ for the document under this agreement.}

\selectlanguage{ngerman}%

\subsubsection*{Links}

Der Text der „\foreignlanguage{english}{GNU Free Documentation License}“
kann auch auf der Seite
\begin{quote}
\url{https://www.gnu.org/licenses/fdl-1.3.de.html}
\end{quote}
nachgelesen werden.

Eine transparente Kopie der aktuellen Version dieses Dokuments kann
von
\begin{quote}
\url{https://github.com/andiv/IK3b}
\end{quote}
heruntergeladen werden.

\newpage{}

\fancyhead[R]{Literatur}


\subsection*{Literatur}
\begin{itemize}
\item \noun{Andreas Schäfer, Florian Rappl}: \foreignlanguage{english}{\emph{Quantum
electrodynamics}}; 2010\\
\url{http://www-nw.uni-regensburg.de/~sca14496/QED/Quantenelektrodynamik.pdf}
\item \noun{Andreas Schäfer}: \foreignlanguage{english}{\emph{Quantum Chromodynamics}};\\
\url{http://www-nw.uni-regensburg.de/~sca14496/index.html}
\item \noun{Elliot Leader, Enrico Predazzi}: \foreignlanguage{english}{\emph{An
introduction to gauge theories and modern particle physics I}}; \foreignlanguage{english}{Cambridge
University Press}, 2004; ISBN: 0-521-46840-X
\item \noun{Elliot Leader, Enrico Predazzi}: \foreignlanguage{english}{\emph{An
introduction to gauge theories and modern particle physics} \emph{II}};
\foreignlanguage{english}{Cambridge University Press}, 2004; ISBN:
0-521-499510-8
\end{itemize}
{\small \newpage{}}\fancyhead[R]{Inhaltsverzeichnis}
\fancyhead[C]{}

\tableofcontents{}\label{END-front}\newpage{}\pagenumbering{arabic}
\fancyfoot[C]{\thepage\quad\!\!\!\slash\quad\!\!\!\pageref{END}} % Seitenzahl des Hauptteils
%\fancyhead[R]{\rightmark}
%\fancyhead[C]{\leftmark}%DATE: Mo 15.04.2013


\chapter*{Motivation}

\fancyhead[R]{Motivation}

Zu Beginn des 20. Jahrhunderts wurden zwei neue fundamentale Theorien
entwickelt:
\begin{itemize}
\item Quantenmechanik mit Heisenbergscher Unschärferelation%
\footnote{Wir verwenden natürliche Einheiten mit $c=1$ und $\hbar=1$.%
}: $\Delta E\cdot\Delta t\ge\frac{1}{2}$
\item Spezielle Relativitätstheorie mit Energie-Impuls-Beziehung: $E=\pm\sqrt{\vec{p}^{2}+m^{2}}$\\
Die Lösungen mit negativer Energie führen zu Antiteilchen:
\begin{align*}
e^{-\ii\left(-\abs Et\right)} & =e^{-\ii\abs E\left(-t\right)}
\end{align*}

\end{itemize}
\begin{figure}[H]
\noindent \begin{centering}
\begin{tikzpicture}
  \draw[decoration={markings,mark=at position 1 with {\arrow[scale=2]{>}};},postaction={decorate}] (0,0) node[left]{$t_i$} -- node[right]{$e^-$} (0,2) node[left]{$t_f$};
  \draw[decoration={markings,mark=at position 1 with {\arrow[scale=2]{>}};},postaction={decorate}] (2,2)  -- node[right]{$e^-\hspace*{5mm}\hat{=}$} (2,0);
  \draw[decoration={markings,mark=at position 1 with {\arrow[scale=2]{>}};},postaction={decorate}] (4,0)  -- node[right]{$e^+$} (4,2);
\end{tikzpicture}
\par\end{centering}

\noindent \begin{centering}
\caption{Antiteilchen}

\par\end{centering}

\end{figure}


Das Problem ist, dass im Vakuum virtuelle Teilchen-Antiteilchen-Paare
entstehen können:

\begin{figure}[H]
\noindent \centering{}\begin{tikzpicture}
  \draw[decoration={markings,mark=at position 0.5 with {\arrow[scale=2]{>}};},postaction={decorate}] (0,0) arc (180:0:1);
  \node at (1,1.5) {$e^-$} ;
  \draw[decoration={markings,mark=at position 0.5 with {\arrow[scale=2]{>}};},postaction={decorate}] (2,0) arc (0:-180:1);
  \node at (1,-1.5) {$e^+$} ;
  \draw[fill=black] (0,0) circle (0.05) (2,0) circle (0.05);
  \node at (4,0) {$\Delta t \le \frac{1}{2E}$};
\end{tikzpicture}\caption{virtuelles Teilchen-Antiteilchen-Paar}
\end{figure}

\begin{itemize}
\item Das Vakuum wird dadurch ein Medium.
\item Bei Rechnungen erhält man Unendlichkeiten.
\end{itemize}
Die Behandlung der Divergenzen führt zur ,,Renormierung``, dem Kern
der Quantenfeldtheorie.

Die Idee dabei ist, dass die Quantenfeldtheorie der nieder-energetische
Grenzfall einer ,,\foreignlanguage{english}{Theory of Everything}``
ist. Da wir diese nicht kennen, fordern wir eine ,,Entkopplung``:
Die Physik bei Laborenergien darf nicht von der \foreignlanguage{english}{Theory
of Everything }abhängen.

Dies wird von Eichtheorien mit oder ohne ,,spontane Symmetriebrechung``
erfüllt. Auf diese Weise erhält man das \emph{Standard-Modell} der
Teilchenphysik.

\fancyhead[R]{\rightmark}


\chapter{Dirac-Gleichung, Klein-Gordon-Gleichung}

Wir verwenden natürliche Einheiten $c=1$ und $\hbar=\frac{h}{2\pi}=1$.
Nützlich für Umrechnungen in das SI-Einheitensystem sind folgende
Konstanten:
\begin{align*}
\hbar c & =197{,}327\,\text{MeV fm} & c & =299792458\,\frac{\text{m}}{\text{s}} & 1\,\text{fm} & =10^{-15}\,\text{m}
\end{align*}
\begin{align*}
\left[\text{Energie}\right]\,\hat{=}\,\left[\text{MeV}\right]\,\hat{=}\,\left[\frac{\text{MeV}}{c}\right] & \,\hat{=}\,\left[\text{Impuls}\right]\,\hat{=}\,\left[\frac{\text{MeV}}{\hbar c}\right]\,\hat{=}\,\left[\frac{1}{\text{fm}}\right]\,\hat{=}\,\left[\frac{c}{\text{fm}}\right]\,\hat{=}\,\left[\frac{1}{\text{s}}\right]
\end{align*}
Sei $t$ eine Zeit in der Einheit $\text{MeV}^{-1}$ gegeben. Die
Umrechnung in Sekunden geht nun wie folgt:
\begin{align*}
t\left[\text{s}\right] & =\frac{t\left[\text{MeV}^{-1}\right]\cdot\hbar c}{c}
\end{align*}
Die Quantenmechanik nutzt die klassische Energie-Impuls-Beziehung
(Dispersionsrelation):
\begin{align}
E_{\text{kin}} & =\frac{\vec{p}^{2}}{2m}
\end{align}
Die Wellenmechanik basiert auf ebenen Wellen:
\begin{align*}
\psi & \sim e^{-\ii\left(Et-\vec{p}\cdot\vec{x}\right)}
\end{align*}
\begin{align*}
\ii\frac{\partial}{\partial t}e^{-\ii\left(Et-\vec{p}\cdot\vec{x}\right)} & =Ee^{-\ii\left(Et-\vec{p}\vec{x}\right)}\\
-\ii\vec{\nabla}e^{-\ii\left(Et-\vec{p}\cdot\vec{x}\right)} & =\vec{p}e^{-\ii\left(Et-\vec{p}\vec{x}\right)}
\end{align*}
Ersetze nun die klassische nun durch die relativistische Dispersionsrelation:
\begin{align}
E^{2}-\vec{p}^{2}-m^{2} & =0
\end{align}
Damit erhält man die \emph{Klein-Gordon-Gleichung}:
\begin{align}
\left[-\frac{\partial^{2}}{\partial t^{2}}+\vec{\nabla}^{2}-m^{2}\right]\phi\left(\vec{x},t\right) & =0
\end{align}
Wieso verwendet man nicht folgende Gleichung?
\begin{align*}
\left[\ii\frac{\partial}{\partial t}\pm\sqrt{-\vec{\nabla}^{2}+m^{2}}\,\right]\psi\left(\vec{x},t\right) & =0
\end{align*}
Die Wurzel lässt sich nur als eine Taylor-Entwicklung mit beliebig
hohen Potenzen in $\vec{\nabla}^{2}$ berechnen.
\begin{align*}
f\left(x+y\right) & =f\left(x\right)+f'\left(x\right)y+\frac{1}{2}f''\left(x\right)y^{2}+\ldots
\end{align*}
Unendlich hohe Ableitungen können zu einer Verletzung der Kausalität
führen. Außerdem ist die Konvergenz nicht gesichert. Die Alternative
ist die Linearisierung, also die Zerlegung der Klein-Gordon-Gleichung
als Differentialgleichung zweiter Ordnung in zwei Differentialgleichungen
erster Ordnung.


\section{Definition \textmd{(Dirac-Matrizen)}}

Seien $\gamma_{0},\gamma_{1},\gamma_{2}$ und $\gamma_{3}$ Elemente
einer Algebra mit folgender Eigenschaft:
\begin{align}
\fbox{\ensuremath{{\displaystyle \gamma_{\mu}\gamma_{\nu}+\gamma_{\nu}\gamma_{\mu}=2g_{\mu\nu}\mathbbm{1}}}}
\end{align}
Die $\gamma_{\mu}$ werden \emph{Dirac-Matrizen} genannt, und sie
erzeugen eine \emph{Clifford-Algebra}. Dabei ist
\begin{align}
\left(g_{\mu\nu}\right) & =\left(\begin{array}{cccc}
1 & 0 & 0 & 0\\
0 & -1 & 0 & 0\\
0 & 0 & -1 & 0\\
0 & 0 & 0 & -1
\end{array}\right)=\left(\left(g^{-1}\right)_{\mu\nu}\right)=:\left(g^{\mu\nu}\right)
\end{align}
die Metrik des Minkowski-Raumes%
\footnote{Mit der Metrik kann man die Indizes verschieben: $a^{\mu}=g^{\mu\nu}a_{\nu}$,
$a_{\mu}=g_{\mu\nu}a^{\nu}$%
}. Wir verwenden die Einsteinsche Summenkonvention. Die Multiplikation
mit $\mathbbm{1}$ schreiben wir gewöhnlich nicht aus. 
\begin{align*}
\left(\hat{p}^{\mu}\gamma_{\mu}-m\right)\left(\hat{p}^{\mu}\gamma_{\mu}+m\right) & =\underbrace{\hat{p}^{\mu}\hat{p}^{\nu}}_{\text{symmetrisch}}\gamma_{\mu}\gamma_{\nu}-m^{2}=\\
 & =\frac{1}{2}\left(\hat{p}^{\mu}\hat{p}^{\nu}\gamma_{\mu}\gamma_{\nu}+\hat{p}^{\nu}\hat{p}^{\mu}\gamma_{\mu}\gamma_{\nu}\right)-m^{2}=\\
 & =\frac{1}{2}\hat{p}^{\mu}\hat{p}^{\nu}\left(\gamma_{\mu}\gamma_{\nu}+\gamma_{\nu}\gamma_{\mu}\right)-m^{2}=\\
 & =\hat{p}^{\mu}\hat{p}^{\nu}g_{\mu\nu}-m^{2}=\hat{p}^{2}-m^{2}
\end{align*}
Wenn $\psi$ die Gleichung
\begin{align}
\fbox{\ensuremath{{\displaystyle \left(\hat{p}^{\mu}\gamma_{\mu}+m\right)\psi=0}}}
\end{align}
erfüllt, so erfüllt $\psi$ auch die Klein-Gordon-Gleichung, ist also
eine Lösung, die der relativistischen Energie-Impuls-Beziehung genügt.
Analoges gilt für:
\begin{align}
\fbox{\ensuremath{{\displaystyle \left(\hat{p}^{\mu}\gamma_{\mu}-m\right)\psi=0}}}
\end{align}
Dies sind die zwei Formen der \emph{Dirac-Gleichung}.

%DATE: Di 16.4.13

Wir verwenden die \emph{Feynman-Slash-Notation}:
\begin{align}
\slashd p & :=p^{\mu}\gamma_{\mu}
\end{align}
Die Dirac-Darstellung der Gamma-Matrizen lautet:
\begin{align}
\gamma^{0} & =\gamma_{0}=\left(\begin{array}{cccc}
1 & 0 & 0 & 0\\
0 & 1 & 0 & 0\\
0 & 0 & -1 & 0\\
0 & 0 & 0 & -1
\end{array}\right) & \gamma^{1} & =-\gamma_{1}=\left(\begin{array}{cccc}
0 & 0 & 0 & 1\\
0 & 0 & 1 & 0\\
0 & -1 & 0 & 0\\
-1 & 0 & 0 & 0
\end{array}\right)\nonumber \\
\gamma^{2} & =-\gamma_{2}=\left(\begin{array}{cccc}
0 & 0 & 0 & -\ii\\
0 & 0 & \ii & 0\\
0 & \ii & 0 & 0\\
-\ii & 0 & 0 & 0
\end{array}\right) & \gamma^{3} & =-\gamma_{3}=\left(\begin{array}{cccc}
0 & 0 & 1 & 0\\
0 & 0 & 0 & -1\\
-1 & 0 & 0 & 0\\
0 & 1 & 0 & 0
\end{array}\right)
\end{align}
Beispiel:
\begin{align*}
\gamma_{1}\gamma_{2}+\gamma_{2}\gamma_{1} & =\left(\begin{array}{cccc}
0 & 0 & 0 & 1\\
0 & 0 & 1 & 0\\
0 & -1 & 0 & 0\\
-1 & 0 & 0 & 0
\end{array}\right)\left(\begin{array}{cccc}
0 & 0 & 0 & -\ii\\
0 & 0 & \ii & 0\\
0 & \ii & 0 & 0\\
-\ii & 0 & 0 & 0
\end{array}\right)+\\
 & \quad+\left(\begin{array}{cccc}
0 & 0 & 0 & -\ii\\
0 & 0 & \ii & 0\\
0 & \ii & 0 & 0\\
-\ii & 0 & 0 & 0
\end{array}\right)\left(\begin{array}{cccc}
0 & 0 & 0 & 1\\
0 & 0 & 1 & 0\\
0 & -1 & 0 & 0\\
-1 & 0 & 0 & 0
\end{array}\right)=\\
 & =\left(\begin{array}{cccc}
-\ii & 0 & 0 & 0\\
0 & \ii & 0 & 0\\
0 & 0 & -\ii & 0\\
0 & 0 & 0 & \ii
\end{array}\right)+\left(\begin{array}{cccc}
\ii & 0 & 0 & 0\\
0 & -\ii & 0 & 0\\
0 & 0 & \ii & 0\\
0 & 0 & 0 & -\ii
\end{array}\right)=0
\end{align*}
Die Dirac-Darstellung ermöglicht die intuitive Interpretation, dass
die ersten beiden Indizes eines \emph{Spinors} $\psi$ einer Teilchenlösung
entsprechen und die letzten beiden einer Antiteilchenlösung.


\section{Die freien Lösungen im Ruhesystem}

Wir wollen eine Lösung $\psi\left(x^{0},\vec{x}\right)$ der Diracgleichung
im Ruhesystem finden, das heißt für $\vec{p}=0$ und $E=m$. Wir machen
einen Wellenansatz:
\begin{align*}
\psi\left(x^{0},\vec{x}\right) & =u\left(E,\vec{p}\right)e^{-\ii\left(Et-\vec{p}\cdot\vec{x}\right)}=u\left(m,\vec{0}\right)e^{-\ii mt}
\end{align*}
Die Dirac-Gleichung für Teilchen ist:
\begin{align*}
\left(m\gamma^{0}-m\right)\psi=\left(\slashd{\hat{p}}-m\right)\psi & =0\\
\left(\begin{array}{cccc}
m-m &  &  & 0\\
 & m-m\\
 &  & -m-m\\
0 &  &  & -m-m
\end{array}\right)\psi & =\left(\begin{array}{c}
0\\
0\\
0\\
0
\end{array}\right)
\end{align*}
\begin{align}
u\left(\vec{p}=\vec{0},+\right) & =a\left(\begin{array}{c}
1\\
0\\
0\\
0
\end{array}\right) & u\left(\vec{p}=\vec{0},-\right) & =b\left(\begin{array}{c}
0\\
1\\
0\\
0
\end{array}\right)
\end{align}
Analog lässt die Dirac-Gleichung für Antiteilchen
\begin{align*}
\left(\slashd{\hat{p}}+m\right)\psi & =0
\end{align*}
folgende Lösungen zu:
\begin{align}
v\left(\vec{p}=\vec{0},-\right) & =c\left(\begin{array}{c}
0\\
0\\
1\\
0
\end{array}\right) & u\left(\vec{p}=\vec{0},+\right) & =d\left(\begin{array}{c}
0\\
0\\
0\\
1
\end{array}\right)
\end{align}



\section{Lorentz-Transformationen der Dirac-Gleichung}

Fordere nun die Invarianz der Dirac-Gleichung unter einer Lorentz-Transformationen%
\footnote{Beachte, dass $\Lambda_{\nu}^{\mu}$ sich unter Lorentz-Transformationen
nicht ändert, also kein Tensor ist.%
} $\Lambda_{\nu}^{\mu}$:

\begin{align}
0 & =\left(\ii\frac{\partial}{\partial x_{\mu}}\gamma_{\mu}-m\right)\psi\left(x\right)=\left(\ii\frac{\partial}{\partial x_{\nu}'}\Lambda_{\nu}^{\mu}\gamma_{\mu}-m\right)\underbrace{S^{-1}\left(\Lambda\right)\psi'\left(x'\right)}_{=\psi\left(x\right)}\qquad/S\cdot\nonumber \\
0 & =\bigg(\ii\frac{\partial}{\partial x_{\nu}'}\underbrace{\Lambda_{\nu}^{\mu}S\gamma_{\mu}S^{-1}}_{\stackrel{!}{=}\gamma_{\nu}}-m\bigg)\psi'\left(x'\right)
\end{align}
Es genügt, eine infinitesimale Lorentz-Transformation zu betrachten,
da man eine endliche Transformation als Hintereinanderausführung von
$N$ infinitesimalen darstellen kann.
\begin{align*}
g^{\nu\nu'}\Lambda_{\nu'}^{\mu}\big|_{\text{inf.}} & =g^{\nu\mu}+\frac{\omega^{\nu\mu}}{N}+\mathcal{O}\left(\frac{1}{N^{2}}\right)
\end{align*}
Wir benutzen dann:
\begin{align*}
\lim_{N\to\infty}\left(1-\ii\frac{a}{N}\right)^{N} & =e^{-\ii a}
\end{align*}
Zur Erinnerung: Die Lorentz-Transformation lässt die Metrik invariant:
\begin{align*}
\Lambda_{\mu'}^{\mu}\Lambda_{\nu'}^{\nu}g_{\mu\nu} & =g_{\mu'\nu'}
\end{align*}
Es folgt:
\begin{align*}
g^{\mu''\mu'}\Lambda_{\mu'}^{\mu}g^{\nu''\nu'}\Lambda_{\nu'}^{\nu}g_{\mu\nu} & =g^{\mu''\nu''}\\
g^{\mu''\mu}g^{\nu''\nu}g_{\mu\nu}+\frac{\omega^{\mu''\mu}}{N}g^{\nu''\nu}g_{\mu\nu}+g^{\mu''\mu}\frac{\omega^{\nu''\nu}}{N}g_{\mu\nu} & =g^{\mu''\nu''}\\
\frac{\omega^{\mu''\mu}}{N}g_{\mu\nu''} & =-\frac{\omega^{\nu''\nu}}{N}g_{\mu''\nu}
\end{align*}
\begin{align}
\Rightarrow\qquad\fbox{\ensuremath{{\displaystyle \omega^{\mu''\nu''}=-\omega^{\nu''\mu''}}}}
\end{align}
Für $S$ machen wir den Ansatz:
\begin{align}
S & =\mathbbm{1}-\frac{\ii}{4}\cdot\frac{\omega^{\mu\nu}}{N}\sigma_{\mu\nu}\label{eq:S-Transformation-Definition}
\end{align}
Dabei sind die $\sigma_{\mu\nu}$ beliebige $\mathbb{C}^{4\times4}$-Matrizen,
ohne Einschränkung mit $\sigma_{\mu\nu}=-\sigma_{\nu\mu}$, da der
symmetrische Anteil wegfällt, weil $\omega^{\mu\nu}$ antisymmetrisch
ist. Einsetzen liefert:
\begin{align*}
\gamma_{\nu}+\mathcal{O}\left(\frac{1}{N^{2}}\right) & \stackrel{!}{=}\left(\mathbbm{1}-\frac{\ii}{4}\sigma_{\mu'\nu'}\frac{\omega^{\mu'\nu'}}{N}\right)\left(\gamma_{\nu}+\frac{\omega_{\nu\mu}}{N}\gamma^{\mu}\right)\left(\mathbbm{1}+\frac{\ii}{4}\sigma_{\mu''\nu''}\frac{\omega^{\mu''\nu''}}{N}\right)=\\
\Rightarrow\qquad0 & =-\frac{\ii}{4}\sigma_{\mu'\nu'}\frac{\omega^{\mu'\nu'}}{N}\gamma_{\nu}+\frac{\omega_{\nu}\msu{\mu}}{N}\gamma_{\mu}+\frac{\ii}{4}\gamma_{\nu}\sigma_{\mu''\nu''}\frac{\omega^{\mu''\nu''}}{N}\\
0 & =-\frac{\ii}{4}\sigma_{\mu'\nu'}\gamma_{\nu}\frac{\omega^{\mu'\nu'}}{N}+g_{\nu\nu'}\gamma_{\mu'}\frac{\omega^{\nu'\mu'}}{N}+\frac{\ii}{4}\gamma_{\nu}\sigma_{\mu'\nu'}\frac{\omega^{\mu'\nu'}}{N}\\
0 & =\frac{\omega^{\mu'\nu'}}{N}\left(-\frac{\ii}{4}\sigma_{\mu'\nu'}\gamma_{\nu}-g_{\nu\nu'}\gamma_{\mu'}+\frac{\ii}{4}\gamma_{\nu}\sigma_{\mu'\nu'}\right)
\end{align*}
Es gilt, da $\omega^{\mu'\nu'}$ antisymmetrisch ist:
\begin{align*}
-\omega^{\mu'\nu'}g_{\nu\nu'}\gamma_{\mu'} & =\left(-\frac{1}{2}g_{\nu\nu'}\gamma_{\mu'}+\frac{1}{2}g_{\nu\mu'}\gamma_{\nu'}\right)\omega^{\mu'\nu'}
\end{align*}
Somit folgt:
\begin{align}
\frac{\ii}{4}\left(\sigma_{\mu'\nu'}\gamma_{\nu}-\gamma_{\nu}\sigma_{\mu'\nu'}\right)\frac{\omega^{\mu'\nu'}}{N} & =\frac{1}{2}\left(-g_{\nu\nu'}\gamma_{\mu'}+g_{\nu\mu'}\gamma_{\nu'}\right)\frac{\omega^{\mu'\nu'}}{N}\nonumber \\
\frac{\ii}{2}\left(\sigma_{\mu'\nu'}\gamma_{\nu}-\gamma_{\nu}\sigma_{\mu'\nu'}\right) & =g_{\mu'\nu}\gamma_{\nu'}-g_{\nu'\nu}\gamma_{\mu'}\label{eq:isigma}
\end{align}
Die $\sigma_{\mu'\nu'}$ sind antisymmetrische Tensoren der Stufe
2, die nur von den $\gamma$-Matrizen abhängen, das heißt mit $A\in\mathbb{C}$
gilt:
\begin{align*}
\sigma_{\mu'\nu'} & =A\cdot\left[\gamma_{\mu'},\gamma_{\nu'}\right]
\end{align*}
Die linke Seite von (\ref{eq:isigma}) ist damit:
\begin{align*}
 & \frac{\ii}{2}A\Big(\gamma_{\mu'}\underbrace{\gamma_{\nu'}\gamma_{\nu}}_{\text{kommutieren}}-\gamma_{\nu'}\underbrace{\gamma_{\mu'}\gamma_{\nu}}_{\text{kommutieren}}-\underbrace{\gamma_{\nu}\gamma_{\mu'}}_{\text{kommutieren}}\gamma_{\nu'}+\underbrace{\gamma_{\nu}\gamma_{\nu'}}_{\text{kommutieren}}\gamma_{\mu'}\Big)=\\
 & \qquad\qquad\qquad=\frac{\ii}{2}A\Big(2g_{\nu'\nu}\gamma_{\mu'}-\underline{\gamma_{\mu'}\gamma_{\nu}\gamma_{\nu'}}-2g_{\mu'\nu}\gamma_{\nu'}+\underleftarrow{\gamma_{\nu'}\gamma_{\nu}\gamma_{\mu'}}-\\
 & \qquad\qquad\qquad\qquad\;-2g_{\nu\mu'}\gamma_{\nu'}+\underline{\gamma_{\mu'}\gamma_{\nu}\gamma_{\nu'}}+2g_{\nu\nu'}\gamma_{\mu'}-\underleftarrow{\gamma_{\nu'}\gamma_{\nu}\gamma_{\mu'}}\Big)\\
 & \qquad\qquad\qquad=2\ii A\left(g_{\nu'\nu}\gamma_{\mu'}-g_{\mu'\nu}\gamma_{\nu'}\right)
\end{align*}
Aus (\ref{eq:isigma}) folgt daher:
\begin{align}
A & =\frac{\ii}{2} & \fbox{\ensuremath{{\displaystyle \sigma_{\mu\nu}=\frac{\ii}{2}\left[\gamma_{\mu},\gamma_{\nu}\right]}}}
\end{align}
Damit haben wir $S$ bestimmt.

Erinnerung: Kugelflächenfunktionen und Drehimpulsoperator
\begin{align*}
Y_{lm} & =\sqrt{\frac{2l+1}{4\pi}\cdot\frac{\left(l-m\right)!}{\left(l+m\right)!}}P_{l}^{m}\left(\cos\left(\vartheta\right)\right)e^{\ii m\varphi}\\
\hat{L}_{z} & =\frac{\hbar}{\ii}\left(x\frac{\partial}{\partial y}-y\frac{\partial}{\partial x}\right)=\frac{\hbar}{\ii}\frac{\partial}{\partial\varphi}
\end{align*}



\subsection{Erster Spezialfall: Rotation}

Betrachte die infinitesimale Transformation für Drehungen um die $z$-Achse:
\begin{align}
\left(\Lambda_{\nu}^{\mu}\right)_{\text{inf. Rot.}} & =\left(\begin{array}{cccc}
1 & 0 & 0 & 0\\
0 & 1 & -\frac{\varphi}{N} & 0\\
0 & \frac{\varphi}{N} & 1 & 0\\
0 & 0 & 0 & 1
\end{array}\right) & \frac{\omega^{12}}{N} & =-\frac{\omega^{21}}{N}=-\frac{\varphi}{N}
\end{align}
Alle anderen $\omega^{\mu\nu}$ verschwinden. Außerdem gilt:
\begin{align*}
\sigma_{12} & =-\sigma_{21}=\frac{\ii}{2}\left(\gamma_{1}\gamma_{2}-\gamma_{2}\gamma_{1}\right)=\ii\gamma_{1}\gamma_{2}=\\
 & =\ii\left(\begin{array}{cccc}
0 & 0 & 0 & -1\\
0 & 0 & -1 & 0\\
0 & 1 & 0 & 0\\
1 & 0 & 0 & 0
\end{array}\right)\left(\begin{array}{cccc}
0 & 0 & 0 & \ii\\
0 & 0 & -\ii & 0\\
0 & -\ii & 0 & 0\\
\ii & 0 & 0 & 0
\end{array}\right)=\left(\begin{array}{cccc}
1 & 0 & 0 & 0\\
0 & -1 & 0 & 0\\
0 & 0 & 1 & 0\\
0 & 0 & 0 & -1
\end{array}\right)
\end{align*}
\begin{align}
S & =\exp\left(-\frac{\ii}{4}N\cdot2\frac{\omega^{12}}{N}\sigma_{12}\right)=\exp\left(\frac{\ii}{2}\varphi\left(\begin{array}{cccc}
1 & 0 & 0 & 0\\
0 & -1 & 0 & 0\\
0 & 0 & 1 & 0\\
0 & 0 & 0 & -1
\end{array}\right)\right)=\nonumber \\
 & =\left(\begin{array}{cccc}
1 &  &  & 0\\
 & 1\\
 &  & 1\\
0 &  &  & 1
\end{array}\right)\cos\left(\frac{\varphi}{2}\right)+\left(\begin{array}{cccc}
1 &  &  & 0\\
 & -1\\
 &  & 1\\
0 &  &  & -1
\end{array}\right)\ii\sin\left(\frac{\varphi}{2}\right)=\nonumber \\
 & =\left(\begin{array}{cccc}
e^{\ii\frac{\varphi}{2}} &  &  & 0\\
 & e^{-\ii\frac{\varphi}{2}}\\
 &  & e^{\ii\frac{\varphi}{2}}\\
0 &  &  & e^{-\ii\frac{\varphi}{2}}
\end{array}\right)
\end{align}
Also haben die erste und dritte Komponenten Spin $\frac{1}{2}$ und
die anderen beiden Spin $-\frac{1}{2}$.


\subsection{Zweiter Spezialfall: Lorentz-Boost}

Für den Lorentz-Boost gilt:
\begin{align*}
\left(x^{0}\right)' & =\gamma\left(x^{0}+\vec{\beta}\vec{x}\right)\\
\left(\vec{x}\right)' & =\gamma\left(\vec{\beta}x^{0}+\vec{x}\right)
\end{align*}
\begin{align*}
\vec{\beta} & =\frac{\vec{v}}{c} & \gamma^{-1} & =\sqrt{1-\beta^{2}}
\end{align*}
\begin{align*}
\frac{\omega^{0}\msd k}{N} & =\frac{\omega^{k}\msd 0}{N}=:\frac{\omega^{k}}{N} & \omega & :=\sqrt{\left(\omega^{1}\right)^{2}+\left(\omega^{2}\right)^{2}+\left(\omega^{3}\right)^{2}}
\end{align*}
\begin{align}
\left(\Lambda\right)_{\text{inf. Boost}} & =\left(\begin{array}{cccc}
1 & \frac{\omega^{1}}{N} & \frac{\omega^{2}}{N} & \frac{\omega^{3}}{N}\\
\frac{\omega^{1}}{N} & 1 & 0 & 0\\
\frac{\omega^{2}}{N} & 0 & 1 & 0\\
\frac{\omega^{3}}{N} & 0 & 0 & 1
\end{array}\right)
\end{align}
Man erhält:
\begin{align*}
\sigma_{10} & =\ii\left(\begin{array}{cccc}
0 & 0 & 0 & 1\\
0 & 0 & 1 & 0\\
0 & 1 & 0 & 0\\
1 & 0 & 0 & 0
\end{array}\right) & \sigma_{20} & =\ii\left(\begin{array}{cccc}
0 & 0 & 0 & -\ii\\
0 & 0 & \ii & 0\\
0 & -\ii & 0 & 0\\
\ii & 0 & 0 & 0
\end{array}\right) & \sigma_{30} & =\ii\left(\begin{array}{cccc}
0 & 0 & 1 & 0\\
0 & 0 & 0 & -1\\
1 & 0 & 0 & 0\\
0 & -1 & 0 & 0
\end{array}\right)
\end{align*}
\begin{align*}
S & =\lim_{N\to\infty}\left(\mathbbm{1}+\frac{1}{2}\cdot\frac{\omega}{N}\underbrace{\left(\begin{array}{cccc}
0 & 0 & p^{3} & p^{1}-\ii p^{2}\\
0 & 0 & p^{1}+\ii p^{2} & -p^{3}\\
p^{3} & p^{1}-\ii p^{2} & 0 & 0\\
p^{1}+\ii p^{2} & -p^{3} & 0 & 0
\end{array}\right)\frac{1}{\norm{\vec{p}}}}_{=:M}\right)^{N}=e^{\frac{\omega}{2}M}
\end{align*}
\begin{align*}
M^{2} & =\left(\begin{array}{cccc}
1 & 0 & 0 & 0\\
0 & 1 & 0 & 0\\
0 & 0 & 1 & 0\\
0 & 0 & 0 & 1
\end{array}\right)
\end{align*}
\begin{align*}
\Rightarrow\qquad M^{2n} & =\mathbbm{1} & M^{2n+1} & =M
\end{align*}
\begin{align}
\Rightarrow\qquad S & =\cosh\left(\frac{\omega}{2}\right)\mathbbm{1}+\sinh\left(\frac{\omega}{2}\right)M
\end{align}
%DATE: Do 18.4.13

Wir betrachten jetzt die $\Lambda_{\alpha}^{\beta}$-Matrix, die sich
aus $N$ infinitesimalen Transformationen ergibt.
\begin{align}
\left(\Lambda_{\alpha}^{\beta}\right) & =\lim_{N\to\infty}\left(1+\frac{\omega}{N}\underbrace{\left(\begin{array}{cccc}
0 & \omega^{1} & \omega^{2} & \omega^{3}\\
\omega^{1} & 0 & 0 & 0\\
\omega^{2} & 0 & 0 & 0\\
\omega^{3} & 0 & 0 & 0
\end{array}\right)\frac{1}{\omega}}_{=:\tilde{M}}\right)^{N}=e^{\omega\tilde{M}}
\end{align}
\begin{align*}
\tilde{M}^{2} & =\left(\begin{array}{cccc}
1 & 0 & 0 & 0\\
0 & \frac{\left(\omega^{1}\right)^{2}}{\omega^{2}} & 0 & 0\\
0 & 0 & \frac{\left(\omega^{2}\right)^{2}}{\omega^{2}} & 0\\
0 & 0 & 0 & \frac{\left(\omega^{3}\right)^{2}}{\omega^{2}}
\end{array}\right)
\end{align*}
Damit folgt:
\begin{align*}
\left(\tilde{M}^{2}\right)_{0}^{0} & =1=\left(\tilde{M}^{2n}\right)_{0}^{0} & \left(\tilde{M}^{2n+1}\right)_{0}^{0} & =0
\end{align*}
Somit ergibt sich:
\begin{align}
\cosh\left(\omega\right) & =\left(\Lambda\right)_{0}^{0}\stackrel{!}{=}\gamma=\frac{E}{m}=\sqrt{\frac{\norm{\vec{p}}^{2}+m^{2}}{m^{2}}}\nonumber \\
\cosh\left(\frac{\omega}{2}\right) & =\sqrt{\frac{\cosh\left(\omega\right)+1}{2}}=\sqrt{\frac{E+m}{2m}}\nonumber \\
\frac{1}{\norm{\vec{p}}}\sinh\left(\frac{\omega}{2}\right) & =\frac{1}{\norm{\vec{p}}}\sqrt{\frac{\cosh\left(\omega\right)-1}{2}}=\frac{1}{\sqrt{E^{2}-m^{2}}}\sqrt{\frac{E-m}{2m}}=\nonumber \\
 & =\frac{1}{\sqrt{2m\left(E+m\right)}}=\frac{1}{E+m}\sqrt{\frac{E+m}{2m}}\nonumber \\
S & =\sqrt{\frac{E+m}{2m}}\left(\begin{array}{cccc}
1 & 0 & \frac{p^{3}}{E+m} & \frac{p^{1}-\ii p^{2}}{E+m}\\
0 & 1 & \frac{p^{1}+\ii p^{2}}{E+m} & \frac{-p^{3}}{E+m}\\
\frac{p^{3}}{E+m} & \frac{p^{1}-\ii p^{2}}{E+m} & 1 & 0\\
\frac{p^{1}+\ii p^{2}}{E+m} & \frac{-p^{3}}{E+m} & 0 & 1
\end{array}\right)
\end{align}
\begin{align*}
\Rightarrow\qquad u\left(\vec{p},+\right) & =\sqrt{\frac{E+m}{2m}}\left(\begin{array}{c}
1\\
0\\
\frac{p^{3}}{E+m}\\
\frac{p^{1}+\ii p^{2}}{E+m}
\end{array}\right)
\end{align*}
Die Lösungen sind also:
\begin{align*}
\psi_{1} & =e^{-\ii\left(Et-\vec{p}\vec{x}\right)}u\left(\vec{p},+\right)=e^{-\ii px}u\left(\vec{p},+\right) & \psi_{2} & =e^{-\ii\left(Et-\vec{p}\vec{x}\right)}u\left(\vec{p},-\right)=e^{-\ii px}u\left(\vec{p},-\right)\\
\psi_{3} & =e^{\ii\left(Et-\vec{p}\vec{x}\right)}v\left(\vec{p},+\right)=e^{\ii px}v\left(\vec{p},+\right) & \psi_{4} & =e^{\ii\left(Et-\vec{p}\vec{x}\right)}v\left(\vec{p},-\right)=e^{\ii px}v\left(\vec{p},-\right)
\end{align*}
\begin{align}
\psi_{1}\left(\vec{x},t\right) & =e^{-\ii px}\sqrt{\frac{E+m}{2m}}\left(\begin{array}{c}
1\\
0\\
\frac{p^{3}}{E+m}\\
\frac{p^{1}+\ii p^{2}}{E+m}
\end{array}\right) & \psi_{2}\left(\vec{x},t\right) & =e^{-\ii px}\sqrt{\frac{E+m}{2m}}\left(\begin{array}{c}
0\\
1\\
\frac{p^{1}-\ii p^{2}}{E+m}\\
\frac{-p^{3}}{E+m}
\end{array}\right)\nonumber \\
\psi_{3}\left(\vec{x},t\right) & =e^{\ii px}\sqrt{\frac{E+m}{2m}}\left(\begin{array}{c}
\frac{p^{3}}{E+m}\\
\frac{p^{1}+\ii p^{2}}{E+m}\\
1\\
0
\end{array}\right) & \psi_{4}\left(\vec{x},t\right) & =e^{\ii px}\sqrt{\frac{E+m}{2m}}\left(\begin{array}{c}
\frac{p^{1}-\ii p^{2}}{E+m}\\
\frac{-p^{3}}{E+m}\\
0\\
1
\end{array}\right)
\end{align}



\subsection{Der 4-Spinvektor}

Wir suchen nun die relativistische Verallgemeinerung des Spinvektors
$\vec{s}$, den 4-Spinvektor $s^{\mu}$. Im Ruhesystem soll gelten:
\begin{align}
s^{\mu} & =\left(0,\vec{s}\right) & p^{\mu} & =\left(E,\vec{0}\right)=\left(m,\vec{0}\right)
\end{align}
Wir verwenden, dass 4-Skalarprodukte invariant unter Lorentz-Transformationen
sind.
\begin{align}
s^{2} & =s_{\mu}s^{\mu}=-\norm{\vec{s}}^{2}\stackrel{\norm{\vec{s}}:=1}{=}-1 & s\cdot p & =s_{\mu}p^{\mu}=0
\end{align}
Wir machen folgenden Ansatz für $s^{\mu}$, da die einzige ausgezeichnete
Raumrichtung $\vec{p}$ ist:
\begin{align*}
s^{\mu} & =\left(s^{0},\alpha\vec{p}\right)
\end{align*}
Dabei ist $\alpha\in\mathbb{R}$ ein beliebiger Proportionalitätsfaktor.
Einsetzen liefert:
\begin{align*}
s^{0}E-\alpha\vec{p}^{2} & =0\\
\Rightarrow\qquad s^{0} & =\frac{\alpha\vec{p}^{2}}{E}
\end{align*}
Damit ergibt die andere Gleichung:
\begin{align*}
-1 & =\left(s^{0}\right)^{2}-\alpha^{2}\vec{p}^{2}=\left(\frac{\alpha\vec{p}^{2}}{E}\right)^{2}-\alpha^{2}\vec{p}^{2}=\\
 & =\alpha^{2}\left(\frac{\vec{p}^{4}}{E^{2}}-\vec{p}^{2}\right)=\alpha^{2}\left(\frac{\vec{p}^{4}}{E^{2}}-\vec{p}^{2}\right)=\\
 & =-\alpha^{2}\cdot\frac{\vec{p}^{2}m^{2}}{E^{2}}\\
\Rightarrow\qquad\alpha & =\pm\frac{E}{m\norm{\vec{p}}}
\end{align*}
Somit ist der Spin 4-Vektor:
\begin{align}
\fbox{\ensuremath{{\displaystyle s^{\mu}=\pm\frac{E}{m}\left(\frac{\norm{\vec{p}}}{E},\frac{\vec{p}}{\norm{\vec{p}}}\right)}}}
\end{align}
Eine wichtige Eigenschaft von $s^{\mu}$ ist:
\begin{align}
\lim_{\frac{\norm{\vec{p}}}{m}\to\infty}s^{\mu} & =\pm\frac{1}{m}\left(E,\vec{p}\right)=\pm\frac{p^{\mu}}{m}
\end{align}
Was ist das Lorentz-invariante Skalarprodukt (und damit die Norm)
für Spinoren? Wir machen folgenden Ansatz:
\begin{align*}
\int\dd^{3}x\left(\psi^{*}\right)^{\TT}\left(t,\vec{x}\right)\Gamma\psi\left(t,\vec{x}\right) & =1
\end{align*}
Dabei ist $\Gamma\in\mathbb{C}^{4\times4}$ eine Matrix. Die Lorentz-Invarianz
bedeutet:
\begin{align*}
S^{\dagger}\Gamma S & \stackrel{!}{=}\Gamma
\end{align*}
Nebenrechnung:
\begin{align*}
-\frac{\omega^{\mu\nu}}{4}\left(\ii\sigma_{\mu\nu}\right)^{\dagger} & =\frac{\omega^{\mu\nu}}{8}\left(\gamma_{\mu}\gamma_{\nu}-\gamma_{\nu}\gamma_{\mu}\right)^{\dagger}=\frac{\omega^{\mu\nu}}{8}\left(\gamma_{\nu}^{\dagger}\gamma_{\mu}^{\dagger}-\gamma_{\mu}^{\dagger}\gamma_{\nu}^{\dagger}\right)
\end{align*}
In der Dirac-Darstellung gilt:
\begin{align*}
\gamma_{0} & =\gamma_{0}^{\dagger} & \gamma_{i} & =-\gamma_{i}^{\dagger}
\end{align*}
Dies kann man aufgrund der Antikommutator-Relationen kurz schreiben
als:
\begin{align*}
\gamma_{\mu}^{\dagger} & =\gamma_{0}\gamma_{\mu}\gamma_{0}
\end{align*}
Es folgt:
\begin{align*}
-\frac{\omega^{\mu\nu}}{4}\left(\ii\sigma_{\mu\nu}\right)^{\dagger} & =\gamma_{0}\frac{\omega^{\mu\nu}}{8}\left(\gamma_{\nu}\gamma_{\mu}-\gamma_{\mu}\gamma_{\nu}\right)\gamma_{0}=\gamma_{0}\frac{\ii}{4}\omega^{\mu\nu}\sigma_{\mu\nu}\gamma_{0}
\end{align*}
Aus (\ref{eq:S-Transformation-Definition}) folgt für eine infinitesimale
Transformation:
\begin{align*}
S^{\dagger} & =\gamma_{0}S^{-1}\gamma_{0}
\end{align*}
Bei einer endlichen Transformation ergibt sich dies ebenfalls:
\begin{align*}
\left(S^{N}\right)^{\dagger} & =\left(S^{\dagger}\right)^{N}=\gamma_{0}\left(S^{-1}\right)^{N}\gamma_{0}=\gamma_{0}\left(S^{N}\right)^{-1}\gamma_{0}
\end{align*}
Damit erhält man für alle $\omega^{\mu\nu}$:
\begin{align*}
\gamma_{0}S^{-1}\gamma_{0}\Gamma S & \stackrel{!}{=}0\\
\Rightarrow\qquad\Gamma & =\gamma_{0}=\left(\begin{array}{cccc}
1 & 0 & 0 & 0\\
0 & 1 & 0 & 0\\
0 & 0 & -1 & 0\\
0 & 0 & 0 & -1
\end{array}\right)
\end{align*}
Die Norm ist also:
\begin{align}
\norm{\psi}^{2} & =\int\dd^{3}x\psi^{\dagger}\left(t,\vec{x}\right)\left(\begin{array}{cccc}
1 & 0 & 0 & 0\\
0 & 1 & 0 & 0\\
0 & 0 & -1 & 0\\
0 & 0 & 0 & -1
\end{array}\right)\psi\left(t,\vec{x}\right)
\end{align}
Im Folgenden verwenden wir die Notation:
\begin{align*}
\overline{\psi} & =\psi^{\dagger}\gamma_{0}
\end{align*}



\subsection{Projektions-Operatoren}

Wir wollen die Lösungen der Dirac-Gleichung jetzt mit Hilfe von Projektions-Operatoren
schreiben. Im euklidischen Raum gilt:
\begin{align*}
\vec{v} & =\sum_{i}\vec{e}_{i}\left(\vec{e}_{i}^{\TT}\cdot\vec{v}\right)=\sum_{i}\underbrace{\left(\vec{e}_{i}\vec{e}_{i}^{\TT}\right)}_{=:P}\cdot\vec{v}
\end{align*}
Die Projektions-Operatoren im Ruhesystem sind einfach:
\begin{align}
\hat{P}_{1} & =\left(\begin{array}{cccc}
1 & 0 & 0 & 0\\
0 & 0 & 0 & 0\\
0 & 0 & 0 & 0\\
0 & 0 & 0 & 0
\end{array}\right) & \hat{P}_{2} & =\left(\begin{array}{cccc}
0 & 0 & 0 & 0\\
0 & 1 & 0 & 0\\
0 & 0 & 0 & 0\\
0 & 0 & 0 & 0
\end{array}\right)\nonumber \\
\hat{P}_{3} & =\left(\begin{array}{cccc}
0 & 0 & 0 & 0\\
0 & 0 & 0 & 0\\
0 & 0 & 1 & 0\\
0 & 0 & 0 & 0
\end{array}\right) & \hat{P}_{4} & =\left(\begin{array}{cccc}
0 & 0 & 0 & 0\\
0 & 0 & 0 & 0\\
0 & 0 & 0 & 0\\
0 & 0 & 0 & 1
\end{array}\right)
\end{align}
Weiter gilt im Ruhesystem:
\begin{align*}
\frac{\slashd p+m}{2m} & =\left(\begin{array}{cccc}
1 & 0 & 0 & 0\\
0 & 1 & 0 & 0\\
0 & 0 & 0 & 0\\
0 & 0 & 0 & 0
\end{array}\right) & \frac{\slashd p-m}{2m} & =\left(\begin{array}{cccc}
0 & 0 & 0 & 0\\
0 & 0 & 0 & 0\\
0 & 0 & 1 & 0\\
0 & 0 & 0 & 1
\end{array}\right)
\end{align*}
Wir definieren:
\begin{align*}
\gamma_{5} & =\gamma^{5}:=\ii\gamma^{0}\gamma^{1}\gamma^{2}\gamma^{3}=-\ii\gamma_{0}\gamma_{1}\gamma_{2}\gamma_{3}=\left(\begin{array}{cccc}
0 & 0 & 1 & 0\\
0 & 0 & 0 & 1\\
1 & 0 & 0 & 0\\
0 & 1 & 0 & 0
\end{array}\right)
\end{align*}
\begin{align*}
\gamma_{5}\gamma_{\mu}s^{\mu}\bigg|_{\vec{s}=\left(0,0,1\right)} & =\gamma_{5}\gamma_{3}=\left(\begin{array}{cccc}
0 & 0 & 1 & 0\\
0 & 0 & 0 & 1\\
1 & 0 & 0 & 0\\
0 & 1 & 0 & 0
\end{array}\right)\left(\begin{array}{cccc}
0 & 0 & -1 & 0\\
0 & 0 & 0 & 1\\
1 & 0 & 0 & 0\\
0 & -1 & 0 & 0
\end{array}\right)=\left(\begin{array}{cccc}
1 & 0 & 0 & 0\\
0 & -1 & 0 & 0\\
0 & 0 & -1 & 0\\
0 & 0 & 0 & 1
\end{array}\right)
\end{align*}
\begin{align}
\frac{1+\gamma_{5}\slashd s}{2} & =\left(\begin{array}{cccc}
1 & 0 & 0 & 0\\
0 & 0 & 0 & 0\\
0 & 0 & 0 & 0\\
0 & 0 & 0 & 1
\end{array}\right)\nonumber \\
\Rightarrow\qquad\hat{P}_{1} & =\frac{\slashd p+m}{2m}\cdot\frac{1+\gamma_{5}\slashd s}{2}=u\left(\vec{p},+\right)\overline{u}\left(\vec{p},+\right)
\end{align}
Analog ergibt sich:
\begin{align}
\hat{P}_{2} & =\frac{\slashd p+m}{2m}\cdot\frac{1-\gamma_{5}\slashd s}{2}\\
\hat{P}_{3} & =\frac{-\slashd p+m}{2m}\cdot\frac{1-\gamma_{5}\slashd s}{2}\\
\hat{P}_{4} & =\frac{-\slashd p+m}{2m}\cdot\frac{1+\gamma_{5}\slashd s}{2}
\end{align}
Für $\norm{\vec{p}}\gg m$ ist $\slashd s\approx\slashd p$ und es
gilt:
\begin{align*}
\left(\slashd p+m\right)\frac{\slashd p}{m} & =\frac{\slashd p^{2}+\slashd pm}{m}=\frac{p^{2}+m\slashd p}{m}=\frac{m^{2}+m\slashd p}{m}=m+\slashd p
\end{align*}
\begin{align}
 & \text{Helizität (engl. helicity):} &  & \frac{1\pm\gamma_{5}\slashd s}{2}\\
 & \text{Chiralität (engl. chirality):} &  & \frac{1\pm\gamma_{5}}{2}
\end{align}
Für $\frac{\norm{\vec{p}}}{m}\to\infty$ stimmt beides überein.

%DATE: Mo 22.4.13


\subsection{Diskrete Lorentz-Transformationen}

Zur Lorentz-Gruppe gehören die diskreten Transformationen $\hat{P}$
(Parität) und $\hat{T}$ (Zeitumkehr).
\begin{align}
\hat{P} & =\left(\begin{array}{cccc}
1 & 0 & 0 & 0\\
0 & -1 & 0 & 0\\
0 & 0 & -1 & 0\\
0 & 0 & 0 & -1
\end{array}\right) & \hat{T} & =\left(\begin{array}{cccc}
-1 & 0 & 0 & 0\\
0 & 1 & 0 & 0\\
0 & 0 & 1 & 0\\
0 & 0 & 0 & 1
\end{array}\right)
\end{align}
\begin{align}
\Lambda_{\mu'}^{\mu}\Lambda_{\nu'}^{\nu}g_{\mu\nu} & =g_{\mu'\nu'}\nonumber \\
\Rightarrow\qquad\left(\det\left(\Lambda\right)\right)^{2} & =1\nonumber \\
\det\left(\Lambda\right) & =\pm1
\end{align}
Für $\mu'=0=\nu'$ gilt:
\begin{align*}
\Lambda_{0}^{0}\Lambda_{0}^{0}-\sum_{i=1}^{3}\Lambda_{0}^{i}\Lambda_{0}^{i} & =1\\
\Rightarrow\qquad\left(\Lambda_{0}^{0}\right)^{2} & =1+\sum_{i=1}^{3}\left(\Lambda_{0}^{i}\right)^{2}\ge1
\end{align*}
\begin{align}
\Lambda_{0}^{0} & \ge1\qquad\text{oder}\qquad\Lambda_{0}^{0}\le1
\end{align}
Es gibt vier nicht zusammenhängende Teilmengen der Lorentzgruppe:
\begin{align}
L_{+}^{\uparrow}:\  & \det\left(\Lambda\right)=+1,\quad\Lambda_{0}^{0}>1 & \stackrel{\hat{T}}{\Rightarrow}\qquad L_{-}^{\downarrow}:\  & \det\left(\Lambda\right)=-1,\quad\Lambda_{0}^{0}<1\nonumber \\
 & \hat{P}\Downarrow & \hat{P}\hat{T}\RotateX{-45}{\Rightarrow}\qquad\qquad & \hat{P}\Downarrow\nonumber \\
L_{-}^{\uparrow}:\  & \det\left(\Lambda\right)=-1,\quad\Lambda_{0}^{0}>1 & \stackrel{\hat{T}}{\Rightarrow}\qquad L_{+}^{\downarrow}:\  & \det\left(\Lambda\right)=+1,\quad\Lambda_{0}^{0}<1
\end{align}
Die schwache Wechselwirkung verletzt die $\hat{P}$- und die $\hat{T}$-Symmetrie.

Stetige Transformationen führen nicht aus $L_{+}^{\uparrow}$ hinaus.
Daher muss man sich zusätzlich das Verhalten unter $\hat{P}$ und
$\hat{T}$ ansehen.
\begin{description}
\item [{Nebenbemerkung:}] $\hat{C}\hat{P}\hat{T}$ ist für jede ,,vernünftige``
Theorie eine Symmetrietransformation, wobei $\hat{C}$ die Ladungskonjugation
ist, die Teilchen und Antiteilchen vertauscht. Daher ist $\hat{C}=\hat{P}\hat{T}$.
\end{description}

\section{Das Eichprinzip}

Die Quantenmechanik ist invariant unter globalen Eichtransformationen
$\psi\left(x\right)\to e^{\ii q\Theta}\psi\left(x\right)$. In der
Quantenelektrodynamik fordert man nun die Invarianz unter den allgemeineren
\emph{lokalen} Eichtransformationen $\psi\left(x\right)\to e^{\ii q\Theta\left(x\right)}$.

Als heuristische Motivation hierfür betrachte zum Beispiel das Aharonov-Bohm-Experiment:

\begin{figure}[H]
\noindent \begin{centering}
\begin{tikzpicture}
  \draw (0,-1) -- (0,1) (0,1.3) -- (0,2.5) (0,-1.3) -- (0,-2.5);
  \draw (4,-2.5) -- (4,2.5);
  \draw (-4,0) node[above]{$e^-$} -- (0,1.15) -- (4,0);
  \draw (-4,0) -- (0,-1.15) -- (4,0);
  \draw (-1,0) node{$\vec{B}$} circle (0.7);
  \draw plot[smooth,tension=.7] coordinates{(4.5,2.5) (5,2) (4.5,1.5) (5,1) (4.5,0.5) (5,0) (4.5,-0.5) (5,-1) (4.5,-1.5) (5,-2) (4.5,-2.5)};
  \draw[orange, thick,<->] (5.5,1) -- node[right]{abhängig von $\vec{B}$} (5.5,-1);
\end{tikzpicture}
\par\end{centering}

\noindent \centering{}\caption{Aharonov-Bohm-Experiment: aufgesammelte Phase $e^{\ii\int_{s}qA_{\mu}\dd x^{\mu}}$}
\end{figure}


Das Vektorpotential $A^{\mu}$ ist nur bis auf eine Eichtransformation
bestimmt. Die Eichphasen dürfen also keine Rolle spielen.

Die freie Dirac-Gleichung
\begin{align*}
\left(\ii\gamma^{\mu}\partial_{\mu}-m\right)\psi\left(x\right) & =0
\end{align*}
ist nicht invariant unter lokalen Eichtransformationen:
\begin{align*}
\left(\ii\gamma^{\mu}\partial_{\mu}-m\right)e^{\ii q\Theta\left(x\right)}\psi\left(x\right) & =e^{\ii q\Theta\left(x\right)}\left(-q\gamma^{\mu}\left(\partial_{\mu}\Theta\right)+\ii\gamma^{\mu}\partial_{\mu}-m\right)
\end{align*}
Aus der Forderung nach Invarianz folgt, dass es etwas geben muss,
was sich so transformiert, dass die Dirac-Gleichung invariant wird.
Da $\partial_{\mu}\Theta\left(x\right)$ ein Vektorfeld ist, muss
das neue Objekt ebenfalls ein Vektorfeld sein, und zwar $A_{\mu}$
mit:
\begin{align}
\left(\ii\gamma^{\mu}\partial_{\mu}-q\gamma^{\mu}A_{\mu}\left(x\right)-m\right)\psi\left(x\right) & =0
\end{align}
Unter einer Eichtransformation transformiert $A_{\mu}$ wie folgt:
\begin{align}
A_{\mu}\left(x\right) & \to A_{\mu}\left(x\right)-\partial_{\mu}\Theta\left(x\right)
\end{align}
Aus der Forderung nach Eichinvarianz folgt die Existenz des Photons.


\subsection{Lagrangedichte der Quantenelektrodynamik}

Um zum Beispiel die Zeitentwicklung eines Elektronfeldes beschreiben
zu können, muss man wissen, wie sich $A_{\mu}\left(x\right)$ dynamisch
entwickelt, das heißt wir brauchen die Lagrangedichte des Photons
$\mathcal{L}_{A}\left(x\right)$.

Wir wissen, dass $\DD_{\mu}=\partial_{\mu}+\ii qA_{\mu}$ ein eichinvarianter
Ableitungsoperator ist, der \emph{kovariante Ableitung} genannt wird.
\begin{itemize}
\item Für die Dimension von $\mathcal{L}$ gilt:
\begin{align}
S & =\int\dd^{4}x\mathcal{L}\left(x\right)\\
1\stackrel{\hbar=1}{=}\left[S\right] & =\left[x\right]^{4}\cdot\left[\mathcal{L}\right]\nonumber \\
\Rightarrow\qquad\left[\mathcal{L}\right] & =\left[x\right]^{-4}\stackrel{\hbar c=1}{=}\left[E\right]^{4}\nonumber 
\end{align}

\item $\mathcal{L}\left(x\right)$ ist eine Lorentz-Skalardichte.
\item $\mathcal{L}\left(x\right)$ ist kein Ableitungsoperator, sondern
eine Funktion (ein Multiplikationsoperator).
\item $\mathcal{L}\left(x\right)$ muss eichinvariant sein, dass heißt sie
muss aus $\DD_{\mu}$-Operatoren aufgebaut sein.
\end{itemize}
Man beobachtet, dass $\left[\hat{D}_{\mu},\hat{D}_{\nu}\right]$ ein
Multiplikations- und kein Ableitungsoperator ist.
\begin{align*}
\left(\partial_{\mu}+\ii qA_{\mu}\right)\left(\partial_{\nu}+\ii qA_{\nu}\right) & =\partial_{\mu}\partial_{\nu}+\ii q\left(\partial_{\mu}A_{\nu}\right)+\ii qA_{\nu}\partial_{\mu}+\ii qA_{\mu}\partial_{\nu}-q^{2}A_{\mu}A_{\nu}\\
\left(\partial_{\nu}+\ii qA_{\nu}\right)\left(\partial_{\mu}+\ii qA_{\mu}\right) & =\partial_{\nu}\partial_{\mu}+\ii q\left(\partial_{\nu}A_{\mu}\right)+\ii qA_{\mu}\partial_{\nu}+\ii qA_{\nu}\partial_{\mu}-q^{2}A_{\nu}A_{\mu}
\end{align*}
\begin{align*}
\Rightarrow\qquad\left[\hat{D}_{\mu},\hat{D}_{\nu}\right] & =\ii q\left(\partial_{\mu}A_{\nu}-\partial_{\nu}A_{\mu}\right)=:\ii qF_{\mu\nu}
\end{align*}
Wählt man $q$ dimensionslos, so hat $A_{\mu}$ die Dimension einer
Energie und somit folgt $\left[F_{\mu\nu}\right]=\left[E\right]^{2}$.
\begin{align}
\stackrel{F^{\mu}\msd{\mu}=0}{\Rightarrow}\qquad\fbox{\ensuremath{{\displaystyle \mathcal{L}\left(x\right)\sim F_{\mu\nu}F^{\mu\nu}}}}
\end{align}
Die Wahl der Konstanten ist willkürlich. Damit die Gleichungen möglichst
einfach werden, wählt man folgende Lagrangedichte: 
\begin{align}
\mathcal{L}\left(x\right) & :=-\frac{1}{4}F_{\mu\nu}F^{\mu\nu}
\end{align}
Die freie Euler-Lagrange-Gleichung ist also:
\begin{align}
\frac{\partial\mathcal{L}}{\partial A_{\sigma}}-\partial^{\lambda}\frac{\partial\mathcal{L}}{\partial\left(\partial^{\lambda}A_{\sigma}\right)} & =0\nonumber \\
\Rightarrow\qquad\partial^{\lambda}F_{\lambda\sigma} & =0
\end{align}
Die Lagrange-Dichte der Quantenelektrodynamik ist somit:
\begin{align}
\mathcal{L}_{\text{QED}}\left(x\right) & =\sum_{\text{Fermionen }j}\overline{\psi}_{j}\left(x\right)\left(\slashd{\hat{p}}-eQ_{j}\slashd A\left(x\right)-m\right)\psi_{j}\left(x\right)-\frac{1}{4}F_{\mu\nu}\left(x\right)F^{\mu\nu}\left(x\right)
\end{align}
Bei der Variation sind $\overline{\psi}$ und $\psi$ als unabhängige
Variablen aufzufassen, weil eine komplexe Größe zwei reelle Freiheitsgrade
hat. Die Euler-Lagrange-Gleichungen für $A_{\sigma}$ sind:
\begin{align}
0 & =\frac{\partial\mathcal{L}_{\text{QED}}}{\partial A_{\sigma}}-\partial_{\lambda}\frac{\partial\mathcal{L}_{\text{QED}}}{\partial\left(\partial_{\lambda}A_{\sigma}\right)}=\nonumber \\
 & =\sum_{\text{Fermionen }j}\overline{\psi}_{j}\left(x\right)\left(-eQ_{j}\gamma^{\sigma}\right)\psi_{j}\left(x\right)-\left(\frac{-1}{4}\cdot4\right)\partial_{\lambda}F^{\lambda\sigma}\nonumber \\
\partial_{\lambda}F^{\lambda\sigma} & =eQ_{j}\underbrace{\sum_{\text{Fermionen }j}\overline{\psi}_{j}\left(x\right)\gamma^{\sigma}\psi_{j}\left(x\right)}_{=j^{\sigma}\left(x\right)}
\end{align}
Die Dirac-Gleichung erhält man als Euler-Lagrange-Gleichung für $\overline{\psi}_{j}$.
\begin{align*}
0 & =\frac{\partial\mathcal{L}_{\text{QED}}}{\partial\overline{\psi}_{j}}-\partial_{\lambda}\underbrace{\frac{\partial\mathcal{L}_{\text{QED}}}{\partial\left(\partial_{\lambda}\overline{\psi}_{j}\right)}}_{=0}=\left(\slashd{\hat{p}}-eQ_{j}\slashd A\left(x\right)-m\right)\psi_{j}\left(x\right)
\end{align*}
Die Euler-Lagrange-Gleichung für $\psi_{j}$ ist die konjugierte Dirac-Gleichung:
\begin{align*}
0 & =\frac{\partial\mathcal{L}_{\text{QED}}}{\partial\psi_{j}}-\partial_{\lambda}\frac{\partial\mathcal{L}_{\text{QED}}}{\partial\left(\partial_{\lambda}\psi_{j}\right)}=\overline{\psi}_{j}\left(-eQ_{j}A^{\lambda}\left(x\right)-m\right)-\partial_{\lambda}\left(\overline{\psi}_{j}\cdot\ii\gamma^{\lambda}\right)=\\
 & =\overline{\psi}_{j}\left(-\overleftarrow{\slashd p}-eQ_{j}\slashd A\left(x\right)-m\right)
\end{align*}
Dabei bedeutet $\overleftarrow{\slashd p}$, dass der Ableitungsoperator
nach links wirkt, also auf $\overline{\psi}_{j}$.


\subsection{Bilinearformen des Dirac-Feldes}

Man nennt $\overline{\psi}\left(x\right)\gamma^{\sigma}\psi\left(x\right)$
eine \emph{Bilinearform}. Physikalische Größen entsprechen Bilinearformen.
Wir kennen bereits die Vektordichte:
\begin{align*}
V^{\sigma}\left(x\right) & =\overline{\psi}\left(x\right)\gamma^{\sigma}\psi\left(x\right)
\end{align*}
$V^{\sigma}\left(x\right)$ ist hermitesch:
\begin{align*}
\left[V^{\sigma}\left(x\right)\right]^{\dagger} & =\psi^{\dagger}\left(x\right)\left(\gamma^{\sigma}\right)^{\dagger}\left(\gamma_{0}\right)^{\dagger}\psi\left(x\right)=\psi^{\dagger}\left(x\right)\left(\gamma_{0}\right)^{2}\left(\gamma^{\sigma}\right)^{\dagger}\left(\gamma_{0}\right)^{\dagger}\psi\left(x\right)=\\
 & =\overline{\psi}\left(x\right)\gamma_{0}\left(\gamma^{\sigma}\right)^{\dagger}\gamma_{0}\psi\left(x\right)=\overline{\psi}\left(x\right)\gamma^{\sigma}\psi\left(x\right)
\end{align*}
Es muss 16 unabhängige hermitesche Bilinearformen geben. Wir wählen
als Basis solche Bilinearformen, die sich aus den $\gamma$-Matrizen
berechnen lassen:
\begin{align}
S\left(x\right) & =\overline{\psi}\left(x\right)\psi\left(x\right) &  & \text{Skalardichte (1 Komponenten)}\\
V^{\sigma}\left(x\right) & =\overline{\psi}\left(x\right)\gamma^{\sigma}\psi\left(x\right) &  & \text{Vektordichte (4 Komponenten)}\\
T^{\mu\nu}\left(x\right) & =\overline{\psi}\left(x\right)\sigma^{\mu\nu}\psi\left(x\right)=\psi^{\dagger}\gamma_{0}\frac{\ii}{2}\left(\gamma^{\mu}\gamma^{\nu}-\gamma^{\nu}\gamma^{\mu}\right)\psi &  & \text{Tensordichte (6 Komponenten)}\\
A^{\sigma}\left(x\right) & =\overline{\psi}\left(x\right)\gamma^{\sigma}\gamma_{5}\psi\left(x\right) &  & \text{Axialvektordichte (4 Komponenten)}\\
P\left(x\right) & =\overline{\psi}\left(x\right)\ii\gamma_{5}\psi\left(x\right) &  & \text{Pseudoskalardichte (1 Komponenten)}
\end{align}
\begin{align*}
P^{\dagger} & =\psi^{\dagger}\left(-\ii\right)\gamma_{5}\gamma_{0}\psi\left(x\right)=\ii\overline{\psi}\gamma_{5}\psi
\end{align*}
\begin{align*}
\left(T^{\mu\nu}\left(x\right)\right)^{\dagger} & =\psi^{\dagger}\left(\gamma^{\nu\dagger}\gamma^{\mu\dagger}-\gamma^{\nu\dagger}\gamma^{\mu\dagger}\right)\left(-\frac{\ii}{2}\right)\gamma_{0}\psi=\\
 & =\psi^{\dagger}\left(\left(\gamma_{0}\right)^{2}\gamma^{\nu\dagger}\left(\gamma_{0}\right)^{2}\gamma^{\mu\dagger}-\left(\gamma_{0}\right)^{2}\gamma^{\nu\dagger}\left(\gamma_{0}\right)^{2}\gamma^{\mu\dagger}\right)\left(-\frac{\ii}{2}\right)\gamma_{0}\psi=\\
 & =-\overline{\psi}\frac{\ii}{2}\left(\gamma^{\nu}\gamma^{\mu}-\gamma^{\mu}\gamma^{\nu}\right)\psi=\overline{\psi}\sigma^{\mu\nu}\psi
\end{align*}
Die Axialvektordichte und die Pseudoskalardichte transformieren sich
unter stetigen Lorentz-Transformationen wie ein Vektor beziehungsweise
wie ein Skalar. Allerdings ändern sie unter der Paritätstransformation
ihr Vorzeichen.

Es gibt unter diesen 16 keine Bilinearform, die ein symmetrischer
Tensor 2. Stufe ist. Daher kann man so nicht an das Gravitationsfeld,
dass durch den symmetrischen Metrik-Tensor $g_{\mu\nu}$ beschrieben
wird koppeln.

%DATE: Di 23.4.13


\subsection{Die Transformationen \texorpdfstring{$\hat{C},\hat{P},\hat{T}$}{C, P, T}}


\subsubsection*{Ladungskonjugation $\hat{C}$}

Die Ladungskonjugation vertauscht Teilchen und Antiteilchen.
\begin{align}
 & \text{Elektron-Dirac-Gleichung:} & \left(\ii\gamma_{\mu}\partial^{\mu}-eA^{\mu}\gamma_{\mu}-m\right)\psi & =0\\
 & \text{Positron-Dirac-Gleichung:} & \left(\ii\gamma_{\mu}\partial^{\mu}+eA^{\mu}\gamma_{\mu}-m\right)\psi & =0
\end{align}
Überlegung: Ein Elektron, das mit Energie $E>0$ und Impuls $\vec{p}$
vorwärts in der Zeit läuft entspricht einem Positron, dass mit Energie
$-E<0$ und Impuls $-\vec{p}$ rückwärts in der Zeit läuft.
\begin{align*}
e^{\ii\left(-Et-\vec{p}\cdot\vec{x}\right)} & =e^{-\ii\left(\left(-E\right)\left(-t\right)+\vec{p}\cdot\vec{x}\right)}
\end{align*}


\begin{figure}[H]
\noindent \begin{centering}
\begin{tikzpicture}[scale=0.5]
  \draw plot[smooth,tension=.7] coordinates{(-2,-3) (-0.5,-2) (0.5,2) (2,3)};
  \node at (-2,-3) [left] {$t_1$};
  \node at (2,3) [right] {$t_2$};
  \draw[thick,->] (-0.9,-1) -- node[left]{$e^-$} (-0.5,1);
  \draw[thick,<-] (0.5,-1) -- node[right]{$e^+$} (0.9,1);
\end{tikzpicture}
\par\end{centering}

\noindent \centering{}\caption{In der Zeit vorwärts laufendes Elektron entspricht in der Zeit rückwärts
laufendem Positron.}
\end{figure}


Hieraus motivieren wir den Ansatz:
\begin{align}
\psi_{C}\left(x\right) & =C\psi^{*}\left(x\right)
\end{align}
Dabei ist $C\in\text{GL}_{4}\left(\mathbb{C}\right)$ eine beliebige
invertierbare Matrix. Komplexe Konjugation der Elektron-Dirac-Gleichung
und Einfügen von $C$ liefert:
\begin{align*}
C\left(-\ii\gamma_{\mu}^{*}\partial^{\mu}-eA^{\mu}\gamma_{\mu}^{*}-m\right)C^{-1}C\psi^{*} & =0\\
\left(-\ii C\gamma_{\mu}^{*}C^{-1}\partial^{\mu}-eA^{\mu}C\gamma_{\mu}^{*}C^{-1}-m\right)\psi_{C} & =0
\end{align*}
Damit dies in die Positron-Dirac-Gleichung übergeht, muss
\begin{align*}
C\gamma_{\mu}^{*}C^{-1} & =-\gamma_{\mu}
\end{align*}
gelten. Die Lösung davon ist:
\begin{align*}
C & =\gamma_{2}e^{\ii\varphi} & C^{-1} & =-\gamma_{2}e^{-\ii\varphi}
\end{align*}
\begin{align*}
C\gamma_{2}^{*}C^{-1} & =\gamma_{2}e^{\ii\varphi}\left(-\gamma_{2}\right)\left(-\gamma_{2}\right)e^{-\ii\varphi}=\gamma_{2}^{3}=-\gamma_{2}
\end{align*}
Für $\mu\not=2$ gilt:
\begin{align*}
C\gamma_{\mu}^{*}C^{-1} & =\gamma_{2}e^{\ii\varphi}\gamma_{\mu}\left(-\gamma_{2}\right)e^{-\ii\varphi}=-\gamma_{2}\gamma_{\mu}\gamma_{2}=\gamma_{\mu}\gamma_{2}\gamma_{2}=-\gamma_{\mu}
\end{align*}
Damit folgt:
\begin{align}
\psi_{C} & =e^{\ii\varphi}\left(\begin{array}{cccc}
0 & 0 & 0 & \ii\\
0 & 0 & -\ii & 0\\
0 & -\ii & 0 & 0\\
\ii & 0 & 0 & 0
\end{array}\right)\psi^{*}
\end{align}
Beispiel:
\begin{align*}
\psi_{4} & =v\left(p,+\right)e^{\ii p\cdot x}=\sqrt{\frac{E+m}{2m}}\left(\begin{array}{c}
\frac{p_{1}-\ii p_{2}}{E+m}\\
\frac{-p_{3}}{E+m}\\
0\\
1
\end{array}\right)e^{\ii px}\\
\psi_{1} & =u\left(p,+\right)e^{-\ii p\cdot x}=\sqrt{\frac{E+m}{2m}}\left(\begin{array}{c}
1\\
0\\
\frac{p_{3}}{E+m}\\
\frac{p_{1}+\ii p_{2}}{E+m}
\end{array}\right)e^{-\ii px}
\end{align*}
\begin{align*}
\left(\psi_{1}\right)_{C} & =e^{\ii\varphi}\left(\begin{array}{cccc}
0 & 0 & 0 & \ii\\
0 & 0 & -\ii & 0\\
0 & -\ii & 0 & 0\\
\ii & 0 & 0 & 0
\end{array}\right)\sqrt{\frac{E+m}{2m}}\left(\begin{array}{c}
1\\
0\\
\frac{p_{3}}{E+m}\\
\frac{p_{1}-\ii p_{2}}{E+m}
\end{array}\right)e^{\ii px}=\\
 & =\ii e^{\ii\varphi}\sqrt{\frac{E+m}{2m}}\left(\begin{array}{c}
\frac{p_{1}-\ii p_{2}}{E+m}\\
\frac{-p_{3}}{E+m}\\
0\\
1
\end{array}\right)e^{-\ii px}=\ii e^{\ii\varphi}\psi_{4}
\end{align*}
Ein physikalisches Positron $\psi_{C}$ mit positiver Energie ist
äquivalent zu einer Lösung $\psi$ mit negativer Energie.


\subsubsection*{Paritätstransformation $\hat{P}$}

Die Paritätstransformation ist eine räumliche Spiegelung am Ursprung:
\begin{align}
x^{\mu} & \to\left(x'\right)^{\mu}=\left(x^{0},-\vec{x}\right)
\end{align}
Wir machen den Ansatz:
\begin{align}
\psi_{P}\left(x'\right) & =\psi\left(x^{0},-\vec{x}\right)=P\psi\left(x\right)
\end{align}
Die Dirac-Gleichung für $x'$ ist:
\begin{align*}
\left(\ii\gamma_{\mu}\frac{\partial}{\partial\left(x'\right)^{\mu}}-e\gamma^{\mu}A_{\mu}\left(x'\right)-m\right)\psi_{P}\left(x'\right) & =0
\end{align*}
Die soll äquivalent sein zu:
\begin{align*}
\left(\ii\gamma^{\mu}\frac{\partial}{\partial x^{\mu}}-e\gamma^{\mu}A_{\mu}\left(x\right)-m\right)\psi\left(x\right) & =0
\end{align*}
Wegen $\vec{E}\stackrel{P}{\to}-\vec{E}$ und $\vec{B}\stackrel{P}{\to}\vec{B}$
sowie $\nabla\stackrel{P}{\to}-\nabla$ und $\partial_{t}\stackrel{P}{\to}\partial_{t}$
folgt $A^{0}\stackrel{P}{\to}A^{0}$ und $A^{j}\stackrel{P}{\to}-A^{j}$.
\begin{align*}
\Rightarrow\qquad P^{-1}\left(\ii\gamma^{0}\frac{\partial}{\partial x^{0}}-\ii\gamma^{j}\frac{\partial}{\partial x^{j}}-e\gamma^{0}A_{0}+e\gamma^{j}A_{j}-m\right)P\psi & =0
\end{align*}
Es muss also gelten:
\begin{align*}
P^{-1}\gamma^{0}P & =\gamma^{0} & P^{-1}\gamma^{j}P & =-\gamma^{j}
\end{align*}
\begin{align}
\Rightarrow\qquad P & =e^{\ii\chi}\gamma_{0} & P^{-1} & =e^{-\ii\chi}\gamma_{0}
\end{align}



\subsubsection*{Zeitumkehrinvarianz $\hat{T}$}

Wegen $\vec{E}\stackrel{T}{\to}\vec{E}$ und $\vec{B}\stackrel{T}{\to}-\vec{B}$
sowie $\nabla\stackrel{T}{\to}\nabla$ und $\partial_{t}\stackrel{T}{\to}-\partial_{t}$
folgt $A^{0}\stackrel{T}{\to}A^{0}$ und $A^{j}\stackrel{T}{\to}-A^{j}$.
Wir machen den Ansatz:
\begin{align}
\psi_{T}\left(x'\right) & =T\psi^{*}\left(x\right)
\end{align}
Die Dirac-Gleichung für $\psi_{T}$ ist:
\begin{align*}
\left(-\ii\gamma^{0}\frac{\partial}{\partial x^{0}}+\ii\gamma^{j}\frac{\partial}{\partial x^{j}}-e\gamma^{0}A_{0}\left(x\right)+e\gamma^{j}A_{j}\left(x\right)-m\right)T\psi^{*}\left(x\right) & =0\qquad/^{*}\\
\left(T^{*}\right)^{-1}\left(\ii\gamma^{0}\frac{\partial}{\partial x^{0}}-\ii\left(\gamma^{j}\right)^{*}\frac{\partial}{\partial x^{j}}-e\gamma^{0}A_{0}\left(x\right)+e\left(\gamma^{j}\right)^{*}A_{j}\left(x\right)-m\right)T^{*}\psi\left(x\right) & =0
\end{align*}
Nun muss gelten:
\begin{align*}
\left(T^{*}\right)^{-1}\gamma^{0}T^{*} & =\gamma^{0} & \left(T^{*}\right)^{-1}\left(\gamma^{j}\right)^{*}T^{*} & =-\gamma^{j}
\end{align*}
Das bedeutet:
\begin{align*}
\left(T^{*}\right)^{-1}\gamma^{0}T^{*} & =\gamma^{0} & \left(T^{*}\right)^{-1}\gamma^{1}T^{*} & =-\gamma^{1}\\
\left(T^{*}\right)^{-1}\gamma^{2}T^{*} & =\gamma^{2} & \left(T^{*}\right)^{-1}\gamma^{3}T^{*} & =-\gamma^{3}
\end{align*}
Die Lösung ist:
\begin{align}
T & =\gamma^{1}\gamma^{3}e^{\ii\xi} & T^{-1} & =\gamma^{3}\gamma^{1}e^{-\ii\xi}=T^{*}
\end{align}
Probe:
\begin{align*}
T^{-1}T & =\gamma^{3}\gamma^{1}\gamma^{1}\gamma^{3}=-\gamma^{3}\gamma^{3}=\mathbbm{1}
\end{align*}
\begin{align*}
\gamma^{1}\gamma^{3}\gamma^{0}\gamma^{3}\gamma^{1} & =\left(-1\right)^{2}\gamma^{0}\gamma^{1}\gamma^{3}\gamma^{3}\gamma^{1}=\gamma^{0}\\
\gamma^{1}\gamma^{3}\gamma^{1}\gamma^{3}\gamma^{1} & =-\gamma^{1}\gamma^{1}\gamma^{3}\gamma^{3}\gamma^{1}=-\gamma^{1}\\
\gamma^{1}\gamma^{3}\gamma^{2}\gamma^{3}\gamma^{1} & =\left(-1\right)^{2}\gamma^{2}\gamma^{1}\gamma^{3}\gamma^{3}\gamma^{1}=\gamma^{2}\\
\gamma^{1}\gamma^{3}\gamma^{3}\gamma^{3}\gamma^{1} & =-\gamma^{1}\gamma^{3}\gamma^{1}=\gamma^{1}\gamma^{1}\gamma^{3}=-\gamma^{3}
\end{align*}



\chapter{Greensche Funktion (Feynman-Propagator)}

Erinnerung: Das elektrische Potential $\phi$ erfüllt in der Elektrostatik
die Differentialgleichung:
\begin{align}
\upDelta\phi & =4\pi\rho
\end{align}
Dabei ist $\rho$ die Ladungsdichte. Die Greensche Funktion $G\left(x,x'\right)$
ist definiert durch:
\begin{align}
\upDelta_{x}G\left(x,x'\right) & =\delta^{\left(3\right)}\left(x-x'\right)
\end{align}
Damit ergibt sich die Lösung obiger Differentialgleichung zu:
\begin{align}
\phi\left(x\right) & =4\pi\int\dd^{3}x'G\left(x,x'\right)\rho\left(x'\right)
\end{align}
Probe:
\begin{align*}
\upDelta_{x}\phi\left(x\right) & =4\pi\int\dd^{3}x'\delta\left(x-x'\right)\rho\left(x'\right)=4\pi\rho\left(x\right)
\end{align*}



\section{Greensche Funktion der Klein-Gordon-Gleichung}

Die definierende Gleichung ist:
\begin{align}
\left(\square_{x}+m^{2}\right)G\left(x-x'\right) & =-\delta^{\left(4\right)}\left(x-x'\right)
\end{align}
Das Minus auf der rechten Seite ist Konvention und wird wegen $\square_{x}=-\hat{p}^{2}$
eingefügt.\\
Die Dirac-Greensche Funktion ist einfach:
\begin{align}
S\left(x-x'\right) & =\left(\ii\gamma^{\mu}\partial_{x,\mu}+m\right)G\left(x-x'\right)
\end{align}
Es gilt nämlich:
\begin{align*}
\left(\ii\gamma^{\nu}\partial_{x,\nu}-m\right)S\left(x-x'\right) & =\left(-\square_{x}-m^{2}\right)G\left(x-x'\right)=\delta^{\left(4\right)}\left(x-x'\right)
\end{align*}
Am einfachsten ist die Bestimmung von $G$ im Impulsraum. Wir führen
also eine Fourier-Transformation durch:
\begin{align}
G\left(x-x'\right) & =\int\frac{\dd^{4}p}{\left(2\pi\right)^{4}}G\left(p\right)e^{-\ii p\left(x-x'\right)}\label{eq:Fouriertrafo-Green}
\end{align}
\begin{align*}
\left(\square_{x}+m^{2}\right)G\left(x-x'\right) & =\int\frac{\dd^{4}p}{\left(2\pi\right)^{4}}\left(-p^{\mu}p_{\mu}+m^{2}\right)G\left(p\right)e^{-\ii p\left(x-x'\right)}\stackrel{!}{=}-\int\frac{\dd^{4}p}{\left(2\pi\right)^{4}}e^{-\ii p\left(x-x'\right)}
\end{align*}
Daher muss gelten:
\begin{align}
\left(p^{2}-m^{2}\right)G\left(p\right) & =1
\end{align}
Ist $p^{2}-m^{2}\not=0$, so folgt:
\begin{align*}
G\left(p\right) & =\frac{1}{p^{2}-m^{2}}
\end{align*}
Was passiert für $p^{2}=\left(p^{0}\right)^{2}-\vec{p}^{2}=m^{2}$?
\begin{align}
p^{0} & =\pm\sqrt{\vec{p}^{2}+m^{2}}
\end{align}
Da für $p^{2}=m^{2}$ Divergenzen auftreten, muss man (\ref{eq:Fouriertrafo-Green})
um eine Zusatzvorschrift ergänzen, wie man diese umschiffen soll.

\begin{figure}[H]
\noindent \begin{centering}
\hfill{}\subfloat[Integrationswege um die Pole herum]{\begin{tikzpicture}
  \draw[thick,->] (-3,0) -- (3,0) node[right]{Re$(p^\mu)$};
  \draw[thick,->] (0,-3) -- (0,3) node[above]{Im$(p^\mu)$};
  \draw[very thick, orange] (-3,0) -- (-2,0) arc (-180:0:0.25) -- (1.5,0) arc(-180:0:0.25) node[below=7pt]{1} -- (3,0);
  \draw[very thick, green] (-3,0) -- (-2.1,0) arc (180:0:0.35) -- (1.4,0) node[below =3pt]{2} arc(-180:0:0.35) -- (3,0);
  \draw[very thick, red] (-3,0) -- (-2.1,0) arc (-180:0:0.35) -- (1.4,0) node[above =3pt]{3} arc(180:0:0.35) -- (3,0);
  \draw[very thick, blue] (-3,0) -- (-2,0) arc (180:0:0.25) -- (1.5,0) arc(180:0:0.25) node[above=7pt]{4} -- (3,0);
  \draw (1.75,0) node[below=15pt]{$\sqrt{\vec{p}^2-m^2}$} +(-0.1,0.1) -- +(0.1,-0.1) +(-0.1,-0.1) -- +(0.1,0.1);
  \draw (-1.75,0) node[below=15pt]{$-\sqrt{\vec{p}^2-m^2}$} +(-0.1,0.1) -- +(0.1,-0.1) +(-0.1,-0.1) -- +(0.1,0.1);
\end{tikzpicture}}\hfill{}\subfloat[äquivalent: Pole verschieben]{{\begin{tikzpicture}
  \draw[thick,->] (-3,0) -- (3,0) node[right]{Re$(p^\mu)$};
  \draw[thick,->] (0,-3) -- (0,3) node[above]{Im$(p^\mu)$};
  \draw[fill, orange] (-1.75,0.25) circle (2pt) (1.75,0.25) circle (2pt);
  \draw[fill, green] (-1.75,-0.5) circle (2pt) (1.75,0.5) circle (2pt);
  \draw[fill, red] (-1.75,0.5) circle (2pt) (1.75,-0.5) circle (2pt);
  \draw[fill, blue] (-1.75,-0.25) circle (2pt) (1.75,-0.25) circle (2pt);
  \draw (1.75,0) node[below=15pt]{$\sqrt{\vec{p}^2-m^2}$} +(-0.1,0.1) -- +(0.1,-0.1) +(-0.1,-0.1) -- +(0.1,0.1);
  \draw (-1.75,0) node[below=15pt]{$-\sqrt{\vec{p}^2-m^2}$} +(-0.1,0.1) -- +(0.1,-0.1) +(-0.1,-0.1) -- +(0.1,0.1);
\end{tikzpicture}}}\hspace*{\fill}
\par\end{centering}

\noindent \centering{}\caption{Die Pole müssen umgangen werden.}
\end{figure}


Mit $\varepsilon,\eta\in\mathbb{R}_{>0}$ lassen sich die vier Möglichkeiten
schreiben als:
\begin{enumerate}
\item ${\displaystyle \frac{1}{p^{2}-m^{2}-\ii\varepsilon\text{sgn}\left(p^{0}\right)}}$:
Pole bei $p^{0}=\pm\sqrt{\vec{p}^{2}+m^{2}}+\ii\eta$
\item ${\displaystyle \frac{1}{p^{2}-m^{2}-\ii\varepsilon}}$: Pole bei
$p^{0}=\pm\sqrt{\vec{p}^{2}+m^{2}}\pm\ii\eta$
\item ${\displaystyle \frac{1}{p^{2}-m^{2}+\ii\varepsilon}}$: Pole bei
$p^{0}=\pm\sqrt{\vec{p}^{2}+m^{2}}\mp\ii\eta$
\item ${\displaystyle \frac{1}{p^{2}-m^{2}+\ii\varepsilon\text{sgn}\left(p^{0}\right)}}$:
Pole bei $p^{0}=\pm\sqrt{\vec{p}^{2}+m^{2}}-\ii\eta$
\end{enumerate}
Mit Hilfe des Residuuensatzes kann man so das Integral

\begin{align*}
\int\frac{\dd^{3}p}{\left(2\pi\right)^{3}}\int_{-\infty}^{\infty}\frac{\dd p^{0}}{2\pi}e^{-\ii p^{0}\left(t-t'\right)+\ii\vec{p}\left(\vec{x}-\vec{x}'\right)}G\left(p^{0},\vec{p}\right)
\end{align*}
ausrechnen. Für $t'>t$ (Propagation von $t'$ nach $t$ rückwärts
in der Zeit) kann man oben schließen, da dann der Faktor $e^{-\ii^{2}\text{Im}\left(p^{0}\right)\left(t-t'\right)}$
exponentiell abfällt. Ebenso kann man für $t>t'$ (vorwärts in der
Zeit) unten schließen. Bei der Propagation vorwärts in der Zeit darf
der Pol bei $p_{0}<0$ nicht im Integrationsbereich sein und $p_{0}>0$
darf nicht rückwärts in der Zeit propagieren. Daher bleibt nur die
3. Lösung. Diese wird \emph{Feynman-Propagator} genannt.


\chapter{Kanonische Quantisierung \texorpdfstring{$\hat{b}^{\dagger},\hat{d}^{\dagger},\hat{b},\hat{d}$}{}}


\chapter{Feynman-Regeln}


\chapter{Elektron-Myon-Streuung}


\chapter{Pauli-Villars-Regularisierung (Renormierung)}


\chapter{\texorpdfstring{$SU\left(N\right)$}{SU(N)} Eichgruppen}


\chapter{Tiefinelastische Streuung}


\chapter{Lagrangedichte der QCD, Feynman-Regeln}


\chapter{DGLAP-Gleichung}


\chapter{Dimensionale Regularisierung \texorpdfstring{$\overline{MS}$}{}}


\chapter{Kopplungskonstante \texorpdfstring{$\alpha_{S}\left(Q^{2}\right)$}{}}




\chapter{Standard-Modell\texorpdfstring{ $SU\left(2\right)\times U\left(1\right)\times SU\left(3\right)$}{},
SSB und Higgs-Feld}


\chapter{Weitere Themen}
\begin{itemize}
\item Anomalien $\to$ ganze Familien
\begin{align*}
\fbox{\ensuremath{{\displaystyle \left(\begin{array}{c}
\nu_{e}\\
e
\end{array}\right),\left(\begin{array}{c}
u\\
d
\end{array}\right)_{\text{r,b,g}}}}}
\end{align*}

\item Inflation (frühes Universum)
\item Supersymmetrie
\item Stringtheorie
\item $\ldots$
\end{itemize}

\part*{Anhang\thispagestyle{empty}}

\addcontentsline{toc}{part}{Anhang}

\fancyhead[R]{Index}
\fancyhead[C]{Anhang}


\chapter*{Danksagungen}

\addcontentsline{toc}{section}{\hspace*{2.7em}Danksagungen}

\fancyhead[R]{Danksagungen}

Mein besonderer Dank geht an Professor Schäfer, der diese Vorlesung
hielt und es mir gestattete, diese Vorlesungsmitschrift zu veröffentlichen.

Außerdem möchte ich mich ganz herzlich bei allen bedanken, die durch
aufmerksames Lesen Fehler gefunden und mir diese mitgeteilt haben.

\vspace{1cm}


\hfill{}Andreas Völklein

\label{END}
\end{document}
