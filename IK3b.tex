%% LyX 2.0.6 created this file.  For more info, see http://www.lyx.org/.
%% Do not edit unless you really know what you are doing.
\documentclass[11pt,english,ngerman]{scrreprt}
\usepackage[T1]{fontenc}
\usepackage[utf8]{inputenc}
\usepackage[a4paper]{geometry}
\geometry{verbose,tmargin=2.6cm,bmargin=3.5cm,lmargin=2.6cm,rmargin=2.6cm}
\usepackage{fancyhdr}
\pagestyle{fancy}
\setlength{\parskip}{\smallskipamount}
\setlength{\parindent}{0pt}
\usepackage{babel}
\usepackage{float}
\usepackage{textcomp}
\usepackage{url}
\usepackage{amsmath}
\usepackage{amssymb}
\usepackage{graphicx}
\usepackage{esint}
\usepackage[unicode=true,
 bookmarks=true,bookmarksnumbered=true,bookmarksopen=false,
 breaklinks=true,pdfborder={0 0 0},backref=page,colorlinks=false]
 {hyperref}
\hypersetup{pdftitle={Quantenfeldtheorie},
 pdfauthor={Andreas Völklein},
 pdfkeywords={Quantenfeldtheorie, Physik}}

\makeatletter

%%%%%%%%%%%%%%%%%%%%%%%%%%%%%% LyX specific LaTeX commands.
\providecommand{\LyX}{\texorpdfstring%
  {L\kern-.1667em\lower.25em\hbox{Y}\kern-.125emX\@}
  {LyX}}
\newcommand{\noun}[1]{\textsc{#1}}

%%%%%%%%%%%%%%%%%%%%%%%%%%%%%% Textclass specific LaTeX commands.
\usepackage{enumitem}		% customizable list environments
\newlength{\lyxlabelwidth}      % auxiliary length 

\@ifundefined{date}{}{\date{}}
%%%%%%%%%%%%%%%%%%%%%%%%%%%%%% User specified LaTeX commands.
\usepackage{tikz,pgfplots}
%\usepackage{tikz-3dplot,cancel,polynom}
\usetikzlibrary{matrix,arrows,calc,decorations.pathmorphing,decorations.markings,decorations.pathreplacing,intersections,trees}
\usetikzlibrary{external}
\tikzexternalize
\usepackage{latexsym,stmaryrd,stackrel,braket,bbm,subfig,framed,esvect,scrhack,calc,slashed,leftidx}
\usepackage [OMLmathrm,OMLmathbf,sfdefault=fav]{isomath}
\usepackage[explicit]{titlesec}
\usepackage[activate]{pdfcprot}

\pgfkeys{/pgf/number format/dec sep={\text{,}}}
\pgfplotsset{compat=newest}

% Inhaltsverzeichnis
\usepackage[subfigure]{tocloft}

\tocloftpagestyle{fancy}

\renewcommand{\cftchapindent}{1 em}
\renewcommand{\cftchapnumwidth}{1.5 em}

\renewcommand{\cftsecindent}{2.7 em}
\renewcommand{\cftsecnumwidth}{2.5em}

\renewcommand{\cftsubsecindent}{5.2 em}
\renewcommand{\cftsubsecnumwidth}{3.8 em}

\renewcommand{\cftsubsubsecindent}{9 em}
\renewcommand{\cftsubsubsecnumwidth}{4.5 em}

% Mathe-Operatoren
\DeclareMathOperator*{\exsop}{\exists}
\DeclareMathOperator*{\exsgop}{\exists!}
\DeclareMathOperator*{\fallop}{\forall}
\DeclareMathOperator*{\bcupdop}{\dot{\bigcup}}
\DeclareMathOperator*{\bcapdop}{\dot{\bigcap}}

%Operatornorm
\newcommand{\opnor}[1]{\abs{\hspace*{-1.1pt}\norm{#1}\hspace*{-1.1pt}}}

% nicht-totales Differential
\newcommand{\dBar}{\mathchar'26\mkern-12mu \textnormal{d}}

% Angström
\newcommand{\ang}{\textup{\AA}}

% schöne Vektorpfeile
\renewcommand{\vec}[1]{\vv{#1}}

% Rotieren
\newcommand{\Rotate}[1]{
\tikzset{external/export next=false}
\begin{tikzpicture}[baseline=(X.base)]
\node[rotate=90] (X){\ensuremath{#1}};
\end{tikzpicture}
}

% Rotieren
\newcommand{\RotateX}[2]{
\tikzset{external/export next=false}
\begin{tikzpicture}[baseline=(X.base)]
\node[rotate=#1] (X) {\ensuremath{#2}};
\end{tikzpicture}
}

%QED-Zeichen (Box)
\newcommand{\qed}{\ensuremath{\Box}}
\newcommand{\qqed}[1][\arabic{chapter}.\arabic{section}\ifnum\arabic{subsection}>0{.\arabic{subsection}}\fi]{\hspace*{1mm}\hfill\qed\ensuremath{_{\text{#1}}}}

% Mengen Modulo
\newcommand{\moduloT}[2]{
\mbox{\raisebox{0.6ex}{\ensuremath{\displaystyle #1}}
{\hspace*{-1.5mm}\Large /}
\raisebox{-0.6ex}{\hspace*{-1.5mm}\ensuremath{\displaystyle #2}}
}}

% Links Modulo
\newcommand{\lmoduloT}[2]{
\mbox{\raisebox{-0.6ex}{\ensuremath{\displaystyle #1}}
{\hspace*{-1.5mm}\Large \ensuremath{\backslash}}
\raisebox{0.6ex}{\hspace*{-1.5mm}\ensuremath{\displaystyle #2}}
}}

% Für Z/2Z, um nicht soviel schreiben zu müssen
\newcommand{\modloT}[2]{\moduloT{ \mathbb{#1}}{#2\mathbb{#1}}}

%Laplace-Beltrami-Operator
\newcommand{\LBO}{
\begin{minipage}{6mm}
 \tikzset{external/export next=false}
 \begin{tikzpicture}
   \node at (0,0){$\Delta$};
   \draw[line width=0.75] (0.25,-0.13) -- (0.1,0.15);
 \end{tikzpicture}
\end{minipage}
}

%Die Modulo-Kommandos in klein, für die Darstellungen unter Quantoren.
\newcommand{\moduloScriptT}[2]{
\mbox{\raisebox{0.4ex}{\scriptsize\ensuremath{\displaystyle #1}}
{\hspace*{-1.5mm}\footnotesize /}
\raisebox{-0.4ex}{\hspace*{-1.5mm}\scriptsize\ensuremath{\displaystyle #2}}
}}

\newcommand{\lmoduloScriptT}[2]{
\mbox{\raisebox{-0.4ex}{\scriptsize\ensuremath{\displaystyle #1}}
{\hspace*{-1.5mm}\footnotesize \ensuremath{\backslash}}
\raisebox{0.4ex}{\hspace*{-1.5mm}\scriptsize\ensuremath{\displaystyle #2}}
}}

\newcommand{\modloScriptT}[2]{\moduloScriptT{ \mathbb{#1}}{#2\mathbb{#1}}}

% stehendes Winkelzeichen
\newcommand{\winkel}{
\tikzset{external/export next=false}
\begin{tikzpicture}[scale=0.25]
\draw ({-2+3^(1/2)},0) -- (0,1) -- ({2-3^(1/2)},0);
\draw ($(0,1) + ({cos(235)*0.7},{sin(315)*0.7})$) arc (235:315:0.7);
\end{tikzpicture}}

% Wurzel mit Häkchen
\newcommand{\hsqrt}[2][{}]{\setbox0=\hbox{$\sqrt[#1]{\phantom{|}\!\! #2\hspace*{1pt}}$}\dimen0=\ht0
  \advance\dimen0-0.2\ht0
  \setbox2=\hbox{\vrule height\ht0 depth -\dimen0}
  {\box0\lower0.4pt\box2}}

% Damit nicht immer "Kapitel 1" etc. über der Kapitelüberschrift steht
\titleformat{\chapter}
  {\huge\bfseries}
  {\textrm{\thechapter} }{0pt}
  {\textrm{#1} \thispagestyle{fancy}
  }

% Neudefinition der Abschnittsmarker für die Kopfzeile
\renewcommand\partmark[1]{\markboth{#1}{}}
\renewcommand\chaptermark[1]{\markright{\arabic{chapter} #1}}
\renewcommand\sectionmark[1]{}
\renewcommand\subsectionmark[1]{}

% Schriften auf Serif umstellen
\addtokomafont{descriptionlabel}{\rmfamily}
\addtokomafont{disposition}{\rmfamily}

% Zeilenumbrüche in Gleichungen
 \allowdisplaybreaks

% Für Feynman-Diagramme
\tikzset{
  position/.style={thick, fill=black,circle,inner sep=1.5pt},
  photon/.style={thick, solid, draw=orange, decorate, decoration={snake}},
  electron/.style={thick, solid, draw=blue, postaction={decorate}, decoration={markings,mark=at position 0.65 with {\arrow[scale=1]{>}} }},
  positron/.style={thick, solid, draw=blue, postaction={decorate}, decoration={markings,mark=at position 0.65 with {\arrow[scale=1]{<}} }},
  scalar/.style={thick, solid, draw=black},
  boson/.style={thick, dashed, draw=black},
  gluon/.style={thick, solid, decorate, draw=cyan, decoration={coil,amplitude=4pt, segment length=5pt}} 
}


% Kopf- und Fußzeile
% Höhe der Kopfzeile
\setlength{\headheight}{14pt}
% obere Trennlinie
%\renewcommand{\headrulewidth}{0.4pt}
\fancyhf{} %alle Kopf- und Fußzeilenfelder bereinigen
\fancyhead[L]{\textbf{IK3b - Quantenfeldtheorie}} %Kopfzeile links
%\fancyhead[C]{\leftmark} %zentrierte Kopfzeile
\fancyhead[R]{\rightmark} %Kopfzeile rechts
\fancyfoot[C]{\thepage\quad\!\!\!\slash\quad\!\!\!\pageref{END-front}} %Seitenzahl der Front-Matter

\AtBeginDocument{
  \def\labelitemi{\normalfont\bfseries{--}}
  \def\labelitemii{\(\circ\)}
  \def\labelitemiii{\(\triangleright\)}
}

\makeatother

\begin{document}






\global\long\def\norm#1{\left\lVert #1\right\rVert }


\global\long\def\abs#1{\left\lvert #1\right\rvert }


\global\long\def\opnorm#1{\opnor{#1}}


\global\long\def\BRA#1{\Bra{#1}}


\global\long\def\KET#1{\Ket{#1}}


\global\long\def\BraKet#1{\Braket{#1}}


\global\long\def\mins{\textnormal{-}}


\global\long\def\LB{\LBO}


\global\long\def\exs{\exsop}


\global\long\def\exsg{\exsgop}


\global\long\def\fall{\fallop}


\global\long\def\bcupd{\bcupdop}


\global\long\def\bcapd{\bcapdop}


\global\long\def\sr#1#2#3{\underset{#3}{\overset{#2}{#1}}}


\global\long\def\dd{\textnormal{d}}


\global\long\def\DD{\textnormal{D}}


\global\long\def\dbar{\dBar}


\global\long\def\angs{\ang}


\global\long\def\TT{\textnormal{T}}


\global\long\def\ii{\textbf{i}}


\global\long\def\slashd#1{\slashed{#1}}


\global\long\def\modulo#1#2{\moduloT{#1}{#2}}


\global\long\def\lmodulo#1#2{\lmoduloT{#1}{#2}}


\global\long\def\modlo#1#2{\modloT{#1}{#2}}


\global\long\def\moduloScript#1#2{\moduloScriptT{#1}{#2}}


\global\long\def\lmoduloScript#1#2{\lmoduloScriptT{#1}{#2}}


\global\long\def\modloScript#1#2{\modloScriptT{#1}{#2}}


\global\long\def\vek#1{\vectorsym{#1}}


\global\long\def\mat#1{\matrixsym{#1}}


\global\long\def\ten#1{\tensorsym{#1}}


\global\long\def\msd#1{\mathstrut_{#1}}


\global\long\def\msu#1{\mathstrut^{#1}}


\pagenumbering{roman}


\title{\hspace*{1mm}\vspace*{-20mm}\\
{\Huge{Integrierter Kurs IIIb}}\\
{\Huge{Quantenfeldtheorie}}}


\author{\vspace*{-5mm}\\
\textit{\small{Vorlesung von}}\\
\textit{\noun{\small{Prof. Dr. Andreas Schäfer}}}\\
\textit{\small{im Sommersemester 2013}}\\
\textit{\small{Überarbeitung und Textsatz in \LyX{} von}}\\
\textit{\noun{\small{Andreas Völklein}}}\\
\vspace*{5mm}\\
\includegraphics[clip,width=15cm]{unir}\\
\vspace*{3mm}\\
{\normalsize{Stand: \today}}\\
\vspace*{-30mm}}

\maketitle
\fancyhead[R]{Lizenz}


\subsubsection*{ACHTUNG}

Diese Mitschrift ersetzt \emph{nicht} die Vorlesung.

Es wird daher \emph{dringend} empfohlen, die Vorlesung zu besuchen.

\vfill{}


\selectlanguage{english}%

\subsubsection*{Copyright Notice}

Copyright © 2013 \noun{Andreas Völklein}

Permission is granted to copy, distribute and/or modify this document
under the terms of the GNU Free Documentation License, Version 1.3
or any later version published by the Free Software Foundation;

with no Invariant Sections, no Front-Cover Texts, and no Back-Cover
Texts.

A copy of the license is included in the document entitled “GFDL”.


\subsubsection*{Disclaimer of Warranty}

\noun{Unless otherwise mutually agreed to by the parties in writing
and to the extent not prohibited by applicable law, }\textbf{\noun{the
Copyright Holders and any other party, who may distribute the Document
as permitted above,   provide the Document “as is}}\textbf{”,}\textbf{\noun{
without warranty of any kind}}\noun{, expressed, implied, statutory
or otherwise, including, but not limited to, the implied warranties
of merchantability, fitness for a particular purpose, non-infringement,
the absence of latent or other defects, accuracy, or the absence of
errors, whether or not discoverable.}


\subsubsection*{Limitation of Liability}

\textbf{\noun{In no event}}\noun{ unless required by applicable law
or agreed to in writing }\textbf{\noun{will the Copyright Holders,
or any other party, who may distribute the Document as permitted above,
be liable to you for any damages}}\noun{, including, but not limited
to, any general, special, incidental, consequential, punitive or exemplary
damages, however caused, regardless of the theory of liability, arising
out of or related to this license or any use of or inability to use
the Document, even if they have been advised of the possibility of
such damages.}

\textbf{\noun{In no event will the Copyright Holders'/Distributor's
liability to you}}\noun{, whether in contract, tort (including negligence),
or otherwise, }\textbf{\noun{exceed the amount you paid the Copyright
Holders/Distributor}}\noun{ for the document under this agreement.}

\selectlanguage{ngerman}%

\subsubsection*{Links}

Der Text der „\foreignlanguage{english}{GNU Free Documentation License}“
kann auch auf der Seite
\begin{quote}
\url{https://www.gnu.org/licenses/fdl-1.3.de.html}
\end{quote}
nachgelesen werden.

Eine transparente Kopie der aktuellen Version dieses Dokuments kann
von
\begin{quote}
\url{https://github.com/andiv/IK3b}
\end{quote}
heruntergeladen werden.

\newpage{}

\fancyhead[R]{Literatur}


\subsection*{Literatur}
\begin{itemize}
\item \noun{Andreas Schäfer, Florian Rappl}: \foreignlanguage{english}{\emph{Quantum
electrodynamics}}; 2010\\
\url{http://www-nw.uni-regensburg.de/~sca14496/QED/Quantenelektrodynamik.pdf}
\item \noun{Andreas Schäfer}: \foreignlanguage{english}{\emph{Quantum Chromodynamics}};\\
\url{http://www-nw.uni-regensburg.de/~sca14496/index.html}
\item \noun{Elliot Leader, Enrico Predazzi}: \foreignlanguage{english}{\emph{An
introduction to gauge theories and modern particle physics I}}; \foreignlanguage{english}{Cambridge
University Press}, 2004; ISBN: 0-521-46840-X
\item \noun{Elliot Leader, Enrico Predazzi}: \foreignlanguage{english}{\emph{An
introduction to gauge theories and modern particle physics} \emph{II}};
\foreignlanguage{english}{Cambridge University Press}, 2004; ISBN:
0-521-499510-8
\end{itemize}
{\small{\newpage{}}}\fancyhead[R]{Inhaltsverzeichnis}
\fancyhead[C]{}

\tableofcontents{}\label{END-front}\newpage{}\pagenumbering{arabic}
\fancyfoot[C]{\thepage\quad\!\!\!\slash\quad\!\!\!\pageref{END}} % Seitenzahl des Hauptteils
%\fancyhead[R]{\rightmark}
%\fancyhead[C]{\leftmark}%DATE: Mo 15.04.2013


\chapter*{Motivation}

\fancyhead[R]{Motivation}

Zu Beginn des 20. Jahrhunderts wurden zwei neue fundamentale Theorien
entwickelt:
\begin{itemize}
\item Quantenmechanik mit Heisenbergscher Unschärferelation%
\footnote{Wir verwenden natürliche Einheiten mit $c=1$ und $\hbar=1$.%
}: $\Delta E\cdot\Delta t\ge\frac{1}{2}$
\item Spezielle Relativitätstheorie mit Energie-Impuls-Beziehung: $E=\pm\sqrt{\vec{p}^{2}+m^{2}}$\\
Die Lösungen mit negativer Energie führen zu Antiteilchen:
\begin{align*}
e^{-\ii\left(-\abs Et\right)} & =e^{-\ii\abs E\left(-t\right)}
\end{align*}

\end{itemize}
\begin{figure}[H]
\noindent \begin{centering}
\begin{tikzpicture}
  \draw[decoration={markings,mark=at position 1 with {\arrow[scale=2]{>}};},postaction={decorate}] (0,0) node[left]{$t_i$} -- node[right]{$e^-$} (0,2) node[left]{$t_f$};
  \draw[decoration={markings,mark=at position 1 with {\arrow[scale=2]{>}};},postaction={decorate}] (2,2)  -- node[right]{$e^-\hspace*{5mm}\hat{=}$} (2,0);
  \draw[decoration={markings,mark=at position 1 with {\arrow[scale=2]{>}};},postaction={decorate}] (4,0)  -- node[right]{$e^+$} (4,2);
\end{tikzpicture}
\par\end{centering}

\noindent \begin{centering}
\caption{Antiteilchen}

\par\end{centering}

\end{figure}


Das Problem ist, dass im Vakuum virtuelle Teilchen-Antiteilchen-Paare
entstehen können:

\begin{figure}[H]
\noindent \centering{}\begin{tikzpicture}
  \draw[decoration={markings,mark=at position 0.5 with {\arrow[scale=2]{>}};},postaction={decorate}] (0,0) arc (180:0:1);
  \node at (1,1.5) {$e^-$} ;
  \draw[decoration={markings,mark=at position 0.5 with {\arrow[scale=2]{>}};},postaction={decorate}] (2,0) arc (0:-180:1);
  \node at (1,-1.5) {$e^+$} ;
  \draw[fill=black] (0,0) circle (0.05) (2,0) circle (0.05);
  \node at (4,0) {$\Delta t \le \frac{1}{2E}$};
\end{tikzpicture}\caption{virtuelles Teilchen-Antiteilchen-Paar}
\end{figure}

\begin{itemize}
\item Das Vakuum wird dadurch ein Medium.
\item Bei Rechnungen erhält man Unendlichkeiten.
\end{itemize}
Die Behandlung der Divergenzen führt zur ,,Renormierung``, dem Kern
der Quantenfeldtheorie.

Die Idee dabei ist, dass die Quantenfeldtheorie der nieder-energetische
Grenzfall einer ,,\foreignlanguage{english}{Theory of Everything}``
ist. Da wir diese nicht kennen, fordern wir eine ,,Entkopplung``:
Die Physik bei Laborenergien darf nicht von der \foreignlanguage{english}{Theory
of Everything }abhängen.

Dies wird von Eichtheorien mit oder ohne ,,spontane Symmetriebrechung``
erfüllt. Auf diese Weise erhält man das \emph{Standard-Modell} der
Teilchenphysik.

\fancyhead[R]{\rightmark}


\chapter{Dirac-Gleichung, Klein-Gordon-Gleichung}

Wir verwenden natürliche Einheiten $c=1$ und $\hbar=\frac{h}{2\pi}=1$.
Nützlich für Umrechnungen in das SI-Einheitensystem sind folgende
Konstanten:
\begin{align*}
\hbar c & =197{,}327\,\text{MeV fm} & c & =299792458\,\frac{\text{m}}{\text{s}} & 1\,\text{fm} & =10^{-15}\,\text{m}
\end{align*}
\begin{align*}
\left[\text{Energie}\right]\,\hat{=}\,\left[\text{MeV}\right]\,\hat{=}\,\left[\frac{\text{MeV}}{c}\right] & \,\hat{=}\,\left[\text{Impuls}\right]\,\hat{=}\,\left[\frac{\text{MeV}}{\hbar c}\right]\,\hat{=}\,\left[\frac{1}{\text{fm}}\right]\,\hat{=}\,\left[\frac{c}{\text{fm}}\right]\,\hat{=}\,\left[\frac{1}{\text{s}}\right]
\end{align*}
Sei $t$ eine Zeit in der Einheit $\text{MeV}^{-1}$ gegeben. Die
Umrechnung in Sekunden geht nun wie folgt:
\begin{align*}
t\left[\text{s}\right] & =\frac{t\left[\text{MeV}^{-1}\right]\cdot\hbar c}{c}
\end{align*}
Die Quantenmechanik nutzt die klassische Energie-Impuls-Beziehung
(Dispersionsrelation):
\begin{align}
E_{\text{kin}} & =\frac{\vec{p}^{2}}{2m}
\end{align}
Die Wellenmechanik basiert auf ebenen Wellen:
\begin{align*}
\psi & \sim e^{-\ii\left(Et-\vec{p}\cdot\vec{x}\right)}
\end{align*}
\begin{align*}
\ii\frac{\partial}{\partial t}e^{-\ii\left(Et-\vec{p}\cdot\vec{x}\right)} & =Ee^{-\ii\left(Et-\vec{p}\vec{x}\right)}\\
-\ii\vec{\nabla}e^{-\ii\left(Et-\vec{p}\cdot\vec{x}\right)} & =\vec{p}e^{-\ii\left(Et-\vec{p}\vec{x}\right)}
\end{align*}
Ersetze nun die klassische nun durch die relativistische Dispersionsrelation:
\begin{align}
E^{2}-\vec{p}^{2}-m^{2} & =0
\end{align}
Damit erhält man die \emph{Klein-Gordon-Gleichung}:
\begin{align}
\left[-\frac{\partial^{2}}{\partial t^{2}}+\vec{\nabla}^{2}-m^{2}\right]\phi\left(\vec{x},t\right) & =0
\end{align}
Wieso verwendet man nicht folgende Gleichung?
\begin{align*}
\left[\ii\frac{\partial}{\partial t}\pm\sqrt{-\vec{\nabla}^{2}+m^{2}}\,\right]\psi\left(\vec{x},t\right) & =0
\end{align*}
Die Wurzel lässt sich nur als eine Taylor-Entwicklung mit beliebig
hohen Potenzen in $\vec{\nabla}^{2}$ berechnen.
\begin{align*}
f\left(x+y\right) & =f\left(x\right)+f'\left(x\right)y+\frac{1}{2}f''\left(x\right)y^{2}+\ldots
\end{align*}
Unendlich hohe Ableitungen können zu einer Verletzung der Kausalität
führen. Außerdem ist die Konvergenz nicht gesichert. Die Alternative
ist die Linearisierung, also die Zerlegung der Klein-Gordon-Gleichung
als Differentialgleichung zweiter Ordnung in zwei Differentialgleichungen
erster Ordnung.


\section{Definition \textmd{(Dirac-Matrizen)}}

Seien $\gamma_{0},\gamma_{1},\gamma_{2}$ und $\gamma_{3}$ Elemente
einer Algebra mit folgender Eigenschaft:
\begin{align}
\fbox{\ensuremath{{\displaystyle \gamma_{\mu}\gamma_{\nu}+\gamma_{\nu}\gamma_{\mu}=2g_{\mu\nu}\mathbbm{1}}}}
\end{align}
Die $\gamma_{\mu}$ werden \emph{Dirac-Matrizen} genannt, und sie
erzeugen eine \emph{Clifford-Algebra}. Dabei ist
\begin{align}
\left(g_{\mu\nu}\right) & =\left(\begin{array}{cccc}
1 & 0 & 0 & 0\\
0 & -1 & 0 & 0\\
0 & 0 & -1 & 0\\
0 & 0 & 0 & -1
\end{array}\right)=\left(\left(g^{-1}\right)_{\mu\nu}\right)=:\left(g^{\mu\nu}\right)
\end{align}
die Metrik des Minkowski-Raumes%
\footnote{Mit der Metrik kann man die Indizes verschieben: $a^{\mu}=g^{\mu\nu}a_{\nu}$,
$a_{\mu}=g_{\mu\nu}a^{\nu}$%
}. Wir verwenden die Einsteinsche Summenkonvention. Die Multiplikation
mit $\mathbbm{1}$ schreiben wir gewöhnlich nicht aus. 
\begin{align*}
\left(\hat{p}^{\mu}\gamma_{\mu}-m\right)\left(\hat{p}^{\mu}\gamma_{\mu}+m\right) & =\underbrace{\hat{p}^{\mu}\hat{p}^{\nu}}_{\text{symmetrisch}}\gamma_{\mu}\gamma_{\nu}-m^{2}=\\
 & =\frac{1}{2}\left(\hat{p}^{\mu}\hat{p}^{\nu}\gamma_{\mu}\gamma_{\nu}+\hat{p}^{\nu}\hat{p}^{\mu}\gamma_{\mu}\gamma_{\nu}\right)-m^{2}=\\
 & =\frac{1}{2}\hat{p}^{\mu}\hat{p}^{\nu}\left(\gamma_{\mu}\gamma_{\nu}+\gamma_{\nu}\gamma_{\mu}\right)-m^{2}=\\
 & =\hat{p}^{\mu}\hat{p}^{\nu}g_{\mu\nu}-m^{2}=\hat{p}^{2}-m^{2}
\end{align*}
Wenn $\psi$ die Gleichung
\begin{align}
\fbox{\ensuremath{{\displaystyle \left(\hat{p}^{\mu}\gamma_{\mu}+m\right)\psi=0}}}
\end{align}
erfüllt, so erfüllt $\psi$ auch die Klein-Gordon-Gleichung, ist also
eine Lösung, die der relativistischen Energie-Impuls-Beziehung genügt.
Analoges gilt für:
\begin{align}
\fbox{\ensuremath{{\displaystyle \left(\hat{p}^{\mu}\gamma_{\mu}-m\right)\psi=0}}}
\end{align}
Dies sind die zwei Formen der \emph{Dirac-Gleichung}.

%DATE: Di 16.4.13

Wir verwenden die \emph{Feynman-Slash-Notation}:
\begin{align}
\slashd p & :=p^{\mu}\gamma_{\mu}
\end{align}
Die Dirac-Darstellung der Gamma-Matrizen lautet:
\begin{align}
\gamma^{0} & =\gamma_{0}=\left(\begin{array}{cccc}
1 & 0 & 0 & 0\\
0 & 1 & 0 & 0\\
0 & 0 & -1 & 0\\
0 & 0 & 0 & -1
\end{array}\right) & \gamma^{1} & =-\gamma_{1}=\left(\begin{array}{cccc}
0 & 0 & 0 & 1\\
0 & 0 & 1 & 0\\
0 & -1 & 0 & 0\\
-1 & 0 & 0 & 0
\end{array}\right)\nonumber \\
\gamma^{2} & =-\gamma_{2}=\left(\begin{array}{cccc}
0 & 0 & 0 & -\ii\\
0 & 0 & \ii & 0\\
0 & \ii & 0 & 0\\
-\ii & 0 & 0 & 0
\end{array}\right) & \gamma^{3} & =-\gamma_{3}=\left(\begin{array}{cccc}
0 & 0 & 1 & 0\\
0 & 0 & 0 & -1\\
-1 & 0 & 0 & 0\\
0 & 1 & 0 & 0
\end{array}\right)
\end{align}
Beispiel:
\begin{align*}
\gamma_{1}\gamma_{2}+\gamma_{2}\gamma_{1} & =\left(\begin{array}{cccc}
0 & 0 & 0 & 1\\
0 & 0 & 1 & 0\\
0 & -1 & 0 & 0\\
-1 & 0 & 0 & 0
\end{array}\right)\left(\begin{array}{cccc}
0 & 0 & 0 & -\ii\\
0 & 0 & \ii & 0\\
0 & \ii & 0 & 0\\
-\ii & 0 & 0 & 0
\end{array}\right)+\\
 & \quad+\left(\begin{array}{cccc}
0 & 0 & 0 & -\ii\\
0 & 0 & \ii & 0\\
0 & \ii & 0 & 0\\
-\ii & 0 & 0 & 0
\end{array}\right)\left(\begin{array}{cccc}
0 & 0 & 0 & 1\\
0 & 0 & 1 & 0\\
0 & -1 & 0 & 0\\
-1 & 0 & 0 & 0
\end{array}\right)=\\
 & =\left(\begin{array}{cccc}
-\ii & 0 & 0 & 0\\
0 & \ii & 0 & 0\\
0 & 0 & -\ii & 0\\
0 & 0 & 0 & \ii
\end{array}\right)+\left(\begin{array}{cccc}
\ii & 0 & 0 & 0\\
0 & -\ii & 0 & 0\\
0 & 0 & \ii & 0\\
0 & 0 & 0 & -\ii
\end{array}\right)=0
\end{align*}
Die Dirac-Darstellung ermöglicht die intuitive Interpretation, dass
die ersten beiden Indizes eines \emph{Spinors} $\psi$ einer Teilchenlösung
entsprechen und die letzten beiden einer Antiteilchenlösung.


\section{Die freien Lösungen im Ruhesystem}

Wir wollen eine Lösung $\psi\left(x^{0},\vec{x}\right)$ der Diracgleichung
im Ruhesystem finden, das heißt für $\vec{p}=0$ und $E=m$. Wir machen
einen Wellenansatz:
\begin{align*}
\psi\left(t,\vec{x}\right) & =u\left(E,\vec{p}\right)e^{-\ii\left(Et-\vec{p}\cdot\vec{x}\right)}=u\left(m,\vec{0}\right)e^{-\ii mt}
\end{align*}
Die Dirac-Gleichung für Teilchen ist:
\begin{align*}
\left(m\gamma^{0}-m\right)\psi=\left(\slashd{\hat{p}}-m\right)\psi & =0\\
\left(\begin{array}{cccc}
m-m &  &  & 0\\
 & m-m\\
 &  & -m-m\\
0 &  &  & -m-m
\end{array}\right)\psi & =\left(\begin{array}{c}
0\\
0\\
0\\
0
\end{array}\right)
\end{align*}
\begin{align}
u\left(\vec{p}=\vec{0},+\right) & =a\left(\begin{array}{c}
1\\
0\\
0\\
0
\end{array}\right) & u\left(\vec{p}=\vec{0},-\right) & =b\left(\begin{array}{c}
0\\
1\\
0\\
0
\end{array}\right)
\end{align}
Analog lässt die Dirac-Gleichung für Antiteilchen
\begin{align*}
\left(\slashd{\hat{p}}+m\right)\psi & =0
\end{align*}
folgende Lösungen zu:
\begin{align}
v\left(\vec{p}=\vec{0},-\right) & =c\left(\begin{array}{c}
0\\
0\\
1\\
0
\end{array}\right) & u\left(\vec{p}=\vec{0},+\right) & =d\left(\begin{array}{c}
0\\
0\\
0\\
1
\end{array}\right)
\end{align}



\section{Lorentz-Transformationen der Dirac-Gleichung}

Fordere nun die Invarianz der Dirac-Gleichung unter einer Lorentz-Transformationen%
\footnote{Beachte, dass $\Lambda_{\nu}^{\mu}$ sich unter Lorentz-Transformationen
nicht ändert, also kein Tensor ist.%
} $\Lambda_{\nu}^{\mu}$:

\begin{align}
0 & =\left(\ii\frac{\partial}{\partial x_{\mu}}\gamma_{\mu}-m\right)\psi\left(x\right)=\left(\ii\frac{\partial}{\partial x_{\nu}'}\Lambda_{\nu}^{\mu}\gamma_{\mu}-m\right)\underbrace{S^{-1}\left(\Lambda\right)\psi'\left(x'\right)}_{=\psi\left(x\right)}\qquad/S\cdot\nonumber \\
0 & =\bigg(\ii\frac{\partial}{\partial x_{\nu}'}\underbrace{\Lambda_{\nu}^{\mu}S\gamma_{\mu}S^{-1}}_{\stackrel{!}{=}\gamma_{\nu}}-m\bigg)\psi'\left(x'\right)
\end{align}
Es genügt, eine infinitesimale Lorentz-Transformation zu betrachten,
da man eine endliche Transformation als Hintereinanderausführung von
$N$ infinitesimalen darstellen kann.
\begin{align*}
g^{\nu\nu'}\Lambda_{\nu'}^{\mu}\big|_{\text{inf.}} & =g^{\nu\mu}+\frac{\omega^{\nu\mu}}{N}+\mathcal{O}\left(\frac{1}{N^{2}}\right)
\end{align*}
Wir benutzen dann:
\begin{align*}
\lim_{N\to\infty}\left(1-\ii\frac{a}{N}\right)^{N} & =e^{-\ii a}
\end{align*}
Zur Erinnerung: Die Lorentz-Transformation lässt die Metrik invariant:
\begin{align*}
\Lambda_{\mu'}^{\mu}\Lambda_{\nu'}^{\nu}g_{\mu\nu} & =g_{\mu'\nu'}
\end{align*}
Es folgt:
\begin{align*}
g^{\mu''\mu'}\Lambda_{\mu'}^{\mu}g^{\nu''\nu'}\Lambda_{\nu'}^{\nu}g_{\mu\nu} & =g^{\mu''\nu''}\\
g^{\mu''\mu}g^{\nu''\nu}g_{\mu\nu}+\frac{\omega^{\mu''\mu}}{N}g^{\nu''\nu}g_{\mu\nu}+g^{\mu''\mu}\frac{\omega^{\nu''\nu}}{N}g_{\mu\nu} & =g^{\mu''\nu''}\\
\frac{\omega^{\mu''\mu}}{N}g_{\mu\nu''} & =-\frac{\omega^{\nu''\nu}}{N}g_{\mu''\nu}
\end{align*}
\begin{align}
\Rightarrow\qquad\fbox{\ensuremath{{\displaystyle \omega^{\mu''\nu''}=-\omega^{\nu''\mu''}}}}
\end{align}
Für $S$ machen wir den Ansatz:
\begin{align}
S & =\mathbbm{1}-\frac{\ii}{4}\cdot\frac{\omega^{\mu\nu}}{N}\sigma_{\mu\nu}\label{eq:S-Transformation-Definition}
\end{align}
Dabei sind die $\sigma_{\mu\nu}$ beliebige $\mathbb{C}^{4\times4}$-Matrizen,
ohne Einschränkung mit $\sigma_{\mu\nu}=-\sigma_{\nu\mu}$, da der
symmetrische Anteil wegfällt, weil $\omega^{\mu\nu}$ antisymmetrisch
ist. Einsetzen liefert:
\begin{align*}
\gamma_{\nu}+\mathcal{O}\left(\frac{1}{N^{2}}\right) & \stackrel{!}{=}\left(\mathbbm{1}-\frac{\ii}{4}\sigma_{\mu'\nu'}\frac{\omega^{\mu'\nu'}}{N}\right)\left(\gamma_{\nu}+\frac{\omega_{\nu\mu}}{N}\gamma^{\mu}\right)\left(\mathbbm{1}+\frac{\ii}{4}\sigma_{\mu''\nu''}\frac{\omega^{\mu''\nu''}}{N}\right)=\\
\Rightarrow\qquad0 & =-\frac{\ii}{4}\sigma_{\mu'\nu'}\frac{\omega^{\mu'\nu'}}{N}\gamma_{\nu}+\frac{\omega_{\nu}\msu{\mu}}{N}\gamma_{\mu}+\frac{\ii}{4}\gamma_{\nu}\sigma_{\mu''\nu''}\frac{\omega^{\mu''\nu''}}{N}\\
0 & =-\frac{\ii}{4}\sigma_{\mu'\nu'}\gamma_{\nu}\frac{\omega^{\mu'\nu'}}{N}+g_{\nu\nu'}\gamma_{\mu'}\frac{\omega^{\nu'\mu'}}{N}+\frac{\ii}{4}\gamma_{\nu}\sigma_{\mu'\nu'}\frac{\omega^{\mu'\nu'}}{N}\\
0 & =\frac{\omega^{\mu'\nu'}}{N}\left(-\frac{\ii}{4}\sigma_{\mu'\nu'}\gamma_{\nu}-g_{\nu\nu'}\gamma_{\mu'}+\frac{\ii}{4}\gamma_{\nu}\sigma_{\mu'\nu'}\right)
\end{align*}
Es gilt, da $\omega^{\mu'\nu'}$ antisymmetrisch ist:
\begin{align*}
-\omega^{\mu'\nu'}g_{\nu\nu'}\gamma_{\mu'} & =\left(-\frac{1}{2}g_{\nu\nu'}\gamma_{\mu'}+\frac{1}{2}g_{\nu\mu'}\gamma_{\nu'}\right)\omega^{\mu'\nu'}
\end{align*}
Somit folgt:
\begin{align}
\frac{\ii}{4}\left(\sigma_{\mu'\nu'}\gamma_{\nu}-\gamma_{\nu}\sigma_{\mu'\nu'}\right)\frac{\omega^{\mu'\nu'}}{N} & =\frac{1}{2}\left(-g_{\nu\nu'}\gamma_{\mu'}+g_{\nu\mu'}\gamma_{\nu'}\right)\frac{\omega^{\mu'\nu'}}{N}\nonumber \\
\frac{\ii}{2}\left(\sigma_{\mu'\nu'}\gamma_{\nu}-\gamma_{\nu}\sigma_{\mu'\nu'}\right) & =g_{\mu'\nu}\gamma_{\nu'}-g_{\nu'\nu}\gamma_{\mu'}\label{eq:isigma}
\end{align}
Die $\sigma_{\mu'\nu'}$ sind antisymmetrische Tensoren der Stufe
2, die nur von den $\gamma$-Matrizen abhängen, das heißt mit $A\in\mathbb{C}$
gilt:
\begin{align*}
\sigma_{\mu'\nu'} & =A\cdot\left[\gamma_{\mu'},\gamma_{\nu'}\right]
\end{align*}
Die linke Seite von (\ref{eq:isigma}) ist damit:
\begin{align*}
 & \frac{\ii}{2}A\Big(\gamma_{\mu'}\underbrace{\gamma_{\nu'}\gamma_{\nu}}_{\text{kommutieren}}-\gamma_{\nu'}\underbrace{\gamma_{\mu'}\gamma_{\nu}}_{\text{kommutieren}}-\underbrace{\gamma_{\nu}\gamma_{\mu'}}_{\text{kommutieren}}\gamma_{\nu'}+\underbrace{\gamma_{\nu}\gamma_{\nu'}}_{\text{kommutieren}}\gamma_{\mu'}\Big)=\\
 & \qquad\qquad\qquad=\frac{\ii}{2}A\Big(2g_{\nu'\nu}\gamma_{\mu'}-\underline{\gamma_{\mu'}\gamma_{\nu}\gamma_{\nu'}}-2g_{\mu'\nu}\gamma_{\nu'}+\underleftarrow{\gamma_{\nu'}\gamma_{\nu}\gamma_{\mu'}}-\\
 & \qquad\qquad\qquad\qquad\;-2g_{\nu\mu'}\gamma_{\nu'}+\underline{\gamma_{\mu'}\gamma_{\nu}\gamma_{\nu'}}+2g_{\nu\nu'}\gamma_{\mu'}-\underleftarrow{\gamma_{\nu'}\gamma_{\nu}\gamma_{\mu'}}\Big)\\
 & \qquad\qquad\qquad=2\ii A\left(g_{\nu'\nu}\gamma_{\mu'}-g_{\mu'\nu}\gamma_{\nu'}\right)
\end{align*}
Aus (\ref{eq:isigma}) folgt daher:
\begin{align}
A & =\frac{\ii}{2} & \fbox{\ensuremath{{\displaystyle \sigma_{\mu\nu}=\frac{\ii}{2}\left[\gamma_{\mu},\gamma_{\nu}\right]}}}
\end{align}
Damit haben wir $S$ bestimmt.

Erinnerung: Kugelflächenfunktionen und Drehimpulsoperator
\begin{align*}
Y_{lm} & =\sqrt{\frac{2l+1}{4\pi}\cdot\frac{\left(l-m\right)!}{\left(l+m\right)!}}P_{l}^{m}\left(\cos\left(\vartheta\right)\right)e^{\ii m\varphi}\\
\hat{L}_{z} & =\frac{\hbar}{\ii}\left(x\frac{\partial}{\partial y}-y\frac{\partial}{\partial x}\right)=\frac{\hbar}{\ii}\frac{\partial}{\partial\varphi}
\end{align*}



\subsection{Erster Spezialfall: Rotation}

Betrachte die infinitesimale Transformation für Drehungen um die $z$-Achse:
\begin{align}
\left(\Lambda_{\nu}^{\mu}\right)_{\text{inf. Rot.}} & =\left(\begin{array}{cccc}
1 & 0 & 0 & 0\\
0 & 1 & -\frac{\varphi}{N} & 0\\
0 & \frac{\varphi}{N} & 1 & 0\\
0 & 0 & 0 & 1
\end{array}\right) & \frac{\omega^{12}}{N} & =-\frac{\omega^{21}}{N}=-\frac{\varphi}{N}
\end{align}
Alle anderen $\omega^{\mu\nu}$ verschwinden. Außerdem gilt:
\begin{align*}
\sigma_{12} & =-\sigma_{21}=\frac{\ii}{2}\left(\gamma_{1}\gamma_{2}-\gamma_{2}\gamma_{1}\right)=\ii\gamma_{1}\gamma_{2}=\\
 & =\ii\left(\begin{array}{cccc}
0 & 0 & 0 & -1\\
0 & 0 & -1 & 0\\
0 & 1 & 0 & 0\\
1 & 0 & 0 & 0
\end{array}\right)\left(\begin{array}{cccc}
0 & 0 & 0 & \ii\\
0 & 0 & -\ii & 0\\
0 & -\ii & 0 & 0\\
\ii & 0 & 0 & 0
\end{array}\right)=\left(\begin{array}{cccc}
1 & 0 & 0 & 0\\
0 & -1 & 0 & 0\\
0 & 0 & 1 & 0\\
0 & 0 & 0 & -1
\end{array}\right)
\end{align*}
\begin{align}
S & =\exp\left(-\frac{\ii}{4}N\cdot2\frac{\omega^{12}}{N}\sigma_{12}\right)=\exp\left(\frac{\ii}{2}\varphi\left(\begin{array}{cccc}
1 & 0 & 0 & 0\\
0 & -1 & 0 & 0\\
0 & 0 & 1 & 0\\
0 & 0 & 0 & -1
\end{array}\right)\right)=\nonumber \\
 & =\left(\begin{array}{cccc}
1 &  &  & 0\\
 & 1\\
 &  & 1\\
0 &  &  & 1
\end{array}\right)\cos\left(\frac{\varphi}{2}\right)+\left(\begin{array}{cccc}
1 &  &  & 0\\
 & -1\\
 &  & 1\\
0 &  &  & -1
\end{array}\right)\ii\sin\left(\frac{\varphi}{2}\right)=\nonumber \\
 & =\left(\begin{array}{cccc}
e^{\ii\frac{\varphi}{2}} &  &  & 0\\
 & e^{-\ii\frac{\varphi}{2}}\\
 &  & e^{\ii\frac{\varphi}{2}}\\
0 &  &  & e^{-\ii\frac{\varphi}{2}}
\end{array}\right)
\end{align}
Also haben die erste und dritte Komponenten Spin $\frac{1}{2}$ und
die anderen beiden Spin $-\frac{1}{2}$.


\subsection{Zweiter Spezialfall: Lorentz-Boost}

Für den Lorentz-Boost gilt:
\begin{align*}
\left(x^{0}\right)' & =\gamma\left(x^{0}+\vec{\beta}\vec{x}\right)\\
\left(\vec{x}\right)' & =\gamma\left(\vec{\beta}x^{0}+\vec{x}\right)
\end{align*}
\begin{align*}
\vec{\beta} & =\frac{\vec{v}}{c} & \gamma^{-1} & =\sqrt{1-\beta^{2}}
\end{align*}
\begin{align*}
\frac{\omega^{0}\msd k}{N} & =\frac{\omega^{k}\msd 0}{N}=:\frac{\omega^{k}}{N} & \omega & :=\sqrt{\left(\omega^{1}\right)^{2}+\left(\omega^{2}\right)^{2}+\left(\omega^{3}\right)^{2}}
\end{align*}
\begin{align}
\left(\Lambda\right)_{\text{inf. Boost}} & =\left(\begin{array}{cccc}
1 & \frac{\omega^{1}}{N} & \frac{\omega^{2}}{N} & \frac{\omega^{3}}{N}\\
\frac{\omega^{1}}{N} & 1 & 0 & 0\\
\frac{\omega^{2}}{N} & 0 & 1 & 0\\
\frac{\omega^{3}}{N} & 0 & 0 & 1
\end{array}\right)
\end{align}
Man erhält:
\begin{align*}
\sigma_{10} & =\ii\left(\begin{array}{cccc}
0 & 0 & 0 & 1\\
0 & 0 & 1 & 0\\
0 & 1 & 0 & 0\\
1 & 0 & 0 & 0
\end{array}\right) & \sigma_{20} & =\ii\left(\begin{array}{cccc}
0 & 0 & 0 & -\ii\\
0 & 0 & \ii & 0\\
0 & -\ii & 0 & 0\\
\ii & 0 & 0 & 0
\end{array}\right) & \sigma_{30} & =\ii\left(\begin{array}{cccc}
0 & 0 & 1 & 0\\
0 & 0 & 0 & -1\\
1 & 0 & 0 & 0\\
0 & -1 & 0 & 0
\end{array}\right)
\end{align*}
\begin{align*}
S & =\lim_{N\to\infty}\left(\mathbbm{1}+\frac{1}{2}\cdot\frac{\omega}{N}\underbrace{\left(\begin{array}{cccc}
0 & 0 & p^{3} & p^{1}-\ii p^{2}\\
0 & 0 & p^{1}+\ii p^{2} & -p^{3}\\
p^{3} & p^{1}-\ii p^{2} & 0 & 0\\
p^{1}+\ii p^{2} & -p^{3} & 0 & 0
\end{array}\right)\frac{1}{\norm{\vec{p}}}}_{=:M}\right)^{N}=e^{\frac{\omega}{2}M}
\end{align*}
\begin{align*}
M^{2} & =\left(\begin{array}{cccc}
1 & 0 & 0 & 0\\
0 & 1 & 0 & 0\\
0 & 0 & 1 & 0\\
0 & 0 & 0 & 1
\end{array}\right)
\end{align*}
\begin{align*}
\Rightarrow\qquad M^{2n} & =\mathbbm{1} & M^{2n+1} & =M
\end{align*}
\begin{align}
\Rightarrow\qquad S & =\cosh\left(\frac{\omega}{2}\right)\mathbbm{1}+\sinh\left(\frac{\omega}{2}\right)M
\end{align}
%DATE: Do 18.4.13

Wir betrachten jetzt die $\Lambda_{\alpha}^{\beta}$-Matrix, die sich
aus $N$ infinitesimalen Transformationen ergibt.
\begin{align}
\left(\Lambda_{\alpha}^{\beta}\right) & =\lim_{N\to\infty}\left(1+\frac{\omega}{N}\underbrace{\left(\begin{array}{cccc}
0 & \omega^{1} & \omega^{2} & \omega^{3}\\
\omega^{1} & 0 & 0 & 0\\
\omega^{2} & 0 & 0 & 0\\
\omega^{3} & 0 & 0 & 0
\end{array}\right)\frac{1}{\omega}}_{=:\tilde{M}}\right)^{N}=e^{\omega\tilde{M}}
\end{align}
\begin{align*}
\tilde{M}^{2} & =\left(\begin{array}{cccc}
1 & 0 & 0 & 0\\
0 & \frac{\left(\omega^{1}\right)^{2}}{\omega^{2}} & 0 & 0\\
0 & 0 & \frac{\left(\omega^{2}\right)^{2}}{\omega^{2}} & 0\\
0 & 0 & 0 & \frac{\left(\omega^{3}\right)^{2}}{\omega^{2}}
\end{array}\right)
\end{align*}
Damit folgt:
\begin{align*}
\left(\tilde{M}^{2}\right)_{0}^{0} & =1=\left(\tilde{M}^{2n}\right)_{0}^{0} & \left(\tilde{M}^{2n+1}\right)_{0}^{0} & =0
\end{align*}
Somit ergibt sich:
\begin{align}
\cosh\left(\omega\right) & =\left(\Lambda\right)_{0}^{0}\stackrel{!}{=}\gamma=\frac{E}{m}=\sqrt{\frac{\norm{\vec{p}}^{2}+m^{2}}{m^{2}}}\nonumber \\
\cosh\left(\frac{\omega}{2}\right) & =\sqrt{\frac{\cosh\left(\omega\right)+1}{2}}=\sqrt{\frac{E+m}{2m}}\nonumber \\
\frac{1}{\norm{\vec{p}}}\sinh\left(\frac{\omega}{2}\right) & =\frac{1}{\norm{\vec{p}}}\sqrt{\frac{\cosh\left(\omega\right)-1}{2}}=\frac{1}{\sqrt{E^{2}-m^{2}}}\sqrt{\frac{E-m}{2m}}=\nonumber \\
 & =\frac{1}{\sqrt{2m\left(E+m\right)}}=\frac{1}{E+m}\sqrt{\frac{E+m}{2m}}\nonumber \\
S & =\sqrt{\frac{E+m}{2m}}\left(\begin{array}{cccc}
1 & 0 & \frac{p^{3}}{E+m} & \frac{p^{1}-\ii p^{2}}{E+m}\\
0 & 1 & \frac{p^{1}+\ii p^{2}}{E+m} & \frac{-p^{3}}{E+m}\\
\frac{p^{3}}{E+m} & \frac{p^{1}-\ii p^{2}}{E+m} & 1 & 0\\
\frac{p^{1}+\ii p^{2}}{E+m} & \frac{-p^{3}}{E+m} & 0 & 1
\end{array}\right)
\end{align}
\begin{align*}
\Rightarrow\qquad u\left(\vec{p},+\right) & =\sqrt{\frac{E+m}{2m}}\left(\begin{array}{c}
1\\
0\\
\frac{p^{3}}{E+m}\\
\frac{p^{1}+\ii p^{2}}{E+m}
\end{array}\right)
\end{align*}
Die Lösungen sind also:
\begin{align*}
\psi_{1} & =e^{-\ii\left(Et-\vec{p}\vec{x}\right)}u\left(\vec{p},+\right)=e^{-\ii px}u\left(\vec{p},+\right) & \psi_{2} & =e^{-\ii\left(Et-\vec{p}\vec{x}\right)}u\left(\vec{p},-\right)=e^{-\ii px}u\left(\vec{p},-\right)\\
\psi_{3} & =e^{\ii\left(Et-\vec{p}\vec{x}\right)}v\left(\vec{p},+\right)=e^{\ii px}v\left(\vec{p},+\right) & \psi_{4} & =e^{\ii\left(Et-\vec{p}\vec{x}\right)}v\left(\vec{p},-\right)=e^{\ii px}v\left(\vec{p},-\right)
\end{align*}
\begin{align}
\psi_{1}\left(\vec{x},t\right) & =e^{-\ii px}\sqrt{\frac{E+m}{2m}}\left(\begin{array}{c}
1\\
0\\
\frac{p^{3}}{E+m}\\
\frac{p^{1}+\ii p^{2}}{E+m}
\end{array}\right) & \psi_{2}\left(\vec{x},t\right) & =e^{-\ii px}\sqrt{\frac{E+m}{2m}}\left(\begin{array}{c}
0\\
1\\
\frac{p^{1}-\ii p^{2}}{E+m}\\
\frac{-p^{3}}{E+m}
\end{array}\right)\nonumber \\
\psi_{3}\left(\vec{x},t\right) & =e^{\ii px}\sqrt{\frac{E+m}{2m}}\left(\begin{array}{c}
\frac{p^{3}}{E+m}\\
\frac{p^{1}+\ii p^{2}}{E+m}\\
1\\
0
\end{array}\right) & \psi_{4}\left(\vec{x},t\right) & =e^{\ii px}\sqrt{\frac{E+m}{2m}}\left(\begin{array}{c}
\frac{p^{1}-\ii p^{2}}{E+m}\\
\frac{-p^{3}}{E+m}\\
0\\
1
\end{array}\right)\label{eq:freie_Loesung}
\end{align}



\subsection{Der 4-Spinvektor}

Wir suchen nun die relativistische Verallgemeinerung des Spinvektors
$\vec{s}$, den 4-Spinvektor $s^{\mu}$. Im Ruhesystem soll gelten:
\begin{align}
s^{\mu} & =\left(0,\vec{s}\right) & p^{\mu} & =\left(E,\vec{0}\right)=\left(m,\vec{0}\right)
\end{align}
Wir verwenden, dass 4-Skalarprodukte invariant unter Lorentz-Transformationen
sind.
\begin{align}
s^{2} & =s_{\mu}s^{\mu}=-\norm{\vec{s}}^{2}\stackrel{\norm{\vec{s}}:=1}{=}-1 & s\cdot p & =s_{\mu}p^{\mu}=0
\end{align}
Wir machen folgenden Ansatz für $s^{\mu}$, da die einzige ausgezeichnete
Raumrichtung $\vec{p}$ ist:
\begin{align*}
s^{\mu} & =\left(s^{0},\alpha\vec{p}\right)
\end{align*}
Dabei ist $\alpha\in\mathbb{R}$ ein beliebiger Proportionalitätsfaktor.
Einsetzen liefert:
\begin{align*}
s^{0}E-\alpha\vec{p}^{2} & =0\\
\Rightarrow\qquad s^{0} & =\frac{\alpha\vec{p}^{2}}{E}
\end{align*}
Damit ergibt die andere Gleichung:
\begin{align*}
-1 & =\left(s^{0}\right)^{2}-\alpha^{2}\vec{p}^{2}=\left(\frac{\alpha\vec{p}^{2}}{E}\right)^{2}-\alpha^{2}\vec{p}^{2}=\\
 & =\alpha^{2}\left(\frac{\vec{p}^{4}}{E^{2}}-\vec{p}^{2}\right)=\alpha^{2}\left(\frac{\vec{p}^{4}}{E^{2}}-\vec{p}^{2}\right)=\\
 & =-\alpha^{2}\cdot\frac{\vec{p}^{2}m^{2}}{E^{2}}\\
\Rightarrow\qquad\alpha & =\pm\frac{E}{m\norm{\vec{p}}}
\end{align*}
Somit ist der Spin 4-Vektor:
\begin{align}
\fbox{\ensuremath{{\displaystyle s^{\mu}=\pm\frac{E}{m}\left(\frac{\norm{\vec{p}}}{E},\frac{\vec{p}}{\norm{\vec{p}}}\right)}}}
\end{align}
Eine wichtige Eigenschaft von $s^{\mu}$ ist:
\begin{align}
\lim_{\frac{\norm{\vec{p}}}{m}\to\infty}s^{\mu} & =\pm\frac{1}{m}\left(E,\vec{p}\right)=\pm\frac{p^{\mu}}{m}
\end{align}
Was ist das Lorentz-invariante Skalarprodukt (und damit die Norm)
für Spinoren? Wir machen folgenden Ansatz:
\begin{align*}
\int\dd^{3}x\left(\psi^{*}\right)^{\TT}\left(t,\vec{x}\right)\Gamma\psi\left(t,\vec{x}\right) & =1
\end{align*}
Dabei ist $\Gamma\in\mathbb{C}^{4\times4}$ eine Matrix. Die Lorentz-Invarianz
bedeutet:
\begin{align*}
S^{\dagger}\Gamma S & \stackrel{!}{=}\Gamma
\end{align*}
Nebenrechnung:
\begin{align*}
-\frac{\omega^{\mu\nu}}{4}\left(\ii\sigma_{\mu\nu}\right)^{\dagger} & =\frac{\omega^{\mu\nu}}{8}\left(\gamma_{\mu}\gamma_{\nu}-\gamma_{\nu}\gamma_{\mu}\right)^{\dagger}=\frac{\omega^{\mu\nu}}{8}\left(\gamma_{\nu}^{\dagger}\gamma_{\mu}^{\dagger}-\gamma_{\mu}^{\dagger}\gamma_{\nu}^{\dagger}\right)
\end{align*}
In der Dirac-Darstellung gilt:
\begin{align*}
\gamma_{0} & =\gamma_{0}^{\dagger} & \gamma_{i} & =-\gamma_{i}^{\dagger}
\end{align*}
Dies kann man aufgrund der Antikommutator-Relationen kurz schreiben
als:
\begin{align*}
\gamma_{\mu}^{\dagger} & =\gamma_{0}\gamma_{\mu}\gamma_{0}
\end{align*}
Es folgt:
\begin{align*}
-\frac{\omega^{\mu\nu}}{4}\left(\ii\sigma_{\mu\nu}\right)^{\dagger} & =\gamma_{0}\frac{\omega^{\mu\nu}}{8}\left(\gamma_{\nu}\gamma_{\mu}-\gamma_{\mu}\gamma_{\nu}\right)\gamma_{0}=\gamma_{0}\frac{\ii}{4}\omega^{\mu\nu}\sigma_{\mu\nu}\gamma_{0}
\end{align*}
Aus (\ref{eq:S-Transformation-Definition}) folgt für eine infinitesimale
Transformation:
\begin{align*}
S^{\dagger} & =\gamma_{0}S^{-1}\gamma_{0}
\end{align*}
Bei einer endlichen Transformation ergibt sich dies ebenfalls:
\begin{align*}
\left(S^{N}\right)^{\dagger} & =\left(S^{\dagger}\right)^{N}=\gamma_{0}\left(S^{-1}\right)^{N}\gamma_{0}=\gamma_{0}\left(S^{N}\right)^{-1}\gamma_{0}
\end{align*}
Damit erhält man für alle $\omega^{\mu\nu}$:
\begin{align*}
\gamma_{0}S^{-1}\gamma_{0}\Gamma S & \stackrel{!}{=}0\\
\Rightarrow\qquad\Gamma & =\gamma_{0}=\left(\begin{array}{cccc}
1 & 0 & 0 & 0\\
0 & 1 & 0 & 0\\
0 & 0 & -1 & 0\\
0 & 0 & 0 & -1
\end{array}\right)
\end{align*}
Die Norm ist also:
\begin{align}
\norm{\psi}^{2} & =\int\dd^{3}x\psi^{\dagger}\left(t,\vec{x}\right)\left(\begin{array}{cccc}
1 & 0 & 0 & 0\\
0 & 1 & 0 & 0\\
0 & 0 & -1 & 0\\
0 & 0 & 0 & -1
\end{array}\right)\psi\left(t,\vec{x}\right)
\end{align}
Im Folgenden verwenden wir die Notation:
\begin{align*}
\overline{\psi} & =\psi^{\dagger}\gamma_{0}
\end{align*}



\subsection{Projektions-Operatoren\label{sub:Projektions-Operatoren}}

Wir wollen die Lösungen der Dirac-Gleichung jetzt mit Hilfe von Projektions-Operatoren
schreiben. Im euklidischen Raum gilt:
\begin{align*}
\vec{v} & =\sum_{i}\vec{e}_{i}\left(\vec{e}_{i}^{\TT}\cdot\vec{v}\right)=\sum_{i}\underbrace{\left(\vec{e}_{i}\vec{e}_{i}^{\TT}\right)}_{=:P}\cdot\vec{v}
\end{align*}
Die Projektions-Operatoren im Ruhesystem sind einfach:
\begin{align}
\hat{P}_{1} & =\left(\begin{array}{cccc}
1 & 0 & 0 & 0\\
0 & 0 & 0 & 0\\
0 & 0 & 0 & 0\\
0 & 0 & 0 & 0
\end{array}\right) & \hat{P}_{2} & =\left(\begin{array}{cccc}
0 & 0 & 0 & 0\\
0 & 1 & 0 & 0\\
0 & 0 & 0 & 0\\
0 & 0 & 0 & 0
\end{array}\right)\nonumber \\
\hat{P}_{3} & =\left(\begin{array}{cccc}
0 & 0 & 0 & 0\\
0 & 0 & 0 & 0\\
0 & 0 & 1 & 0\\
0 & 0 & 0 & 0
\end{array}\right) & \hat{P}_{4} & =\left(\begin{array}{cccc}
0 & 0 & 0 & 0\\
0 & 0 & 0 & 0\\
0 & 0 & 0 & 0\\
0 & 0 & 0 & 1
\end{array}\right)
\end{align}
Weiter gilt im Ruhesystem:
\begin{align*}
\frac{\slashd p+m}{2m} & =\left(\begin{array}{cccc}
1 & 0 & 0 & 0\\
0 & 1 & 0 & 0\\
0 & 0 & 0 & 0\\
0 & 0 & 0 & 0
\end{array}\right) & \frac{\slashd p-m}{2m} & =\left(\begin{array}{cccc}
0 & 0 & 0 & 0\\
0 & 0 & 0 & 0\\
0 & 0 & 1 & 0\\
0 & 0 & 0 & 1
\end{array}\right)
\end{align*}
Wir definieren:
\begin{align*}
\gamma_{5} & =\gamma^{5}:=\ii\gamma^{0}\gamma^{1}\gamma^{2}\gamma^{3}=-\ii\gamma_{0}\gamma_{1}\gamma_{2}\gamma_{3}=\left(\begin{array}{cccc}
0 & 0 & 1 & 0\\
0 & 0 & 0 & 1\\
1 & 0 & 0 & 0\\
0 & 1 & 0 & 0
\end{array}\right)
\end{align*}
\begin{align*}
\gamma_{5}\gamma_{\mu}s^{\mu}\bigg|_{\vec{s}=\left(0,0,1\right)} & =\gamma_{5}\gamma_{3}=\left(\begin{array}{cccc}
0 & 0 & 1 & 0\\
0 & 0 & 0 & 1\\
1 & 0 & 0 & 0\\
0 & 1 & 0 & 0
\end{array}\right)\left(\begin{array}{cccc}
0 & 0 & -1 & 0\\
0 & 0 & 0 & 1\\
1 & 0 & 0 & 0\\
0 & -1 & 0 & 0
\end{array}\right)=\left(\begin{array}{cccc}
1 & 0 & 0 & 0\\
0 & -1 & 0 & 0\\
0 & 0 & -1 & 0\\
0 & 0 & 0 & 1
\end{array}\right)
\end{align*}
\begin{align}
\frac{1+\gamma_{5}\slashd s}{2} & =\left(\begin{array}{cccc}
1 & 0 & 0 & 0\\
0 & 0 & 0 & 0\\
0 & 0 & 0 & 0\\
0 & 0 & 0 & 1
\end{array}\right)\nonumber \\
\Rightarrow\qquad\hat{P}_{1} & =\frac{\slashd p+m}{2m}\cdot\frac{1+\gamma_{5}\slashd s}{2}=u\left(\vec{p},+\right)\overline{u}\left(\vec{p},+\right)\label{eq:P1}
\end{align}
Analog ergibt sich:
\begin{align}
\hat{P}_{2} & =\frac{\slashd p+m}{2m}\cdot\frac{1-\gamma_{5}\slashd s}{2}\label{eq:P2}\\
\hat{P}_{3} & =\frac{-\slashd p+m}{2m}\cdot\frac{1-\gamma_{5}\slashd s}{2}\\
\hat{P}_{4} & =\frac{-\slashd p+m}{2m}\cdot\frac{1+\gamma_{5}\slashd s}{2}
\end{align}
Für $\norm{\vec{p}}\gg m$ ist $\slashd s\approx\slashd p$ und es
gilt:
\begin{align*}
\left(\slashd p+m\right)\frac{\slashd p}{m} & =\frac{\slashd p^{2}+\slashd pm}{m}=\frac{p^{2}+m\slashd p}{m}=\frac{m^{2}+m\slashd p}{m}=m+\slashd p
\end{align*}
\begin{align}
 & \text{Helizität (engl. helicity):} &  & \frac{1\pm\gamma_{5}\slashd s}{2}\\
 & \text{Chiralität (engl. chirality):} &  & \frac{1\pm\gamma_{5}}{2}
\end{align}
Für $\frac{\norm{\vec{p}}}{m}\to\infty$ stimmt beides überein.

%DATE: Mo 22.4.13


\subsection{Diskrete Lorentz-Transformationen}

Zur Lorentz-Gruppe gehören die diskreten Transformationen $\hat{P}$
(Parität) und $\hat{T}$ (Zeitumkehr).
\begin{align}
\hat{P} & =\left(\begin{array}{cccc}
1 & 0 & 0 & 0\\
0 & -1 & 0 & 0\\
0 & 0 & -1 & 0\\
0 & 0 & 0 & -1
\end{array}\right) & \hat{T} & =\left(\begin{array}{cccc}
-1 & 0 & 0 & 0\\
0 & 1 & 0 & 0\\
0 & 0 & 1 & 0\\
0 & 0 & 0 & 1
\end{array}\right)
\end{align}
\begin{align}
\Lambda_{\mu'}^{\mu}\Lambda_{\nu'}^{\nu}g_{\mu\nu} & =g_{\mu'\nu'}\nonumber \\
\Rightarrow\qquad\left(\det\left(\Lambda\right)\right)^{2} & =1\nonumber \\
\det\left(\Lambda\right) & =\pm1
\end{align}
Für $\mu'=0=\nu'$ gilt:
\begin{align*}
\Lambda_{0}^{0}\Lambda_{0}^{0}-\sum_{i=1}^{3}\Lambda_{0}^{i}\Lambda_{0}^{i} & =1\\
\Rightarrow\qquad\left(\Lambda_{0}^{0}\right)^{2} & =1+\sum_{i=1}^{3}\left(\Lambda_{0}^{i}\right)^{2}\ge1
\end{align*}
\begin{align}
\Lambda_{0}^{0} & \ge1\qquad\text{oder}\qquad\Lambda_{0}^{0}\le1
\end{align}
Es gibt vier nicht zusammenhängende Teilmengen der Lorentzgruppe:
\begin{align}
L_{+}^{\uparrow}:\  & \det\left(\Lambda\right)=+1,\quad\Lambda_{0}^{0}>1 & \stackrel{\hat{T}}{\Rightarrow}\qquad L_{-}^{\downarrow}:\  & \det\left(\Lambda\right)=-1,\quad\Lambda_{0}^{0}<1\nonumber \\
 & \hat{P}\Downarrow & \hat{P}\hat{T}\RotateX{-45}{\Rightarrow}\qquad\qquad & \hat{P}\Downarrow\nonumber \\
L_{-}^{\uparrow}:\  & \det\left(\Lambda\right)=-1,\quad\Lambda_{0}^{0}>1 & \stackrel{\hat{T}}{\Rightarrow}\qquad L_{+}^{\downarrow}:\  & \det\left(\Lambda\right)=+1,\quad\Lambda_{0}^{0}<1
\end{align}
Die schwache Wechselwirkung verletzt die $\hat{P}$- und die $\hat{T}$-Symmetrie.

Stetige Transformationen führen nicht aus $L_{+}^{\uparrow}$ hinaus.
Daher muss man sich zusätzlich das Verhalten unter $\hat{P}$ und
$\hat{T}$ ansehen.
\begin{description}
\item [{Nebenbemerkung:}] $\hat{C}\hat{P}\hat{T}$ ist für jede ,,vernünftige``
Theorie eine Symmetrietransformation, wobei $\hat{C}$ die Ladungskonjugation
ist, die Teilchen und Antiteilchen vertauscht. Daher ist $\hat{C}=\hat{P}\hat{T}$.
\end{description}

\section{Das Eichprinzip}

Die Quantenmechanik ist invariant unter globalen Eichtransformationen
$\psi\left(x\right)\to e^{\ii q\Theta}\psi\left(x\right)$. In der
Quantenelektrodynamik fordert man nun die Invarianz unter den allgemeineren
\emph{lokalen} Eichtransformationen $\psi\left(x\right)\to e^{\ii q\Theta\left(x\right)}$.

Als heuristische Motivation hierfür betrachte zum Beispiel das Aharonov-Bohm-Experiment:

\begin{figure}[H]
\noindent \begin{centering}
\begin{tikzpicture}
  \draw (0,-1) -- (0,1) (0,1.3) -- (0,2.5) (0,-1.3) -- (0,-2.5);
  \draw (4,-2.5) -- (4,2.5);
  \draw (-4,0) node[above]{$e^-$} -- (0,1.15) -- (4,0);
  \draw (-4,0) -- (0,-1.15) -- (4,0);
  \draw (-1,0) node{$\vec{B}$} circle (0.7);
  \draw plot[smooth,tension=.7] coordinates{(4.5,2.5) (5,2) (4.5,1.5) (5,1) (4.5,0.5) (5,0) (4.5,-0.5) (5,-1) (4.5,-1.5) (5,-2) (4.5,-2.5)};
  \draw[orange, thick,<->] (5.5,1) -- node[right]{abhängig von $\vec{B}$} (5.5,-1);
\end{tikzpicture}
\par\end{centering}

\noindent \centering{}\caption{Aharonov-Bohm-Experiment: aufgesammelte Phase $e^{\ii\int_{s}qA_{\mu}\dd x^{\mu}}$}
\end{figure}


Das Vektorpotential $A^{\mu}$ ist nur bis auf eine Eichtransformation
bestimmt. Die Eichphasen dürfen also keine Rolle spielen.

Die freie Dirac-Gleichung
\begin{align*}
\left(\ii\gamma^{\mu}\partial_{\mu}-m\right)\psi\left(x\right) & =0
\end{align*}
ist nicht invariant unter lokalen Eichtransformationen:
\begin{align*}
\left(\ii\gamma^{\mu}\partial_{\mu}-m\right)e^{\ii q\Theta\left(x\right)}\psi\left(x\right) & =e^{\ii q\Theta\left(x\right)}\left(-q\gamma^{\mu}\left(\partial_{\mu}\Theta\right)+\ii\gamma^{\mu}\partial_{\mu}-m\right)\psi\left(x\right)
\end{align*}
Aus der Forderung nach Invarianz folgt, dass es etwas geben muss,
was sich so transformiert, dass die Dirac-Gleichung invariant wird.
Da $\partial_{\mu}\Theta\left(x\right)$ ein Vektorfeld ist, muss
das neue Objekt ebenfalls ein Vektorfeld sein, und zwar $A_{\mu}$
mit:
\begin{align}
\left(\ii\gamma^{\mu}\partial_{\mu}-q\gamma^{\mu}A_{\mu}\left(x\right)-m\right)\psi\left(x\right) & =0
\end{align}
Unter einer Eichtransformation transformiert $A_{\mu}$ wie folgt:
\begin{align}
A_{\mu}\left(x\right) & \to A_{\mu}\left(x\right)-\partial_{\mu}\Theta\left(x\right)
\end{align}
Aus der Forderung nach Eichinvarianz folgt die Existenz des Photons.


\subsection{Lagrangedichte der Quantenelektrodynamik}

Um zum Beispiel die Zeitentwicklung eines Elektronfeldes beschreiben
zu können, muss man wissen, wie sich $A_{\mu}\left(x\right)$ dynamisch
entwickelt, das heißt wir brauchen die Lagrangedichte des Photons
$\mathcal{L}_{A}\left(x\right)$.

Wir wissen, dass $\DD_{\mu}=\partial_{\mu}+\ii qA_{\mu}$ ein eichinvarianter
Ableitungsoperator ist, der \emph{kovariante Ableitung} genannt wird.
\begin{itemize}
\item Für die Dimension von $\mathcal{L}$ gilt:
\begin{align}
S & =\int\dd^{4}x\mathcal{L}\left(x\right)\\
1\stackrel{\hbar=1}{=}\left[S\right] & =\left[x\right]^{4}\cdot\left[\mathcal{L}\right]\nonumber \\
\Rightarrow\qquad\left[\mathcal{L}\right] & =\left[x\right]^{-4}\stackrel{\hbar c=1}{=}\left[E\right]^{4}\nonumber 
\end{align}

\item $\mathcal{L}\left(x\right)$ ist eine Lorentz-Skalardichte.
\item $\mathcal{L}\left(x\right)$ ist kein Ableitungsoperator, sondern
eine Funktion (ein Multiplikationsoperator).
\item $\mathcal{L}\left(x\right)$ muss eichinvariant sein, dass heißt sie
muss aus $\DD_{\mu}$-Operatoren aufgebaut sein.
\end{itemize}
Man beobachtet, dass $\left[\hat{D}_{\mu},\hat{D}_{\nu}\right]$ ein
Multiplikations- und kein Ableitungsoperator ist.
\begin{align*}
\left(\partial_{\mu}+\ii qA_{\mu}\right)\left(\partial_{\nu}+\ii qA_{\nu}\right) & =\partial_{\mu}\partial_{\nu}+\ii q\left(\partial_{\mu}A_{\nu}\right)+\ii qA_{\nu}\partial_{\mu}+\ii qA_{\mu}\partial_{\nu}-q^{2}A_{\mu}A_{\nu}\\
\left(\partial_{\nu}+\ii qA_{\nu}\right)\left(\partial_{\mu}+\ii qA_{\mu}\right) & =\partial_{\nu}\partial_{\mu}+\ii q\left(\partial_{\nu}A_{\mu}\right)+\ii qA_{\mu}\partial_{\nu}+\ii qA_{\nu}\partial_{\mu}-q^{2}A_{\nu}A_{\mu}
\end{align*}
\begin{align*}
\Rightarrow\qquad\left[\hat{D}_{\mu},\hat{D}_{\nu}\right] & =\ii q\left(\partial_{\mu}A_{\nu}-\partial_{\nu}A_{\mu}\right)=:\ii qF_{\mu\nu}
\end{align*}
Wählt man $q$ dimensionslos, so hat $A_{\mu}$ die Dimension einer
Energie und somit folgt $\left[F_{\mu\nu}\right]=\left[E\right]^{2}$.
\begin{align}
\stackrel{F^{\mu}\msd{\mu}=0}{\Rightarrow}\qquad\fbox{\ensuremath{{\displaystyle \mathcal{L}\left(x\right)\sim F_{\mu\nu}F^{\mu\nu}}}}
\end{align}
Die Wahl der Konstanten ist willkürlich. Damit die Gleichungen möglichst
einfach werden, wählt man folgende Lagrangedichte: 
\begin{align}
\mathcal{L}\left(x\right) & :=-\frac{1}{4}F_{\mu\nu}F^{\mu\nu}
\end{align}
Die freie Euler-Lagrange-Gleichung ist also:
\begin{align}
\frac{\partial\mathcal{L}}{\partial A_{\sigma}}-\partial^{\lambda}\frac{\partial\mathcal{L}}{\partial\left(\partial^{\lambda}A_{\sigma}\right)} & =0\nonumber \\
\Rightarrow\qquad\partial^{\lambda}F_{\lambda\sigma} & =0
\end{align}
Die Lagrange-Dichte der Quantenelektrodynamik ist somit:
\begin{align}
\mathcal{L}_{\text{QED}}\left(x\right) & =\sum_{\text{Fermionen }j}\overline{\psi}_{j}\left(x\right)\left(\slashd{\hat{p}}-q_{e}Q_{j}\slashd A\left(x\right)-m\right)\psi_{j}\left(x\right)-\frac{1}{4}F_{\mu\nu}\left(x\right)F^{\mu\nu}\left(x\right)
\end{align}
Dabei ist $q_{e}$ die Elementarladung und $Q_{j}$ die relative Ladung
des $j$-ten Fermions.

Bei der Variation sind $\overline{\psi}$ und $\psi$ als unabhängige
Variablen aufzufassen, weil eine komplexe Größe zwei reelle Freiheitsgrade
hat. Die Euler-Lagrange-Gleichungen für $A_{\sigma}$ sind:
\begin{align}
0 & =\frac{\partial\mathcal{L}_{\text{QED}}}{\partial A_{\sigma}}-\partial_{\lambda}\frac{\partial\mathcal{L}_{\text{QED}}}{\partial\left(\partial_{\lambda}A_{\sigma}\right)}=\nonumber \\
 & =\sum_{\text{Fermionen }j}\overline{\psi}_{j}\left(x\right)\left(-q_{e}Q_{j}\gamma^{\sigma}\right)\psi_{j}\left(x\right)-\left(\frac{-1}{4}\cdot4\right)\partial_{\lambda}F^{\lambda\sigma}\nonumber \\
\partial_{\lambda}F^{\lambda\sigma} & =q_{e}\underbrace{\sum_{\text{Fermionen }j}Q_{j}\overline{\psi}_{j}\left(x\right)\gamma^{\sigma}\psi_{j}\left(x\right)}_{=j^{\sigma}\left(x\right)}\label{eq:Maxwell-inhomogen}
\end{align}
Die Dirac-Gleichung erhält man als Euler-Lagrange-Gleichung für $\overline{\psi}_{j}$.
\begin{align*}
0 & =\frac{\partial\mathcal{L}_{\text{QED}}}{\partial\overline{\psi}_{j}}-\partial_{\lambda}\underbrace{\frac{\partial\mathcal{L}_{\text{QED}}}{\partial\left(\partial_{\lambda}\overline{\psi}_{j}\right)}}_{=0}=\left(\slashd{\hat{p}}-q_{e}Q_{j}\slashd A\left(x\right)-m\right)\psi_{j}\left(x\right)
\end{align*}
Die Euler-Lagrange-Gleichung für $\psi_{j}$ ist die konjugierte Dirac-Gleichung:
\begin{align*}
0 & =\frac{\partial\mathcal{L}_{\text{QED}}}{\partial\psi_{j}}-\partial_{\lambda}\frac{\partial\mathcal{L}_{\text{QED}}}{\partial\left(\partial_{\lambda}\psi_{j}\right)}=\overline{\psi}_{j}\left(-q_{e}Q_{j}A^{\lambda}\left(x\right)-m\right)-\partial_{\lambda}\left(\overline{\psi}_{j}\cdot\ii\gamma^{\lambda}\right)=\\
 & =\overline{\psi}_{j}\left(-\overleftarrow{\slashd p}-q_{e}Q_{j}\slashd A\left(x\right)-m\right)
\end{align*}
Dabei bedeutet $\overleftarrow{\slashd p}$, dass der Ableitungsoperator
nach links wirkt, also auf $\overline{\psi}_{j}$.


\subsection{Bilinearformen des Dirac-Feldes}

Man nennt $\overline{\psi}\left(x\right)\gamma^{\sigma}\psi\left(x\right)$
eine \emph{Bilinearform}. Physikalische Größen entsprechen Bilinearformen.
Wir kennen bereits die Vektordichte:
\begin{align*}
V^{\sigma}\left(x\right) & =\overline{\psi}\left(x\right)\gamma^{\sigma}\psi\left(x\right)
\end{align*}
$V^{\sigma}\left(x\right)$ ist hermitesch:
\begin{align*}
\left[V^{\sigma}\left(x\right)\right]^{\dagger} & =\psi^{\dagger}\left(x\right)\left(\gamma^{\sigma}\right)^{\dagger}\left(\gamma_{0}\right)^{\dagger}\psi\left(x\right)=\psi^{\dagger}\left(x\right)\left(\gamma_{0}\right)^{2}\left(\gamma^{\sigma}\right)^{\dagger}\left(\gamma_{0}\right)^{\dagger}\psi\left(x\right)=\\
 & =\overline{\psi}\left(x\right)\gamma_{0}\left(\gamma^{\sigma}\right)^{\dagger}\gamma_{0}\psi\left(x\right)=\overline{\psi}\left(x\right)\gamma^{\sigma}\psi\left(x\right)
\end{align*}
Es muss 16 unabhängige hermitesche Bilinearformen geben. Wir wählen
als Basis solche Bilinearformen, die sich aus den $\gamma$-Matrizen
berechnen lassen:
\begin{align}
S\left(x\right) & =\overline{\psi}\left(x\right)\psi\left(x\right) &  & \text{Skalardichte (1 Komponenten)}\\
V^{\sigma}\left(x\right) & =\overline{\psi}\left(x\right)\gamma^{\sigma}\psi\left(x\right) &  & \text{Vektordichte (4 Komponenten)}\\
T^{\mu\nu}\left(x\right) & =\overline{\psi}\left(x\right)\sigma^{\mu\nu}\psi\left(x\right)=\psi^{\dagger}\gamma_{0}\frac{\ii}{2}\left(\gamma^{\mu}\gamma^{\nu}-\gamma^{\nu}\gamma^{\mu}\right)\psi &  & \text{Tensordichte (6 Komponenten)}\\
A^{\sigma}\left(x\right) & =\overline{\psi}\left(x\right)\gamma^{\sigma}\gamma_{5}\psi\left(x\right) &  & \text{Axialvektordichte (4 Komponenten)}\\
P\left(x\right) & =\overline{\psi}\left(x\right)\ii\gamma_{5}\psi\left(x\right) &  & \text{Pseudoskalardichte (1 Komponenten)}
\end{align}
\begin{align*}
P^{\dagger} & =\psi^{\dagger}\left(-\ii\right)\gamma_{5}\gamma_{0}\psi\left(x\right)=\ii\overline{\psi}\gamma_{5}\psi
\end{align*}
\begin{align*}
\left(T^{\mu\nu}\left(x\right)\right)^{\dagger} & =\psi^{\dagger}\left(\gamma^{\nu\dagger}\gamma^{\mu\dagger}-\gamma^{\nu\dagger}\gamma^{\mu\dagger}\right)\left(-\frac{\ii}{2}\right)\gamma_{0}\psi=\\
 & =\psi^{\dagger}\left(\left(\gamma_{0}\right)^{2}\gamma^{\nu\dagger}\left(\gamma_{0}\right)^{2}\gamma^{\mu\dagger}-\left(\gamma_{0}\right)^{2}\gamma^{\nu\dagger}\left(\gamma_{0}\right)^{2}\gamma^{\mu\dagger}\right)\left(-\frac{\ii}{2}\right)\gamma_{0}\psi=\\
 & =-\overline{\psi}\frac{\ii}{2}\left(\gamma^{\nu}\gamma^{\mu}-\gamma^{\mu}\gamma^{\nu}\right)\psi=\overline{\psi}\sigma^{\mu\nu}\psi
\end{align*}
Die Axialvektordichte und die Pseudoskalardichte transformieren sich
unter stetigen Lorentz-Transformationen wie ein Vektor beziehungsweise
wie ein Skalar. Allerdings ändern sie unter der Paritätstransformation
ihr Vorzeichen.

Es gibt unter diesen 16 keine Bilinearform, die ein symmetrischer
Tensor 2. Stufe ist. Daher kann man so nicht an das Gravitationsfeld,
dass durch den symmetrischen Metrik-Tensor $g_{\mu\nu}$ beschrieben
wird koppeln.

%DATE: Di 23.4.13


\subsection{Die Transformationen \texorpdfstring{$\hat{C},\hat{P},\hat{T}$}{C, P, T}}


\subsubsection*{Ladungskonjugation $\hat{C}$}

Die Ladungskonjugation vertauscht Teilchen und Antiteilchen.
\begin{align}
 & \text{Elektron-Dirac-Gleichung:} & \left(\ii\gamma_{\mu}\partial^{\mu}+q_{e}A^{\mu}\gamma_{\mu}-m\right)\psi & =0\\
 & \text{Positron-Dirac-Gleichung:} & \left(\ii\gamma_{\mu}\partial^{\mu}-q_{e}A^{\mu}\gamma_{\mu}-m\right)\psi & =0
\end{align}
Überlegung: Ein Elektron, das mit Energie $E>0$ und Impuls $\vec{p}$
vorwärts in der Zeit läuft entspricht einem Positron, dass mit Energie
$-E<0$ und Impuls $\vec{p}$ rückwärts in der Zeit läuft.
\begin{align*}
e^{-\ii\left(Et-\vec{p}\vec{x}\right)} & =e^{-\ii\left(\left(-E\right)\left(-t\right)-\vec{p}\vec{x}\right)}
\end{align*}


\begin{figure}[H]
\noindent \begin{centering}
\begin{tikzpicture}[scale=0.5]
  \draw plot[smooth,tension=.7] coordinates{(-2,-3) (-0.5,-2) (0.5,2) (2,3)};
  \node at (-2,-3) [left] {$t_1$};
  \node at (2,3) [right] {$t_2$};
  \draw[thick,->] (-0.9,-1) -- node[left]{$e^-$} (-0.5,1);
  \draw[thick,<-] (0.5,-1) -- node[right]{$e^+$} (0.9,1);
\end{tikzpicture}
\par\end{centering}

\noindent \centering{}\caption{In der Zeit vorwärts laufendes Elektron entspricht in der Zeit rückwärts
laufendem Positron.}
\end{figure}


Hieraus motivieren wir den Ansatz:
\begin{align}
\psi_{C}\left(x\right) & =C\psi^{*}\left(x\right)
\end{align}
Dabei ist $C\in\text{GL}_{4}\left(\mathbb{C}\right)$ eine beliebige
invertierbare Matrix. Komplexe Konjugation der Elektron-Dirac-Gleichung
und Einfügen von $C$ liefert:
\begin{align*}
C\left(-\ii\gamma_{\mu}^{*}\partial^{\mu}+q_{e}A^{\mu}\gamma_{\mu}^{*}-m\right)C^{-1}C\psi^{*} & =0\\
\left(-\ii C\gamma_{\mu}^{*}C^{-1}\partial^{\mu}+q_{e}A^{\mu}C\gamma_{\mu}^{*}C^{-1}-m\right)\psi_{C} & =0
\end{align*}
Damit dies in die Positron-Dirac-Gleichung übergeht, muss
\begin{align*}
C\gamma_{\mu}^{*}C^{-1} & =-\gamma_{\mu}
\end{align*}
gelten. Die Lösung davon ist:
\begin{align*}
C & =\gamma_{2}e^{\ii\varphi} & C^{-1} & =-\gamma_{2}e^{-\ii\varphi}
\end{align*}
\begin{align*}
C\gamma_{2}^{*}C^{-1} & =\gamma_{2}e^{\ii\varphi}\left(-\gamma_{2}\right)\left(-\gamma_{2}\right)e^{-\ii\varphi}=\gamma_{2}^{3}=-\gamma_{2}
\end{align*}
Für $\mu\not=2$ gilt:
\begin{align*}
C\gamma_{\mu}^{*}C^{-1} & =\gamma_{2}e^{\ii\varphi}\gamma_{\mu}\left(-\gamma_{2}\right)e^{-\ii\varphi}=-\gamma_{2}\gamma_{\mu}\gamma_{2}=\gamma_{\mu}\gamma_{2}\gamma_{2}=-\gamma_{\mu}
\end{align*}
Damit folgt:
\begin{align}
\psi_{C} & =e^{\ii\varphi}\left(\begin{array}{cccc}
0 & 0 & 0 & \ii\\
0 & 0 & -\ii & 0\\
0 & -\ii & 0 & 0\\
\ii & 0 & 0 & 0
\end{array}\right)\psi^{*}
\end{align}
Beispiel:
\begin{align*}
\psi_{4} & =v\left(p,+\right)e^{\ii p\cdot x}=\sqrt{\frac{E+m}{2m}}\left(\begin{array}{c}
\frac{p_{1}-\ii p_{2}}{E+m}\\
\frac{-p_{3}}{E+m}\\
0\\
1
\end{array}\right)e^{\ii px}\\
\psi_{1} & =u\left(p,+\right)e^{-\ii p\cdot x}=\sqrt{\frac{E+m}{2m}}\left(\begin{array}{c}
1\\
0\\
\frac{p_{3}}{E+m}\\
\frac{p_{1}+\ii p_{2}}{E+m}
\end{array}\right)e^{-\ii px}
\end{align*}
\begin{align*}
\left(\psi_{1}\right)_{C} & =e^{\ii\varphi}\left(\begin{array}{cccc}
0 & 0 & 0 & \ii\\
0 & 0 & -\ii & 0\\
0 & -\ii & 0 & 0\\
\ii & 0 & 0 & 0
\end{array}\right)\sqrt{\frac{E+m}{2m}}\left(\begin{array}{c}
1\\
0\\
\frac{p_{3}}{E+m}\\
\frac{p_{1}-\ii p_{2}}{E+m}
\end{array}\right)e^{\ii px}=\\
 & =\ii e^{\ii\varphi}\sqrt{\frac{E+m}{2m}}\left(\begin{array}{c}
\frac{p_{1}-\ii p_{2}}{E+m}\\
\frac{-p_{3}}{E+m}\\
0\\
1
\end{array}\right)e^{-\ii px}=\ii e^{\ii\varphi}\psi_{4}
\end{align*}
Ein physikalisches Positron $\psi_{C}$ mit positiver Energie ist
äquivalent zu einer Lösung $\psi$ mit negativer Energie.


\subsubsection*{Paritätstransformation $\hat{P}$}

Die Paritätstransformation ist eine räumliche Spiegelung am Ursprung:
\begin{align}
x^{\mu} & \to\left(x'\right)^{\mu}=\left(x^{0},-\vec{x}\right)
\end{align}
Wir machen den Ansatz:
\begin{align}
\psi_{P}\left(x'\right) & =\psi\left(x^{0},-\vec{x}\right)=P\psi\left(x\right)
\end{align}
Die Dirac-Gleichung für $x'$ ist:
\begin{align*}
\left(\ii\gamma_{\mu}\frac{\partial}{\partial\left(x'\right)^{\mu}}+q_{e}\gamma^{\mu}A_{\mu}\left(x'\right)-m\right)\psi_{P}\left(x'\right) & =0
\end{align*}
Die soll äquivalent sein zu:
\begin{align*}
\left(\ii\gamma^{\mu}\frac{\partial}{\partial x^{\mu}}+q_{e}\gamma^{\mu}A_{\mu}\left(x\right)-m\right)\psi\left(x\right) & =0
\end{align*}
Wegen $\vec{E}\stackrel{P}{\to}-\vec{E}$ und $\vec{B}\stackrel{P}{\to}\vec{B}$
sowie $\nabla\stackrel{P}{\to}-\nabla$ und $\partial_{t}\stackrel{P}{\to}\partial_{t}$
folgt $A^{0}\stackrel{P}{\to}A^{0}$ und $A^{j}\stackrel{P}{\to}-A^{j}$.
\begin{align*}
\Rightarrow\qquad P^{-1}\left(\ii\gamma^{0}\frac{\partial}{\partial x^{0}}-\ii\gamma^{j}\frac{\partial}{\partial x^{j}}+q_{e}\gamma^{0}A_{0}+q_{e}\gamma^{j}A_{j}-m\right)P\psi & =0
\end{align*}
Es muss also gelten:
\begin{align*}
P^{-1}\gamma^{0}P & =\gamma^{0} & P^{-1}\gamma^{j}P & =-\gamma^{j}
\end{align*}
\begin{align}
\Rightarrow\qquad P & =e^{\ii\chi}\gamma_{0} & P^{-1} & =e^{-\ii\chi}\gamma_{0}
\end{align}



\subsubsection*{Zeitumkehrinvarianz $\hat{T}$}

Wegen $\vec{E}\stackrel{T}{\to}\vec{E}$ und $\vec{B}\stackrel{T}{\to}-\vec{B}$
sowie $\nabla\stackrel{T}{\to}\nabla$ und $\partial_{t}\stackrel{T}{\to}-\partial_{t}$
folgt $A^{0}\stackrel{T}{\to}A^{0}$ und $A^{j}\stackrel{T}{\to}-A^{j}$.
Wir machen den Ansatz:
\begin{align}
\psi_{T}\left(x'\right) & =T\psi^{*}\left(x\right)
\end{align}
Die Dirac-Gleichung für $\psi_{T}$ ist:
\begin{align*}
\left(-\ii\gamma^{0}\frac{\partial}{\partial x^{0}}+\ii\gamma^{j}\frac{\partial}{\partial x^{j}}+q_{e}\gamma^{0}A_{0}\left(x\right)+q_{e}\gamma^{j}A_{j}\left(x\right)-m\right)T\psi^{*}\left(x\right) & =0\qquad/^{*}\\
\left(T^{*}\right)^{-1}\left(\ii\gamma^{0}\frac{\partial}{\partial x^{0}}-\ii\left(\gamma^{j}\right)^{*}\frac{\partial}{\partial x^{j}}+q_{e}\gamma^{0}A_{0}\left(x\right)+q_{e}\left(\gamma^{j}\right)^{*}A_{j}\left(x\right)-m\right)T^{*}\psi\left(x\right) & =0
\end{align*}
Nun muss gelten:
\begin{align*}
\left(T^{*}\right)^{-1}\gamma^{0}T^{*} & =\gamma^{0} & \left(T^{*}\right)^{-1}\left(\gamma^{j}\right)^{*}T^{*} & =-\gamma^{j}
\end{align*}
Das bedeutet:
\begin{align*}
\left(T^{*}\right)^{-1}\gamma^{0}T^{*} & =\gamma^{0} & \left(T^{*}\right)^{-1}\gamma^{1}T^{*} & =-\gamma^{1}\\
\left(T^{*}\right)^{-1}\gamma^{2}T^{*} & =\gamma^{2} & \left(T^{*}\right)^{-1}\gamma^{3}T^{*} & =-\gamma^{3}
\end{align*}
Die Lösung ist:
\begin{align}
T & =\gamma^{1}\gamma^{3}e^{\ii\xi} & T^{-1} & =\gamma^{3}\gamma^{1}e^{-\ii\xi}=T^{*}
\end{align}
Probe:
\begin{align*}
T^{-1}T & =\gamma^{3}\gamma^{1}\gamma^{1}\gamma^{3}=-\gamma^{3}\gamma^{3}=\mathbbm{1}
\end{align*}
\begin{align*}
\gamma^{1}\gamma^{3}\gamma^{0}\gamma^{3}\gamma^{1} & =\left(-1\right)^{2}\gamma^{0}\gamma^{1}\gamma^{3}\gamma^{3}\gamma^{1}=\gamma^{0}\\
\gamma^{1}\gamma^{3}\gamma^{1}\gamma^{3}\gamma^{1} & =-\gamma^{1}\gamma^{1}\gamma^{3}\gamma^{3}\gamma^{1}=-\gamma^{1}\\
\gamma^{1}\gamma^{3}\gamma^{2}\gamma^{3}\gamma^{1} & =\left(-1\right)^{2}\gamma^{2}\gamma^{1}\gamma^{3}\gamma^{3}\gamma^{1}=\gamma^{2}\\
\gamma^{1}\gamma^{3}\gamma^{3}\gamma^{3}\gamma^{1} & =-\gamma^{1}\gamma^{3}\gamma^{1}=\gamma^{1}\gamma^{1}\gamma^{3}=-\gamma^{3}
\end{align*}



\chapter{Greensche Funktion (Feynman-Propagator)}

Erinnerung: Das elektrische Potential $\phi$ erfüllt in der Elektrostatik
die Differentialgleichung:
\begin{align}
\upDelta\phi & =4\pi\rho
\end{align}
Dabei ist $\rho$ die Ladungsdichte. Die Greensche Funktion $G\left(x,x'\right)$
ist definiert durch:
\begin{align}
\upDelta_{x}G\left(x,x'\right) & =\delta^{\left(3\right)}\left(x-x'\right)
\end{align}
Damit ergibt sich die Lösung obiger Differentialgleichung zu:
\begin{align}
\phi\left(x\right) & =4\pi\int\dd^{3}x'G\left(x,x'\right)\rho\left(x'\right)
\end{align}
Probe:
\begin{align*}
\upDelta_{x}\phi\left(x\right) & =4\pi\int\dd^{3}x'\delta\left(x-x'\right)\rho\left(x'\right)=4\pi\rho\left(x\right)
\end{align*}



\section{Greensche Funktion der Klein-Gordon-Gleichung}

Die definierende Gleichung ist:
\begin{align}
\left(\square_{x}+m^{2}\right)G\left(x-x'\right) & =-\delta^{\left(4\right)}\left(x-x'\right)
\end{align}
Das Minus auf der rechten Seite ist Konvention und wird wegen $\square_{x}=-\hat{p}^{2}$
eingefügt.\\
Die Dirac-Greensche Funktion ist einfach:
\begin{align}
S\left(x-x'\right) & =\left(\ii\gamma^{\mu}\partial_{x,\mu}+m\right)G\left(x-x'\right)
\end{align}
Es gilt nämlich:
\begin{align*}
\left(\ii\gamma^{\nu}\partial_{x,\nu}-m\right)S\left(x-x'\right) & =\left(-\square_{x}-m^{2}\right)G\left(x-x'\right)=\delta^{\left(4\right)}\left(x-x'\right)
\end{align*}
Am einfachsten ist die Bestimmung von $G$ im Impulsraum. Wir führen
also eine Fourier-Transformation durch:
\begin{align}
G\left(x-x'\right) & =\int\frac{\dd^{4}p}{\left(2\pi\right)^{4}}G\left(p\right)e^{-\ii p\left(x-x'\right)}\label{eq:Fouriertrafo-Green}
\end{align}
\begin{align*}
\left(\square_{x}+m^{2}\right)G\left(x-x'\right) & =\int\frac{\dd^{4}p}{\left(2\pi\right)^{4}}\left(-p^{\mu}p_{\mu}+m^{2}\right)G\left(p\right)e^{-\ii p\left(x-x'\right)}\stackrel{!}{=}-\int\frac{\dd^{4}p}{\left(2\pi\right)^{4}}e^{-\ii p\left(x-x'\right)}
\end{align*}
Daher muss gelten:
\begin{align}
\left(p^{2}-m^{2}\right)G\left(p\right) & =1
\end{align}
Ist $p^{2}-m^{2}\not=0$, so folgt:
\begin{align*}
G\left(p\right) & =\frac{1}{p^{2}-m^{2}}
\end{align*}
Was passiert für $p^{2}=\left(p^{0}\right)^{2}-\vec{p}^{2}=m^{2}$?
\begin{align}
p^{0} & =\pm\sqrt{\vec{p}^{2}+m^{2}}
\end{align}
Da für $p^{2}=m^{2}$ Divergenzen auftreten, muss man (\ref{eq:Fouriertrafo-Green})
um eine Zusatzvorschrift ergänzen, wie man diese umschiffen soll.

\begin{figure}[H]
\noindent \begin{centering}
\hfill{}\subfloat[Integrationswege um die Pole herum]{\begin{tikzpicture}
  \draw[thick,->] (-3,0) -- (3,0) node[right]{Re$(p^\mu)$};
  \draw[thick,->] (0,-3) -- (0,3) node[above]{Im$(p^\mu)$};
  \draw[very thick, orange] (-3,0) -- (-2,0) arc (-180:0:0.25) -- (1.5,0) arc(-180:0:0.25) node[below=7pt]{1} -- (3,0);
  \draw[very thick, green] (-3,0) -- (-2.1,0) arc (180:0:0.35) -- (1.4,0) node[below =3pt]{2} arc(-180:0:0.35) -- (3,0);
  \draw[very thick, red] (-3,0) -- (-2.1,0) arc (-180:0:0.35) -- (1.4,0) node[above =3pt]{3} arc(180:0:0.35) -- (3,0);
  \draw[very thick, blue] (-3,0) -- (-2,0) arc (180:0:0.25) -- (1.5,0) arc(180:0:0.25) node[above=7pt]{4} -- (3,0);
  \draw (1.75,0) node[below=15pt]{$\sqrt{\vec{p}^2-m^2}$} +(-0.1,0.1) -- +(0.1,-0.1) +(-0.1,-0.1) -- +(0.1,0.1);
  \draw (-1.75,0) node[below=15pt]{$-\sqrt{\vec{p}^2-m^2}$} +(-0.1,0.1) -- +(0.1,-0.1) +(-0.1,-0.1) -- +(0.1,0.1);
\end{tikzpicture}}\hfill{}\subfloat[äquivalent: Pole verschieben]{\begin{tikzpicture}
  \draw[thick,->] (-3,0) -- (3,0) node[right]{Re$(p^\mu)$};
  \draw[thick,->] (0,-3) -- (0,3) node[above]{Im$(p^\mu)$};
  \draw[fill, orange] (-1.75,0.25) circle (2pt) (1.75,0.25) circle (2pt);
  \draw[fill, green] (-1.75,-0.5) circle (2pt) (1.75,0.5) circle (2pt);
  \draw[fill, red] (-1.75,0.5) circle (2pt) (1.75,-0.5) circle (2pt);
  \draw[fill, blue] (-1.75,-0.25) circle (2pt) (1.75,-0.25) circle (2pt);
  \draw (1.75,0) node[below=15pt]{$\sqrt{\vec{p}^2-m^2}$} +(-0.1,0.1) -- +(0.1,-0.1) +(-0.1,-0.1) -- +(0.1,0.1);
  \draw (-1.75,0) node[below=15pt]{$-\sqrt{\vec{p}^2-m^2}$} +(-0.1,0.1) -- +(0.1,-0.1) +(-0.1,-0.1) -- +(0.1,0.1);
\end{tikzpicture}}\hspace*{\fill}
\par\end{centering}

\noindent \centering{}\caption{Die Pole müssen umgangen werden.}
\end{figure}


Mit $\varepsilon,\eta\in\mathbb{R}_{>0}$ lassen sich die vier Möglichkeiten
schreiben als:
\begin{enumerate}
\item ${\displaystyle \frac{1}{p^{2}-m^{2}-\ii\varepsilon\text{sgn}\left(p^{0}\right)}}$:
Pole bei $p^{0}=\pm\sqrt{\vec{p}^{2}+m^{2}}+\ii\eta$
\item ${\displaystyle \frac{1}{p^{2}-m^{2}-\ii\varepsilon}}$: Pole bei
$p^{0}=\pm\sqrt{\vec{p}^{2}+m^{2}}\pm\ii\eta$
\item ${\displaystyle \frac{1}{p^{2}-m^{2}+\ii\varepsilon}}$: Pole bei
$p^{0}=\pm\sqrt{\vec{p}^{2}+m^{2}}\mp\ii\eta$
\item ${\displaystyle \frac{1}{p^{2}-m^{2}+\ii\varepsilon\text{sgn}\left(p^{0}\right)}}$:
Pole bei $p^{0}=\pm\sqrt{\vec{p}^{2}+m^{2}}-\ii\eta$
\end{enumerate}
Mit Hilfe des Residuensatzes kann man so das Integral
\begin{align*}
\int\frac{\dd^{3}p}{\left(2\pi\right)^{3}}\int_{-\infty}^{\infty}\frac{\dd p^{0}}{2\pi}e^{-\ii p^{0}\left(t-t'\right)+\ii\vec{p}\left(\vec{x}-\vec{x}'\right)}G\left(p^{0},\vec{p}\right)
\end{align*}
ausrechnen. Für $t'>t$ (Propagation von $t'$ nach $t$ rückwärts
in der Zeit) kann man oben schließen, da dann der Faktor $e^{-\ii^{2}\text{Im}\left(p^{0}\right)\left(t-t'\right)}$
exponentiell abfällt. Ebenso kann man für $t>t'$ (vorwärts in der
Zeit) unten schließen. Bei der Propagation vorwärts in der Zeit darf
der Pol bei $p_{0}<0$ nicht im Integrationsbereich sein und $p_{0}>0$
darf nicht rückwärts in der Zeit propagieren. Daher bleibt nur die
3. Lösung. Diese wird \emph{Feynman-Propagator} genannt und Berechnungen
ergeben:
\begin{align}
G\left(x\right) & =-\frac{1}{4\pi}\delta\left(x^{2}\right)+\frac{m}{8\pi\sqrt{x^{2}}}\Theta\left(x^{2}\right)\left(J_{1}\left(m\sqrt{x^{2}}\right)-\ii Y_{1}\left(m\sqrt{x^{2}}\right)\right)-\nonumber \\
 & \qquad-\frac{\ii m}{4\pi^{2}\sqrt{-x^{2}}}\Theta\left(-x^{2}\right)K_{1}\left(m\sqrt{-x^{2}}\right)
\end{align}
Dabei ist $\delta$ die Delta-Distribution, $\Theta$ die Heaviside-Sprungfunktion,
$J_{1}$ die Bessel-Funktion erster Ordnung erster Gattung und $Y_{1}$
die Bessel-Funktion erster Ordnung zweiter Gattung, sowie $K_{1}$
die modifizierte Bessel-Funktion erster Ordnung zweiter Gattung.

\begin{figure}[H]
\noindent \begin{centering}
\begin{tikzpicture}
 \begin{axis}[width=13cm, axis x line=middle, axis y line=middle, xlabel=$x$, x label style={ yshift=6pt}, ylabel=$\hspace*{2mm}y$, xmin=0, ymin=-5.2, ymax=5.2, xmax=9.8, samples=300]
  \addplot[mark=none] file {besj1.txt};
  \addlegendentry{$J_1(x)$}
  \addplot[mark=none, dashed, blue] file {besy1.txt};
  \addlegendentry{$Y_1(x)$}
  \addplot[mark=none, thick, red] file {besk1.txt};
  \addlegendentry{$K_1(x)$}
 \end{axis}
\end{tikzpicture}
\par\end{centering}

\noindent \centering{}\caption{Bessel-Funktionen}
\end{figure}


Dieses Ergebnis lässt sich direkt physikalisch interpretieren. Da
$x^{2}=0$ den Lichtkegel beschreibt, entspricht der Term mit $\delta\left(x^{2}\right)$
der Ausbreitung mit Lichtgeschwindigkeit. $J_{1}$ und $Y_{1}$ sind
oszillierende Funktionen, die die kausale Ausbreitung massiver Teilchen
beschreiben. Die modifizierte Bessel-Funktion $K_{1}\left(x\right)\stackrel{x\to\infty}{\sim}x^{-\frac{1}{2}}e^{-x}$
fällt exponentiell ab und erscheint wie eine nicht-kausale Ausbreitung.
Dies kann man physikalisch dadurch erklären, dass aufgrund der Unschärferelation
der Ort des Teilchens schon zu Beginn nicht genauer als die Compton-Wellenlänge
festgelegt war, und somit sich ein exponentiell kleiner Teil der Wellenfunktion
von einem anderen Ort her ausgebreitet hat. Dies ist keine Ausbreitung
mit Überlichtgeschwindigkeit!

%DATE: Mo 29.4.13 --> Anhang A

%DATE: Di 30.4.13


\section{Normierung der Wellenfunktionen}

\begin{align*}
u\left(\vec{p},+\right) & =\sqrt{\frac{E+m}{2m}}\left(\begin{array}{c}
1\\
0\\
\frac{p^{3}}{E+m}\\
\frac{p^{1}+\ii p^{2}}{E+m}
\end{array}\right)
\end{align*}
\begin{align*}
\overline{u}u & =\frac{E+m}{2m}\cdot\frac{\left(E+m\right)^{2}-\vec{p}^{2}}{\left(E+m\right)^{2}}=\frac{E+m}{2m}\cdot\frac{E^{2}+2mE+m^{2}-\vec{p}^{2}}{\left(E+m\right)^{2}}=\\
 & =\frac{E+m}{2m}\cdot\frac{2mE+2m^{2}}{\left(E+m\right)^{2}}=\frac{E+m}{E+m}=1
\end{align*}
Führe nun einen möglichen Normierungsfaktor $\mathcal{N}$ für die
Wellenfunktion $\psi$ ein:
\begin{align}
\psi_{1}\left(\vec{p},x\right) & =\mathcal{N}u\left(\vec{p},+\right)e^{-\ii px}
\end{align}
\begin{align*}
\int_{\mathbb{R}^{3}}\dd^{3}x\overline{\psi_{1}}\left(\vec{p},x\right)\psi_{1}\left(\vec{p},x\right) & \stackrel{?}{=}\infty
\end{align*}
Man müsste also für die Normierung $\mathcal{N}=0$ wählen, was keinen
Sinn ergibt. In einem endlichen Volumen $V$ wäre dieses Integral
proportional zu $V$. Beachte nun die Analogie:
\begin{align}
\int_{\mathbb{R}^{3}}\frac{\dd^{3}}{\left(2\pi\right)^{3}}e^{\ii\vec{p}\vec{x}} & =\delta^{\left(3\right)}\left(\vec{p}\right)\nonumber \\
\int_{V}\frac{\dd^{3}}{\left(2\pi\right)^{3}}e^{0} & =\frac{V}{\left(2\pi\right)^{3}}\:\widehat{=}\:\delta^{\left(3\right)}\left(0\right)
\end{align}
Wir machen daher folgenden Ansatz für die Normierung:
\begin{align}
\int_{\mathbb{R}^{3}}\dd^{3}x\overline{\psi}_{1}\left(\vec{p},x\right)\psi\left(\vec{p}',x\right) & =\abs{\mathcal{N}}^{2}\int_{\mathbb{R}^{3}}\dd^{3}xe^{-\ii\left(p'-p\right)x}=\abs{\mathcal{N}}^{2}\left(2\pi\right)^{3}\delta^{\left(3\right)}\left(\vec{p}'-\vec{p}\right)=:I
\end{align}
Wie muss man $\abs{\mathcal{N}}^{2}$ wählen, sodass $I$ eine Lorentz-Invariante
ist? Das Integral
\begin{align*}
\int\frac{\dd^{4}p}{\left(2\pi\right)^{3}}\delta\left(p^{2}-m^{2}\right)
\end{align*}
ist wegen $\abs{\det\left(\Lambda\right)}=1$ Lorentz-invariant und
es gilt:
\begin{align*}
\int\frac{\dd^{4}p}{\left(2\pi\right)^{3}}\delta\left(p^{2}-m^{2}\right) & =\int\frac{\dd^{3}p}{\left(2\pi\right)^{3}}\int\dd E\delta\left(E^{2}-\vec{p}^{2}-m^{2}\right)=\\
 & =\int\frac{\dd^{3}p}{\left(2\pi\right)^{3}}\int\dd E\delta\left(\left(E-\sqrt{\vec{p}^{2}+m^{2}}\right)\left(E+\sqrt{\vec{p}^{2}+m^{2}}\right)\right)=\\
 & =\int\frac{\dd^{3}p}{\left(2\pi\right)^{3}}\int\dd E\frac{1}{2\abs E}\left(\delta\left(E-\sqrt{\vec{p}^{2}+m^{2}}\right)+\delta\left(E+\sqrt{\vec{p}^{2}+m^{2}}\right)\right)
\end{align*}
Nun sind
\begin{align*}
\int\frac{\dd^{3}p}{2\abs E}\cdot2\abs E\delta^{\left(3\right)}\left(p\right)=\int\dd^{3}p\delta^{\left(3\right)}\left(p\right) & =1\qquad\text{und}\qquad\frac{\dd^{3}p}{2\abs E}
\end{align*}
Lorentz-Invarianten und somit auch $2\abs E\delta^{\left(3\right)}\left(p\right)$.
Man kann $I$ explizit Lorentz-invariant machen, indem man wählt:
\begin{align}
\mathcal{N} & =\sqrt{\frac{\abs E}{m}}
\end{align}
Diese Wahl sorgt dafür, dass im Ruhesystem die $u\left(\vec{p},+\right)$
etc. nach wie vor Einheitsspinoren sind. Wir ersetzen also:
\begin{align}
\psi\left(\vec{p},m\right) & \to\tilde{\psi}\left(\vec{p},m\right)=\sqrt{\frac{\abs E}{m}}\psi\left(\vec{p},m\right)\label{eq:Normierung-psi}\\
\Rightarrow\qquad\sum_{s}\tilde{u}\left(\vec{p},s\right)\overline{\tilde{u}}\left(\vec{p},s\right) & =\frac{m}{\abs E}\cdot\frac{\slashd p+m}{2m}=\frac{\slashd p+m}{2\abs E}
\end{align}
Dies ist eine sehr vorteilhafte Normierung für Hochenergiephysik,
da man jetzt bereits von Anfang an den Limes $m\to0$ bilden kann
und nicht erst ganz am Ende.


\chapter{Kanonische Quantisierung \texorpdfstring{$\hat{b}^{\dagger},\hat{d}^{\dagger},\hat{b},\hat{d}$}{}}

\emph{Quantisierung} ist ein Satz von Regeln, der die korrekten Greenschen
Funktionen liefert.


\section{Umformung des Feynman-Propagators}

Der Feynman-Propagator ist:
\begin{align}
S_{F}\left(x-y\right) & =\int\frac{\dd^{4}p}{\left(2\pi\right)^{4}}e^{-\ii p\left(x-y\right)}\frac{\slashd p+m}{p^{2}-m^{2}+\ii\varepsilon}\left(\Theta\left(x^{0}-y^{0}\right)+\Theta\left(y^{0}-x^{0}\right)\right)
\end{align}
Der erste Summand gibt bei der $p^{0}$-Integration:
\begin{align*}
 & \int_{-\infty}^{\infty}\frac{\dd E}{\left(E-\sqrt{p^{2}-m^{2}+\ii\varepsilon}\right)\left(E+\sqrt{p^{2}-m^{2}+\ii\varepsilon}\right)}e^{-\ii E\left(x^{0}-y^{0}\right)}
\end{align*}
\begin{figure}[H]
\noindent \begin{centering}
\begin{tikzpicture}
  \draw[decoration={markings,mark=at position 1 with {\arrow[scale=1.75]{>}};},postaction={decorate}] (-3,0) -- (3,0) node[right]{$x$};
  \draw[decoration={markings,mark=at position 1 with {\arrow[scale=1.75]{>}};},postaction={decorate}] (0,-3) -- (0,3) node[above]{$y$};
  \draw[fill] (1.5,-0.2) circle (2pt);
  \draw[fill] (-1.5,0.2) circle (2pt);
  \draw[thick,decoration={markings,mark=at position 0.7 with {\arrow[scale=1.75]{>}};},postaction={decorate}] (2,0) arc (0:-180:2);
  \draw[thick,decoration={markings,mark=at position 0.7 with {\arrow[scale=1.75]{>}};},postaction={decorate}] (-1,0) -- (-0.98,0);
  \draw[orange, thick,decoration={markings,mark=at position 0.7 with {\arrow[scale=1.75]{>}};},postaction={decorate}] (2,0) arc (0:180:2);
  \draw[orange, thick,decoration={markings,mark=at position 0.7 with {\arrow[scale=1.75]{>}};},postaction={decorate}] (0.98,0) -- (1,0);
\end{tikzpicture}
\par\end{centering}

\noindent \centering{}\caption{Pole bei der Integration}
\end{figure}


Mit dem Residuensatz folgt:
\begin{align*}
\left(x^{0}-y^{0}\right)>0 & \quad\Rightarrow\quad\frac{-2\pi\ii}{2\sqrt{\vec{p}^{2}+m^{2}}}\\
\left(x^{0}-y^{0}\right)>0 & \quad\Rightarrow\quad\frac{2\pi\ii}{-2\sqrt{\vec{p}^{2}+m^{2}}}
\end{align*}
Mit $E_{p}:=\sqrt{\vec{p}^{2}+m^{2}}$ gibt dies:
\begin{align*}
S_{F}\left(x-y\right) & =-\ii\int\frac{\dd^{3}p}{\left(2\pi\right)^{3}}e^{-\ii E_{p}\left(x^{0}-y^{0}\right)+\ii\vec{p}\left(\vec{x}-\vec{y}\right)}\frac{E_{p}\gamma_{0}-\vec{p}\cdot\vec{\gamma}+m}{2E_{p}}\Theta\left(x^{0}-y^{0}\right)-\\
 & \quad-\ii\int\frac{\dd^{3}p}{\left(2\pi\right)^{3}}e^{\ii E_{p}\left(x^{0}-y^{0}\right)+\ii\vec{p}\left(\vec{x}-\vec{y}\right)}\frac{-E_{p}\gamma_{0}-\vec{p}\cdot\vec{\gamma}+m}{2E_{p}}\Theta\left(x^{0}-y^{0}\right)
\end{align*}
Im zweiten Integral gehe nun von $\vec{p}$ nach $-\vec{p}$ über
und erhalte:
\begin{align}
S_{F}\left(x-y\right) & =-\ii\int\frac{\dd^{3}p}{\left(2\pi\right)^{3}}e^{-\ii E_{p}\left(x^{0}-y^{0}\right)+\ii\vec{p}\left(\vec{x}-\vec{y}\right)}\frac{\overbrace{E_{p}\gamma_{0}-\vec{p}\cdot\vec{\gamma}}^{\slashd p}+m}{2E_{p}}\Theta\left(x^{0}-y^{0}\right)-\nonumber \\
 & \quad-\ii\int\frac{\dd^{3}p}{\left(2\pi\right)^{3}}e^{\ii E_{p}\left(x^{0}-y^{0}\right)-\ii\vec{p}\left(\vec{x}-\vec{y}\right)}\frac{\overbrace{-E_{p}\gamma_{0}+\vec{p}\cdot\vec{\gamma}}^{-\slashd p}+m}{2E_{p}}\Theta\left(x^{0}-y^{0}\right)=\nonumber \\
 & =-\ii\int\frac{\dd^{3}p}{\left(2\pi\right)^{3}}\left(\frac{\left(\slashd p+m\right)}{2E_{p}}e^{-\ii p\left(x-y\right)}\Theta\left(x^{0}-y^{0}\right)-\frac{\left(\slashd p-m\right)}{2E_{p}}e^{\ii p\left(x-y\right)}\Theta\left(y^{0}-x^{0}\right)\right)=\nonumber \\
 & =-\ii\int\frac{\dd^{3}p}{\left(2\pi\right)^{3}}\Bigg(\sum_{r\in\left\{ 1,2\right\} }\tilde{\psi}_{r}\left(\vec{p},x\right)\overline{\tilde{\psi}}_{r}\left(\vec{p},y\right)\Theta\left(x^{0}-y^{0}\right)-\nonumber \\
 & \qquad\qquad\qquad\qquad-\sum_{r\in\left\{ 3,4\right\} }\tilde{\psi}_{r}\left(\vec{p},x\right)\overline{\tilde{\psi}}_{r}\left(\vec{p},y\right)\Theta\left(y^{0}-x^{0}\right)\Bigg)\label{eq:Feynman-Propagator-psi}
\end{align}
Hierbei sind die $\tilde{\psi}_{r}$ die freien Lösungen (\ref{eq:freie_Loesung}).
Produkte der Form $\tilde{\psi}_{r}\left(\vec{p},x\right)\overline{\tilde{\psi}}_{r}\left(\vec{p},y\right)$
sind Projektionen, ähnlich wie das Tensorprodukt
\begin{align*}
\vec{e}_{i}\otimes\vec{e}_{i} & :=\vec{e}_{i}\vec{e}_{i}^{\TT}
\end{align*}
im euklidischen Raum:
\begin{align*}
\vec{v} & =\sum_{i}\vec{e}_{i}\left(\vec{e}_{i}\cdot\vec{v}\right)=\sum_{i}\left(\vec{e}_{i}\vec{e}_{i}^{\TT}\right)\vec{v}
\end{align*}
Diese Darstellung von $\slashd p\pm m$ haben wir in \ref{sub:Projektions-Operatoren}
schon kennen gelernt, beachte hier allerdings die andere Normierung
(\ref{eq:Normierung-psi}).


\section{Definition der Kanonischen Quantisierung}

Die \emph{kanonische Quantisierung} definieren wir wie folgt:
\begin{align}
\hat{\psi}\left(x\right) & :=\sum_{s\in\left\{ \pm\right\} }\int_{\mathbb{R}^{3}}\frac{\dd^{3}p}{\left(2\pi\right)^{3}\cdot2E_{p}}\left(\hat{b}\left(\vec{p},s\right)u\left(\vec{p},s\right)e^{-\ii px}+\hat{d}^{\dagger}\left(\vec{p},s\right)v\left(\vec{p},s\right)e^{\ii px}\right)\\
\hat{\psi}^{\dagger}\left(x\right)\gamma_{0} & :=\sum_{s\in\left\{ \pm\right\} }\int_{\mathbb{R}^{3}}\frac{\dd^{3}p}{\left(2\pi\right)^{3}\cdot2E_{p}}\left(\hat{b}^{\dagger}\left(\vec{p},s\right)\overline{u}\left(\vec{p},s\right)e^{\ii px}+\hat{d}\left(\vec{p},s\right)\overline{v}\left(\vec{p},s\right)e^{-\ii px}\right)
\end{align}
Definiere noch ein zeitgeordnetes Produkt ($j,l$: Spinorindizes).
\begin{align}
\mathcal{T}\left\{ \hat{\psi}_{j}\left(x\right)\hat{\overline{\psi}}_{l}\left(y\right)\right\}  & :=\Theta\left(x^{0}-y^{0}\right)\hat{\psi}_{j}\left(x\right)\hat{\overline{\psi}}_{l}\left(y\right)-\Theta\left(y^{0}x^{0}\right)\hat{\overline{\psi}}_{l}\left(y\right)\hat{\psi}_{j}\left(x\right)
\end{align}
Die Anti-Kommutatoren der Erzeugunsoperatoren $\hat{b}^{\dagger}$,
$\hat{d}^{\dagger}$ und der Vernichtungsoperatoren $\hat{b},\hat{d}$
sollen alle verschwinden, außer:
\begin{align}
\left\{ \hat{b}\left(\vec{p},s\right),\hat{b}^{\dagger}\left(\vec{p}',s'\right)\right\}  & =\left\{ \hat{d}\left(\vec{p},s\right),\hat{d}^{\dagger}\left(\vec{p}',s'\right)\right\} =\delta_{ss'}\delta^{\left(3\right)}\left(\vec{p}-\vec{p}'\right)\left(2\pi\right)^{3}\cdot2p^{0}
\end{align}
Zudem fordern wir für den normierten Vakuumzustand $\KET 0$:
\begin{align}
\hat{b}\KET 0 & =0 & \hat{d}\KET 0 & =0\\
\BRA 0\hat{b}^{\dagger} & =0 & \BRA 0\hat{d}^{\dagger} & =0
\end{align}

\begin{description}
\item [{Behauptung:}] Für die kanonische Quantisierung gilt:
\begin{align}
\BraKet{0|\mathcal{T}\left\{ \hat{\psi}_{j}\left(x\right)\hat{\overline{\psi}}_{l}\left(y\right)\right\} |0} & =\left(\ii S_{F}\left(x-y\right)\right)_{jl}
\end{align}

\item [{Beweis:}] Wir rechnen nach:
\begin{align*}
\negthickspace\negthickspace\negthickspace\negthickspace\negthickspace\negthickspace\negthickspace\negthickspace\negthickspace\negthickspace & \BraKet{0|\mathcal{T}\left\{ \hat{\psi}_{j}\left(x\right)\hat{\overline{\psi}}_{l}\left(y\right)\right\} |0}=\\
\negthickspace\negthickspace\negthickspace\negthickspace\negthickspace\negthickspace\negthickspace\negthickspace\negthickspace\negthickspace & =\Theta\left(x^{0}-y^{0}\right)\sum_{s,s'}\int\frac{\dd^{3}p\dd^{3}p'}{\left(2\pi\right)^{6}\cdot2E\cdot2E'}\underbrace{\BraKet{0|\hat{b}\left(\vec{p},s\right)\hat{b}^{\dagger}\left(\vec{p}',s'\right)|0}}_{=\BraKet{0|\delta_{ss'}\delta^{\left(3\right)}\left(\vec{p}-\vec{p}'\right)\left(2\pi\right)^{3}\cdot2E|0}}u_{j}\left(\vec{p},s\right)\overline{u}_{l}\left(\vec{p}',s'\right)e^{-\ii px}e^{\ii p'y}-\\
\negthickspace\negthickspace\negthickspace\negthickspace\negthickspace\negthickspace\negthickspace\negthickspace\negthickspace\negthickspace & \ -\Theta\left(y^{0}-x^{0}\right)\sum_{s,s'}\int\frac{\dd^{3}p\dd^{3}p'}{\left(2\pi\right)^{6}\cdot2E\cdot2E'}\underbrace{\BraKet{0|\hat{d}\left(\vec{p},s\right)\hat{d}^{\dagger}\left(\vec{p}',s'\right)|0}}_{=\BraKet{0|\delta_{ss'}\delta^{\left(3\right)}\left(\vec{p}-\vec{p}'\right)\left(2\pi\right)^{3}\cdot2E|0}}\overline{v}_{l}\left(\vec{p},s\right)v_{j}\left(\vec{p}',s'\right)e^{\ii px}e^{-\ii p'y}=\\
\negthickspace\negthickspace\negthickspace\negthickspace\negthickspace\negthickspace\negthickspace\negthickspace\negthickspace\negthickspace & =\Theta\left(x^{0}-y^{0}\right)\sum_{s}\int\frac{\dd^{3}p}{\left(2\pi\right)^{3}\cdot2E}\left(u\left(\vec{p},s\right)\overline{u}\left(\vec{p}',s'\right)\right)_{jl}e^{-\ii p\left(x-y\right)}-\\
\negthickspace\negthickspace\negthickspace\negthickspace\negthickspace\negthickspace\negthickspace\negthickspace\negthickspace\negthickspace & \ -\Theta\left(y^{0}-x^{0}\right)\sum_{s}\int\frac{\dd^{3}p}{\left(2\pi\right)\cdot2E}\left(v\left(\vec{p}',s'\right)\overline{v}_{l}\left(\vec{p},s\right)\right)_{jl}e^{\ii p\left(x-y\right)}=\\
\negthickspace\negthickspace\negthickspace\negthickspace\negthickspace\negthickspace\negthickspace\negthickspace\negthickspace\negthickspace & =\int\frac{\dd^{3}p}{\left(2\pi\right)^{3}}\Bigg(\sum_{r\in\left\{ 1,2\right\} }\left(\tilde{\psi}_{r}\left(\vec{p},x\right)\overline{\tilde{\psi}}_{r}\left(\vec{p},y\right)\right)_{jl}\Theta\left(x^{0}-y^{0}\right)-\\
\negthickspace\negthickspace\negthickspace\negthickspace\negthickspace\negthickspace\negthickspace\negthickspace\negthickspace\negthickspace & \qquad\qquad\qquad\quad-\sum_{r\in\left\{ 3,4\right\} }\left(\tilde{\psi}_{r}\left(\vec{p},x\right)\overline{\tilde{\psi}}_{r}\left(\vec{p},y\right)\right)_{jl}\Theta\left(y^{0}-x^{0}\right)\Bigg)=\left(\ii S_{F}\left(x-y\right)\right)_{jl}
\end{align*}
\qqed[Behauptung]
\end{description}
Außerdem gilt die \emph{Mikrokausalität}:
\begin{align}
\left\{ \hat{\psi}_{j}\left(\vec{x},t\right),\hat{\psi}_{l}^{\dagger}\left(\vec{y},t\right)\right\}  & =\delta_{jl}\delta^{\left(3\right)}\left(\vec{x}-\vec{y}\right)
\end{align}


%DATE: Mo 6.5.13


\section{Übergang von der Kanonische Quantisierung zur Quantenfeldtheorie}

Nun wollen wir auch kompliziertere Prozesse mit mehr als zwei Feldoperatoren
berechnen. Aus der Quantenmechanik ist bekannt, dass die Wahrscheinlichkeiten
proportional zum Betragsquadrat der Summe der Amplituden ist. Die
Amplitude berechnen wir mit Störungstheorie. Die verschiedenen Ordnungen
kann man durch Feynman-Diagramme wie in Abbildung \ref{abb:Feynman-Diagramm}
beschreiben.

\begin{figure}[H]
\noindent \hfill{}\begin{tikzpicture}[thick, level/.style={level distance=1.7cm}]
  \path (0,0) node [left]{$x_1$} node[position] {}
    child [grow=right] {node[position]{}
      child [grow=down] {node[position]{}
        child [grow=right] {node[position]{}
          child [grow=north east, level distance=4.7cm,edge from parent path={(\tikzparentnode.north) .. controls +(0,2)  and +(-2,0) .. (\tikzchildnode.west)}] {node[position]{} node [right]{$x_3$} edge from parent [electron] node[above] {$e^-$}}
          child [grow=south east, level distance=4.7cm,edge from parent path={(\tikzparentnode.south) .. controls  +(0,-2) and +(-2,0) .. (\tikzchildnode.west)}] {node[position] {} node [right]{$x_6$}
            edge from parent [positron] node (a){} node[below left]{$e^+$}}
            edge from parent [photon]}
        child [grow=left] {node[left]{$x_2$} node[position]{} edge from parent [electron] node[below]{$e^+$}}
        edge from parent [electron]}
      child [grow=right, level distance=5cm] {node[position]{} node [right]{$x_7$}
        edge from parent [photon]}
      edge from parent [electron] node[above]{$e^-$}};
  \path (a) node[position]{}
    child [grow=north east,level distance=1.3cm] {node[position]{}
      child [grow=25, level distance=1.6cm,edge from parent path={(\tikzparentnode.north east) .. controls +(0.3,0.6)  and +(-0.8,-0.1) .. (\tikzchildnode.west)}] {node[position]{} node [right]{$x_4$} edge from parent [electron]  node[position] (b){} node[above right=2pt]{$e^-$}}
      child [grow=-25, level distance=1.6cm,edge from parent path={(\tikzparentnode.south east) .. controls +(0.3,-0.6)  and +(-0.8,0.1) .. (\tikzchildnode.west)}] {node[position]{} node [right]{$x_5$} edge from parent [positron] node[position] (c){} node[below right]{$e^+$}}
      edge from parent [photon]};
  \path (b) edge[photon] (c);
\end{tikzpicture}\hfill{}\begin{tikzpicture}[thick, level/.style={level distance=1.7cm}]
  \path (0,0) node [left]{$x_1$} node[position] {}
    child [grow=right] {node[position] {}
      child [grow=down] {node[position] {}
        child [grow=right] {node[position] (u){}
          child [grow=south east, level distance=3.7cm] {node[position] {} node [right]{$x_6$}
            edge from parent [positron] node[below left]{$e^+$}}
            edge from parent [photon]}
        child [grow=left] {node[left]{$x_2$} node[position]{} edge from parent [electron] node[below]{$e^+$}}
        edge from parent [electron]}
      child [grow=right] {node[position] (o){}
          child [grow=north east, level distance=3.7cm] {node[position] {} node [right]{$x_3$}
            edge from parent [electron] node[above left]{$e^-$}}
        edge from parent [photon]} 
      edge from parent [electron] node[above]{$e^-$} };
  \path (u) edge[electron] node (m) {} (o);
  \path (m) node[position]{}
    child [grow=east,level distance=1.1cm] {node[position]{}
      child [grow=25, level distance=1.6cm,edge from parent path={(\tikzparentnode.north east) .. controls +(0.3,0.6)  and +(-0.8,-0.1) .. (\tikzchildnode.west)}] {node[position]{} node [right]{$x_4$} edge from parent [electron] node[position,pos=0.15] (b){} node[above right]{$e^-$}}
      child [grow=-25, level distance=1.6cm,edge from parent path={(\tikzparentnode.south east) .. controls +(0.3,-0.6) and +(-0.8,0.1) .. (\tikzchildnode.west)}] {node[position]{} node [right]{$x_5$} edge from parent [positron] node[below right]{$e^+$}}
      edge from parent [photon]};
  \path (b) node[inner sep=0]{} child[grow=right,level distance=1.25cm] {node[position]{} node[right]{$x_7$} edge from parent [photon]};
\end{tikzpicture}\hfill{}\hspace*{1mm}\caption{Zwei verschiedene Feynman-Diagramme für einen Prozess}
\label{abb:Feynman-Diagramm}
\end{figure}


Rechts und links befinden sich asymptotische Zustände, also ebene
Wellen. Die Wechselwirkungsamplitude ist:
\begin{align*}
\leftidx{_{H}}{\BraKet{0|\mathcal{T}\left\{ \hat{\overline{\psi}}_{H}\left(x_{1}\right)\hat{\psi}_{H}\left(x_{2}\right)\hat{\psi}_{H}\left(x_{3}\right)\hat{\psi}_{H}\left(x_{4}\right)\hat{\overline{\psi}}_{H}\left(x_{5}\right)\hat{\overline{\psi}}_{H}\left(x_{6}\right)\hat{A}_{H,\mu}\left(x_{7}\right)\right\} |0}}{_{H}}
\end{align*}
Die Zeitordnung ist nötig, damit man die Propagatoren bekommt. Also
ist die Reihenfolge im zeitgeordneten Produkt willkürlich bis auf
ein totales Vorzeichen $\left(-1\right)^{n}$.

Wegen der Forderung, dass der Vakuumzustand zeitunabhängig sein soll,
ist dieser Erwartungswert nur im Heisenberg-Bild zu verstehen. Im
Wechselwirkungsbild (\foreignlanguage{english}{interaction picture})
mit $H=H_{0}+H_{I}$ erhält man:
\begin{align*}
 & \leftidx{_{H}}{\BraKet{0|\mathcal{T}\left\{ \hat{\overline{\psi}}_{H}\left(x_{1}\right)\hat{\psi}_{H}\left(x_{2}\right)\hat{\psi}_{H}\left(x_{3}\right)\hat{\psi}_{H}\left(x_{4}\right)\hat{\overline{\psi}}_{H}\left(x_{5}\right)\hat{\overline{\psi}}_{H}\left(x_{6}\right)\hat{A}_{H,\mu}\left(x_{7}\right)\right\} |0}}{_{H}}=\\
 & \qquad=\BraKet{0|\mathcal{T}\left\{ \hat{\overline{\psi}}_{I}\left(x_{1}\right)\hat{\psi}_{I}\left(x_{2}\right)\hat{\psi}_{I}\left(x_{3}\right)\hat{\psi}_{I}\left(x_{4}\right)\hat{\overline{\psi}}_{I}\left(x_{5}\right)\hat{\overline{\psi}}_{I}\left(x_{6}\right)\hat{A}_{H,\mu}\left(x_{7}\right)e^{-\ii\int_{-\infty}^{\infty}\hat{H}_{I}\left(x\right)\dd^{4}x}\right\} |0}
\end{align*}



\subsection{Schrödinger-, Heisenberg- und Wechselwirkungsbild in der Quantenmechanik}

Nur Matrixelemente sind physikalisch relevant:
\begin{align*}
O_{\phi'\phi} & =\BraKet{\phi'|\hat{O}|\phi}
\end{align*}
Diese hängen im Allgemeinen von der Zeit ab:
\begin{align*}
\frac{\dd}{\dd t}O_{\phi'\phi} & =\left(\frac{\dd}{\dd t}\BRA{\phi'}\right)\hat{O}\KET{\phi}+\BraKet{\phi'|\left(\frac{\dd}{\dd t}\hat{O}\right)|\phi}+\BRA{\phi'}\hat{O}\left(\frac{\dd}{\dd t}\KET{\phi}\right)
\end{align*}
Im Schrödingerbild sind die Operatoren nicht zeitabhängig:
\begin{align*}
\frac{\dd}{\dd t}\hat{O}_{S} & =0
\end{align*}
Die Zeitabhängigkeit der Zustände wird durch den Hamilton-Operator
beschrieben:
\begin{align*}
\ii\frac{\dd}{\dd t}\KET{\phi_{S}\left(t\right)} & =:\hat{H}\KET{\phi_{S}\left(t\right)} & -\ii\frac{\dd}{\dd t}\BRA{\phi_{S}'\left(t\right)} & =:\BRA{\phi_{S}'\left(t\right)}\hat{H}
\end{align*}
Es folgt:
\begin{align*}
\frac{\dd}{\dd t}O_{\phi'\phi} & =\ii\BraKet{\phi'|\hat{H}\hat{O}|\phi}-\ii\BraKet{\phi'|\hat{O}\hat{H}|\phi}=\ii\BraKet{\phi_{S}'|\left[\hat{H},\hat{O}\right]|\phi_{S}}
\end{align*}
Im Heisenbergbild sind die Zustände zeitunabhängig:
\begin{align*}
\frac{\dd}{\dd t}\KET{\phi} & =\frac{\dd}{\dd t}\BRA{\phi'}=0 & \frac{\dd}{\dd t}\hat{O}_{H} & =\ii\left[\hat{H},\hat{O}_{H}\right]
\end{align*}
Im Wechselwirkungsbild zerlegt man $\hat{H}=\hat{H}_{0}+\hat{H}_{I}$.
Dabei sei $\hat{H}_{0}$ nicht explizit zeitabhängig.
\begin{align*}
-\ii\frac{\dd}{\dd t}\BRA{\phi_{I}'} & =\BRA{\phi_{I}'}\hat{H}_{I} & -\ii\frac{\dd}{\dd t}\hat{O}_{I} & =\left[\hat{H}_{0},\hat{O}_{I}\right]\\
\ii\frac{\dd}{\dd t}\KET{\phi_{I}} & =-\hat{H}_{I}\KET{\phi_{I}}
\end{align*}
Mache den Ansatz:
\begin{align}
\fbox{\ensuremath{{\displaystyle \hat{O}_{I}\left(t\right)=e^{\ii\hat{H}_{0}\left(t-t_{0}\right)}\hat{O}_{S}e^{-\ii\hat{H}_{0}\left(t-t_{0}\right)}}}}
\end{align}
\begin{align*}
-\ii\frac{\dd}{\dd t}\hat{O}_{I}\left(t\right) & =e^{\ii\hat{H}_{0}\left(t-t_{0}\right)}\left(\hat{H}_{0}\hat{O}_{S}-\hat{O}_{S}\hat{H}_{0}\right)e^{-\ii\hat{H}_{0}\left(t-t_{0}\right)}=\left[\hat{H}_{0},\hat{O}_{I}\right]
\end{align*}
Für den Zusammenhang zum Heisenbergbild machen wir folgenden Ansatz:
\begin{align}
\fbox{\ensuremath{{\displaystyle \hat{O}_{H}\left(t\right)=\left(\mathcal{T}\exp\left(-\ii\int_{-\infty}^{t}\hat{H}_{I}\left(\tau\right)\dd\tau\right)\right)^{\dagger}\hat{O}_{I}\left(t\right)\mathcal{T}\exp\left(-\ii\int_{-\infty}^{t}\hat{H}_{I}\left(\tau'\right)\dd\tau'\right)}}}
\end{align}
\begin{align*}
-\ii\frac{\dd}{\dd t}\hat{O}_{H} & =-\ii\left(-\ii H_{I}\left(t\right)\mathcal{T}\exp\left(-\ii\int_{-\infty}^{t}\hat{H}_{I}\left(\tau\right)\dd\tau\right)\right)^{\dagger}\hat{O}_{I}\left(t\right)\mathcal{T}\exp\left(-\ii\int_{-\infty}^{t}\hat{H}_{I}\left(\tau'\right)\dd\tau'\right)+\\
 & \qquad+\left(\mathcal{T}\exp\left(-\ii\int_{-\infty}^{t}\hat{H}_{I}\left(\tau\right)\dd\tau\right)\right)^{\dagger}\left[\hat{H}_{0},\hat{O}_{I}\left(t\right)\right]\mathcal{T}\exp\left(-\ii\int_{-\infty}^{t}\hat{H}_{I}\left(\tau'\right)\dd\tau'\right)+\\
 & \qquad+\left(\mathcal{T}\exp\left(-\ii\int_{-\infty}^{t}\hat{H}_{I}\left(\tau\right)\dd\tau\right)\right)^{\dagger}\hat{O}_{I}\left(t\right)\left(-H_{I}\left(t\right)\right)\mathcal{T}\exp\left(-\ii\int_{-\infty}^{t}\hat{H}_{I}\left(\tau'\right)\dd\tau'\right)=\\
 & =\left(\mathcal{T}\exp\left(-\ii\int_{-\infty}^{t}\hat{H}_{I}\left(\tau\right)\dd\tau\right)\right)^{\dagger}\left[\hat{H},\hat{O}_{I}\left(t\right)\right]\mathcal{T}\exp\left(-\ii\int_{-\infty}^{t}\hat{H}_{I}\left(\tau'\right)\dd\tau'\right)
\end{align*}



\subsection{Beispiel: Quantenelektrodynamik}

Die Lagrangedichte ist:
\begin{align*}
\mathcal{L}\left(x\right) & =\underbrace{\overline{\psi}\left(x\right)\left(\hat{\slashd p}-m\right)\psi\left(x\right)-\frac{1}{4}F_{\mu\nu}\left(x\right)F^{\mu\nu}\left(x\right)}_{=:\mathcal{L}_{0}}\underbrace{-q_{e}\overline{\psi}\left(x\right)\hat{Q}\slashd A\left(x\right)\psi\left(x\right)}_{=:\mathcal{L}_{I}}
\end{align*}
\begin{align*}
L\left(t\right) & =\int\dd^{3}x\mathcal{L}\left(x\right)=\underbrace{L_{0}}_{\widehat{=}\, T}+\underbrace{L_{I}}_{\widehat{=}\,-V}\\
H\left(t\right) & =\int\dd^{3}x\mathcal{H}\left(x\right)=H_{0}+\underbrace{H_{I}}_{=-L_{I}}
\end{align*}
\begin{align*}
\mathcal{H}_{0}\left(x\right) & =\overline{\psi}\left(x\right)\left(\hat{\slashd p}-m\right)\psi\left(x\right)-\frac{1}{4}F_{\mu\nu}\left(x\right)F^{\mu\nu}\left(x\right)\\
\mathcal{H}_{I}\left(x\right) & =q_{e}\overline{\psi}\left(x\right)\hat{Q}\slashd A\left(x\right)\psi\left(x\right)
\end{align*}
\begin{align*}
\hat{H}_{I} & \to\int\dd^{3}x\hat{\mathcal{H}}_{I}\left(x\right)
\end{align*}



\subsection{Zusammenfassung der Exponentialfaktoren am Beispiel einer Skalaren
Feldtheorie}

Zu berechnen ist:
\begin{align*}
\leftidx{_{H}}{\BraKet{0|\mathcal{T}\left\{ \hat{\phi}_{H}\left(x_{1}\right)\ldots\hat{\phi}_{H}\left(x_{n}\right)\right\} |0}}{_{H}}
\end{align*}
Im Wechselwirkungsbild haben wir:
\begin{align*}
\hat{\phi}_{H}\left(x_{1}\right) & \to\left(\mathcal{T}\exp\left(-\ii\int_{-\infty}^{t}\dd^{3}x\dd\tau\mathcal{H}_{I}\left(x\right)\right)\right)^{\dagger}\hat{\phi}_{I}\left(x_{1}\right)\mathcal{T}\exp\left(-\ii\int_{-\infty}^{t}\dd^{3}x\dd\tau\mathcal{H}_{I}\left(x\right)\right)
\end{align*}
Wegen $\KET{\psi}_{H}=\KET{\psi}_{I,t_{-\infty}}$ folgt:
\begin{align*}
\KET{\phi}_{I} & =\mathcal{T}\exp\left(-\ii\int_{-\infty}^{t}\dd^{3}x\dd\tau\mathcal{H}_{I}\left(x\right)\right)\KET{\phi}_{H}
\end{align*}
Insbesondere gilt für das Vakuum:
\begin{align*}
\KET 0_{I} & =\mathcal{T}\exp\left(-\ii\int_{-\infty}^{t}\dd^{3}x\dd\tau\mathcal{H}_{I}\left(x\right)\right)\KET 0_{H}\\
\KET 0_{H} & =\left(\mathcal{T}\exp\left(-\ii\int_{-\infty}^{t}\dd^{3}x\dd\tau\mathcal{H}_{I}\left(x\right)\right)\right)^{\dagger}\KET 0_{I}\\
\Rightarrow\qquad\KET 0_{I,t_{\infty}} & =\mathcal{T}\exp\left(-\ii\int_{-\infty}^{\infty}\dd^{3}x\dd\tau\mathcal{H}_{I}\left(x\right)\right)\KET 0_{H}
\end{align*}
Damit folgt:
\begin{align*}
_{H}\BraKet{0|\mathcal{T}\left\{ \hat{\phi}_{H}\left(x_{1}\right)\ldots\hat{\phi}_{H}\left(x_{n}\right)\right\} |0}_{H} & =\leftidx{_{I,t_{\infty}}}{\bigg\langle0\bigg|\mathcal{T}\exp\left(-\ii\int_{-\infty}^{\infty}\dd\tau H_{I}\left(\tau\right)\right)\cdot}\\
 & \qquad\cdot\mathcal{T}\left\{ \left(\mathcal{T}\exp\left(-\ii\int_{-\infty}^{t_{1}}\dd\tau H_{I}\left(\tau\right)\right)\right)^{\dagger}\hat{\phi}_{I}\left(x_{1}\right)\ldots\right\} \cdot\\
 & \qquad\cdot\underbrace{\left(\mathcal{T}\exp\left(\int_{-\infty}^{-\infty}\dd\tau H_{I}\left(t\right)\right)\right)^{\dagger}}_{=\mathbbm{1}}\bigg|0\bigg\rangle_{I,t_{-\infty}}
\end{align*}
Ohne Beschränkung der Allgemeinheit sei $y_{i}^{0}\ge y_{i+1}^{0}$
für $i\in\mathbb{N}_{\ge1}$.
\begin{align*}
 & \mathcal{T}\exp\left(-\ii\int_{-\infty}^{\infty}\dd\tau H_{I}\left(\tau\right)\right)\left(\mathcal{T}\exp\left(-\ii\int_{-\infty}^{y_{1}^{0}}\dd\tau H_{I}\left(\tau\right)\right)\right)^{\dagger}\\
 & \qquad=\mathcal{T}\exp\left(-\ii\int_{y_{1}^{0}}^{\infty}\dd\tau H_{I}\left(\tau\right)\right)\underbrace{\mathcal{T}\exp\left(-\ii\int_{-\infty}^{y_{1}^{0}}\dd\tau H_{I}\left(\tau\right)\right)\left(\mathcal{T}\exp\left(-\ii\int_{-\infty}^{y_{1}^{0}}\dd\tau H_{I}\left(\tau\right)\right)\right)^{\dagger}}_{=\mathbbm{1}}=\\
 & \qquad=\mathcal{T}\exp\left(-\ii\int_{y_{1}^{0}}^{\infty}\dd\tau H_{I}\left(\tau\right)\right)
\end{align*}
Man erhält also:
\begin{align*}
 & \BraKet{0|\mathcal{T}\left\{ \hat{\phi}_{H}\left(x_{1}\right)\ldots\hat{\phi}_{H}\left(x_{n}\right)\right\} |0}={}_{H}\BraKet{0|\mathcal{T}\left\{ \hat{\phi}_{H}\left(y_{1}\right)\ldots\hat{\phi}_{H}\left(y_{n}\right)\right\} |0}_{H}=\\
 & \quad=\leftidx{_{I,t_{\infty}\negthickspace\negthickspace}}{\BraKet{0|\mathcal{T}\exp\left(\frac{1}{\ii}\int_{y_{1}^{0}}^{\infty}\dd\tau H_{I}\left(\tau\right)\right)\hat{\phi}_{I}\left(y_{1}\right)\mathcal{T}\exp\left(\frac{1}{\ii}\int_{y_{2}^{0}}^{y_{1}^{0}}\dd\tau H_{I}\left(\tau\right)\right)\hat{\phi}_{I}\left(y_{2}\right)\ldots|0}}{_{\negthickspace I,t_{-\infty}}}=\\
 & \quad=\BraKet{0|\mathcal{T}\left\{ \hat{\phi}_{I}\left(y_{1}\right)\hat{\phi}_{I}\left(y_{2}\right)\ldots\hat{\phi}_{I}\left(y_{n}\right)e^{-\ii\int_{-\infty}^{\infty}\dd\tau\hat{H}_{I}\left(\tau\right)}\right\} |0}=\\
 & \quad=\BraKet{0|\mathcal{T}\left\{ \hat{\phi}_{I}\left(x_{1}\right)\hat{\phi}_{I}\left(x_{2}\right)\ldots\hat{\phi}_{I}\left(x_{n}\right)e^{-\ii\int_{-\infty}^{\infty}\dd\tau\hat{H}_{I}\left(\tau\right)}\right\} |0}
\end{align*}
Dies ist eine zentrale Gleichung:
\begin{align}
\fbox{\ensuremath{{\displaystyle \BraKet{0|\mathcal{T}\left\{ \hat{\phi}_{H}\left(x_{1}\right)\ldots\hat{\phi}_{H}\left(x_{n}\right)\right\} |0}=\BraKet{0|\mathcal{T}\left\{ \hat{\phi}_{I}\left(x_{1}\right)\ldots\hat{\phi}_{I}\left(x_{n}\right)e^{-\ii\int_{-\infty}^{\infty}\dd\tau\hat{H}_{I}\left(\tau\right)}\right\} |0}}}}
\end{align}



\section{Das Wicksche Theorem}


\subsection{Definition \textmd{(Normalgeordnetes Produkt)}}

Das \emph{normalgeordnete Produkt} ist dadurch definiert, dass alle
Vernichtungsoperatoren rechts von allen Erzeugungsoperatoren stehen.
Insbesondere gilt:
\begin{align}
:\ldots:\KET 0 & =0 & \BRA 0:\ldots: & =0
\end{align}



\subsection{Das Wicksche Theorem}

\begin{align}
 & \mathcal{T}\left\{ \hat{\phi}\left(x_{1}\right)\ldots\hat{\phi}\left(x_{N}\right)\right\} =:\hat{\phi}\left(x_{1}\right)\ldots\hat{\phi}\left(x_{N}\right):+\BraKet{0|\mathcal{T}\left\{ \hat{\phi}\left(x_{1}\right)\hat{\phi}\left(x_{2}\right)\right\} |0}:\hat{\phi}\left(x_{3}\right)\ldots\hat{\phi}\left(x_{N}\right):+\nonumber \\
 & \qquad\qquad\qquad\qquad+\text{ Permutationen }+\nonumber \\
 & \qquad\qquad\qquad\qquad+\BraKet{0|\mathcal{T}\left\{ \hat{\phi}\left(x_{1}\right)\hat{\phi}\left(x_{2}\right)\right\} |0}\BraKet{0|\mathcal{T}\left\{ \hat{\phi}\left(x_{3}\right)\hat{\phi}\left(x_{4}\right)\right\} |0}:\hat{\phi}\left(x_{5}\right)\ldots\hat{\phi}\left(x_{N}\right):+\nonumber \\
 & \qquad\qquad\qquad\qquad+\text{ Permutationen }+\ldots
\end{align}


%DATE: Di 7.5.13


\subsubsection*{Beweis}

Führe eine vollständige Induktion über $N$ durch.
\begin{itemize}
\item Induktionsanfang bei $N=1$:
\begin{align*}
\mathcal{T}\left\{ \hat{\phi}\left(x_{1}\right)\right\}  & =\hat{\phi}\left(x_{1}\right)=:\hat{\phi}\left(x_{1}\right):
\end{align*}

\item Veranschaulichung der Beweisidee am Fall $N=2$: Für Bosonen ist das
zeitgeordnete Produkt mit Plus, damit sich für $x_{1}=x_{2}$ das
einfache Produkt ergibt:
\begin{align*}
\mathcal{T}\left\{ \hat{\phi}\left(x_{1}\right)\hat{\phi}\left(x_{2}\right)\right\}  & =\Theta\left(x_{1}^{0}-x_{2}^{0}\right)\hat{\phi}\left(x_{1}\right)\hat{\phi}\left(x_{2}\right)+\Theta\left(x_{2}^{0}-x_{1}^{0}\right)\hat{\phi}\left(x_{2}\right)\hat{\phi}\left(x_{1}\right)
\end{align*}
Hier sind die Feldoperatoren:
\begin{align*}
\hat{\phi}\left(\vec{x},t\right) & =\int\frac{\dd^{3}p}{\left(2\pi\right)^{3}2p^{0}}\left(\hat{a}\left(p\right)e^{-\ii px}+\hat{b}^{\dagger}\left(p\right)e^{\ii px}\right)\\
\hat{\phi}^{*}\left(\vec{x},t\right) & =\int\frac{\dd^{3}p}{\left(2\pi\right)^{3}2p^{0}}\left(\hat{a}^{\dagger}\left(p\right)e^{\ii px}+\hat{b}\left(p\right)e^{-\ii px}\right)
\end{align*}
Für eine skalares Feld gilt $\hat{a}=\hat{b}$, da es nur einen Freiheitsgrad
gibt, weil es ein reelles Feld ist, sodass das Teilchen sein eigenes
Antiteilchen ist. Weiter muss gelten:
\begin{align*}
\left[\hat{a}\left(p\right),\hat{a}^{\dagger}\left(p'\right)\right] & =\delta^{\left(3\right)}\left(\vec{p}-\vec{p}'\right)\left(2\pi\right)^{3}2p^{0}=\left[\hat{b}\left(p\right),\hat{b}^{\dagger}\left(p'\right)\right]
\end{align*}
\begin{align*}
\BraKet{0|\mathcal{T}\left\{ \phi\left(x\right)\phi^{*}\left(y\right)\right\} |0} & =\ii D_{F}\left(x-y\right)
\end{align*}
Andere Bezeichnungen für den Klein-Gordon-Propagator sind $G\left(x-y\right)$
oder $\Delta_{F}\left(x-y\right)$. Das normalgeordnete Produkt unterscheidet
sich in diesem Fall nur um eine Distribution $f\left(x\right)$, da
bei der Vertauschung der Erzeugungs- und Vernichtunsoperatoren der
Kommutator hinzukommt, der eine Distribution ist.
\begin{align*}
\mathcal{T}\left\{ \hat{\phi}\left(x_{1}\right)\hat{\phi}\left(x_{2}\right)\right\}  & =:\hat{\phi}\left(x_{1}\right)\hat{\phi}\left(x_{2}\right):+f\left(x_{1},x_{2}\right)\\
\Rightarrow\qquad\BraKet{0|\mathcal{T}\left\{ \hat{\phi}\left(x_{1}\right)\hat{\phi}\left(x_{2}\right)\right\} |0} & =0+f\left(x_{1},x_{2}\right)\underbrace{\BraKet{0|0}}_{=1}
\end{align*}

\item Induktionsschritt $N\leadsto N+1$: Ohne Beschränkung der Allgemeinheit
sei $x_{N+1}^{0}\le x_{i}^{0}$ für $i\in\left\{ 1,\ldots,N\right\} $.
Mit
\begin{align*}
\hat{\phi}^{v}\left(x_{N+1}\right) & :=\int\frac{\dd^{3}p}{\left(2\pi\right)^{3}2p^{0}}\hat{a}\left(p\right)e^{-\ii px}\\
\hat{\phi}^{e}\left(x_{N+1}\right) & :=\int\frac{\dd^{3}p}{\left(2\pi\right)^{3}2p^{0}}\hat{a}^{\dagger}\left(p\right)e^{\ii px}
\end{align*}
folgt:
\begin{align*}
 & \mathcal{T}\left\{ \hat{\phi}\left(x_{1}\right)\ldots\hat{\phi}\left(x_{N+1}\right)\right\} =\mathcal{T}\left\{ \hat{\phi}\left(x_{1}\right)\ldots\hat{\phi}\left(x_{N}\right)\right\} \underbrace{\hat{\phi}\left(x_{N+1}\right)}_{=\hat{\phi}^{v}\left(x_{N+1}\right)+\hat{\phi}^{e}\left(x_{N+1}\right)}=\\
 & \quad=\bigg(:\hat{\phi}\left(x_{1}\right)\ldots\hat{\phi}\left(x_{N}\right):+\BraKet{0|\mathcal{T}\left\{ \hat{\phi}\left(x_{1}\right)\hat{\phi}\left(x_{2}\right)\right\} |0}:\hat{\phi}\left(x_{3}\right)\ldots\hat{\phi}\left(x_{N}\right):+\\
 & \quad\qquad+\text{ Permutationen }+\\
 & \quad\qquad\BraKet{0|\mathcal{T}\left\{ \hat{\phi}\left(x_{1}\right)\hat{\phi}\left(x_{2}\right)\right\} |0}\BraKet{0|\mathcal{T}\left\{ \hat{\phi}\left(x_{3}\right)\hat{\phi}\left(x_{4}\right)\right\} |0}:\hat{\phi}\left(x_{5}\right)\ldots\hat{\phi}\left(x_{N}\right):+\\
 & \quad\qquad+\text{Permutationen}+\ldots\bigg)\cdot\left(\hat{\phi}^{v}\left(x_{N+1}\right)+\hat{\phi}^{e}\left(x_{N+1}\right)\right)
\end{align*}
Was wir brauchen ist:
\begin{align*}
:\hat{\phi}\left(x_{1}\right)\ldots\hat{\phi}\left(x_{M}\right):\hat{\phi}^{e}\left(x_{N+1}\right)
\end{align*}
Die Indexmenge $E$ sei eine Teilmenge von $\left\{ 1,2,\ldots,M\right\} $;
$V=\left\{ 1,2,\ldots,M\right\} \setminus E$.
\begin{align*}
:\hat{\phi}\left(x_{1}\right)\ldots\hat{\phi}\left(x_{M}\right):\hat{\phi}^{e}\left(x_{N+1}\right) & =\sum_{E,V}\left(\prod_{i\in E}\phi^{e}\left(x_{i}\right)\right)\left(\prod_{j\in V}\phi^{v}\left(x_{j}\right)\right)\hat{\phi}^{e}\left(x_{N+1}\right)
\end{align*}
Da der Kommutator $\left[\hat{\phi}^{v}\left(x_{k}\right),\hat{\phi}^{e}\left(x_{N+1}\right)\right]$
eine Distribution ist, erhält man:
\begin{align*}
\left[\hat{\phi}^{v}\left(x_{k}\right),\hat{\phi}^{e}\left(x_{N+1}\right)\right] & =\BraKet{0|\left[\hat{\phi}^{v}\left(x_{k}\right),\hat{\phi}^{e}\left(x_{N+1}\right)\right]|0}=\BraKet{0|\hat{\phi}^{v}\left(x_{k}\right)\hat{\phi}^{e}\left(x_{N+1}\right)|0}=\\
 & =\BraKet{0|\hat{\phi}\left(x_{k}\right)\hat{\phi}\left(x_{N+1}\right)|0}=\BraKet{0|\mathcal{T}\left\{ \hat{\phi}\left(x_{k}\right)\hat{\phi}\left(x_{N+1}\right)\right\} |0}
\end{align*}
Dadurch erhält man zu allen Summanden noch alle möglichen Kombinationen
\begin{align*}
\BraKet{0|\mathcal{T}\left\{ \hat{\phi}\left(x_{k}\right)\hat{\phi}\left(x_{N+1}\right)\right\} |0}
\end{align*}
und somit den Ausdruck auf der rechten Seite für $N+1$.\qqed
\end{itemize}
Wir wissen:
\begin{align*}
\BraKet{0|\mathcal{T}\left\{ \hat{\phi}_{H}\left(x_{1}\right)\ldots\hat{\phi}_{H}\left(x_{N}\right)\right\} |0} & =\BraKet{0|\mathcal{T}\left\{ \hat{\phi}_{I}\left(x_{1}\right)\ldots\hat{\phi}_{I}\left(x_{N}\right)e^{-\ii\int_{-\infty}^{\infty}H_{I}\left(\tau\right)\dd\tau}\right\} |0}
\end{align*}
Nun kann man die Exponentialfunktion zu einer beliebigen Ordnung nähern
und den Erwartungswert des zeitgeordneten Produktes können wir mit
dem Wickschen Theorem auf bekannten Vakuumerwartungswerte, die Propagatoren
freier Teilchen, reduzieren:
\begin{align*}
\BraKet{0|\mathcal{T}\left\{ \hat{\phi}_{I}\left(x_{1}\right)\hat{\phi}_{I}\left(x_{2}\right)\right\} |0} & =\ii D_{F}\left(x_{1}-x_{2}\right)
\end{align*}
Da der Vakuumerwartungswert eines normalgeordneten Produktes verschwindet,
bleibt beim Bilden des Vakuumerwartungswertes nur der Term, wo nur
noch Distributionen stehen.

Für eine ungerade Anzahl von $\hat{\phi}$ erhält man Null, da auf
der rechten Seite dann jeder Term proportional zu einem normalgeordneten
Produkt ist.


\subsection{Beispiel: Skalare Feldtheorie}

Betrachte die $\phi^{4}$-Theorie:

\begin{align}
\mathcal{H}_{I} & =\frac{\lambda}{4!}\phi_{I}^{4}\left(x\right)\\
H_{I}\left(\tau\right) & =\int\dd^{3}x\mathcal{H}_{I}\left(x\right)
\end{align}
Diese Theorie wird betrachtet, da ungerade Potenzen nicht möglich
sind, weil sonst die Hamilton-Funktion nicht nach unten beschränkt
ist und $\phi^{2}$ im kinetischen Anteil enthalten ist. Höhere Potenzen
können nicht auftreten, da diese nicht normierbar sind, wie wir später
sehen werden.

\begin{figure}[H]
\noindent \centering{}\begin{tikzpicture}[thick, level/.style={level distance=1.7cm}]
  \path (0,0) node[position]{} 
    child [grow=north west] {node[position]{} node[left]{$x_1$} edge from parent [scalar]}
    child [grow=south west] {node[position]{} node[left]{$x_1$} edge from parent [scalar]}
    child [grow=north east] {node[position]{} node[right]{$x_3$} edge from parent [scalar]}
    child [grow=south east] {node[position]{} node[right]{$x_4$} edge from parent [scalar]}
  node[above=3pt]{$\lambda$};
  \node at (2.5,0) {$=$};
  \path[level/.style={level distance=1cm}] (3.5,0.5) node[position]{} 
    child [grow=south] {node[position]{} edge from parent [scalar]};
  \path[level/.style={level distance=1cm}] (4.5,0.5) node[position]{} 
    child [grow=south] {node[position]{} edge from parent [scalar]};
  \node at (5.25,0) {$+$};
  \path[level/.style={level distance=1cm}] (6,0.5) node[position]{} 
    child [grow=east] {node[position]{} edge from parent [scalar]};
  \path[level/.style={level distance=1cm}] (6,-0.5) node[position]{} 
    child [grow=east] {node[position]{} edge from parent [scalar]};
  \node at (7.75,0) {$+$};
  \path[level/.style={level distance=1.41cm}] (8.5,0.5) node[position]{} 
    child [grow=south east] {node[position]{} edge from parent [scalar]};
  \path[level/.style={level distance=0.5cm}] (8.5,-0.5) node[position]{} 
    child [grow=north east] {
      child [grow=north east,edge from parent path={(\tikzparentnode.45) .. controls +(-0.3,0.3)  and +(-0.3,0.3) .. (\tikzchildnode.west)}] {
        child [grow=north east,edge from parent path=(\tikzparentnode) -- (\tikzchildnode)]{node[position]{} edge from parent [scalar]}
        edge from parent [scalar]}
      edge from parent [scalar]};
  \node at (10.5,0) {$+ \quad \ldots $};
  \node at (3,-2) {$+$};
  \path[level/.style={level distance=0.5cm}] (4,-2) node[position]{} 
    child [level distance=0, edge from parent path={(\tikzparentnode.30) .. controls +(0.6,0.5)  and  +(0.6,-0.5) .. (\tikzchildnode.-30)}] {
    child [level distance=0, edge from parent path={(\tikzparentnode.150) .. controls +(-0.6,0.5)  and +(-0.6,-0.5) .. (\tikzchildnode.-150)}] {}};
  \node at (7,-2) {Vakuumfluktuation};
  \node at (3,-4) {$+$};
  \path[level/.style={level distance=1cm}] (3.5,-3.5) node[position]{} 
    child [grow=south] {node[position]{} edge from parent [scalar]};
  \path[level/.style={level distance=0.5cm}] (4.5,-3.5) node[position]{} 
    child [grow=south] {node[position]{}
    child [level distance=0,edge from parent path={(\tikzparentnode) .. controls +(-0.4,0.7)  and +(-0.4,-0.7) .. (\tikzchildnode)}] {node[position]{}
      child [grow=south,level distance=0.5cm,edge from parent path={(\tikzparentnode) -- (\tikzchildnode)}] {node[position]{}} }
    edge from parent [scalar]};
  \node at (5.25,-4) {$+$};
  \path[level/.style={level distance=0.5cm}] (6,-3.5) node[position]{}    child [grow=east] {node[position]{}
    child [level distance=0,edge from parent path={(\tikzparentnode) .. controls +(-0.7,0.4)  and +(0.7,0.4) .. (\tikzchildnode)}] {node[position]{}
      child [grow=east,level distance=0.5cm,edge from parent path={(\tikzparentnode) -- (\tikzchildnode)}] {node[position]{}} }
    edge from parent [scalar]};
  \path[level/.style={level distance=1cm}] (6,-4.5) node[position]{} 
    child [grow=east] {node[position]{} edge from parent [scalar]};
  \node at (7.75,-4) {$+$};
  \path[level/.style={level distance=1cm}] (9.5,-3.5) node[position]{} 
    child [grow=south] {node[position]{} edge from parent [scalar]};
  \path[level/.style={level distance=0.5cm}] (8.5,-3.5) node[position]{} 
    child [grow=south] {node[position]{}
    child [level distance=0,edge from parent path={(\tikzparentnode) .. controls +(0.4,0.7)  and +(0.4,-0.7) .. (\tikzchildnode)}] {node[position]{}
      child [grow=south,level distance=0.5cm,edge from parent path={(\tikzparentnode) -- (\tikzchildnode)}] {node[position]{}} }
    edge from parent [scalar]};
  \node at (10.5,-4) {$+ \quad \ldots $};
\end{tikzpicture}\caption{Verschiedene Ordnungen dargestellt durch entsprechende Feynman-Diagramme}
\end{figure}


Die Amplitude erhält man durch Entwicklung der Exponentialfunktion:
\begin{align*}
 & \BraKet{0|\mathcal{T}\left\{ \hat{\phi}\left(x_{1}\right)\hat{\phi}\left(x_{2}\right)\hat{\phi}\left(x_{3}\right)\hat{\phi}\left(x_{4}\right)\left(1-\ii\int\dd^{4}x\frac{\lambda}{4!}\hat{\phi}^{4}\left(x\right)\right)\right\} |0}=\\
 & \qquad=\ii D_{F}\left(x_{1}-x_{2}\right)\ii D_{F}\left(x_{3}-x_{4}\right)+\ii D_{F}\left(x_{1}-x_{3}\right)\ii D_{F}\left(x_{2}-x_{4}\right)+\\
 & \qquad\quad+\ii D_{F}\left(x_{1}-x_{4}\right)\ii D_{F}\left(x_{2}-x_{3}\right)-\\
 & \qquad\quad-\ii\int\dd^{4}x\frac{\lambda}{4!}\left(\ii D_{F}\left(x_{1}-x_{2}\right)\ii D_{F}\left(x_{3}-x_{4}\right)+\ldots\right)\ii D_{F}\left(x-x\right)\ii D_{F}\left(x-x\right)\cdot3-\\
 & \qquad\quad-\ii\int\dd^{4}x\frac{\lambda}{4!}\ii D_{F}\left(x_{1}-x_{2}\right)\ii D_{F}\left(x_{3}-x\right)\ii D_{F}\left(x_{4}-x\right)\ii D_{F}\left(x-x\right)\cdot4\cdot3+\ldots-\\
 & \qquad\quad-\ii\int\dd^{4}x\frac{\lambda}{4!}\ii D_{F}\left(x_{1}-x\right)\ii D_{F}\left(x_{2}-x\right)\ii D_{F}\left(x_{3}-x\right)\ii D_{F}\left(x_{4}-x\right)\cdot4!+\ldots
\end{align*}
Beachte hier, dass $D_{F}\left(0\right)\approx\delta\left(0\right)$
ist, aber eine Produkt von Distributionen wie $\left(\delta\left(0\right)\right)^{2}$
nicht definiert ist. Der einzige Beitrag, bei dem Impuls ausgetauscht
wird, der also zur Streuung beiträgt ist, ist der letzte Term. Die
Streuamplitude ist also:
\begin{align*}
-\ii\lambda\int\dd^{4}xD_{F}\left(x_{1}-x\right)D_{F}\left(x_{2}-x\right)D_{F}\left(x_{3}-x\right)D_{F}\left(x_{4}-x\right)
\end{align*}



\subsection{Beispiel: Der Elektron-Photon-Vertex}

\begin{figure}[H]
\noindent \centering{}\hfill{}\subfloat[trägt nicht bei]{ \begin{tikzpicture}[level/.style={level distance=4cm}]
  \path (-1,0) node [position] {}
     child [grow=east] {node [position] {} node[right]{$x_1$} edge from parent[electron] node[above]{$e^-$}}     node[left]{$x_2$};
  \path (1,-3) node [position] {}
    child [grow=north,level distance=1.5cm] {node[position] {}
      child [grow=north,level distance=1cm, edge from parent path={(\tikzparentnode) .. controls +(0.5,0)  and +(0.5,0) .. (\tikzchildnode)}]{node[inner sep=0,fill=blue] {}
      child [grow=south,level distance=1cm, edge from parent path={(\tikzparentnode) .. controls +(-0.5,0)  and +(-0.5,0) .. (\tikzchildnode)}] {node[position]{}}
      edge from parent[electron]}
    node[below right]{$y$} edge from parent [photon]} node[right]{$x_3$};
\end{tikzpicture}}\hfill{}\subfloat[Trilevel-Diagramm]{\begin{tikzpicture}[level/.style={level distance=1.7cm}]
  \draw[dashed, ->] (-0.6,1) -- node[above]{Zeit} (4,1);
  \path (0,0) node [position] {}
    child [grow=east] {node [position] {}
      child [grow=south] {node [position] {} node[below]{$x_3$}
edge from parent[photon] node[right]{$A_\nu$}}
      child [grow=east] {node [position] {} node[right]{$x_1$}
edge from parent[electron] node[above]{$e^-$}}       node[above]{$y$}
edge from parent[electron] node[above]{$e^-$}}     node[left]{$x_2$};
\end{tikzpicture}}\hfill{}\hspace*{1mm}\caption{Feynman-Diagramme des Elektron-Photon-Vertex}
\end{figure}


Die Hamiltondichte der Wechselwirkung ist für ein Elektron mit Ladung
$q=-q_{e}$:
\begin{align*}
\mathcal{H}_{I}\left(x\right) & =-q_{e}\overline{\psi}\left(x\right)\slashd A\left(x\right)\psi\left(x\right)
\end{align*}
Die Amplitude ergibt sich somit zu:
\begin{align*}
I & :=\BraKet{0|\mathcal{T}\left\{ \hat{\psi}_{j_{1}}\left(x_{1}\right)\hat{\overline{\psi}}_{j_{2}}\left(x_{2}\right)\hat{A}_{\nu}\left(x_{3}\right)e^{-\ii\int\dd^{4}y\hat{\overline{\psi}}\left(y\right)\left(-q_{e}\right)\hat{\slashd A}\left(y\right)\hat{\psi}\left(y\right)}\right\} |0}\approx\\
 & \sr{\approx}{\text{1. Ordnung}}{\text{0. Ordnung}\leadsto0}\int\dd^{4}y\BraKet{0|\mathcal{T}\left\{ \hat{\psi}_{j_{1}}\left(x_{1}\right)\hat{\overline{\psi}}_{j_{2}}\left(x_{2}\right)\hat{A}_{\nu}\left(x_{3}\right)\hat{\overline{\psi}}_{k_{2}}\left(y\right)\left(\ii q_{e}\gamma^{\mu}\right)_{k_{2}k_{1}}\hat{\psi}_{k_{1}}\left(y\right)\hat{A}_{\mu}\left(y\right)\right\} |0}=\\
 & =\left(\ii q_{e}\gamma^{\mu}\right)_{k_{2}k_{1}}\int\dd^{4}y\BraKet{0|\mathcal{T}\left\{ \hat{A}_{\nu}\left(x_{3}\right)\hat{A}_{\mu}\left(y\right)\right\} |0}\cdot\\
 & \qquad\cdot\bigg(\underbrace{\BraKet{0|\mathcal{T}\left\{ \hat{\psi}_{j_{1}}\left(x_{1}\right)\hat{\overline{\psi}}_{j_{2}}\left(x_{2}\right)\right\} |0}\left(-1\right)\BraKet{0|\mathcal{T}\left\{ \hat{\psi}_{k_{1}}\left(y\right)\hat{\overline{\psi}}_{k_{2}}\left(y\right)\right\} |0}}_{\leadsto0}\cdot\\
 & \qquad\cdot\left(-1\right)\BraKet{0|\mathcal{T}\left\{ \hat{\psi}_{j_{1}}\left(x_{1}\right)\hat{\overline{\psi}}_{k_{2}}\left(y\right)\right\} |0}\left(-1\right)\BraKet{0|\mathcal{T}\left\{ \hat{\psi}_{k_{1}}\left(y\right)\hat{\overline{\psi}}_{j_{2}}\left(x_{2}\right)\right\} |0}=\\
 & =\left(\ii q_{e}\gamma^{\mu}\right)_{k_{2}k_{1}}\int\ii D_{F\mu\nu}\left(x_{3}-y\right)\ii\left(S_{F}\left(x_{1}-y\right)\right)_{j_{1}k_{2}}\ii\left(S_{F}\left(y-x_{2}\right)\right)_{k_{1}j_{2}}\dd^{4}y
\end{align*}
Für $I$ erhält man also im Allgemeinen eine Distribution, was auch
nicht weiter verwundert, da $I_{\nu}\left(x_{1},x_{2},x_{3}\right)$
die Wechselwirkung beschreibt, die zwischen Teilchen in den Raumzeitpunkten
$x_{i}$ besteht. Man nennt $I\left(x_{1},\ldots,x_{n}\right)$ eine
$n$-Punkt-Greens-Funktion. Real sind die Teilchen aber nicht lokalisiert
und somit muss man $I$ auf ein- und auslaufende Wellenfunktionen
$\psi_{e},\overline{\psi}_{e},\left(A_{\gamma}^{*}\right)^{\nu}$
anwenden und die $x_{i}$ über einen Bereich $U\subseteq\left(\mathbb{R}^{4}\right)^{3}$
integrieren, sodass nur
\begin{align*}
 & \int_{U}\overline{\psi}_{e}\left(x_{1}\right)\left(A_{\gamma}^{*}\right)^{\nu}\left(x_{3}\right)I_{\nu}\left(x_{1},x_{2},x_{3}\right)\psi_{e}\left(x_{2}\right)\dd^{4}x_{1}\dd^{4}x_{2}\dd^{4}x_{3}
\end{align*}
eine reale Bedeutung hat.

Das Trilevel-Diagramm liefert einen nicht verschwindenden Beitrag,
der auch nicht divergiert. Die Propagatoren sind:
\begin{align}
S_{F}\left(x_{1}-y\right) & =\int\frac{\dd^{4}p_{1}}{\left(2\pi\right)^{4}}e^{-\ii p_{1}\left(x_{1}-y\right)}\frac{\slashd p_{1}+m}{p_{1}^{2}-m^{2}+\ii\varepsilon}\\
S_{F}\left(y-x_{2}\right) & =\int\frac{\dd^{4}p_{2}}{\left(2\pi\right)^{4}}e^{-\ii p_{2}\left(y-x_{2}\right)}\frac{\slashd p_{2}+m}{p_{2}^{2}-m^{2}+\ii\varepsilon}\\
D_{F\mu\nu}\left(x_{3}-y\right) & =-\int\frac{\dd^{4}p_{3}}{\left(2\pi\right)^{4}}e^{-\ii p_{3}\left(x_{3}-y\right)}\frac{g_{\mu\nu}}{p_{3}^{2}+\ii\varepsilon}
\end{align}
Es folgt:
\begin{align*}
I & =\left(\ii q_{e}\gamma^{\mu}\right)_{k_{2}k_{1}}\int\frac{\dd^{4}p_{1}\dd^{4}p_{2}\dd^{4}p_{3}}{\left(2\pi\right)^{12}}\underbrace{\int\dd^{4}ye^{-\ii y\left(-p_{1}+p_{2}-p_{3}\right)}}_{=\left(2\pi\right)^{4}\delta^{\left(4\right)}\left(p_{2}-p_{1}-p_{3}\right);\text{E-p-Erhaltung}}e^{-\ii p_{1}x_{1}}e^{-\ii p_{3}x_{3}}e^{\ii p_{2}x_{2}}\cdot\\
 & \qquad\cdot\ii\frac{g_{\mu\nu}}{p_{3}^{2}+\ii\varepsilon}\ii\left(\frac{\slashd p_{1}+m}{p_{1}^{2}-m^{2}+\ii\varepsilon}\right)_{j_{1}k_{2}}\ii\left(\frac{\slashd p_{2}+m}{p_{2}^{2}-m^{2}+\ii\varepsilon}\right)_{k_{1}j_{2}}=\\
 & =\left(\ii q_{e}\gamma^{\mu}\right)_{k_{2}k_{1}}\int\frac{\dd^{4}p_{2}\dd^{4}p_{3}}{\left(2\pi\right)^{8}}e^{-\ii\left(p_{2}-p_{3}\right)x_{1}}e^{-\ii p_{3}x_{3}}e^{\ii p_{2}x_{2}}\cdot\\
 & \qquad\cdot\ii\frac{g_{\mu\nu}}{p_{3}^{2}+\ii\varepsilon}\ii\left(\frac{\slashd p_{2}-\slashd p_{3}+m}{\left(p_{2}-p_{3}\right)^{2}-m^{2}+\ii\varepsilon}\right)_{j_{1}k_{2}}\ii\left(\frac{\slashd p_{2}+m}{p_{2}^{2}-m^{2}+\ii\varepsilon}\right)_{k_{1}j_{2}}
\end{align*}
Der essentielle Teil ist also:
\begin{align*}
\ii\frac{g_{\mu\nu}}{p_{3}^{2}+\ii\varepsilon}\left(\ii\frac{\slashd p_{2}-\slashd p_{3}+m}{\left(p_{2}-p_{3}\right)^{2}-m^{2}+\ii\varepsilon}\left(\ii q_{e}\gamma^{\mu}\right)\ii\left(\frac{\slashd p_{2}+m}{p_{2}^{2}-m^{2}+\ii\varepsilon}\right)\right)_{j_{1}j_{2}}
\end{align*}
Im Impulsraum erhält man diesen direkt aus dem Feynman-Diagramm, wenn
man die Energie-Impuls-Erhaltung berücksichtigt:
\begin{align*}
\ii\frac{\slashd p_{2}-\slashd p_{3}+m}{\left(p_{2}-p_{3}\right)^{2}-m^{2}+\ii\varepsilon}\ \bigg|\ \left(\ii\left(-q_{e}\right)\gamma^{\mu}\right)\ \bigg|\ \ii\frac{\slashd p_{2}+m}{p_{2}^{2}-m^{2}+\ii\varepsilon}\ \bigg|\ \ii\frac{g_{\mu\nu}}{p_{3}^{2}+\ii\varepsilon}
\end{align*}


\begin{figure}[H]
\noindent \centering{}\begin{tikzpicture}[level/.style={level distance=1.7cm}]
  \path (0,0) node [position] {}
    child [grow=east] {node [position] {}
      child [grow=south] {node [position] {} node[below]{$x_3$}
edge from parent[photon] node[right]{$\,\downarrow$} node[left]{$q\,$}}
      child [grow=east] {node [position] {} node[right]{$x_1$}
edge from parent[electron] node[above]{$p_2 -p_3$} node[below]{$\to$}}       node[above]{$y$}
edge from parent[electron] node[above]{$p_2$} node[below]{$\to $}}     node[left]{$x_2$};
\end{tikzpicture}\caption{Energie-Impuls-Erhaltung}
\end{figure}


Den Feynman-Propagator kann man auf unterschiedlich schreiben, vergleiche
dazu (\ref{eq:Feynman-Propagator-psi}).
\begin{align*}
\ii S_{F}\left(y-x\right) & =\int\frac{\dd^{4}p}{\left(2\pi\right)^{4}}e^{-\ii p\left(y-x\right)}\ii\frac{\slashd p+m}{p^{2}-m^{2}+\ii\varepsilon}=\\
 & =\int\frac{\dd^{3}p}{\left(2\pi\right)^{3}}\Bigg(\sum_{r\in\left\{ 1,2\right\} }\tilde{\psi}_{r}\left(\vec{p},y\right)\overline{\tilde{\psi}}_{r}\left(\vec{p},x\right)\Theta\left(y^{0}-x^{0}\right)-\\
 & \qquad\qquad\qquad\qquad-\sum_{r\in\left\{ 3,4\right\} }\tilde{\psi}_{r}\left(\vec{p},y\right)\overline{\tilde{\psi}}_{r}\left(\vec{p},x\right)\Theta\left(x^{0}-y^{0}\right)\Bigg)=\\
 & =\int\frac{\dd^{3}p}{\left(2\pi\right)^{3}}\Bigg(e^{-\ii p\left(y-x\right)}\sum_{s\in\left\{ \pm1\right\} }\tilde{u}\left(\vec{p},s\right)\overline{\tilde{u}}\left(\vec{p},s\right)\Theta\left(y^{0}-x^{0}\right)-\\
 & \qquad\qquad\qquad\qquad-e^{\ii p\left(y-x\right)}\sum_{s\in\left\{ \pm1\right\} }\tilde{v}\left(\vec{p},s\right)\overline{\tilde{v}}\left(\vec{p},s\right)\Theta\left(x^{0}-y^{0}\right)\Bigg)
\end{align*}
Für Streuprozesse betrachtet man asymptotische ein- und auslaufende
Zustände. Für ein von $x$ einlaufendes Teilchen mit Wellenfunktion
$\psi\left(x\right)=\tilde{u}\left(\vec{p}',s'\right)e^{-\ii p'x}$
und $U_{x}=\left(t-\varepsilon,t+\varepsilon\right)\times\mathbb{R}^{3}$,
dass in $y$ wechselwirkt, gilt $x^{0}<y^{0}$, weshalb nur der vordere
Term mit $\Theta\left(y^{0}-x^{0}\right)$ beiträgt.
\begin{align*}
 & \int_{t-\varepsilon}^{t+\varepsilon}\dd x^{0}\int_{\mathbb{R}^{3}}\dd^{3}x\int\frac{\dd^{3}p}{\left(2\pi\right)^{3}}e^{-\ii p\left(y-x\right)}\sum_{s\in\left\{ \pm1\right\} }\tilde{u}\left(\vec{p},s\right)\overline{\tilde{u}}\left(\vec{p},s\right)\tilde{u}\left(\vec{p}',s'\right)e^{-\ii p'x}=\\
 & \quad=\int_{t-\varepsilon}^{t+\varepsilon}\dd x^{0}\int\frac{\dd^{3}p}{\left(2\pi\right)^{3}}e^{-\ii py}e^{-\ii\left(E_{p}'-E_{p}\right)x^{0}}\left(2\pi\right)^{3}\delta^{\left(3\right)}\left(\vec{p}'-\vec{p}\right)\sum_{s\in\left\{ \pm1\right\} }\tilde{u}\left(\vec{p},s\right)\overline{\tilde{u}}\left(\vec{p},s\right)\tilde{u}\left(\vec{p}',s'\right)=\\
 & \quad=\int_{t-\varepsilon}^{t+\varepsilon}\dd x^{0}e^{-\ii py}\underbrace{e^{-\ii\left(E_{p}'-E_{p}'\right)x^{0}}}_{=1}\underbrace{\sum_{s\in\left\{ \pm1\right\} }\tilde{u}\left(\vec{p}',s\right)\overline{\tilde{u}}\left(\vec{p}',s\right)\tilde{u}\left(\vec{p}',s'\right)}_{=\tilde{u}\left(\vec{p}',s'\right)}=\\
 & \quad=2\varepsilon e^{-\ii py}\tilde{u}\left(\vec{p}',s'\right)
\end{align*}
Der Faktor $e^{-\ii py}$ sorgt für die Impulserhaltung am Vertex.
Der Propagator $S_{F}\left(x-y\right)$ beschreibt neben einlaufenden
Teilchen gleichzeitig auch auslaufende Antiteilchen. Denn ist $x^{0}>y^{0}$,
so bleibt nur der Term proportional zu $ $$\Theta\left(x^{0}-y^{0}\right)$,
also:
\begin{align*}
\sum_{s\in\left\{ \pm1\right\} }\tilde{v}\left(\vec{p},s\right)\overline{\tilde{v}}\left(\vec{p},s\right)
\end{align*}
Für $\overline{\psi}\left(x\right)=\overline{\tilde{v}}\left(\vec{p}',s'\right)e^{\ii p'x}$
bleibt dann $-2\varepsilon\overline{\tilde{v}}\left(\vec{p}',s'\right)$.
Dreht man nun $x$ und $y$ um, so sieht man, dass $2\varepsilon\overline{\tilde{u}}\left(\vec{p},s\right)$
auslaufende Teilchen und $\tilde{v}\left(\vec{p},s\right)$ einlaufende
Antiteilchen beschreibt.

Solche Regeln im Impulsraum heißen \emph{Feynman-Regeln}.




\chapter{Feynman-Regeln}
\begin{itemize}
\item Einlaufendes Teilchen:
\begin{align}
u\left(\vec{p},s\right) & \cdot\left[\frac{1}{\sqrt{2E}}\right]
\end{align}
Den hinteren Faktor in eckigen Klammern schreibt man für gewöhnlich
nicht hin, da er im Wesentlichen nur die Energie-Impuls-Erhaltung
liefert.
\item Auslaufendes Teilchen:
\begin{align}
\overline{u}\left(\vec{p},s\right) & \cdot\left[\frac{1}{\sqrt{2E}}\right]
\end{align}

\item Einlaufendes Antiteilchen:
\begin{align}
\overline{v}\left(\vec{p},s\right) & \cdot\left[\frac{1}{\sqrt{2E}}\right]
\end{align}

\item Auslaufendes Antiteilchen:
\begin{align}
v\left(\vec{p},s\right) & \cdot\left[\frac{1}{\sqrt{2E}}\right]
\end{align}

\item Für ein einlaufendes Photon:
\begin{align}
\epsilon^{\mu} & \cdot\left[\frac{1}{\sqrt{2E}}\right]
\end{align}

\item Für ein auslaufendes Photon:
\begin{align}
\left(\epsilon^{\mu}\right)^{*} & \cdot\left[\frac{1}{\sqrt{2E}}\right]
\end{align}

\item Für jeden Fermion-Photon-Vertex (Fermionenladung $q_{f}$):
\begin{align}
-\ii q_{f}\gamma_{\mu} & \cdot\left[\left(2\pi\right)^{4}\delta^{\left(4\right)}\left(p-p'-q\right)\right]
\end{align}

\item Für jeden Fermion-Propagator (Linie):
\begin{align}
{\displaystyle \ii\frac{\slashd p+m}{p^{2}-m^{2}+\ii\varepsilon}} & \cdot\left[\frac{\dd^{4}p}{\left(2\pi\right)^{4}}\right]
\end{align}

\item Für jeden Photon-Propagator (Linie):
\begin{align}
{\displaystyle -\ii\frac{g_{\mu\nu}-\frac{1-c}{c}\frac{q_{\mu}q_{\nu}}{q^{2}}}{q^{2}+\ii\varepsilon}} & \cdot\left[\frac{\dd^{4}q}{\left(2\pi\right)^{4}}\right]
\end{align}
Dabei ist $c\in\mathbb{R}\setminus\left\{ 0\right\} $ eine beliebige
Konstante.
\item Aufgrund der Antikommutatoren erhalten Fermion-Loops einen Faktor
$\left(-1\right)$. Dies sei am Beispiel eines Loops mit zwei Vertizes
gezeigt:
\begin{align*}
 & \BraKet{0|\mathcal{T}\left\{ \hat{\overline{\psi}}_{j_{1}}\left(y\right)\left(e\hat{\slashd A}\right)_{j_{1}j_{3}}\left(\hat{\psi}_{j_{3}}\left(x\right)\hat{\overline{\psi}}_{j_{4}}\left(x'\right)\right)\left(e\hat{\slashd A}\right)_{j_{4}j_{5}}\hat{\psi}_{j_{5}}\left(y'\right)\right\} |0}\\
 & \qquad=\left(-1\right)^{3}\BraKet{0|\mathcal{T}\left\{ \left(e\slashd A\right)_{j_{1}j_{3}}\left(\hat{\psi}_{j_{3}}\left(x\right)\hat{\overline{\psi}}_{j_{4}}\left(x'\right)\right)\left(e\slashd A\right)_{j_{4}j_{5}}\left(\psi_{j_{5}}\left(y'\right)\overline{\psi}_{j_{1}}\left(y\right)\right)\right\} |0}=\\
 & \qquad=-\left(e\slashd A\right)_{j_{1}j_{3}}\left(S_{F}\left(x-x'\right)\right)_{j_{3}j_{4}}\left(e\slashd A\right)_{j_{4}j_{5}}\left(S_{F}\left(y-y'\right)\right)_{j_{5}j_{1}}=\\
 & \qquad=-\text{tr}\left(e\slashd AS_{F}\left(x-x'\right)e\slashd AS_{F}\left(y-y'\right)\right)
\end{align*}
Außerdem sieht man, dass bei einem Loops nur Spuren auftreten. Daher
spielt die Reihenfolge der Loops bei komplizierten Feynman-Diagrammen
keine Rolle.
\end{itemize}
\begin{figure}[H]
\noindent \centering{}\begin{tikzpicture}[thick, level/.style={level distance=1.7cm},scale=0.8]
  \path (0,0) node [position] {}
    child [grow=north,level distance=1.5cm] {node[position] {}
      child [grow=north east,level distance=1cm, edge from parent path={(\tikzparentnode) .. controls +(0.3,0)  and +(0,-0.3) .. (\tikzchildnode)}]{ node[position](rechts){}
        child [grow=north west,level distance=1cm, edge from parent path={(\tikzparentnode) .. controls +(0,0.3)  and +(0.3,0) .. (\tikzchildnode)}]{node[inner sep=0,fill=blue] (oben) {}
          child [grow=south west,level distance=1cm, edge from parent path={(\tikzparentnode) .. controls +(-0.3,0)  and +(0,0.3) .. (\tikzchildnode)}] {node[position](links){} 
            child [grow=south east,level distance=1cm, edge from parent path={(\tikzparentnode) .. controls +(0,-0.3)  and +(-0.3,0) .. (\tikzchildnode)}] {node[position]{} 
            edge from parent[electron]}
          edge from parent[electron]}
        edge from parent[electron]}
      edge from parent[electron]}
    edge from parent [photon]};
  \path (links) node[position]{} child [grow=south west]{node[position]{} edge from parent[photon]};
  \path (rechts) node[position]{} child [grow=south east]{node[position]{} edge from parent[photon]};
  \path (oben) node[position]{}
    child [grow=north]{node[position] (oben-2) {}
      child [grow=150,level distance=10mm,edge from parent path={(\tikzparentnode) .. controls +(-0.3,0)  and +(0.2,-0.3) .. (\tikzchildnode)}] {node[position] (phl) {}
        child [grow=80,level distance=10mm,edge from parent path={(\tikzparentnode) .. controls +(-0.2,0.3)  and +(-0.2,-0.2) .. (\tikzchildnode)}] {node[position] {}
          child [grow=120,level distance=10mm,edge from parent path={(\tikzparentnode) -- (\tikzchildnode)}] {node[position]{} edge from parent[photon]}
          child [grow=0,level distance=10mm,edge from parent path={(\tikzparentnode) .. controls +(0.2,0.3)  and +(-0.2,0.2) .. (\tikzchildnode)}] {node[position] (pho) {}
            child [grow=-60,level distance=7mm,edge from parent path={(\tikzparentnode) .. controls +(0.2,-0.2)  and +(-0.1,0.4) .. (\tikzchildnode)}] {node[position] (phu) {}
              child [grow=-100,level distance=7mm,edge from parent path={(\tikzparentnode) .. controls +(0.1,-0.4)  and +(0.2,0.3) .. (\tikzchildnode)}] {node[position] (phr) {}
              edge from parent[electron]}
            edge from parent[electron]}
          edge from parent[electron]}
        edge from parent[electron]}
      edge from parent[electron]}
    edge from parent[photon]};
  \draw[electron] (phr) .. controls +(-0.2mm,-0.3mm) and +(0.2,0) .. (oben-2);
  \draw[photon] (phl) -- (phr);
  \path (pho) node[position] {}
    child [grow=east,level distance=2cm]{node[position] (a) {}
      child [grow=south,level distance=8mm,edge from parent path={(\tikzparentnode) .. controls +(-0.2,-0.2)  and +(-0.2,0.2) .. (\tikzchildnode)}]{node[position] (phur) {}
        child [grow=east,level distance=10mm,edge from parent path={(\tikzparentnode) .. controls +(0.2,-0.3)  and +(-0.2,-0.4) .. (\tikzchildnode)}]{node[position] (b) {}
          child [grow=east,level distance=12mm,edge from parent path={(\tikzparentnode) -- (\tikzchildnode)}]{node[position] {}
          edge from parent[photon]}
        edge from parent[positron]}
      edge from parent[positron]}
    edge from parent[photon]};
  \draw[photon] (phu) -- (phur);
  \draw[electron] (a) .. controls +(0.4,0.4) and +(0.2,0.8) .. (b);
\end{tikzpicture}\caption{Kompliziertes Feynman-Diagramm}
\end{figure}


%DATE: Mi 8.5.13


\section{Die Quantisierung des Photonenfeldes}

Wie viele Freiheitsgrade hat $A_{\mu}\left(x\right)$, das heißt wie
viele Erzeugungs- und Vernichtungsoperatoren muss man einführen?

Die Plancksche Strahlungsformel für die spektrale Energiedichte lautet:
\begin{align}
u\left(\omega\right) & =\frac{N}{2}\frac{\hbar}{c^{3}\pi^{2}}\frac{\omega^{3}}{e^{\frac{\hbar\omega}{k_{B}T}}-1}
\end{align}
Dabei ist $N$ die Anzahl der Freiheitsgrade. Aus Messungen erhält
man $N=2$. Ebenso weiß man aus der Optik, dass es zwei Polarisationsfreiheitsgrade
gibt.

\begin{figure}[H]
\noindent \centering{}\begin{tikzpicture}[scale=1.5]
  \draw[photon] (-1,0) -- (1,0);
  \draw[->] (1,0) -- (1.5,0) node[right]{$\vec{q}$};
  \draw[->] (1,0) -- (1,1) node[above]{$E_1$};
  \draw[->] (1,0) -- (0.7,-0.7) node[left]{$E_2$}; 
\end{tikzpicture}\caption{Das Photon hat zwei Polarisationen senkrecht zur Ausbreitungsrichtung.}
\end{figure}


Naiv könnte man meinen, dass nur die transversalen Freiheitsgrade
quantisiert werden sollten. Dafür benötigt man ein jedoch begleitendes
Dreibein, was sehr unpraktisch ist.

Wir quantisieren alle vier Freiheitsgrade und zeigen, dass die Beiträge
der beiden unphysikalischen Freiheitsgrade sich gegenseitig aufheben.
So können wir alles explizit Lorentz-invariant formulieren.
\begin{description}
\item [{Nebenbemerkung:}] Die Wirkung des elektromagnetischen Feldes ist:
\begin{align*}
S & =\int\dd^{4}x\bigg(\underbrace{\frac{1}{2}\left(\vec{E}^{2}+\vec{B}^{2}\right)+\left(\partial^{0}\vec{A}\right)\cdot\vec{E}}_{\widehat{=}\,\sum_{i}\dot{q}_{i}p_{i}}-A^{0}\left(x\right)\left(\vec{\nabla}\cdot\vec{E}\right)\bigg)
\end{align*}
Hier entspricht $\vec{A}\,\widehat{=}\,\vec{q}$ und $\vec{E}\,\widehat{=}\,\vec{p}$.
Der kanonisch konjugierte Impuls
\begin{align*}
\frac{\partial\mathcal{L}}{\partial\left(\partial_{\lambda}A^{0}\right)} & =0
\end{align*}
von $A^{0}$ verschwindet, weshalb $A^{0}$ keine Dynamik hat. $A^{0}$
kann daher als Lagrangeparameter zur Zwangsbedingung $\vec{\nabla}\cdot\vec{E}=0$
aufgefasst werden.
\end{description}
Wir zeigen zunächst, dass die unphysikalischen Freiheitsgrade proportional
zum Impuls sind. Aus der Lagrangedichte
\begin{align}
\mathcal{L}\left(x\right) & =-\frac{1}{4}F_{\mu\nu}\left(x\right)F^{\mu\nu}\left(x\right)
\end{align}
folgen Euler-Lagrange-Gleichungen:
\begin{align}
0 & =\partial_{\mu}F^{\mu\nu}\left(x\right)=\partial_{\mu}\partial^{\mu}A^{\nu}\left(x\right)-\partial_{\mu}\partial^{\nu}A^{\mu}\left(x\right)
\end{align}
Normalerweise wählt man hier die Lorentz-Eichung $\partial_{\mu}A^{\mu}=0$
und erhält $\square A^{\nu}\left(x\right)=0$. Im Impulsraum heißt
dies:
\begin{align*}
q^{0}A^{0}-\vec{q}\vec{A} & =0\\
\Rightarrow\qquad A^{0}-\frac{\vec{q}}{q^{0}}\vec{A} & =0
\end{align*}
Wir gehen nun per Fourier-Transformation ohne Eichung in den Impulsraum:
\begin{align}
A^{\mu}\left(x\right) & =\int\frac{\dd^{4}q}{\left(2\pi\right)^{4}}e^{-\ii qx}A^{\mu}\left(q\right)
\end{align}
\begin{align}
q_{\mu}q^{\mu}A^{\nu}\left(q\right)-q_{\mu}q^{\nu}A^{\mu}\left(q\right) & =0\label{eq:Maxwell-Impulsraum}
\end{align}
Ohne Beschränkung der Allgemeinheit wählen wir die $z$-Achse als
Impulsrichtung, das heißt $q^{1}=q^{2}=0$.
\begin{enumerate}
\item Fall: $q_{\mu}q^{\mu}=0=\left(q^{0}\right)^{2}-\left(q^{3}\right)^{2}$
\begin{align*}
q_{\mu}q^{\nu}A^{\mu}\left(q\right) & =0
\end{align*}
Für $\nu=0$ kann man $q^{0}\not=0$ heraus dividieren und erhält:
\begin{align*}
0 & =\left(\begin{array}{c}
q^{0}\\
0\\
0\\
q^{3}
\end{array}\right)\left(\begin{array}{c}
A^{0}\\
-A^{1}\\
-A^{2}\\
-A^{3}
\end{array}\right)=q^{0}A^{0}-q^{3}A^{3}
\end{align*}
Aus $q^{0}=\pm q^{3}$ folgt $A^{0}=\pm A^{3}$.
\begin{align*}
\left(\begin{array}{c}
A^{0}\left(q\right)\\
0\\
0\\
A^{3}\left(q\right)
\end{array}\right) & \sim\left(\begin{array}{c}
q^{0}\\
0\\
0\\
q^{3}
\end{array}\right)\sim\left(\begin{array}{c}
1\\
0\\
0\\
\pm1
\end{array}\right)
\end{align*}
Der unphysikalische Anteil von $A^{\mu}$ ist also proportional zu
$q^{\mu}$.
\item Fall: $q_{\mu}q^{\mu}\not=0$
\begin{align*}
A^{\nu}\left(q\right) & =q^{\nu}\underbrace{\frac{q^{\mu}A_{\mu}}{q_{\sigma}q^{\sigma}}}_{=:a\left(q\right)}=a\left(q\right)\cdot q^{\nu}
\end{align*}

\end{enumerate}
Daher sind alle unphysikalischen Terme proportional zu $q^{\nu}$.
\begin{description}
\item [{Behauptung:}] Die Eichinvarianz garantiert, dass alle Terme proportional
zu $q^{\nu}$ keinen physikalischen Beitrag liefern.
\item [{Beweis:}] Betrachte den allgemeinen Fall eines Loops:


\begin{figure}[H]
\noindent \centering{}\begin{tikzpicture}
  \path (0,0) node[position] (unten){}
      child [grow=-90,level distance=10mm,edge from parent path={(\tikzparentnode) -- (\tikzchildnode)}] {node[position]{} edge from parent[electron]}
      child [grow=150,level distance=10mm,edge from parent path={(\tikzparentnode) .. controls +(-0.3,0)  and +(0.2,-0.3) .. (\tikzchildnode)}] {node[position] {}
        child [grow=220,level distance=10mm,edge from parent path={(\tikzparentnode) -- (\tikzchildnode)}] {node[position]{} edge from parent[boson]}
        child [grow=80,level distance=10mm,edge from parent path={(\tikzparentnode) .. controls +(-0.2,0.3)  and +(-0.2,-0.2) .. (\tikzchildnode)}] {node[position] {}
          child [grow=120,level distance=10mm,edge from parent path={(\tikzparentnode) -- (\tikzchildnode)}] {node[position]{} node[above]{$A^{\mu_1}$} edge from parent[photon]}
          child [grow=0,level distance=10mm,edge from parent path={(\tikzparentnode) .. controls +(0.2,0.3)  and +(-0.2,0.2) .. (\tikzchildnode)}] {node[position]  {}
            child [grow=80,level distance=10mm,edge from parent path={(\tikzparentnode) -- (\tikzchildnode)}] {node[position]{}  node[above]{$A^{\mu_{n-1}}$} edge from parent[photon]}
            child [grow=-60,level distance=7mm,edge from parent path={(\tikzparentnode) .. controls +(0.2,-0.2)  and +(-0.1,0.4) .. (\tikzchildnode)}] {node[position] {}
              child [grow=40,level distance=10mm,edge from parent path={(\tikzparentnode) -- (\tikzchildnode)}] {node[position]{} node[above]{$A^{\mu_n}$} edge from parent[photon]}
              child [grow=-100,level distance=7mm,edge from parent path={(\tikzparentnode) .. controls +(0.1,-0.4)  and +(0.2,0.3) .. (\tikzchildnode)}] {node[position] (rrr) {}
                child [grow=-60,level distance=10mm,edge from parent path={(\tikzparentnode) -- (\tikzchildnode)}] {node[position]{} edge from parent[positron]}
              edge from parent[electron]}
            edge from parent[electron]}
          edge from parent[electron]}
        edge from parent[electron]}
      edge from parent[electron]};
  \draw[electron] (rrr) .. controls +(-0.2mm,-0.3mm) and +(0.2,0) .. (unten);
  \node at (-0.2,2.2) {$\ldots$};
  \node at (-0.7,-0.4) {$\ldots$};
  \node at (1.7,0.2) {$W^{\mu_1 \mu_2\ldots \mu_n}$};
\end{tikzpicture}\caption{Allgemeiner Loop}
\end{figure}
Man erhält immer Terme folgender Art:
\begin{align*}
 & \sim\int\dd^{4}x_{1}\ldots\dd^{4}x_{n}A_{\mu_{1}}\left(x_{1}\right)A_{\mu_{2}}\left(x_{2}\right)\ldots A_{\mu_{n}}\left(x_{n}\right)W^{\mu_{1}\mu_{2}\ldots\mu_{n}}\left(x_{1},x_{2},\ldots\right)=\\
 & \sr ={\text{Eichinvarianz}}{}\int\dd^{4}x_{1}\ldots\dd^{4}x_{n}\left(A_{\mu_{1}}\left(x_{1}\right)-\frac{\partial}{\partial x_{1}^{\mu_{1}}}\Lambda\left(x_{1}\right)\right)\cdot\ldots\cdot\\
 & \qquad\cdot\left(A_{\mu_{n}}\left(x_{n}\right)-\frac{\partial}{\partial x_{n}^{\mu_{n}}}\Lambda\left(x_{n}\right)\right)W^{\mu_{1}\mu_{2}\ldots\mu_{n}}\left(x_{1},x_{2},\ldots,x_{n}\right)
\end{align*}
Die Terme proportional zu ${\displaystyle \frac{\partial}{\partial x_{i}^{\mu_{i}}}\Lambda\left(x_{i}\right)}$
müssen Null liefern. Nun führen wir eine partielle Integration nach
$x_{i}^{\mu_{i}}$ durch erhalten:
\begin{align*}
0 & =\frac{\partial}{\partial x_{1}^{\mu_{1}}}W^{\mu_{1}\ldots\mu_{n}}\left(x_{1},\ldots,x_{n}\right)=\frac{\partial}{\partial x_{2}^{\mu_{2}}}W^{\mu_{1}\ldots\mu_{n}}\left(x_{1},\ldots,x_{n}\right)=\ldots
\end{align*}
Eine $n$-fache Fourier-Transformation liefert:
\begin{align*}
0 & =\left(q_{1}\right)_{\mu_{1}}W^{\mu_{1}\ldots\mu_{n}}\left(q_{1},\ldots,q_{n}\right)=\left(q_{2}\right)_{\mu_{2}}W^{\mu_{1}\ldots\mu_{n}}\left(q_{1},\ldots,q_{n}\right)=\ldots
\end{align*}
Im Spezialfall $n=1$ erhält man die Kontinuitätsgleichung:
\begin{align*}
\partial_{\mu}j^{\mu}\left(x\right) & =0 & q_{\mu}j^{\mu}\left(q\right) & =0
\end{align*}
\begin{figure}[H]
\noindent \centering{} \begin{tikzpicture}[level/.style={level distance=15mm}, scale=1.5]
  \path (0,0) node[position] {}
    child [grow=60,edge from parent path={(\tikzparentnode) .. controls +(0.2,0.2)  and +(-0.3,-0.8) .. (\tikzchildnode)}] {node[inner sep=0pt]{} edge from parent[scalar]}
    child [grow=-140,edge from parent path={(\tikzparentnode) .. controls +(-0.2,-0.2)  and +(0.2,0.1) .. (\tikzchildnode)}] {node[inner sep=0pt]{} edge from parent[scalar]}
    child [grow=-50,level distance=15mm,edge from parent path={(\tikzparentnode) -- (\tikzchildnode)}] {node[position] {}
      child [grow=south east,level distance=1cm, edge from parent path={(\tikzparentnode) .. controls +(0.5,0.5)  and +(0.5,0.5) .. (\tikzchildnode)}]{node[inner sep=0,fill=blue] {}
        child [grow=north west,level distance=1cm, edge from parent path={(\tikzparentnode) .. controls +(-0.5,-0.5)  and +(-0.5,-0.5) .. (\tikzchildnode)}] {node[position]{}}
      edge from parent[electron]}
    node[above=5pt] {$\mu_1$} edge from parent[photon] node[below left]{$q_1$}}  node [above left]{$\nu $};
  \node at (1.3,-1.5) {$W^{\mu_1}(q)$};
\end{tikzpicture}\caption{Loop für den Spezialfall $n=1$}
\end{figure}
Terme im Propagator, die proportional zu $\left(q_{1}\right)_{\mu_{1}}$
sind, tragen wegen
\begin{align*}
\left(q_{1}\right)_{\mu_{1}}W^{\mu_{1}\ldots\mu_{n}}\left(q_{1},\ldots,q_{n}\right) & =0
\end{align*}
nicht bei. In 
\begin{align*}
\BraKet{0|\mathcal{T}\left\{ \hat{A}_{\mu}\left(x\right)\hat{A}_{\nu}\left(y\right)\right\} |0}
\end{align*}
tragen Terme, die proportional zu $q^{\mu}$ oder $q^{\nu}$ sind,
nicht bei. Also ist in
\begin{align}
\left(A^{\mu}\left(q\right)\right) & =\left(\begin{array}{c}
0\\
A^{1}\left(q\right)\\
A^{2}\left(q\right)\\
0
\end{array}\right)+a\left(q\right)q^{\mu}
\end{align}
die Wahl von $a\left(q\right)$ beliebig.\qqed[Behauptung]

\end{description}
Wir können also alle Polarisationsfreiheitsgrade $\epsilon^{\mu}\left(\vec{q},\lambda\right)$
für die kanonische Quantisierung verwenden:
\begin{align}
\hat{A}^{\mu} & =\int\frac{\dd^{3}q}{\left(2\pi\right)^{3}2q_{0}}\sum_{\lambda=0}^{3}\left(\epsilon^{\mu}\left(\vec{q},\lambda\right)e^{-\ii qx}\hat{a}\left(q,\lambda\right)+\left(\epsilon^{*}\right)^{\mu}\left(\vec{q},\lambda\right)e^{\ii qx}\hat{a}^{\dagger}\left(q,\lambda\right)\right)
\end{align}
\begin{align}
\left[\hat{a}\left(\vec{q},\lambda\right),\hat{a}^{\dagger}\left(\vec{q}',\lambda'\right)\right] & =-g^{\lambda\lambda'}2q^{0}\left(2\pi\right)^{3}\delta^{\left(3\right)}\left(\vec{q}-\vec{q}'\right)
\end{align}
Für einlaufende Photonen steht $\epsilon^{\mu}\left(\vec{q},\lambda\right)$
und für auslaufende Photonen $\left(\epsilon^{*}\right)^{\mu}\left(\vec{q},\lambda\right)$.
Dabei sind:
\begin{align}
\epsilon^{\mu}\left(\vec{q},1\right) & =\left(\begin{array}{c}
0\\
1\\
0\\
0
\end{array}\right) & \epsilon^{\mu}\left(\vec{q},2\right) & =\left(\begin{array}{c}
0\\
0\\
1\\
0
\end{array}\right) &  & \text{physikalisch}\\
\epsilon^{\mu}\left(\vec{q},0\right) & =\left(\begin{array}{c}
1\\
0\\
0\\
0
\end{array}\right) & \epsilon^{\mu}\left(\vec{q},3\right) & =\left(\begin{array}{c}
0\\
0\\
0\\
1
\end{array}\right) &  & \text{unphysikalisch}
\end{align}
Nun gilt:
\begin{align}
\sum_{\lambda}\epsilon^{\mu}\left(\vec{q},\lambda\right)\left(\epsilon^{*}\right)^{\nu}\left(\vec{q},\lambda\right) & =g^{\mu\nu}
\end{align}
Damit folgt:
\begin{align}
\sum_{\lambda}\epsilon^{\mu}\left(\vec{q},\lambda\right)\cdot\epsilon_{\mu}^{*}\left(\vec{q},\lambda\right) & =\sum_{\lambda}\epsilon^{\mu}\left(\vec{q},\lambda\right)\cdot\left(\epsilon^{*}\right)^{\nu}\left(\vec{q},\lambda\right)g_{\mu\nu}=\nonumber \\
 & =g^{\mu\nu}g_{\mu\nu}=\underbrace{+1-1}_{=0}-1-1=-2
\end{align}
Auch hier kann man alle Polarisationsfreiheitsgrade mitnehmen, da
die unphysikalischen Anteile aufheben. Damit wird das Rechnen viel
einfacher!


\section{Der Photon-Propagator}

Die inhomogenen Maxwell-Gleichung (\ref{eq:Maxwell-inhomogen}) ist:
\begin{align}
\partial_{\mu}F^{\mu\nu}\left(x\right)=\partial_{\mu}\left(\partial^{\mu}A^{\nu}-\partial^{\nu}A^{\mu}\right)\left(x\right) & =q_{e}j^{\nu}\left(x\right)=q_{e}\sum_{\text{Fermionen }j}Q_{j}\overline{\psi}_{j}\left(x\right)\gamma^{\nu}\psi_{j}\left(x\right)\nonumber \\
\left(\partial_{\mu}\partial^{\mu}g^{\nu}\msd{\lambda}-\partial_{\lambda}\partial^{\nu}\right)A^{\lambda}\left(x\right) & =q_{e}j^{\nu}\left(x\right)
\end{align}
Die Greensche Funktion $D_{F}\left(x-y\right)$ ist definiert durch:
\begin{align}
\left(\partial_{\mu}\partial^{\mu}g^{\nu}\msd{\lambda}-\partial_{\lambda}\partial^{\nu}\right)\left(D_{F}\left(x-y\right)\right)^{\lambda}\msd{\sigma} & =g^{\nu}\msd{\sigma}\delta^{\left(4\right)}\left(x-y\right)
\end{align}
Durch Fourier-Transformation mit
\begin{align*}
A^{\lambda}\left(x\right) & =\int\frac{\dd^{4}q}{\left(2\pi\right)^{4}}e^{-\ii qx}A^{\lambda}\left(q\right)\\
j^{\nu}\left(x\right) & =\int\frac{\dd^{4}q}{\left(2\pi\right)^{4}}e^{-\ii qx}j^{\nu}\left(q\right)\\
\left(D_{F}\left(x-y\right)\right)^{\lambda}\msd{\sigma} & =\int\frac{\dd^{4}q}{\left(2\pi\right)^{4}}e^{-\ii q\left(x-y\right)}\left(D_{F}\left(q\right)\right)^{\lambda}\msd{\sigma}
\end{align*}
ergibt sich analog zu(\ref{eq:Maxwell-Impulsraum}):
\begin{align}
\left(q_{\mu}q^{\mu}g^{\nu}\msd{\lambda}-q_{\lambda}q^{\nu}\right)A^{\lambda}\left(q\right) & =q_{e}j^{\nu}\left(q\right)\\
\Rightarrow\qquad\left(q_{\mu}q^{\mu}g^{\nu}\msd{\lambda}-q_{\lambda}q^{\nu}\right)\left(D_{F}\left(q\right)\right)^{\lambda}\msd{\sigma} & =-g^{\nu}\msd{\sigma}+b\frac{q^{\nu}q_{\sigma}}{q_{\mu}q^{\mu}}
\end{align}
Der Term mit $b$ kann hinzugefügt werden, da Terme proportional zu
$q_{\sigma}$ nichts beitragen:
\begin{align}
A^{\lambda}\left(x\right) & =q_{e}\int\dd^{4}y\left(D_{F}\left(x-y\right)\right)^{\lambda}\msd{\sigma}j^{\sigma}\left(y\right)=\nonumber \\
 & =q_{e}\int\dd^{4}y\int\frac{\dd^{4}q}{\left(2\pi\right)^{4}}e^{-\ii q\left(x-y\right)}\left(D_{F}\left(q\right)\right)^{\lambda}\msd{\sigma}\int\frac{\dd^{4}\tilde{q}}{\left(2\pi\right)^{4}}e^{-\ii\tilde{q}y}j^{\sigma}\left(\tilde{q}\right)=\nonumber \\
 & =q_{e}\int\frac{\dd^{4}q}{\left(2\pi\right)^{4}}\frac{\dd^{4}\tilde{q}}{\left(2\pi\right)^{4}}\dd^{4}ye^{-\ii\left(\tilde{q}-q\right)y}e^{-\ii qx}\left(D_{F}\left(q\right)\right)^{\lambda}\msd{\sigma}j^{\sigma}\left(\tilde{q}\right)=\nonumber \\
 & =q_{e}\int\frac{\dd^{4}q}{\left(2\pi\right)^{4}}\frac{\dd^{4}\tilde{q}}{\left(2\pi\right)^{4}}\left(2\pi\right)^{4}\delta\left(\tilde{q}-q\right)e^{-\ii qx}\left(D_{F}\left(q\right)\right)^{\lambda}\msd{\sigma}j^{\sigma}\left(\tilde{q}\right)=\nonumber \\
 & =q_{e}\int\frac{\dd^{4}q}{\left(2\pi\right)^{4}}e^{-\ii qx}\left(D_{F}\left(q\right)\right)^{\lambda}\msd{\sigma}j^{\sigma}\left(q\right)
\end{align}
Für den Term proportional zu $q_{\sigma}$ liefert die Kontinuitätsgleichung:
\begin{align*}
q_{\sigma}j^{\sigma}\left(q\right) & =0
\end{align*}
Also spielt dieser Term für $A^{\lambda}\left(x\right)$ keine Rolle.

Nun müssen wir den Tensor auf der linken Seite invertieren. Da $D_{F}\left(q\right)$
nur vom 4-Vektor $q^{\lambda}$ abhängt, und wegen $\epsilon^{\lambda}\msd{\sigma\alpha\beta}q^{\alpha}q^{\beta}=0$
ist aufgrund der Lorentz-Symmetrie der allgemeinste Ansatz:
\begin{align*}
\left(D_{F}\left(q\right)\right)^{\lambda}\msd{\sigma} & =B\left(q^{2}\right)g^{\lambda}\msd{\sigma}+A\left(q^{2}\right)\frac{q^{\lambda}q_{\sigma}}{q^{2}}
\end{align*}
Einsetzen liefert:
\begin{align*}
\left(q^{2}g^{\nu}\msd{\lambda}-q_{\lambda}q^{\nu}\right)q^{\lambda}A\left(q^{2}\right)\frac{q_{\sigma}}{q^{2}} & =\left(q^{2}q^{\nu}-q^{2}q^{\nu}\right)A\left(q^{2}\right)\frac{q_{\sigma}}{q^{2}}=0\\
\left(q^{2}g^{\nu}\msd{\lambda}-q_{\lambda}q^{\nu}\right)B\left(q^{2}\right)g^{\lambda}\msd{\sigma} & =\left(q^{2}g^{\nu}\msd{\sigma}-q_{\sigma}q^{\nu}\right)B\left(q^{2}\right)\stackrel{!}{=}-g^{\nu}\msd{\sigma}+b\frac{q^{\nu}q_{\sigma}}{q_{\mu}q^{\mu}}
\end{align*}
Daher folgt $B\left(q^{2}\right)=-\frac{1}{q^{2}}$ und $b=1$ und
somit:
\begin{align}
\left(D_{F}\left(q\right)\right)^{\lambda}\msd{\sigma} & =-\frac{g^{\lambda}\msd{\sigma}-A\left(q^{2}\right)\frac{q^{\lambda}q_{\sigma}}{q^{2}}}{q^{2}+\ii\varepsilon}
\end{align}
Die Freiheit zur Wahl von $A\left(q^{2}\right)$ ist gerade die Eichfreiheit.
Die bekanntesten Eichungen sind:
\begin{enumerate}
\item \emph{Feynman Eichung}: $A=0$
\begin{align}
\left(D_{F}\left(q\right)\right)^{\lambda}\msd{\sigma} & =-\frac{g^{\lambda}\msd{\sigma}}{q^{2}+\ii\varepsilon}
\end{align}

\item \emph{Landau Eichung}: $A=1$
\begin{align}
\left(D_{F}\left(q\right)\right)^{\lambda}\msd{\sigma} & =-\frac{g^{\lambda}\msd{\sigma}-\frac{q^{\lambda}q_{\sigma}}{q^{2}}}{q^{2}+\ii\varepsilon}
\end{align}
Dies kann bei komplizierten Problemen die Rechnung vereinfachen.
\item Allgemeinere Eichung: $A=\frac{1-c}{c}\in\mathbb{R}$
\begin{align}
\left(D_{F}\left(q\right)\right)^{\lambda}\msd{\sigma} & =-\frac{g^{\lambda}\msd{\sigma}-\frac{1-c}{c}\frac{q^{\lambda}q_{\sigma}}{q^{2}}}{q^{2}+\ii\varepsilon}
\end{align}

\end{enumerate}
%DATE: 8.5.13 (2. Vorlesung)


\chapter{Berechnung Physikalischer Prozesse}


\section{Elektron-Myon-Streuung}

Die Elektron-Myon-Streuung und die Elektron-Quark-Streuung können
gleichzeitig behandelt werden. Der einzige Unterschied ist die Ladung
des Stoßpartners. Im Folgenden betrachten wir die Streuung mit einem
Quark der $Q_{q}$-fachen Elementarladung $q_{e}$, das heißt $q_{q}=Q_{q}q_{e}$.

\begin{figure}[H]
\noindent \centering{}\begin{tikzpicture}[level/.style={level distance=15mm}, scale=1.5]
  \path (0,0) node[position] {}
    child [grow=40] {node[inner sep=0pt]{} node[below right]{$p_e', s_e'$} edge from parent[electron] node[above]{$e^-$}}
    child [grow=-180] {node[inner sep=0pt]{} node[below]{$p_e, s_e$} edge from parent[positron] node[above]{$e^-$}}
    child [grow=-50] {node[position] {}
      child [grow=10] {node[inner sep=0pt]{} node[below right]{$p_q', s_q'$} edge from parent[electron] node[above]{$q$}}
      child [grow=-140] {node[inner sep=0pt]{} node[below]{$p_q, s_q$} edge from  parent[positron] node[above]{$q$}}
    node[above=5pt] {$\nu$} edge from parent[photon] node[above right]{$p_p=p_e-p_e'=p_q'-p_q$}}  node [above left]{$\varrho $};
\end{tikzpicture}\caption{Feynman-Diagramm der ersten Ordnung}
\end{figure}


Das Matrixelement ist nach den Feynman-Regeln mit allen Faktoren:
\begin{align*}
M & =\int\left(\frac{\overline{u}_{e}\left(\vec{p}_{e}',s_{e}'\right)}{\sqrt{2E_{e}'}}\left(-\ii\left(-q_{e}\right)\gamma_{\varrho}\right)\left(2\pi\right)^{4}\delta^{\left(4\right)}\left(p_{e}-p_{e}'-p_{p}\right)\frac{u_{e}\left(\vec{p}_{e},s_{e}\right)}{\sqrt{2E_{e}}}\right)\cdot\\
 & \qquad\cdot\left(\frac{\overline{u}_{q}\left(\vec{p}_{q}',s_{q}'\right)}{\sqrt{2E_{q}'}}\left(-\ii Q_{q}q_{e}\gamma_{\nu}\right)\left(2\pi\right)^{4}\delta^{\left(4\right)}\left(p_{q}+p_{p}-p_{q}'\right)\frac{u_{q}\left(\vec{p}_{q},s_{q}\right)}{\sqrt{2E_{q}}}\right)\frac{\left(-\ii\right)g^{\varrho\nu}}{p_{p}^{2}+\ii\varepsilon}\frac{\dd^{4}p_{p}}{\left(2\pi\right)^{4}}=\\
 & =-\ii Q_{q}q_{e}^{2}\left(\overline{u}_{e}\left(\vec{p}_{e}',s_{e}'\right)\frac{1}{\sqrt{2E_{e}'}}\gamma_{\varrho}u_{e}\left(\vec{p}_{e},s_{e}\right)\frac{1}{\sqrt{2E_{e}}}\right)\cdot\frac{g^{\varrho\nu}}{\left(p_{e}-p_{e}'\right)^{2}+\ii\varepsilon}\cdot\\
 & \qquad\cdot\left(\overline{u}_{q}\left(\vec{p}_{q}',s_{q}'\right)\frac{1}{\sqrt{2E_{q}'}}\gamma_{\nu}\left(2\pi\right)^{4}\delta^{\left(4\right)}\left(p_{q}+p_{e}-p_{e}'-p_{q}'\right)u_{q}\left(\vec{p}_{q},s_{q}\right)\frac{1}{\sqrt{2E_{q}}}\right)=\\
\\
 & =\underbrace{-\ii Q_{q}q_{e}^{2}\overline{u}_{e}\left(\vec{p}_{e}',s_{e}'\right)\gamma_{\varrho}u_{e}\left(\vec{p}_{e},s_{e}\right)\overline{u}_{q}\left(\vec{p}_{q}',s_{q}'\right)\gamma_{\nu}u_{q}\left(\vec{p}_{q},s_{q}\right)\cdot\frac{g^{\varrho\nu}}{\left(p_{e}-p_{e}'\right)^{2}+\ii\varepsilon}\cdot}_{=:\mathcal{M}}\\
 & \qquad\cdot\left(2\pi\right)^{4}\delta^{\left(4\right)}\left(p_{e}+p_{q}-p_{e}'-p_{q}'\right)\frac{1}{4\sqrt{E_{e}E_{e}'E_{q}E_{q}'}}
\end{align*}
Das verkürzte Matrixelement $\mathcal{M}$ kann man direkt erhalten,
wenn man die Feynman-Regeln mit Energie-Impuls-Erhaltung ohne die
zusätzlichen Faktoren benutzt.

Für die Wahrscheinlichkeit muss man das Betragsquadrat bilden.
\begin{align*}
\abs M^{2} & =\abs{\mathcal{M}}^{2}\cdot\left(2\pi\right)^{4}\delta^{\left(4\right)}\left(p_{e}+p_{q}-p_{e}'-p_{q}'\right)\frac{1}{16E_{e}E_{e}'E_{q}E_{q}'}\left(2\pi\right)^{4}\underbrace{\delta^{\left(4\right)}\left(0\right)}_{=\text{n.def.}}
\end{align*}
Das Problem ist, dass das Produkt von Distributionen nicht definiert
ist. Man interpretiert dieses ,,$\delta^{\left(4\right)}\left(0\right)$``
wie folgt:
\begin{align*}
\left(2\pi\right)^{4}\delta^{\left(4\right)}\left(q\right) & =\int_{-\infty}^{\infty}\dd^{4}xe^{-\ii xq}=\lim_{V,T\to\infty}\int_{V,T}\dd^{4}xe^{-\ii xq}\\
\Rightarrow\qquad\left(2\pi\right)^{4}\delta^{\left(4\right)}\left(0\right) & \,\widehat{=}\, VT
\end{align*}
Das Betragsquadrat des verkürzte Matrixelement ist: 
\begin{align*}
\abs{\mathcal{M}}^{2} & =\left(\overline{u}_{e}\left(\vec{p}_{e}',s_{e}'\right)\gamma_{\varrho}u_{e}\left(\vec{p}_{e},s_{e}\right)\right)^{\dagger}\overline{u}_{e}\left(\vec{p}_{e}',s_{e}'\right)\gamma_{\varrho'}u_{e}\left(\vec{p}_{e},s_{e}\right)\cdot\\
 & \quad\cdot\left(\overline{u}_{q}\left(\vec{p}_{q}',s_{q}'\right)\gamma_{\nu}u_{q}\left(\vec{p}_{q},s_{q}\right)\right)^{\dagger}\overline{u}_{q}\left(\vec{p}_{q}',s_{q}'\right)\gamma_{\nu'}u_{q}\left(\vec{p}_{q},s_{q}\right)\cdot\\
 & \quad\cdot g^{\varrho\nu}g^{\varrho'\nu'}\cdot\frac{1}{\left(\left(p_{e}-p_{e}'\right)^{2}+\ii\varepsilon\right)\left(\left(p_{e}-p_{e}'\right)^{2}-\ii\varepsilon\right)}q_{e}^{4}Q_{q}^{2}=\\
 & =\overline{u}_{e}\left(\vec{p}_{e},s_{e}\right)\gamma_{\varrho}u_{e}\left(\vec{p}_{e}',s_{e}'\right)\overline{u}_{e}\left(\vec{p}_{e}',s_{e}'\right)\gamma_{\varrho'}u_{e}\left(\vec{p}_{e},s_{e}\right)\cdot\\
 & \quad\cdot\overline{u}_{q}\left(\vec{p}_{q},s_{q}\right)\gamma_{\nu}u_{q}\left(\vec{p}_{q}',s_{q}'\right)\overline{u}_{q}\left(\vec{p}_{q}',s_{q}'\right)\gamma_{\nu'}u_{q}\left(\vec{p}_{q},s_{q}\right)\cdot\\
 & \quad\cdot\frac{q_{e}^{4}Q_{q}^{2}g^{\varrho\nu}g^{\varrho'\nu'}}{\left(p_{e}-p_{e}'\right)^{4}+\varepsilon^{2}}=\\
 & =\text{tr}\left(\overline{u}_{e}\left(\vec{p}_{e},s_{e}\right)\gamma_{\varrho}u_{e}\left(\vec{p}_{e}',s_{e}'\right)\overline{u}_{e}\left(\vec{p}_{e}',s_{e}'\right)\gamma_{\varrho'}u_{e}\left(\vec{p}_{e},s_{e}\right)\right)\cdot\\
 & \qquad\text{tr}\left(\overline{u}_{q}\left(\vec{p}_{q},s_{q}\right)\gamma_{\nu}u_{q}\left(\vec{p}_{q}',s_{q}'\right)\overline{u}_{q}\left(\vec{p}_{q}',s_{q}'\right)\gamma_{\nu'}u_{q}\left(\vec{p}_{q},s_{q}\right)\right)\cdot\\
 & \qquad\cdot\frac{q_{e}^{4}Q_{q}^{2}g^{\varrho\nu}g^{\varrho'\nu'}}{\left(p_{e}-p_{e}'\right)^{4}+\varepsilon^{2}}=\\
 & =\text{tr}\left(\gamma_{\varrho}u_{e}\left(\vec{p}_{e}',s_{e}'\right)\overline{u}_{e}\left(\vec{p}_{e}',s_{e}'\right)\gamma_{\varrho'}u_{e}\left(\vec{p}_{e},s_{e}\right)\overline{u}_{e}\left(\vec{p}_{e},s_{e}\right)\right)\cdot\\
 & \qquad\text{tr}\left(\gamma_{\nu}u_{q}\left(\vec{p}_{q}',s_{q}'\right)\overline{u}_{q}\left(\vec{p}_{q}',s_{q}'\right)\gamma_{\nu'}u_{q}\left(\vec{p}_{q},s_{q}\right)\overline{u}_{q}\left(\vec{p}_{q},s_{q}\right)\right)\cdot\\
 & \qquad\cdot\frac{q_{e}^{4}Q_{q}^{2}g^{\varrho\nu}g^{\varrho'\nu'}}{\left(p_{e}-p_{e}'\right)^{4}+\varepsilon^{2}}=\\
 & =\text{tr}\left(\gamma_{\varrho}\left(\slashd p_{e}'+m_{e}\right)\frac{1+\alpha_{e}'\gamma_{5}\slashd s_{e}'}{2}\gamma_{\varrho'}\left(\slashd p_{e}+m_{e}\right)\frac{1+\alpha_{e}\gamma_{5}\slashd s_{e}}{2}\right)\cdot\\
 & \qquad\text{tr}\left(\gamma^{\varrho}\left(\slashd p_{q}'+m_{q}\right)\frac{1+\alpha_{q}'\gamma_{5}\slashd s_{q}'}{2}\gamma^{\varrho'}\left(\slashd p_{q}+m_{q}\right)\frac{1+\alpha_{q}'\gamma_{5}\slashd s_{q}}{2}\right)\cdot\\
 & \qquad\cdot\frac{q_{e}^{4}Q_{q}^{2}}{\left(p_{e}-p_{e}'\right)^{4}+\varepsilon^{2}}
\end{align*}
Dabei sind die $\alpha\in\left\{ \pm1\right\} $, je nachdem, ob der
Spin positiv oder negativ ist (vergleiche (\ref{eq:P1}) und (\ref{eq:P2})).
Da wir uns nur für den unpolarisierten Prozess interessieren, summieren
wir über die Endkanäle und mitteln über die Anfangskanäle:
\begin{align*}
\frac{1}{2}\sum_{\alpha_{e}\in\left\{ \pm1\right\} }\frac{1}{2}\sum_{\alpha_{q}\in\left\{ \pm1\right\} }\sum_{\alpha_{e}'\in\left\{ \pm1\right\} }\sum_{\alpha_{q}'\in\left\{ \pm1\right\} } & =\frac{1}{4}\sum_{\alpha_{e},\alpha_{q},\alpha_{e}',\alpha_{q}'\in\left\{ \pm1\right\} }
\end{align*}
Es folgt
\begin{align*}
I_{1} & :=\frac{1}{4}\sum_{\alpha_{e},\alpha_{q},\alpha_{e}',\alpha_{q}'\in\left\{ \pm1\right\} }\abs{\mathcal{M}}^{2}=\\
 & =\frac{1}{4}\text{tr}\left(\gamma_{\varrho}\left(\slashd p_{e}'+m_{e}\right)\underbrace{\left(\sum_{\alpha_{e}'\in\left\{ \pm1\right\} }\frac{1+\alpha_{e}'\gamma_{5}\slashd s_{e}'}{2}\right)}_{=\mathbbm{1}}\gamma_{\varrho'}\left(\slashd p_{e}+m_{e}\right)\sum_{\alpha_{e}\in\left\{ \pm1\right\} }\frac{1+\alpha_{e}\gamma_{5}\slashd s_{e}}{2}\right)\cdot\\
 & \qquad\text{tr}\left(\gamma^{\varrho}\left(\slashd p_{q}'+m_{q}\right)\sum_{\alpha_{q}'\in\left\{ \pm1\right\} }\frac{1+\alpha_{q}'\gamma_{5}\slashd s_{q}'}{2}\gamma^{\varrho'}\left(\slashd p_{q}+m_{q}\right)\sum_{\alpha_{q}\in\left\{ \pm1\right\} }\frac{1+\alpha_{q}\gamma_{5}\slashd s_{q}}{2}\right)\cdot\\
 & \qquad\cdot\frac{q_{e}^{4}Q_{q}^{2}}{\left(p_{e}-p_{e}'\right)^{4}+\varepsilon^{2}}=\\
 & =\frac{q_{e}^{4}Q_{q}^{2}}{4\left(\left(p_{e}-p_{e}'\right)^{4}+\varepsilon^{2}\right)}\text{tr}\left(\gamma_{\varrho}\left(\slashd p_{e}'+m_{e}\right)\gamma_{\varrho'}\left(\slashd p_{e}+m_{e}\right)\right)\cdot\text{tr}\left(\gamma^{\varrho}\left(\slashd p_{q}'+m_{q}\right)\gamma^{\varrho'}\left(\slashd p_{q}+m_{q}\right)\right)
\end{align*}
Das $\varepsilon^{2}$ kann man weglassen, da $\left(p_{e}-p_{e}'\right)^{2}\not=0$
ist. Für hohe Energien können wir $m_{e}\approx0\approx m_{q}$ nähern
und mit
\begin{align*}
\text{tr}\left(\gamma_{\varrho}\slashd p_{e}'\gamma_{\varrho'}\slashd p_{e}\right) & =4\left(p_{e,\varrho}'p_{e,\varrho'}-g_{\varrho\varrho'}p_{e}'\cdot p_{e}+p_{e,\varrho}p_{e,\varrho'}'\right)
\end{align*}
folgt dann:
\begin{align*}
I_{1} & \approx\frac{q_{e}^{4}Q_{q}^{2}\cdot16}{4\left(p_{e}-p_{e}'\right)^{4}}\left(p_{e,\varrho}p_{e,\varrho'}'-g_{\varrho\varrho'}\left(p_{e}\cdot p_{e}'\right)+p_{e,\varrho}'p_{e,\varrho'}\right)\left(p_{q}^{\varrho}\left(p_{q}'\right)^{\varrho'}-g^{\varrho\varrho'}\left(p_{q}\cdot p_{q}'\right)+\left(p_{q}'\right)^{\varrho}p_{q}^{\varrho'}\right)=\\
 & =\frac{4q_{e}^{4}Q_{q}^{2}}{\left(p_{e}-p_{e}'\right)^{4}}\left(2\left(p_{e}p_{q}\right)\left(p_{e}'p_{q}'\right)-4\left(p_{e}p_{e}'\right)\left(p_{q}p_{q}'\right)+2\left(p_{e}p_{q}'\right)\left(p_{e}'p_{q}\right)+g^{\varrho}\msd{\varrho}\left(p_{e}p_{e}'\right)\left(p_{q}p_{q}'\right)\right)=\\
 & =\frac{8q_{e}^{4}Q_{q}^{2}}{\left(p_{e}-p_{e}'\right)^{4}}\left(\left(p_{e}\cdot p_{q}\right)\left(p_{e}'\cdot p_{q}'\right)+\left(p_{e}\cdot p_{q}'\right)\left(p_{e}'\cdot p_{q}\right)\right)
\end{align*}
Führe nun die \emph{Mandelstamm-Variablen} ein:
\begin{align*}
s & :=\left(p_{e}+p_{q}\right)^{2}\sr ={m\approx0}{}2p_{e}p_{q}=\left(p_{e}'+p_{q}'\right)^{2}=2p_{e}'p_{q}'\\
t & :=\left(p_{e}-p_{e}'\right)^{2}\sr ={m\approx0}{}-2p_{e}p_{e}'=\left(p_{q}-p_{q}'\right)^{2}=-2p_{q}p_{q}'\\
u & :=\left(p_{e}-p_{q}'\right)^{2}\sr ={m\approx0}{}-2p_{e}p_{q}'=\left(p_{q}'-p_{e}\right)^{2}=-2p_{e}'p_{q}
\end{align*}
\begin{align*}
s+t+u & \sr ={m\approx0}{}0
\end{align*}
Damit folgt:
\begin{align*}
I_{1} & =\frac{2q_{e}^{4}Q_{q}^{2}}{t^{2}}\left(s^{2}+u^{2}\right)
\end{align*}
Das heißt:
\begin{align*}
\frac{1}{4}\sum_{\sr{}{\alpha_{e},\alpha_{q}}{\alpha_{e}',\alpha_{q}'}}\abs M^{2} & =2q_{e}^{4}Q_{q}^{2}\cdot\frac{s^{2}+u^{2}}{t^{2}}\frac{1}{16E_{e}E_{e'}E_{q}E_{q'}}\left(2\pi\right)^{4}\delta^{\left(4\right)}\left(p_{e}+p_{q}-p_{e}'-p_{q}'\right)VT
\end{align*}
Das Quark sei auf einer Fläche $A$ lokalisiert.

\begin{figure}[H]
\noindent \centering{}\begin{tikzpicture}
  \draw (0,0) ellipse (1 and 2);
  \node at (0,2.5) {$A$};
  \draw[orange] (0,0) ellipse (0.25 and 0.5);
  \node[orange] at (0,0.75) {$\sigma$};
  \draw[fill=black] (0,0) ellipse (0.05 and 0.1);
  \node at (1.8,0) {Quark};
  \draw (0,2) -- (-6,2) arc (90:270:1 and 2) -- (0,-2);
  \draw[dashed] (-6,-2) arc (-90:90:1 and 2);
\draw [decorate,decoration={brace,amplitude=10pt},xshift=-4pt,yshift=0pt] (0,-2.5) -- node [below=10pt]{$v\cdot T$} (-6,-2.5);
  \draw[->,thick] (-4,0) node[left]{$e^-$} -- node[above]{$\vec{v}$} (-2,0);
\end{tikzpicture}\caption{Zusammenhang zum Wirkungsquerschnitt des Quarks}
\end{figure}


Wir betrachten zunächst die Wahrscheinlichkeit, dass ein Elektron
in der Beobachtungszeit $T$ die Fläche $A$ trifft.
\begin{align*}
{\displaystyle \frac{TvA}{V}}
\end{align*}
Die Wahrscheinlichkeit, dass, wenn das Elektron die Fläche $A$ trifft,
die gewünschte Reaktion erfolgt, ergibt sich mit dem Wirkungsquerschnitt
$\sigma$ zu:
\begin{align*}
\frac{\sigma}{A}
\end{align*}
Die Gesamtwahrscheinlichkeit ist somit:
\begin{align*}
\frac{TvA}{V}\cdot\frac{\sigma}{A} & =\frac{Tv\sigma}{V}
\end{align*}
Wir wählen als Normierung:
\begin{align*}
\int_{V}\dd^{3}x\overline{\psi}\psi & =1
\end{align*}
Gehe nun von $\psi$ über zu $\frac{1}{\sqrt{V}}\psi$. Die Diskretisierung
der Impulsintegrale liefert:
\begin{align*}
V\int\frac{\dd^{3}p}{\left(2\pi\right)^{3}} & \leftrightarrow\sum_{\vec{p}}
\end{align*}
Damit folgt:
\begin{align*}
\frac{TvA}{V}\cdot\frac{\sigma}{A} & =\frac{Tv\sigma}{V}\stackrel{!}{=}\underbrace{V^{2}\int\frac{\dd^{3}p_{e}'\dd^{3}p_{q}'}{\left(2\pi\right)^{6}}}_{\text{Diskterisierung der Impulsintegrale}}\frac{1}{4}\sum_{\sr{}{\alpha_{e},\alpha_{q}}{\alpha_{e}',\alpha_{q}'}}\abs M^{2}\cdot\underbrace{\left(\frac{1}{\sqrt{V}}\right)^{8}}_{\text{andere Normierung von }\psi}
\end{align*}
\begin{align*}
\sigma & =\frac{1}{vTV}\int\frac{\dd^{3}p_{e}'\dd^{3}p_{q}'}{\left(2\pi\right)^{6}}\frac{1}{4}\sum_{\sr{}{\alpha_{e},\alpha_{q}}{\alpha_{e}',\alpha_{q}'}}\abs M^{2}=\\
 & =\frac{1}{v}\int\frac{\dd^{3}p_{e}'\dd^{3}p_{q}'}{\left(2\pi\right)^{6}}2q_{e}^{4}Q_{q}^{2}\cdot\frac{s^{2}+u^{2}}{t^{2}}\frac{1}{16E_{e}E_{e'}E_{q}E_{q'}}\left(2\pi\right)^{4}\delta^{\left(4\right)}\left(p_{e}+p_{q}-p_{e}'-p_{q}'\right)=\\
 & =\frac{q_{e}^{4}Q_{q}^{2}}{v}\int\frac{\dd^{3}p_{e}'}{\left(4\pi\right)^{2}}\frac{s^{2}+u^{2}}{t^{2}}\frac{1}{2E_{e}E_{e}'E_{q}E_{q}'}\delta\left(E_{e}+E_{q}-\sqrt{\left(\vec{p}_{e}'\right)^{2}}-\sqrt{\left(\vec{p}_{e}+\vec{p}_{q}-\vec{p}_{e}'\right)^{2}}\right)
\end{align*}
Dies sollte Lorentz-invariant sein.
\begin{align*}
v\cdot2E_{e}E_{q} & \sr ={v_{q}=0}{}\left(v_{e}+v_{q}\right)2E_{e}E_{q}=2\left(p_{e}E_{q}+p_{q}E_{e}\right)
\end{align*}
\begin{align*}
\left(p_{e}\cdot p_{q}\right)^{2}-m_{e}^{2}m_{q}^{2} & =\left(E_{e}E_{q}+\norm{\vec{p}_{e}}\norm{\vec{p}_{q}}\right)^{2}-\left(E_{e}^{2}-\norm{\vec{p}_{e}}^{2}\right)\left(E_{q}^{2}-\norm{\vec{p}_{q}}^{2}\right)=\\
 & =2\norm{\vec{p}_{e}}\norm{\vec{p}_{q}}E_{e}E_{q}+\norm{\vec{p}_{e}}^{2}E_{q}^{2}+\norm{\vec{p}_{q}}^{2}E_{e}^{2}=\\
 & =\left(\norm{\vec{p}_{e}}E_{q}+\norm{\vec{p}_{q}}E_{e}\right)^{2}
\end{align*}
Also gilt:
\begin{align*}
v\cdot2E_{e}E_{q} & =2\sqrt{\left(p_{e}\cdot p_{q}\right)^{2}-m_{e}^{2}m_{q}^{2}}
\end{align*}
Damit folgt in der Näherung $m\approx0$:
\begin{align*}
\sigma & =\frac{q_{e}^{4}Q_{q}^{2}}{\sqrt{\left(2p_{e}p_{q}\right)^{2}}}\int\frac{\dd^{3}p_{e}'}{\left(4\pi\right)^{2}}\frac{s^{2}+u^{2}}{t^{2}}\frac{1}{E_{e}'E_{q}'}\delta\left(E_{e}+E_{q}-\sqrt{\left(\vec{p}_{e}'\right)^{2}}-\sqrt{\left(\vec{p}_{e}+\vec{p}_{q}-\vec{p}_{e}'\right)^{2}}\right)\\
\dd\sigma & \sr ={\alpha=\frac{q_{e}^{2}}{4\pi}}{s=2p_{e}p_{q}}4\alpha^{2}Q_{q}^{2}\frac{\dd^{3}p_{e}'}{2E_{e}'}\frac{s^{2}+u^{2}}{st^{2}}\frac{1}{2E_{q}'}\delta\left(E_{e}+E_{q}-E_{e}'-E_{q}'\right)
\end{align*}
Nun sind $s,t,u$, $\frac{\dd^{3}p_{e}'}{2E_{e}'}$ und $\frac{1}{2E_{q}'}\delta\left(E_{e}+E_{q}-E_{e}'-E_{q}'\right)$
Lorentz-invariant, und damit auch $\dd\sigma$. Daher kann man in
das Schwerpunkt-Bezugssystem mit $E_{q}=E_{e}=E_{q}'=E_{e}'$ übergehen.
Dort kann man den Wirkungsquerschnitt explizit Lorentz-invariant schreiben.
Das Endergebnis ist:
\begin{align*}
\dd\sigma & =\frac{\abs{\mathcal{M}}^{2}}{4\sqrt{\left(p_{e}p_{q}\right)^{2}-m_{e}^{2}m_{q}^{2}}}\frac{\dd^{3}p_{e}'\dd^{3}p_{q}'}{\left(2\pi\right)^{6}2E_{e}'2E_{q}'}\left(2\pi\right)^{4}\delta^{\left(4\right)}\left(p_{e}+p_{q}-p_{e}'-p_{q}'\right)
\end{align*}
 %DATE: Mo 13.5.13

Allgemeine Formel für $2\to n$ Prozesse lautet:
\begin{align*}
\dd\sigma & =\frac{\abs{\mathcal{M}}^{2}}{4\sqrt{\left(p_{1}\cdot p_{2}\right)^{2}-m_{1}^{2}m_{2}^{2}}}\underbrace{\left(2\pi\right)^{4}\delta^{4}\left(p_{1}+p_{2}-p_{3}-\ldots-p_{n}\right)\prod_{j=3}^{n+2}\frac{\dd^{3}p_{j}}{\left(2\pi\right)^{3}\cdot2E_{j}}}_{\text{Lorentz-invariantes Phasenraumelement}}
\end{align*}
\begin{figure}[H]
\noindent \centering{}\begin{tikzpicture}
\begin{scope}
  \clip[draw] (0,0) circle(1);
  \foreach \x in {-3,-2.5,...,2}{
    \draw (-1,\x) -- (1,{\x+1});}
\end{scope}
  \draw[decoration={markings,mark=at position 0.6 with {\arrow[scale=2]{>}};},postaction={decorate}] ({cos(60)},{sin(60)}) node[position]{} -- ({2*cos(60)},{2*sin(60)}) node[above right]{$p_3$};
  \draw[decoration={markings,mark=at position 0.6 with {\arrow[scale=2]{>}};},postaction={decorate}] ({cos(30)},{sin(30)}) node[position]{} -- ({2*cos(30)},{2*sin(30)}) node[right]{$p_4$};
  \node at (1.4,-0.2) {$\vdots$};
  \draw[decoration={markings,mark=at position 0.6 with {\arrow[scale=2]{>}};},postaction={decorate}] ({cos(-50)},{sin(-50)}) node[position]{} -- ({2*cos(-50)},{2*sin(-50)}) node[right]{$p_{n+2}$};
  \draw[decoration={markings,mark=at position 0.6 with {\arrow[scale=2]{<}};},postaction={decorate}] ({cos(160)},{sin(160)}) node[position]{} -- ({2*cos(160)},{2*sin(160)}) node[above left]{$p_1$};
\end{tikzpicture}\caption{Allgemeiner $2\to n$ Prozess}
\end{figure}
Zur Berechnung von $\abs{\mathcal{M}}$ verwendet man die Feynman-Diagramme.

Zum Beispiel sind die Feynman-Diagramme für Elektron-Myon-Streuung
in dritter Ordnung:

\begin{figure}[H]
\noindent \centering{}\begin{tikzpicture}[level/.style={level distance=15mm}, scale=1.5]
  \path (0,0) node[position] {}
    child [grow=40] {node[inner sep=0pt]{} node[below right]{$p_e'$} edge from parent[electron]}
    child [grow=-180] {node[inner sep=0pt]{} node[below]{$p_e$} edge from parent[positron]}
    child [grow=-50] {node[position] {}
      child [grow=10] {node[inner sep=0pt]{} node[below right]{$p_q'$} edge from parent[electron]}
      child [grow=-140] {node[inner sep=0pt]{} node[below]{$p_q$} edge from  parent[positron]}
    edge from parent[photon]};

  \node at (4,1) {2 weitere Ordnungen};
  \path[level/.style={level distance=5mm}] (3.5,0) node[position] {}
    child [grow=40] {node[inner sep=0pt]{} edge from parent[electron]}
    child [grow=-220] {node[inner sep=0pt]{} edge from parent[positron]}
    child [grow=south] {node[position] {} edge from parent[photon]};
  \path[level/.style={level distance=5mm}] (4.5,0) node[position] {}
    child [grow=40] {node[inner sep=0pt]{} edge from parent[electron]}
    child [grow=-220] {node[inner sep=0pt]{} edge from parent[positron]}
    child [grow=south] {node[position] {} edge from parent[photon]};

 \node at (-1,-4.5) {$\to$};
  \path[level/.style={level distance=5mm}] (0,-3) node[position] {}
    child [grow=40] {node[inner sep=0pt]{} edge from parent[electron]}
    child [grow=-180] {node[inner sep=0pt]{} edge from parent[positron]}
    child [grow=-45] {node[position] {}
      child [grow=-45] {node[position]{}
        child [grow=135] {node[position]{} edge from parent[electron, edge from parent path={(\tikzparentnode) .. controls +(-0.25,-0.25)  and +(-0.25,-0.25) .. (\tikzchildnode)}]}
        child [grow=-45] {node[position] {}
          child [grow=10] {node[inner sep=0pt]{} edge from parent[electron]}
          child [grow=-140] {node[inner sep=0pt]{} edge from  parent[positron]}
        edge from parent[photon]}
      edge from parent[electron,edge from parent path={(\tikzparentnode) .. controls +(0.25,0.25)  and +(0.25,0.25) .. (\tikzchildnode)}]}
    edge from parent[photon]};
  \path[level/.style={level distance=6mm}] (0.7,-5.5) node[position] {}
    child [grow=40] {node (r) [position]{}
      child [grow=40] {node[inner sep=0pt]{} edge from parent[electron]}
    edge from parent[electron]}
    child [grow=-180] {node[position] (l) {}
      child [grow=-180] {node[inner sep=0pt]{} edge from parent[positron]}
    edge from parent[positron]}
    child [grow=-45] {node[position] {}
      child [grow=10] {node[inner sep=0pt]{} edge from parent[electron]}
      child [grow=-140] {node[inner sep=0pt]{} edge from  parent[positron]}
    edge from parent[photon]};
  \draw[photon] (l) .. controls +(0.25,0.4) and +(-0.4,0.25) ..  (r);
  \path[level/.style={level distance=6mm}] (3,-3) node[position] {}
    child [grow=0] {node (r) [position]{}
      child [grow=0] {node[inner sep=0pt]{} edge from parent[electron]}
    edge from parent[electron]}
    child [grow=-180] {node[inner sep=0pt]{} edge from parent[positron]}
    child [grow=-90] {node[position] {}
      child [grow=0] {node[position] (ru) {}
        child [grow=0] {node[inner sep=0pt]{} edge from parent[electron]}
      edge from parent[electron]}
      child [grow=-180] {node[inner sep=0pt]{} edge from parent[positron]}
    edge from parent[photon]};
  \draw[photon] (r) -- (ru);
  \path[level/.style={level distance=6mm}] (3,-5) node[position] (l) {}
    child [grow=0] {node (r) [position]{}
      child [grow=0] {node[inner sep=0pt]{} edge from parent[electron]}
    edge from parent[electron]}
    child [grow=-180] {node[inner sep=0pt]{} edge from parent[positron]}
    child [grow=-90] {node[position] (lu) {}
      child [grow=0] {node[position] (ru) {}
        child [grow=0] {node[inner sep=0pt]{} edge from parent[electron]}
      edge from parent[electron]}
      child [grow=-180] {node[inner sep=0pt]{} edge from parent[positron]}
    edge from parent[draw=none]};
  \draw[photon] (r) -- (lu);
  \draw[photon] (l) -- (ru);
\end{tikzpicture}\caption{Feynman-Diagramme der dritten Ordnung}
\end{figure}


Das Ergebnis einer Berechnung kann man für andere Prozesse wiederverwenden,
zum Beispiel durch Crossing:

\begin{figure}[H]
\noindent \centering{}\begin{tikzpicture}[level/.style={level distance=15mm}, scale=1.5]
  \path (0,0) node[position] {}
    child [grow=20] {node[inner sep=0pt]{} node[right]{$p_e'$} edge from parent[electron]}
    child [grow=-200] {node[inner sep=0pt]{} node[left]{$p_e$} edge from parent[positron]}
    child [grow=-90] {node[position] {}
      child [grow=-20] {node[inner sep=0pt]{} node[right]{$p_q'$} edge from parent[electron]}
      child [grow=-160] {node[inner sep=0pt]{} node[left]{$p_q$} edge from  parent[positron]}     edge from parent[photon]};
  \draw[thick,decoration={markings,mark=at position 1 with {\arrow[scale=2]{>}};},postaction={decorate}] (2,-1) -- node[above=2pt]{Crossing} node[above=16pt]{$p_e' \to -p_e'$} node[below=8pt]{$p_q \to -p_q$} (3,-1);
\begin{scope}[rotate=90,yshift=-4cm,xshift=-1cm]
  \path (0,0) node[position] {}
    child [grow=20] {node[inner sep=0pt]{} edge from parent[electron] node[left]{$e^+$}}
    child [grow=-200] {node[inner sep=0pt]{} edge from parent[positron] node[left]{$e^-$}}
    child [grow=-90] {node[position] {}
      child [grow=-20] {node[inner sep=0pt]{} edge from parent[electron] node[right]{$q$}}
      child [grow=-160] {node[inner sep=0pt]{} edge from  parent[positron] node[right]{$\overline{q}$}}     edge from parent[photon]};
\end{scope}
\end{tikzpicture}\caption{Crossing}
\end{figure}


Wir hatten für die Elektron-Quark-Streuung:
\begin{align*}
\sigma & =4\alpha^{2}Q_{q}^{2}\int\frac{\dd^{3}p_{e}'\delta\left(E_{e}+E_{q}-\sqrt{\left(\vec{p}_{e}'\right)^{2}}-\sqrt{\left(\vec{p}_{e}+\vec{p}_{q}-\vec{p}_{e}'\right)^{2}}\right)}{2\underbrace{\sqrt{\left(\vec{p}_{e}'\right)^{2}}}_{=E_{e}'}2\underbrace{\sqrt{\left(\vec{p}_{e}+\vec{p}_{q}-\vec{p}_{e}'\right)^{2}}}_{=E_{q}'}}\cdot\frac{s^{2}+u^{2}}{st^{2}}
\end{align*}
Zur Vereinfachung kann man wegen der Lorentz-Invarianz diesen Ausdruck
in einem besonders günstigen Bezugssystem weiter vereinfachen und
dann wieder Lorentz-invariant schreiben. Wir wählen das Schwerpunktsystem
(engl. \foreignlanguage{english}{center of mass system, cms}).
\begin{align*}
\vec{p}_{e}+\vec{p}_{q} & =0=\vec{p}_{e}'+\vec{p}_{q}'
\end{align*}


\begin{figure}[H]
\noindent \centering{}\begin{tikzpicture}[scale=1.5]
  \draw[thick,->] (-2,0) -- node[above]{$e^-$} (-0.5,0);
  \draw[thick,<-] ({2*cos(120)},{2*sin(120)}) -- node[above right]{$e^-$} ({0.5*cos(120)},{0.5*sin(120)});
  \draw[thick,->] (2,0) -- node[above]{$q$} (0.5,0);
  \draw[thick,<-] ({-2*cos(120)},{-2*sin(120)}) -- node[right]{$q$} ({-0.5*cos(120)},{-0.5*sin(120)});
  \draw (1,0) arc (0:120:1);
  \draw[dashed] (0.5,0) -- (0,0) -- ({0.5*cos(120)},{0.5*sin(120)});
  \node at (0.3,0.5) {$\vartheta$};
\end{tikzpicture}\caption{Impulse im Schwerpunktsystem}
\end{figure}


\begin{align*}
\sigma & =4\alpha^{2}Q_{q}^{2}\int\frac{\dd\varphi\dd\left(\cos\left(\vartheta\right)\right)\dd\norm{\vec{p}_{e}'}\cdot\norm{\vec{p}_{e}'}^{2}}{4E^{2}}\delta\left(2E-2\sqrt{\left(\vec{p}_{e}'\right)^{2}}\right)\cdot\frac{s^{2}+u^{2}}{st^{2}}=\\
 & =4\alpha^{2}Q_{q}^{2}\frac{1}{8}\cdot2\pi\int_{-1}^{1}\dd\left(\cos\left(\vartheta\right)\right)\frac{s^{2}+u^{2}}{st^{2}}=\pi\alpha^{2}Q_{q}^{2}\int_{-1}^{1}\dd\left(\cos\left(\vartheta\right)\right)\frac{s^{2}+u^{2}}{st^{2}}
\end{align*}
Unter Vernachlässigung der Masse folgt aus
\begin{align*}
t & =\left(p_{e}-p_{e}'\right)^{2}=-2p_{e}p_{e}'=-2E^{2}\left(1-\cos\left(\vartheta\right)\right)\\
s & =\left(p_{e}+p_{q}\right)^{2}=\left(2E\right)^{2}=4E^{2}\\
\Rightarrow\qquad t & =-\frac{s}{2}\left(1-\cos\left(\vartheta\right)\right)\\
\dd t & =\frac{s}{2}\dd\left(\cos\left(\vartheta\right)\right)
\end{align*}
nun:
\begin{align}
\fbox{\ensuremath{{\displaystyle \frac{\dd\sigma}{\dd t}=2\pi\alpha^{2}Q_{q}^{2}\frac{s^{2}+u^{2}}{s^{2}t^{2}}}}}
\end{align}



\section{Compton-Streuung}

Bei der Compton-Streuung $\gamma+e^{-}\to\gamma+e^{-}$ sind in niedrigster
Ordnung zwei Feynman-Diagramme wichtig.

\begin{figure}[H]
\noindent \centering{} \begin{tikzpicture}[level/.style={level distance=20mm}]
  \path (0,0) node[position] {}
    child [grow=-180] {node[inner sep=0pt]{} edge from parent[positron] node[above=2pt]{$e^-$}}
    child [grow=110] {node[inner sep=0pt]{} edge from parent[photon] node[right]{$\gamma$}}
    child [grow=0] {node[position]{}
      child [grow=-70] {node[inner sep=0pt]{} edge from parent[photon] node[left]{$\gamma$}}
      child [grow=0] {node[inner sep=0pt]{} edge from parent[electron] node[above]{$e^-$}}
    edge from parent[electron]};
  \path (7,0) node[position] {}
    child [grow=-180] {node[inner sep=0pt]{} edge from parent[positron] node[above=2pt]{$e^-$}}
    child [grow=-40,level distance=3cm] {node[inner sep=0pt]{} edge from parent[photon] node[below left]{$\gamma$}}
    child [grow=0] {node[position]{}
      child [grow=140,level distance=3cm] {node[inner sep=0pt]{} edge from parent[photon] node[above right]{$\gamma$}}
      child [grow=0] {node[inner sep=0pt]{} edge from parent[electron] node[above]{$e^-$}}
    edge from parent[electron]};
\end{tikzpicture}\caption{Feynman-Diagramme für Compton-Streuung in zweiter Ordnung}
\end{figure}


Dies ist ein wesentlich schwierigerer Fall, weil die Massen nicht
vernachlässigt werden können und Spuren von bis zu acht Gamma-Matrizen
zu berechnen sind. Die folgende Rechnung soll insbesondere zeigen,
wie wichtig es geeignete Koordinatensysteme und Eichungen zu wählen,
um die Berechnung zu vereinfachen. Außerdem ist das Ergebnis wichtig,
weil man durch Crossing die Wechselwirkungsquerschnitte für andere
Prozesse, zum Beispiel die Quark-Antiquark-Erzeugung aus Gluonen.

\begin{figure}[H]
\noindent \centering{} \begin{tikzpicture}[level/.style={level distance=20mm}]
  \path (0,0) node[position] {}
    child [grow=0] {node[inner sep=0pt]{} edge from parent[electron] node[above]{$q$}}
    child [grow=-180] {node[inner sep=0pt]{} edge from parent[gluon] node[above=2pt]{$g$}}
    child [grow=-90] {node[position] {}
      child [grow=0] {node[inner sep=0pt]{} edge from parent[electron] node[above]{$\overline{q}$}}
      child [grow=-180] {node[inner sep=0pt]{} edge from parent[gluon] node[above=2pt]{$g$}}
    edge from parent[positron]};
\end{tikzpicture}\caption{Quark-Antiquark-Erzeugung aus Gluonen}
\end{figure}







\chapter{Renormierung, Regularisierung, Vakuumpolarisation}


\section{Vakuumpolarisation}

Betrachte einen Vakuumpolarisations-Loop irgendwo in einem größeren
Feynman-Diagramm.

\begin{figure}[H]
\noindent \centering{}\begin{tikzpicture}[level/.style={level distance=20mm}]
  \path (0,0)   node [above left]{$q, \mu , y $} node[above=6mm,xshift=-1mm]{$e^{\ii qy}$} node[position] {}
    child [grow=70] {node[inner sep=0pt]{} edge from parent[dashed,edge from parent path={(\tikzparentnode) .. controls +(0.1,0.2)  and +(0.1,-0.4) .. (\tikzchildnode)}]}
    child [grow=-130] {node[inner sep=0pt]{} edge from parent[dashed,edge from parent path={(\tikzparentnode) .. controls +(-0.1,-0.2)  and +(0.4,0.1) .. (\tikzchildnode)}]}
    child [grow=0] {node[position] {}
      child [grow=east,level distance=1cm]{node[position] {}
        child [grow=0,level distance=2cm] {node[position] {}
          child [grow=110] {node[inner sep=0pt]{} edge from parent[thin,black,dashed,edge from parent  path={(\tikzparentnode) .. controls +(-0.1,0.2)  and +(-0.1,-0.4) .. (\tikzchildnode)}]}
          child [grow=-50] {node[inner sep=0pt]{} edge from parent[thin,black,dashed,edge from parent path={(\tikzparentnode) .. controls +(0.1,-0.2)  and +(-0.4,0.1) .. (\tikzchildnode)}]}
        node [above right]{$\tilde{q}, \nu , z $} node[above=6mm,xshift=1mm]{$e^{-\ii \tilde{q}z}$} edge from parent[photon]}
        child [grow=west,level distance=1cm] {node[position]{} edge from parent[positron,edge from parent path={(\tikzparentnode) .. controls +(0,-0.5)  and +(0,-0.5) .. (\tikzchildnode)}] node[below]{$k$} }
      edge from parent[positron,edge from parent path={(\tikzparentnode) .. controls +(0,0.5)  and +(0,0.5) .. (\tikzchildnode)}] node[above]{$k-q$}}
    edge from parent[photon]};
\end{tikzpicture}\caption{Loop mit freiem Impuls $k$}
\end{figure}


Das Wicksche Theorem liefert in zweiter Ordnung:
\begin{align*}
M & =\int\dd^{4}y\dd^{4}ze^{\ii qy}e^{-\ii\tilde{q}z}\cdot\frac{1}{2}q_{e}^{2}\int\dd^{4}x_{1}\dd^{4}x_{2}\left(-\ii\right)^{2}\cdot\\
 & \quad\cdot\BraKet{0|\mathcal{T}\left\{ \hat{A}_{\mu}\left(y\right)\hat{A}_{\nu}\left(z\right)\hat{A}^{\rho}\left(x_{1}\right)\hat{\overline{\psi}_{j}}\left(x_{1}\right)\left(\gamma_{\rho}\right)_{jk}\hat{\psi}_{k}\left(x_{1}\right)\hat{A}^{\lambda}\left(x_{2}\right)\hat{\overline{\psi}_{j}}\left(x_{2}\right)\left(\gamma_{\lambda}\right)_{lm}\hat{\psi}_{m}\left(x_{2}\right)\right\} |0}=\\
 & =\int\dd^{4}y\dd^{4}ze^{\ii qy}e^{-\ii\tilde{q}z}\cdot\frac{-1}{2}q_{e}^{2}\int\dd^{4}x_{1}\dd^{4}x_{2}\cdot\bigg\langle0\bigg|\mathcal{T}\bigg\{\hat{A}_{\mu}\left(y\right)\hat{A}_{\nu}\left(z\right)\hat{A}^{\rho}\left(x_{1}\right)\left(-1\right)\\
 & \qquad\cdot\text{tr}\left(\left(\gamma_{\rho}\right)_{jk}\hat{\psi}_{k}\left(x_{1}\right)\hat{A}^{\lambda}\left(x_{2}\right)\hat{\overline{\psi}_{j}}\left(x_{2}\right)\left(\gamma_{\lambda}\right)_{lm}\hat{\psi}_{m}\left(x_{2}\right)\hat{\overline{\psi}_{j}}\left(x_{1}\right)\right)\bigg\}\bigg|0\bigg\rangle=\\
 & =q_{e}^{2}\frac{2}{2}\int\dd^{4}y\dd^{4}z\dd^{4}x_{1}\dd^{4}x_{2}\cdot\ii D_{F\mu}\msu{\rho}\left(y-x_{1}\right)\ii D_{F\nu}\msu{\lambda}\left(z-x_{2}\right)\ii\left(S_{F}\left(x_{2}-x_{1}\right)\right)_{mj}\left(\gamma_{\rho}\right)_{jk}\cdot\\
 & \qquad\cdot\ii\left(S_{F}\left(x_{1}-x_{2}\right)\right)_{kl}\left(\gamma_{\lambda}\right)_{lm}e^{-\ii\tilde{q}z+\ii qy}+\text{irrelevante Terme}
\end{align*}


Ein Beispiel für irrelevanten Term ist in Abbildung \ref{abb:irrelevanter-Term}
zu sehen.

\begin{figure}[H]
\noindent \centering{}\begin{tikzpicture}[level/.style={level distance=20mm}]
  \path (-1.5,0) node[position]{}
    child [grow=0,level distance=30mm] {node[position]{} edge from parent [photon]};
  \path (-1,-1.5) node[position]{}
    child [grow=0] {node[position]{} edge from parent [photon]}
    child [grow=0] {node[position]{} edge from parent [electron,edge from parent path={(\tikzparentnode) .. controls +(0,1)  and +(0,1) .. (\tikzchildnode)}]}
    child [grow=0] {node[position]{} edge from parent [positron,edge from parent path={(\tikzparentnode) .. controls +(0,-1)  and +(0,-1) .. (\tikzchildnode)}]};
\end{tikzpicture}\caption{Feynman-Diagramm eines irrelevanten Terms}
\label{abb:irrelevanter-Term}
\end{figure}


Betrachte nun den inneren Teil und führe eine Fourier-Transformation
durch:
\begin{align*}
\tilde{M} & =q_{e}^{2}\int\dd^{4}x_{1}\dd^{4}x_{2}e^{-\ii\tilde{q}x_{1}+\ii qx_{2}}\ii\left(S_{F}\left(x_{2}-x_{1}\right)\right)_{mj}\left(\gamma_{\rho}\right)_{jk}\ii\left(S_{F}\left(x_{1}-x_{2}\right)\right)_{kl}\left(\gamma_{\lambda}\right)_{lm}=\\
 & =-q_{e}^{2}\int\dd^{4}x_{1}\dd^{4}x_{2}\frac{\dd^{4}k'}{\left(2\pi\right)^{4}}\frac{\dd^{4}k}{\left(2\pi\right)^{4}}e^{-\ii\tilde{q}x_{1}+\ii qx_{2}}\text{tr}\left(S_{F}\left(k'\right)\gamma_{\rho}S_{F}\left(k\right)\gamma_{\lambda}\right)e^{-\ii\left(x_{2}-x_{1}\right)k}e^{-\ii\left(x_{1}-x_{2}\right)k'}=\\
 & =-q_{e}^{2}\int\frac{\dd^{4}k'}{\left(2\pi\right)^{4}}\frac{\dd^{4}k}{\left(2\pi\right)^{4}}\left(2\pi\right)^{4}\delta^{\left(4\right)}\left(k-k'-\tilde{q}\right)\left(2\pi\right)^{4}\delta^{\left(4\right)}\big(q\underbrace{-k+k'}_{=-\tilde{q}}\big)\text{tr}\left(S_{F}\left(k'\right)\gamma_{\rho}S_{F}\left(k\right)\gamma_{\lambda}\right)=\\
 & =-q_{e}^{2}\delta^{\left(4\right)}\left(q-\tilde{q}\right)\int\frac{\dd^{4}k}{\left(2\pi\right)^{4}}\text{tr}\left(S_{F}\left(k-q\right)\gamma_{\rho}S_{F}\left(k\right)\gamma_{\lambda}\right)
\end{align*}
Ein Integral ist übrig geblieben, das heißt der im Kreis laufende
Impuls $k$ kann beliebige Werte annehmen. Es folgt:
\begin{align*}
\int\frac{\dd^{4}k}{\left(2\pi\right)^{4}}\text{tr}\left(\frac{\left(\slashd k-\slashd q+m\right)\gamma_{\rho}}{\left(k-q\right)^{2}-m^{2}+\ii\varepsilon}\frac{\left(\slashd k+m\right)\gamma_{\lambda}}{k^{2}-m^{2}+\ii\varepsilon}\right) & \sim\int\frac{\dd^{4}k}{\left(2\pi\right)^{4}}\cdot\frac{g_{\varrho\lambda}k^{2}}{\left(\left(k-q\right)^{2}-m^{2}\right)\left(k^{2}-m^{2}\right)+\ii\eta}
\end{align*}
Dies ist quadratisch divergent! Selbst der Term
\begin{align*}
\int\frac{\dd^{4}k}{\left(2\pi\right)^{4}}\frac{m^{2}}{\left(\left(k-q\right)^{2}-m^{2}+\ii\varepsilon\right)\left(k^{2}-m^{2}+\ii\varepsilon\right)}
\end{align*}
ist logarithmisch divergent. Da dieser Loop in jedem Feynman-Diagramm
hinzugefügt werden kann, ist jede Amplitude divergent, wenn man genügend
viele Ordnungen entwickelt. Wir müssen also einen Weg finden, mit
diesen Divergenzen umzugehen, da sonst die Theorie unbrauchbar ist.


\section{Behandlung der Divergenzen}

Es gibt verschiedene Typen von Divergenzen:
\begin{enumerate}
\item Quadratisch divergent:\\
\begin{figure}[H]
\noindent \centering{}\begin{tikzpicture}[level/.style={level distance=20mm},scale=0.5]
  \path node[position]{}
    child [grow=180] {node[position]{} edge from parent [photon]}
    child [grow=0] {node[position]{}
      child [grow=0] {node[position]{} edge from parent [photon]}     edge from parent [electron,edge from parent path={(\tikzparentnode) .. controls +(0,1)  and +(0,1) .. (\tikzchildnode)}]}
    child [grow=0] {node[position]{} edge from parent [positron,edge from parent path={(\tikzparentnode) .. controls +(0,-1)  and +(0,-1) .. (\tikzchildnode)}]};
\end{tikzpicture}\caption{quadratisch divergentes Feynman-Diagramm}
\end{figure}
Dies wird als Null definiert, da sie sonst die Eichinvarianz verletzen.
\item Linear divergent:
\begin{align*}
\sim\int\frac{\dd^{4}k}{\left(2\pi\right)^{4}}\frac{\left(\slashd k\right)^{3}}{\left(k^{2}\right)^{3}}
\end{align*}
\begin{figure}[H]
\noindent \centering{}\begin{tikzpicture}[level/.style={level distance=20mm},scale=0.5]
  \path node[position]{}
    child [grow=180] {node[position]{} edge from parent [photon]}
    child [grow=0] {node[position]{}
      child [grow=0] {node[position]{} edge from parent [photon]}
    edge from parent [electron,edge from parent path={(\tikzparentnode) .. controls +(0,1.3)  and +(0,1.3) .. (\tikzchildnode)}]}
    child [grow=-45,level distance=14mm] {node[position]{}
      child [grow=-90] {node[position]{} edge from parent [photon]}
      child [grow=45,level distance=14mm] {node[position]{} edge from parent [positron,edge from parent path={(\tikzparentnode) .. controls +(0.65,0)  and +(0,-0.65) .. (\tikzchildnode)}]}
    edge from parent [positron,edge from parent path={(\tikzparentnode) .. controls +(0,-0.65)  and +(-0.65,0) .. (\tikzchildnode)}]};
\end{tikzpicture}\caption{linear divergentes Feynman-Diagramm}
\end{figure}
Solche Beiträge verschwinden in der Quantenelektrodynamik immer und
führen zu einer Anomalie in der Schwachen Wechselwirkung.
\item Logarithmisch divergent:\\
\begin{figure}[H]
\noindent \centering{}\begin{tikzpicture}[level/.style={level distance=20mm},scale=0.5]
   \path node[position]{}
    child [grow=180] {node[position]{} edge from parent [photon]}
    child [grow=45,level distance=14mm] {node[position]{}
      child [grow=90] {node[position]{} edge from parent [photon]}
      child [grow=-45,level distance=14mm] {node[position]{}
      edge from parent [positron,edge from parent path={(\tikzparentnode) .. controls +(0.65,0)  and +(0,0.65) .. (\tikzchildnode)}]}
    edge from parent [electron,edge from parent path={(\tikzparentnode) .. controls +(0,0.65)  and +(-0.65,0) .. (\tikzchildnode)}]}
    child [grow=-45,level distance=14mm] {node[position]{}
      child [grow=-90] {node[position]{} edge from parent [photon]}
      child [grow=45,level distance=14mm] {node[position]{}
        child [grow=0] {node[position]{} edge from parent [photon]}
      edge from parent [positron,edge from parent path={(\tikzparentnode) .. controls +(0.65,0)  and +(0,-0.65) .. (\tikzchildnode)}]}
    edge from parent [positron,edge from parent path={(\tikzparentnode) .. controls +(0,-0.65)  and +(-0.65,0) .. (\tikzchildnode)}] }; 
\end{tikzpicture}\caption{logarithmisch divergentes Feynman-Diagramm}
\end{figure}

\item Logarithmisch divergent mit Bosonen:
\begin{align*}
\int\frac{\dd^{4}k}{\left(2\pi\right)^{4}}\frac{1}{k^{2}-M^{2}+\ii\varepsilon}\frac{1}{\left(k-q\right)^{2}-M^{2}+\ii\varepsilon}
\end{align*}
\begin{figure}[H]
\noindent \centering{}\begin{tikzpicture}[level/.style={level distance=20mm},scale=0.5]
  \path node[position]{}
    child [grow=180] {node[position]{} edge from parent [photon]}
    child [grow=0] {node[position]{}
      child [grow=0] {node[position]{} edge from parent [photon]}     edge from parent [boson,edge from parent path={(\tikzparentnode) .. controls +(0,1)  and +(0,1) .. (\tikzchildnode)}]}
    child [grow=0] {node[position]{} edge from parent [boson,edge from parent path={(\tikzparentnode) .. controls +(0,-1)  and +(0,-1) .. (\tikzchildnode)}]};
\end{tikzpicture}\caption{logarithmisch divergentes Feynman-Diagramm}
\end{figure}

\item Logarithmisch divergent:
\begin{align*}
\int\frac{\dd^{4}k}{\left(2\pi\right)^{k}}\frac{\slashd k+m}{k^{2}-m^{2}+\ii\varepsilon}\frac{1}{\left(k-q\right)^{2}-M^{2}+\ii\varepsilon}
\end{align*}
\begin{figure}[H]
\noindent \centering{}\begin{tikzpicture}[level/.style={level distance=20mm}, scale=0.5]
  \path (0,0) node[position] (l) {}
    child [grow=-180] {node[inner sep=0pt]{} edge from parent[positron]}
    child [grow=0] {node[position] (r) {}
      child [grow=0] {node[inner sep=0pt]{} edge from parent[electron]}
    edge from parent[electron]};
  \draw[photon] (l) .. controls +(0,-1.35) and +(0,-1.35) .. (r);
\end{tikzpicture}\caption{logarithmisch divergentes Feynman-Diagramm}
\end{figure}
Terme mit ungerader $k$-Potenz verschwinden, da über einen symmetrischen
Bereich integriert wird. Es bleibt eine logarithmische Divergenz.
\end{enumerate}
%DATE: Di 14.5.13


\subsection{Logarithmische Divergenzen}

Die logarithmischen Divergenzen kann man renormieren. Die Idee dabei
ist, dass die Divergenzen nur Ausdruck unseres begrenzten Wissens
sind. Die Quantenfeldtheorie ist unvollständig, da sie die Gravitation
nicht enthält!

Vergleiche das Coulomb-Gesetz und das Gravitationsgesetz:
\begin{align*}
V_{\text{Cb}} & =\frac{\alpha Q_{1}Q_{2}}{r} & V_{\text{G}} & =G\frac{M_{1}M_{2}}{r}
\end{align*}
Für die Gravitation liefern die Einsteinschen Feldgleichungen bessere
Ergebnisse:
\begin{align*}
G_{\mu\nu} & =8\pi GT_{\mu\nu}
\end{align*}
Dabei ist $G_{\mu\nu}$ der Einsteintensor, der sich aus der Krümmung
des Raumes berechnet, und $T_{\mu\nu}$ der Energie-Impuls-Tensor,
der durch die Materieverteilung bestimmt ist. Also ist der Grenzfall
niedriger Energie eher:
\begin{align*}
V_{G} & =G\frac{E_{1}E_{2}}{r}
\end{align*}
Eigentlich muss man also $\alpha$ mit $GE^{2}$ vergleichen und das
Stärkeverhältnis der Kräfte abzuschätzen. Für $GE^{2}>1$ ist die
Gravitation genauso wichtig wie jede andere Wechselwirkung. So erhält
man die Planck-Energie:
\begin{align*}
E_{\text{Pl}} & =\frac{1}{\sqrt{G}}\sr ={\hbar c=1}{}\sqrt{\frac{\hbar c}{G}}\stackrel{\text{SI}}{=}1{,}22\cdot10^{19}\text{GeV}
\end{align*}
Für $E\approx E_{\text{Pl}}$ müsste man im Rahmen der Allgemeinen
statt der Speziellen Relativitätstheorie rechnen. Physikalische Gesetze
existieren jedoch auch bei niedrigen Energien. Wir fordern, dass die
\foreignlanguage{english}{Theory of Everything} ,,entkoppelt``,
das heißt, um Ergebnisse verschiedener Experimente vergleichen zu
können, darf die \foreignlanguage{english}{Theory of Everything} keine
Rolle spielen.

Die Differenz zwischen der gemessenen Ladung eines Elektrons im Experiment
1 und im Experiment 2 muss endlich und berechenbar sein. In der ultrarelativistischen
Näherung gilt:
\begin{align*}
q_{e}\left(\text{Exp 1}\right)-q_{e}\left(\text{Exp 2}\right) & =\left(c+c'\ln\left(\frac{p^{2}}{\Lambda^{2}}\right)\right)-\left(c+c'\ln\left(\frac{\tilde{p}^{2}}{\Lambda^{2}}\right)\right)=c'\ln\left(\frac{p^{2}}{\tilde{p}^{2}}\right)
\end{align*}
\emph{Regularisierung} nennt man eine Methode, um die Logarithmen
endlich zu machen. Dazu gibt es mehre Möglichkeiten:
\begin{enumerate}
\item \emph{Cutoff-Regularisierung}: Berechne nur einen Teil des divergenten
Integrals:
\begin{align*}
\int_{-\Lambda}^{\Lambda}\int_{-\Lambda}^{\Lambda}\int_{-\Lambda}^{\Lambda}\int_{-\Lambda}^{\Lambda}\dd^{4}k
\end{align*}
Der Nachteil ist, dass die Lorentz-Invarianz nicht offensichtlich
ist.\\
Eine Variante hiervon ist die Gitter-QFT, bei der eine Quantisierung
des Raumes durchgeführt wird. Ein Gitter mit Konstante $a$ entspricht
einer Cutoff-Grenze von $\Lambda=a^{-1}$.\\
\begin{figure}[H]
\noindent \centering{}\begin{tikzpicture}
  \foreach \x in {0,2,4}{
    \foreach \y in {0,2,4}{
      \draw[fill=black] (\x,\y) circle (0.1);
  }}
\draw [decorate,decoration={brace,amplitude=10pt},yshift=-5pt] (2,0) -- node [below=10pt]{$a$} (0,0);
\end{tikzpicture}\caption{Wellenfunktionen werden auf einem diskreten Gitter berechnet}
\end{figure}
Der Vorteil hiervon ist, dass man dies numerisch berechnen kann. Außerdem
funktioniert das Verfahren immer und anschaulich kann man $a\,\widehat{=}\, l_{\text{Pl}}=E_{\text{Pl}}^{-1}$
wählen.
\item \emph{Pauli-Villar-Regularisierung}: Man verwendet die Idee der Supersymmetrie,
dass jeder divergente Loop von einem anderen (symmetrischen) Teilchen
aufgehoben wird. Für jedes Fermion muss es also ein supersymmetrisches
Boson geben und andersherum.\\
\begin{figure}[H]
\noindent \centering{}\begin{tikzpicture}[level/.style={level distance=20mm},scale=0.5]
  \path (0,0) node[position]{}
    child [grow=180] {node[position]{} edge from parent [photon]}
    child [grow=0] {node[position]{}
      child [grow=0] {node[position]{} edge from parent [photon]}     edge from parent [electron,edge from parent path={(\tikzparentnode) .. controls +(0,1)  and +(0,1) .. (\tikzchildnode)}]}
    child [grow=0] {node[position]{} edge from parent [positron,edge from parent path={(\tikzparentnode) .. controls +(0,-1)  and +(0,-1) .. (\tikzchildnode)}]};
  \node at (5,0) {+};
  \path (8,0) node[position]{}
    child [grow=180] {node[position]{} edge from parent [photon]}
    child [grow=0] {node[position]{}
      child [grow=0] {node[position]{} edge from parent [photon]}     edge from parent [boson,edge from parent path={(\tikzparentnode) .. controls +(0,1)  and +(0,1) .. (\tikzchildnode)}]}
    child [grow=0] {node[position]{} edge from parent [boson,edge from parent path={(\tikzparentnode) .. controls +(0,-1)  and +(0,-1) .. (\tikzchildnode)}]};
\end{tikzpicture}\caption{Supersymmetrischer Loop hebt Divergenzen auf}
\end{figure}
Das andere Teilchen muss die gleichen Feynman-Regeln haben, aber das
umgekehrte Loop-Vorzeichen (Statistik). Man betrachtet dann den Grenzwert,
dass die Masse des symmetrischen Teilchens gegen unendlich geht. Dies
ist explizit Lorentz-invariant.
\item \emph{Dimensionale Regularisierung}: Das Integral
\begin{align*}
\int\frac{\dd x}{x^{\alpha}}
\end{align*}
divergiert für $\alpha=1$ logarithmisch, konvergiert aber für $\alpha>1$.
Verallgemeinere nun die Integration auf fraktale Dimension $d\in\mathbb{R}$.
\begin{align*}
\int\frac{\dd^{d}x}{\sqrt{x^{2}}+a^{2}}
\end{align*}
Dieses $d$-dimensionale Integral konvergiert für $d<1$. Setzt man
dies analytisch auf $d\in\mathbb{C}$ fort und bildet die Laurent-Reihe,
so ist das Residuum des Pols $\frac{1}{d-4}$ dann das regularisierte
Ergebnis.
\end{enumerate}

\subsection{Lineare Divergenzen}

\begin{figure}[H]
\noindent \centering{} \begin{tikzpicture}[level/.style={level distance=20mm}]
  \path node[position]{}
    child [grow=90]{node[position]{} edge from parent[photon]}
    child [grow=-60]{node[position]{}
      child [grow=-45]{node[position]{} edge from parent[photon]}
      child [grow=180]{node[position]{} edge from parent[electron] node[below=2pt]{$k_2$}}
    node[above right]{$\mu_2$} edge from parent[electron] node[right=2pt]{$k_1$}}
    child [grow=-120]{node[position]{}
      child [grow=-135]{node[position]{} edge from parent[photon]}
    node[above left]{$\mu_1$} edge from parent[positron] node[left=2pt]{$k_3$}} node[right]{$\mu_3$};
\end{tikzpicture}\caption{linear divergentes Feynman-Diagramm}
\end{figure}


Die Parität unter Ladungskonjugation $C$ liefert:
\begin{align*}
\psi_{C} & =e^{\ii\varphi}\gamma_{2}\gamma_{0}\overline{\psi}^{\TT}
\end{align*}
Mit
\begin{align*}
\overline{\psi} & =\psi^{\dagger}\gamma_{0}
\end{align*}
folgt:
\begin{align*}
\psi_{C} & =e^{\ii\varphi}\gamma_{2}\gamma_{0}\overline{\psi}^{\TT}\\
\gamma_{2}\psi_{C}e^{-\ii\varphi} & =\gamma_{2}\gamma_{2}\gamma_{0}\overline{\psi}^{\TT}=-\gamma_{0}\overline{\psi}^{\TT}\\
-\gamma_{0}\gamma_{2}\psi_{C}e^{-\ii\varphi} & =\gamma_{0}\gamma_{0}\overline{\psi}^{\TT}=\overline{\psi}^{\TT}\\
-\gamma_{0}\gamma_{2}\psi_{C}e^{-\ii\varphi} & =\left(\psi^{\dagger}\gamma_{0}\right)^{\TT}=\gamma_{0}^{\TT}\left(\psi^{\dagger}\right)^{\TT}=\gamma_{0}\psi^{*}\\
\gamma_{2}\gamma_{0}\gamma_{0}\psi_{C}^{*}e^{\ii\varphi} & =\psi\\
\gamma_{2}\gamma_{0}\left(\overline{\psi}_{C}\right)^{\TT}e^{\ii\varphi} & =\psi\\
\overline{\psi}_{C}\gamma_{0}\gamma_{2}e^{\ii\varphi} & =\psi^{\TT}
\end{align*}
\begin{align*}
\overline{\psi}^{\TT}\psi^{\TT} & =\left(-\gamma_{0}\gamma_{2}\psi_{C}e^{-\ii\varphi}\right)\left(\overline{\psi}_{C}\gamma_{0}\gamma_{2}e^{\ii\varphi}\right)=-\gamma_{0}\gamma_{2}S_{F}^{C}\gamma_{0}\gamma_{2}=\\
 & =-\left(-1\right)^{2}\gamma_{2}\gamma_{0}S_{F}^{C}\gamma_{2}\gamma_{0}
\end{align*}
Nun ist $S_{F}\sim\psi\overline{\psi}$ und somit folgt:
\begin{align*}
S_{F}^{\TT} & \sim\overline{\psi}^{\TT}\psi^{\TT}\stackrel{\text{C}}{\sim}\gamma_{2}\gamma_{0}S_{F}^{\text{C}}\gamma_{2}\gamma_{0}e^{\ii\varphi-\ii\varphi}
\end{align*}
\begin{align*}
\text{tr}\left(\gamma_{\mu_{1}}S_{F}\left(k_{2}\right)\gamma_{\mu_{2}}S_{F}\left(k_{1}\right)\gamma_{\mu_{3}}S_{F}\left(k\right)\right) & =\text{tr}\left(\left(\gamma_{\mu_{1}}S_{F}\left(k_{2}\right)\gamma_{\mu_{2}}S_{F}\left(k_{1}\right)\gamma_{\mu_{3}}S_{F}\left(k\right)\right)^{\TT}\right)=\\
 & =\text{tr}\left(\gamma_{2}\gamma_{0}S_{F}^{\text{C}}\left(k\right)\gamma_{2}\gamma_{0}\gamma_{\mu_{3}}^{\TT}\gamma_{2}\gamma_{0}S_{F}^{\text{C}}\left(k_{1}\right)\gamma_{2}\gamma_{0}\gamma_{\mu_{2}}^{\TT}\gamma_{2}\gamma_{0}S_{F}^{\text{C}}\left(k_{2}\right)\gamma_{2}\gamma_{0}\gamma_{\mu_{1}}^{\TT}\right)
\end{align*}
Nun gilt $S_{F}^{\text{C}}\left(k\right)=S_{F}\left(-k\right)$ und:
\begin{align*}
\gamma_{2}\gamma_{0}\gamma_{\mu_{3}}^{\TT}\gamma_{2}\gamma_{0} & =\gamma_{2}\gamma_{0}\left\{ \begin{array}{c}
-\gamma_{\mu_{3}}\text{ falls }\mu_{3}\in\left\{ 1,3\right\} \\
\gamma_{\mu_{3}}\text{ falls }\mu_{3}\in\left\{ 0,2\right\} 
\end{array}\right\} \gamma_{2}\gamma_{0}=-\gamma_{\mu_{3}}
\end{align*}
Es folgt:
\begin{align*}
\text{tr}\left(\gamma_{\mu_{1}}S_{F}\left(k_{2}\right)\gamma_{\mu_{2}}S_{F}\left(k_{1}\right)\gamma_{\mu_{3}}S_{F}\left(k\right)\right) & =\left(-1\right)^{3}\text{tr}\left(S_{F}\left(-k\right)\gamma_{\mu_{3}}S_{F}\left(-k_{1}\right)\gamma_{\mu_{2}}S_{F}\left(-k_{2}\right)\gamma_{\mu_{1}}\right)
\end{align*}
TODO: Abb43

Mit zusätzlichem $\gamma^{5}$:
\begin{align*}
\text{tr}\left(\gamma^{2}\gamma^{0}S_{F}^{\text{C}}\left(k\right)\gamma^{2}\gamma^{0}\left(\gamma^{5}\right)^{\TT}\gamma_{\mu_{3}}^{\TT}\gamma^{2}\gamma^{0}S_{F}^{\text{C}}\left(k_{1}\right)\gamma^{2}\gamma^{0}\gamma_{\mu_{2}}^{\TT}\gamma^{2}\gamma^{0}S_{F}^{\text{C}}\left(k_{2}\right)\gamma^{2}\gamma^{0}\gamma_{\mu_{1}}^{\TT}\right)
\end{align*}
\begin{align*}
\left(\gamma^{5}\right)^{\TT}\gamma_{\mu_{3}}^{\TT} & =\gamma^{5}\gamma_{\mu_{3}}^{\TT}=-\gamma_{\mu_{3}}^{\TT}\gamma^{5}
\end{align*}
Dieses zusätzliche Minus verhindert, dass der Term wegfallen muss.


\section{Explizite Berechnung der Vakuum-Polarisation mit Pauli-Villars Regularisierung}

\begin{figure}[H]
\noindent \centering{}\begin{tikzpicture}[level/.style={level distance=20mm}]
  \path (0,0) node[position]{}
    child [grow=180] {node[position]{} edge from parent [photon] node[above]{$\leftarrow q$}}
    child [grow=0] {node[position]{}
      child [grow=0] {node[position]{} edge from parent [photon]  node[above]{$\leftarrow q$}}
     node[above right]{$\nu$} edge from parent [electron,edge from parent path={(\tikzparentnode) .. controls +(0,1)  and +(0,1) .. (\tikzchildnode)}] node[above]{$k-q$}}
    child [grow=0] {node[position]{} edge from parent [positron,edge from parent path={(\tikzparentnode) .. controls +(0,-1)  and +(0,-1) .. (\tikzchildnode)}] node[below]{$k$} }  node[above left]{$\mu$};
\end{tikzpicture}\caption{Feynman-Diagramm der Vakuum-Polarisation}
\end{figure}


\begin{align*}
I_{\mu\nu} & =-q_{e}^{2}\int\frac{\dd^{4}k}{\left(2\pi\right)^{4}}\text{tr}\left(\frac{1}{\slashd k-m+\ii\varepsilon}\gamma_{\nu}\frac{1}{\slashd k-\slashd q-m+\ii\varepsilon}\gamma_{\mu}\right)
\end{align*}
Die Eichinvarianz liefert:
\begin{align*}
q^{\mu}I_{\mu\nu} & =0\\
q^{\nu}I_{\mu\nu} & =0
\end{align*}
Dies ist fundamental!
\begin{align*}
q^{\mu}I_{\mu\nu} & =-q_{e}^{2}\text{tr}\left(\int\frac{\dd^{4}k}{\left(2\pi\right)^{4}}\frac{1}{\slashd k-\slashd q-m+\ii\varepsilon}\left(\left(\slashd k-m+\ii\varepsilon\right)-\left(\slashd k-\slashd q-m+\ii\varepsilon\right)\right)\frac{1}{\slashd k-m+\ii\varepsilon}\gamma_{\nu}\right)=\\
 & =-q_{e}^{2}\text{tr}\left(\int\frac{\dd^{4}k}{\left(2\pi\right)^{4}}\left(\frac{1}{\slashd k-\slashd q-m+\ii\varepsilon}-\frac{1}{\slashd k-m+\ii\varepsilon}\right)\gamma_{\nu}\right)
\end{align*}
Variablentransformation $k_{\mu}\to k_{\mu}+q_{\mu}$ im ersten Summand,
führt auf das gleiche Integral wie das zweite, sodass man Null erhält?
Aber die quadratische Divergenzen liefern Oberflächenterme:
\begin{align*}
\lim_{\Lambda\to\infty}\int_{-\Lambda-a}^{\Lambda-a}\dd x\cdot x & =\lim_{\Lambda\to\infty}\left(\frac{1}{2}\left(\Lambda-a\right)^{2}-\frac{1}{2}\left(\Lambda+a\right)^{2}\right)=\lim_{\Lambda\to\infty}\left(-2\Lambda a\right)\not=0
\end{align*}
Die Theorie muss mit einem neuen Integralbegriff definiert werden,
der sich vom Riemann-Integral ausschließlich für das quadratisch divergen
te Integral unterscheidet.

Das Problem existiert nicht für ein logarithmisches Integral:
\begin{align*}
\lim_{\Lambda\to\infty}\int_{-\Lambda-a}^{\Lambda-a}\dd x\cdot\frac{1}{x} & =\lim_{\Lambda\to\infty}\left(\ln\left(\Lambda-a\right)-\ln\left(-\Lambda-a\right)\right)=\lim_{\Lambda\to\infty}\ln\left(\frac{\Lambda-a}{-\Lambda-a}\right)=\lim_{\Lambda\to\infty}\ln\left(\frac{\Lambda}{-\Lambda}\right)
\end{align*}


%DATE: Mi 22.5.13

\begin{align*}
I_{\mu\nu} & =-q_{e}^{2}\int\frac{\dd^{4}k}{\left(2\pi\right)^{4}}\underbrace{\text{tr}\left(\frac{1}{\slashd k-m+\ii\varepsilon}\gamma_{\nu}\frac{1}{\slashd k-\slashd q-m+\ii\varepsilon}\gamma_{\mu}\right)}_{=:\alpha_{m}}-\alpha_{M}
\end{align*}
\begin{align*}
\alpha_{m} & =\text{tr}\left(\frac{\left(\slashd k+m\right)\gamma_{\nu}\left(\slashd k-\slashd q+m\right)\gamma_{\mu}}{\left(k^{2}-m^{2}+\ii\varepsilon\right)\left(\left(k-q\right)^{2}-m^{2}+\ii\varepsilon\right)}\right)=\\
 & =4\left(m^{2}g_{\mu\nu}+k_{\nu}\left(k-q\right)_{\mu}-k\cdot\left(k-1\right)g_{\nu\mu}+k_{\mu}\left(k-q\right)_{\nu}\right)
\end{align*}
Wende den Trick mit Feynman-Parametern (2 äquivalente Varianten) an:
\begin{align*}
\text{1. Variante} &  & \frac{\ii}{k^{2}-m^{2}+\ii\varepsilon} & =\int_{0}^{\infty}\dd ze^{\ii z\left(k^{2}-m^{2}+\ii\varepsilon\right)}\\
\text{2. Variante} &  & \frac{1}{A_{1}\cdot\ldots A_{n}} & =\left(n-1\right)!\int_{0}^{1}\dd z_{1}\ldots\int_{0}^{1}\dd z_{n}\frac{\delta\left(1-z_{1}-\ldots-z_{n}\right)}{\left(A_{1}z_{1}+\ldots+A_{n}z_{n}\right)^{n}}
\end{align*}
Wir verwenden die 1. Variante:
\begin{align*}
I_{\mu\nu} & =4q_{e}^{2}\int\frac{\dd^{4}k}{\left(2\pi\right)^{4}}\int_{0}^{\infty}\dd z_{1}\int_{0}^{\infty}\dd z_{2}e^{\ii z_{1}\left(k^{2}-m^{2}+\ii\varepsilon\right)+\ii z_{2}\left(\left(k-q\right)^{2}-m^{2}+\ii\varepsilon\right)}\cdot\\
 & \qquad\cdot\left(\left(k-q\right)_{\mu}k_{\nu}+\left(k-q\right)_{\nu}k_{\mu}-g_{\mu\nu}\left(k\left(k-q\right)-m^{2}\right)-\alpha_{M}\right)
\end{align*}
Nach Subtraktion des $M$-Terms ist das Integral konvergent, da das
Integral über $\alpha_{m}$ logarithmisch divergiert, und somit können
wir die Integrationsvariablen verschieben.
\begin{align*}
k_{\mu} & \to l_{\mu}+q_{\mu}\frac{z_{2}}{z_{1}+z_{2}}\\
\left(k-q\right)_{\mu} & \to l_{\mu}-q_{\mu}\frac{z_{1}}{z_{1}+z_{2}}
\end{align*}
Damit erhält man für den Exponenten:
\begin{align*}
z_{1}\left(k^{2}-m^{2}+\ii\varepsilon\right)+z_{2}\left(\left(k-q\right)^{2}-m^{2}+\ii\varepsilon\right) & =z_{1}\left(l^{2}+\left(\frac{z_{2}}{z_{1}+z_{2}}\right)^{2}q^{2}+2l\cdot q\frac{z_{2}}{z_{1}+z_{2}}-m^{2}+\ii\varepsilon\right)+\\
 & \qquad+z_{2}\left(l^{2}+\left(\frac{z_{1}}{z_{1}+z_{2}}\right)^{2}q^{2}-2l\cdot q\frac{z_{1}}{z_{1}+z_{2}}-m^{2}+\ii\varepsilon\right)=\\
 & =z_{1}\left(l^{2}+\left(\frac{z_{2}}{z_{1}+z_{2}}\right)^{2}q^{2}-m^{2}+\ii\varepsilon\right)+\\
 & \qquad+z_{2}\left(l^{2}+\left(\frac{z_{1}}{z_{1}+z_{2}}\right)^{2}q^{2}-m^{2}+\ii\varepsilon\right)=\\
 & =\left(z_{1}+z_{2}\right)\left(l^{2}-m^{2}+\ii\varepsilon\right)+q^{2}\frac{z_{1}z_{2}}{z_{1}+z_{2}}
\end{align*}
Dies ist eine gerade Funktion von $l$. Daher bleiben vom Rest des
Integranden auch nur noch die geraden Potenzen von $l$, denn bei
ungeraden Potenzen von $l$ wird der Beitrag von $l$ durch den von
$-l$ aufgehoben, da sich der Exponent beim Übergang $l\to-l$ nicht
ändert. Diesen Trick kann man auch einzeln auf die Komponenten anwenden,
weshalb von $l_{\mu}l_{\nu}$ nur ein Term proportional zu $g_{\mu\nu}l^{2}$
beiträgt. Den Vorfaktor erhält man durch Multiplikation mit $g^{\mu\nu}$:
\begin{align*}
l_{\mu}l_{\nu} & \leadsto cg_{\mu\nu}l^{2}\\
l^{2}=l_{\mu}l_{\nu}g^{\mu\nu} & \leadsto cg_{\mu\nu}l^{2}g^{\mu\nu}=4cl^{2}
\end{align*}
Also ist $c=\frac{1}{4}$. Mit all diesen Tricks erhält man:
\begin{align*}
I_{\mu\nu} & =4q_{e}^{2}\int_{0}^{\infty}\dd z_{1}\int_{0}^{\infty}\dd z_{2}\int\frac{\dd^{4}l}{\left(2\pi\right)^{4}}e^{\ii\left(z_{1}+z_{2}\right)\left(l^{2}-m^{2}+\ii\varepsilon\right)+\ii q^{2}\frac{z_{1}z_{2}}{z_{1}+z_{2}}}\cdot\\
 & \qquad\cdot\left(2\cdot\left(\frac{1}{4}l^{2}g_{\mu\nu}-q_{\mu}q_{\nu}\frac{z_{1}z_{2}}{\left(z_{1}+z_{2}\right)^{2}}\right)-g_{\mu\nu}l^{2}+g_{\mu\nu}q^{2}\frac{z_{1}z_{2}}{\left(z_{1}+z_{2}\right)^{2}}+g_{\mu\nu}m^{2}-\alpha_{M}\right)
\end{align*}
Diese Gaußschen Integrale kann man ausrechnen. Nach längerer Rechnung
erhält man im Limes $\frac{q^{2}}{M^{2}}\to0$: {[}QED-Skript 8.1{]}
\begin{align*}
I_{\mu\nu} & =-\ii\frac{\alpha}{3\pi}\left(q^{2}g_{\mu\nu}-q_{\mu}q_{\nu}\right)\ln\left(\frac{M^{2}}{m^{2}}\right)+\ii\left(q^{2}g_{\mu\nu}-q_{\mu}q_{\nu}\right)\left(\text{endlicher Term}\right)
\end{align*}


Der endliche Term hängt ab von $q^{2}$, je nach $q^{2}\ge4m$, $0\le q^{2}\le4m$
oder $q^{2}\le0$.

Die Bedingung der Eichinvarianz war:
\begin{align*}
q^{\mu}I_{\mu\nu} & =q^{\nu}I_{\mu\nu}=0
\end{align*}
Dies ist hier erfüllt.

Jetzt kommen wir zur Renormierung:

TODO: Abb45
\begin{align*}
=electron & \left(g_{\mu\lambda}+\left(-\ii\frac{\alpha}{3\pi}\right)\ln\left(\frac{M^{2}}{m^{2}}\right)\left(q^{2}g_{\mu\nu}+q_{\mu}q_{\nu}\right)\frac{-\ii g^{\nu}\msd{\lambda}}{q^{2}+\ii\varepsilon}+\ldots\right)
\end{align*}
Der Faktor mit $q_{\mu}q_{\nu}$ trägt nicht bei, da $q_{\mu}W^{\mu}=0$
ist. Man erhält:
\begin{align*}
 & =electron\left(g_{\mu\lambda}-\frac{\alpha}{3\pi}\ln\left(\frac{M^{2}}{m^{2}}\right)g_{\mu\lambda}+\ldots\right)
\end{align*}
Nehme nun den Faktor auf mit in die Definition der renormierten Ladung:
\begin{align*}
e & \to e\left(1-\frac{\alpha}{3\pi}\ln\left(\frac{M^{2}}{m^{2}}\right)+\ldots\right)=e_{R}
\end{align*}
Die physikalisch gemessene Ladung ist die renormierte Ladung $e_{R}$.
Der endlichen Term ist:
\begin{align*}
 & \ii\left(q^{2}g_{\mu\nu}-q_{\mu}q_{\nu}\right)\left(-\frac{5\alpha}{9\pi}-\frac{4\alpha}{3\pi}\frac{m^{2}}{q^{2}}-\frac{\alpha}{3\pi}\left(1+\frac{2m^{2}}{q^{2}}\right)f\left(q^{2}\right)\right)
\end{align*}
\begin{align*}
f\left(q^{2}\right) & =\begin{cases}
\sqrt{1-\frac{4m^{2}}{q^{2}}}\ln\left(\frac{\sqrt{1-\frac{4m^{2}}{q^{2}}}+1}{\sqrt{1-\frac{4m^{2}}{q^{2}}}-1}\right) & \text{für }q^{2}<0\\
2\sqrt{\frac{4m^{2}}{q^{2}}-1}\arctan\left(\frac{1}{\sqrt{\frac{4m^{2}}{q^{2}}-1}}\right) & \text{für }0\le q^{2}\le4m^{2}\\
\sqrt{1-\frac{4m^{2}}{q^{2}}}\ln\left(\frac{\sqrt{1-\frac{4m^{2}}{q^{2}}}+1}{1-\sqrt{1-\frac{4m^{2}}{q^{2}}}}\right)-\ii\pi\sqrt{1-\frac{4m^{2}}{q^{2}}} & \text{für }4m^{2}\le q^{2}
\end{cases}
\end{align*}
Der Imaginärteil bedeutet, dass die Streumatrix nicht unitär ist,
was daran liegt, dass aus dem Photon ein reelles Teilchen-Antiteilchen-Paar
entstehen kann (Fermi's Goldene Regel).

%DATE: Mi 22.5.13 (2. Vorlesung)

Herleitung von Fermi's goldener Regel (vgl. Jackson):

TODO: Abb46; Der Körper sei von endlicher Ausdehnung, sodass man ihn
in eine Fläche $S_{1}$ einschließen kann.
\begin{align*}
\vec{E} & =\vec{E}_{i}+\vec{E}_{s}\\
\vec{B} & =\vec{B}_{i}+\vec{B}_{s}
\end{align*}
Wieviel Energie wird von dem Körper absorbiert? Mit der äußeren Normalen
$\vec{n}'$ von $S_{1}$ gilt.
\begin{align*}
P_{\text{abs}} & =-\frac{c}{8\pi}\oint_{S_{1}}\text{Re}\left(\vec{E}\times\vec{B}^{*}\right)\cdot\vec{n}'\dd a'\\
P_{\text{scatt}} & =\frac{c}{8\pi}\oint_{s_{1}}\text{Re}\left(\vec{E}_{s}\times\vec{B}_{s}^{*}\right)\cdot\vec{n}'\dd a'
\end{align*}
Die gesamte Verlustleistung der einlaufenden Welle ist:
\begin{align*}
P & =-\frac{c}{8\pi}\oint_{S_{1}}\text{Re}\left(\left(\vec{E}_{i}+\vec{E}_{s}\right)\times\left(\vec{B}_{i}^{*}+\vec{B}_{s}^{*}\right)-\vec{E}_{s}\times\vec{B}_{s}^{*}-\vec{E}_{i}\times\vec{B}_{i}^{*}\right)\vec{n}'\dd a'=\\
 & =-\frac{c}{8\pi}\oint_{S_{1}}\text{Re}\left(\vec{E}_{s}\times\vec{B}_{i}^{*}+\vec{E}_{i}\times\vec{B}_{s}^{*}\right)\vec{n}'\dd a'=\\
 & \sr ={\text{Re}\left(z\right)=\text{Re}\left(z^{*}\right)}{}-\frac{c}{8\pi}\oint_{S_{1}}\text{Re}\left(\vec{E}_{s}\times\vec{B}_{i}^{*}+\vec{E}_{i}^{*}\times\vec{B}_{s}\right)\vec{n}'\dd a'
\end{align*}
Mit
\begin{align*}
\vec{E}_{i} & =E\vec{\epsilon}_{0}e^{\ii\vec{k}_{0}\vec{x}} & \vec{B}_{i} & =\frac{\vec{k}_{0}}{k_{0}}\times\vec{E}_{i}
\end{align*}
und dem Ansatz $\frac{e^{\ii kr}}{r}\vec{F}\left(\vec{k}\right)$
für die gestreute Welle erhält man:
\begin{align*}
P & =\frac{c}{2k}\text{Im}\left(E_{0}\vec{\epsilon}_{0}^{*}\cdot\vec{F}\left(\vec{k},\vec{k}_{0}\right)\right)
\end{align*}
Definition des Streuquerschnitts:
\begin{align*}
\frac{P}{c\frac{\norm{\vec{E}_{0}}^{2}}{8\pi}} & \sim\frac{\text{Energie}}{\text{Zeit}\cdot\frac{\text{Energie}}{c\text{Zeit}\text{Fläche}}}\sim\text{Fläche}=:\sigma
\end{align*}
Dieser Wirkungsquerschnitt ist gegeben durch:
\begin{align*}
\sigma_{\text{abs}} & =\frac{4\pi}{k}\text{Im}\left[\vec{\epsilon}_{0}^{*}\cdot\frac{\vec{F}\left(\vec{k},\vec{k}_{0}\right)}{E_{0}}\right]_{\vec{k}=\vec{k}_{0}}
\end{align*}
Dies ist das optische Theorem. Die Streuamplitude $F$ berechnet durch
Betrachtung von
\begin{align*}
Ae^{\ii\vec{k}\cdot\vec{x}}+F\left(\vec{k},\vec{k}_{0}\right)\frac{e^{\ii\norm{\vec{k}}r}}{r}
\end{align*}
im Limes $r\to\infty$.

TODO: Abb47; $q^{2}=\left(k_{e}^{0}-\left(k_{e}'\right)^{0}\right)^{2}-\norm{\vec{k}_{e}-\vec{k}_{e}'}^{2}<0$

Zurück zur Renormierung:

TODO: Abb48, Abb49
\begin{align*}
e_{R}=e\left(1-\frac{1}{2}\frac{\alpha}{3\pi}\ln\left(\frac{M^{2}}{m^{2}}\right)+\frac{1}{2}\text{endlicher Anteil}\right) & \approx\underbrace{\left(1-\frac{1}{2}\frac{\alpha}{3\pi}\ln\left(\frac{M^{2}}{m^{2}}\right)\right)}_{\tilde{e}_{R}}\left(1+\frac{1}{2}\text{endlicher Anteil}\right)
\end{align*}


Messen kann man:
\begin{align*}
\frac{\dd\sigma}{\dd q^{2}} & =\underbrace{\frac{e_{R}\left(q^{2}\right)}{4\pi}}_{=\alpha_{R}}f\left(q^{2}\right)
\end{align*}
Dies macht sich zum Beispiel durch den Lamb-Shift bemerkbar:

TODO: Abb50; $-q^{2}\left(2p_{1/2}\right)<-q^{2}\left(2s_{1/2}\right)$;
$\alpha\left(2p_{1/2}\right)<\alpha\left(2s_{1/2}\right)$

Daher spalten diese vormals entarteten Niveaus auf:

TODO: Abb51

Zur Berechnung verwendet man die Diracgleichung mit Elektronenladung
$q_{e}\left(p\right)$ bzw. $q_{e}\left(r\right)$.
\begin{align*}
\left(\slashd p-q_{e}\left(r\right)\slashd A\left(r\right)-m\right)\psi & =0
\end{align*}
$A\left(r\right)$ berechnet man aus der gemessenen Ladungsverteilung
des Kerns.

TODO: Abb52; $-\ii e\gamma_{\mu}+\frac{\alpha}{2\pi}\ldots\sigma_{\mu\nu}q^{\nu}$

Daher wird das magnetische Moment des Elektrons durch die Vakuumfluktionen
geändert.

zurück zum Thema:

Ist der Graph symmetrisch um die Vakuumfluktionen, so schreibt man
an beiden Vertizes die Wurzel des zusätzlichen Faktors hin.

Für ein-/auslaufende Teilchen wird ein Faktor $\sqrt{1-\frac{\alpha}{\pi}\ldots}$
in die ,,Wellekfunktionsrenormierung`` absorbiert.


\section{Power-Counting}

Hamilton-Dichten wie

\begin{align*}
\mathcal{H}_{I} & =g\overline{\psi}\left(x\right)\gamma_{\mu}\psi\left(x\right)\overline{\psi}\left(x\right)\gamma^{\mu}\psi\left(x\right)
\end{align*}
sind nicht möglich, da sie nicht renormierbar sind.
\begin{align*}
\mathcal{H}_{0} & =\overline{\psi}\left(x\right)\left(\hat{\slashd p}-m\right)\psi\left(x\right)\\
\text{dim}\left(\mathcal{H}\right) & =\frac{1}{\text{Länge}^{4}}=\text{Energie}^{4}\\
\Rightarrow\qquad\text{dim}\left(\psi\right) & =\text{Energie}^{\frac{3}{2}}\\
\Rightarrow\qquad\text{dim}\left(g\right) & =\text{Energie}^{-2}
\end{align*}
TODO: Abb53

\begin{align*}
g^{2}\int_{-\Lambda}^{\Lambda}\dd^{4}l_{1}\int_{-\Lambda}^{\Lambda}\dd^{4}l_{2}\frac{1}{l^{4}} & \sim g^{2}\Lambda^{4}
\end{align*}
Daher divergiert das quartisch, was nicht zu renormieren ist.
\begin{align*}
\mathcal{L}_{0}^{\text{KG}} & =\phi\left(\hat{p}^{2}-m^{2}+\ii\varepsilon\right)\phi\\
\Rightarrow\qquad\text{dim}\left(\phi\right) & =\text{Energie}
\end{align*}
Bosonen:
\begin{align*}
\text{dim}\left(\frac{1}{4}F_{\mu\nu}F^{\mu\nu}\right) & =\text{Energie}^{4}\\
F^{\mu\nu} & =\partial_{\mu}A_{\nu}-\partial_{\nu}A_{\mu}
\end{align*}
Verboten sind Terme wie $F_{\mu\nu}F^{\mu\nu}F_{\alpha\beta}F^{\alpha\beta}$,
$F_{\mu\nu}F^{\mu\nu}\phi$. Möglich sind $A_{\mu}A^{\mu}A_{\nu}A^{\nu}$,
$\partial_{\mu}A_{\nu}A^{\mu}A^{\nu}$ etc..

Fermionen:
\begin{align*}
\text{dim}\left(\overline{\psi}\left(\slashd p-e\slashd A-m\right)\psi\right) & =\text{Energie}^{4}
\end{align*}
\begin{align*}
e^{2}\overline{\psi}A_{\mu}A^{\mu}\psi & \text{ ist verboten}
\end{align*}
TODO: Abb54; $\dd^{4}l\frac{1}{l^{4}}$; benötigt Ladung für Acht-Fermionen-Vertex,
dieser für 16-Vertex und so weiter und so fort...

Daher hat man unendlich viele freie Parameter.


\chapter{\texorpdfstring{$SU\left(N\right)$}{SU(N)} Eichgruppen}


\chapter{Tiefinelastische Streuung}


\chapter{Lagrangedichte der QCD, Feynman-Regeln}


\chapter{DGLAP-Gleichung}


\chapter{Dimensionale Regularisierung \texorpdfstring{$\overline{MS}$}{}}


\chapter{Kopplungskonstante \texorpdfstring{$\alpha_{S}\left(Q^{2}\right)$}{}}




\chapter{Standard-Modell, SSB und Higgs-Feld}




\chapter{Weitere Themen}
\begin{itemize}
\item Anomalien $\to$ ganze Familien
\begin{align*}
\fbox{\ensuremath{{\displaystyle \left(\begin{array}{c}
\nu_{e}\\
e
\end{array}\right),\left(\begin{array}{c}
u\\
d
\end{array}\right)_{\text{r,b,g}}}}}
\end{align*}

\item Inflation (frühes Universum)
\item Supersymmetrie
\item Stringtheorie
\item $\ldots$
\end{itemize}
\appendix

\part*{Anhang\thispagestyle{empty}}

\addcontentsline{toc}{part}{Anhang}

\fancyhead[C]{Anhang}

%DATE: Mo 29.4.13


\chapter{Analysis komplexer Zahlen}

\fancyhead[R]{A Analysis komplexer Zahlen}


\section{Komplexe Stetigkeit und Differenzierbarkeit}

Eine komplexe Zahl $z$ lässt sich in Realteil $x$ und Imaginärteil
$y$ zerlegen:
\begin{align*}
z & =x+\ii y
\end{align*}
Betrachte nun eine Abbildung:
\begin{align*}
f:\mathbb{C} & \to\mathbb{C}\\
z=x+\ii y & \mapsto w=u\left(x,y\right)+\ii v\left(x,y\right)
\end{align*}



\subsection{Definition \textmd{(Stetigkeit)}}

Eine Funktion $f\left(z\right)$ ist in einem abgeschlossenen und
beschränkten (also kompakten) Gebiet $\overline{G}$ \emph{gleichmäßig
stetig} bezüglich der Norm $\sqrt{zz^{*}}$, wenn gilt:
\begin{align}
\fall_{\varepsilon\in\mathbb{R}_{>0}}\,\exs_{\delta\left(\varepsilon\right)\in\mathbb{R}_{>0}}\,\fall_{\sr{}{z'',z'\in\overline{G}}{\abs{z''-z'}<\delta}}:\abs{f\left(z''\right)-f\left(z'\right)} & <\varepsilon
\end{align}



\subsection{Definition \textmd{(Differenzierbarkeit)}}

Sei $\left(z_{i}\right)_{i\in\mathbb{N}}\subseteq G$ eine Zahlenfolge
mit $z_{i}\not=\zeta$ und $\lim_{i\to\infty}z_{i}=\zeta$, so ist
die Funktion $f\left(z\right)$ im Punkt $\zeta$ differenzierbar,
wenn die Folge
\begin{align*}
g_{n} & =\frac{f\left(z_{n}\right)-f\left(\zeta\right)}{z_{n}-\zeta}
\end{align*}
für $n\to\infty$ konvergiert und der Grenzwert für jede solche Folge
$\left(z_{i}\right)_{i\in\mathbb{N}}$ gleich ist.

\begin{figure}[H]
\noindent \begin{centering}
\begin{tikzpicture}
  \draw[postaction={decorate},decoration={markings, mark=at position 1 with {\arrow[scale=1.75]{>}}}] (-2,0) -- (2,0) node[right]{$\text{Re}(z)$};
  \draw[postaction={decorate},decoration={markings, mark=at position 1 with {\arrow[scale=1.75]{>}}}] (0,-2) -- (0,2) node[above]{$\text{Im}(z)$};
  \draw (1,1.5) +(0.1,0.1) -- +(-0.1,-0.1) +(0.1,-0.1) -- +(-0.1,0.1);
  \draw[postaction={decorate},decoration={markings, mark=at position 0.5 with {\arrow[scale=1.75]{<}}}, dashed] (1,1.5) -- (1,0) node[below]{$x_0$};
  \draw[postaction={decorate},decoration={markings, mark=at position 0.5 with {\arrow[scale=1.75]{<}}}, dashed] (1,1.5) -- (0,1.5) node[left]{$y_0$};
\end{tikzpicture}
\par\end{centering}

\noindent \centering{}\caption{Der Limes muss entlang $x=x_{0}$ und $y=y_{0}$ denselben Wert ergeben.}
\end{figure}

\begin{enumerate}[label=\roman*)]
\item Für $x=x_{0}$ ist der Grenzwert:
\begin{align*}
 & \lim_{\Delta y\to0}\frac{u\left(x_{0},y_{0}+\Delta y\right)-u\left(x_{0},y_{0}\right)+\ii v\left(x_{0},y_{0}+\Delta y\right)-\ii v\left(x_{0},y_{0}\right)}{\ii\Delta y}=\\
 & \qquad\qquad\qquad\qquad\qquad\qquad\qquad\qquad\qquad\qquad\qquad\qquad=-\ii\partial_{2}u\left(x_{0},y_{0}\right)+\partial_{2}v\left(x_{0},y_{0}\right)
\end{align*}

\item Für $y=y_{0}$ ist der Grenzwert:
\begin{align*}
 & \lim_{\Delta x\to0}\frac{u\left(x_{0}+\Delta x,y_{0}\right)-u\left(x_{0},y_{0}\right)+\ii v\left(x_{0}+\Delta x,y_{0}\right)-\ii v\left(x_{0},y_{0}\right)}{\Delta x}=\\
 & \qquad\qquad\qquad\qquad\qquad\qquad\qquad\qquad\qquad\qquad\qquad\qquad=\partial_{1}u\left(x_{0},y_{0}\right)+\ii\partial_{1}v\left(x_{0},y_{0}\right)
\end{align*}

\end{enumerate}
Da der Grenzwert unabhängig von der Folge sein muss, folgen die \emph{Cauchy-Riemannschen
Differentialgleichungen}:
\begin{align}
\partial_{2}v\left(x_{0},y_{0}\right) & =\partial_{1}u\left(x_{0},y_{0}\right)\\
-\partial_{2}u\left(x_{0},y_{0}\right) & =\partial_{1}v\left(x_{0},y_{0}\right)
\end{align}
Wenn $u$ und $v$ zweimal stetig differenzierbar sind, ergeben sich
Laplace-Gleichungen:
\begin{align*}
\partial_{1}^{2}u & =\partial_{1}\partial_{2}v=\partial_{2}\partial_{1}v=-\partial_{2}^{2}u & \Rightarrow\qquad\upDelta u & =0\\
\partial_{1}^{2}v & =-\partial_{1}\partial_{2}u=-\partial_{2}\partial_{1}u=-\partial_{2}^{2}v & \Rightarrow\qquad\upDelta v & =0
\end{align*}



\section{Komplexe Integration}


\subsection{Definition \textmd{(Rektifizierbare Kurve, Integral)}}

Eine Kurve $J$ ist \emph{rektifizierbar}, wenn für jedes Sehnenpolygon
gilt:
\begin{align*}
\sum_{i=1}^{n}L_{i} & \le L<\infty
\end{align*}


\begin{figure}[H]
\noindent \begin{centering}
\begin{tikzpicture}
  \draw plot[smooth,tension=.7] coordinates{(0,-1) (1,0) (0,1) (0,1.8) (1,2)};
  \draw (0,-1) +(0.1,0.1) -- +(-0.1,-0.1) +(0.1,-0.1) -- +(-0.1,0.1);
  \draw (1,0) +(0.1,0.1) -- +(-0.1,-0.1) +(0.1,-0.1) -- +(-0.1,0.1);
  \draw (0,1) +(0.1,0.1) -- +(-0.1,-0.1) +(0.1,-0.1) -- +(-0.1,0.1);
  \draw (0,1.8) +(0.1,0.1) -- +(-0.1,-0.1) +(0.1,-0.1) -- +(-0.1,0.1);
  \draw (1,2) +(0.1,0.1) -- +(-0.1,-0.1) +(0.1,-0.1) -- +(-0.1,0.1);
  \draw[blue] (0,-1) -- (1,0) node[right]{$z_i$} -- node[below left]{$L_i$} (0,1) node[right]{$z_{i+1}$} -- (0,1.8) -- (1,2);
  \draw[fill,blue] ($ (1,0)!0.5!(0,1) $) circle (2pt) node[right=5pt]{$\zeta_i$};
\end{tikzpicture}
\par\end{centering}

\noindent \centering{}\caption{Sehnenpolygon}
\end{figure}


Nun charakterisiere $l$ ein Polygon mit $m\left(l\right)$ Kanten
und $\zeta_{i}^{\left(l\right)}$ einen Zwischenpunkt auf dem $i$-ten
Teilstück. $\mathfrak{Z}^{l}$ sei eine Polygonfolge mit $L_{i}^{\left(l\right)}\xrightarrow{l\to\infty}0$.
Wenn der Grenzwert
\begin{align*}
S_{J} & =\lim_{l\to\infty}\left(\sum_{i=1}^{m\left(l\right)}f\left(\zeta_{i}^{\left(l\right)}\right)\cdot\left(z_{i+1}^{\left(l\right)}-z_{i}^{\left(l\right)}\right)\right)
\end{align*}
existiert, dann definiert dieser das \emph{Integral}:
\begin{align}
\int_{J}f\left(z\right)\dd z & =-\int_{-J}f\left(z\right)\dd z:=S_{J}
\end{align}



\subsection{Hauptsatz der Funktionentheorie}

Die Funktion $f\left(z\right)$ sei in dem einfach zusammenhängenden
Gebiet $G$ \emph{regulär} (auch \emph{holomorph} genannt) und sei
$C$ eine geschlossene Kurve in $G$, dann gilt:
\begin{align}
\int_{C}f\left(z\right)\dd z & =0
\end{align}
Äquivalent dazu ist, dass das Integral $\int_{z_{1}}^{z_{2}}f\left(z\right)\dd z$
wegunabhängig ist:
\begin{align*}
\int_{C_{1}}f\left(z\right)\dd z-\int_{C_{2}}f\left(z\right)\dd z & =\int_{C}f\left(z\right)\dd z=0
\end{align*}
\begin{figure}[H]
\noindent \begin{centering}
\begin{tikzpicture}
  \draw plot[smooth cycle,tension=.7] coordinates{(0,-1) (1,0) (0,1) (-0.7,1.2) (-1.2,0) (-1,-1)};
  \draw (0,-1) node[below=2pt]{$z_1$} +(0.1,0.1) -- +(-0.1,-0.1) +(0.1,-0.1) -- +(-0.1,0.1);
  \draw (-0.7,1.2) node[above=2pt]{$z_2$} +(0.1,0.1) -- +(-0.1,-0.1) +(0.1,-0.1) -- +(-0.1,0.1);
  \draw[thick,decoration={markings,mark=at position 1 with {\arrow[scale=1.75]{>}};},postaction={decorate}] (1,-0.025) -- node[right]{$C_2$} (1,0.025);
  \node at (1.2,0) {};
  \node at (-1.3,0) {};
  \draw[thick,decoration={markings,mark=at position 1 with {\arrow[scale=1.75]{>}};},postaction={decorate}] (-1.2,-0.025) -- node[left]{$C_1$} (-1.195,0.025);
  \draw[blue,thick,decoration={markings,mark=at position 1 with {\arrow[scale=1.75]{>}};},postaction={decorate}] (-0.7,-1.135) -- node[below=3pt]{$C$} (-0.725,-1.13);
  \draw[blue,thick,decoration={markings,mark=at position 1 with {\arrow[scale=1.75]{>}};},postaction={decorate}] (-0.98,0.75) -- (-0.96,0.81);
  \draw[blue,thick,decoration={markings,mark=at position 1 with {\arrow[scale=1.75]{>}};},postaction={decorate}] (0,1) -- (0.05,0.96);
  \draw[blue,thick,decoration={markings,mark=at position 1 with {\arrow[scale=1.75]{>}};},postaction={decorate}] (0.5,-0.69) -- (0.45,-0.725);
\end{tikzpicture}
\par\end{centering}

\noindent \centering{}\caption{Wegunabhängigkeit}
\end{figure}



\subsubsection*{Beweis}

Wir gehen wie folgt vor:
\begin{enumerate}
\item $\int_{C}\dd z=0$
\item $\int_{C}z\dd z=0$
\item Triangulation
\end{enumerate}
Fangen wir an:
\begin{enumerate}
\item Es gilt:
\begin{align*}
S_{C} & =\lim_{l\to\infty}\sum_{i=1}^{m\left(l\right)}\left(z_{i+1}^{\left(l\right)}-z_{i}^{\left(l\right)}\right)\cdot1=\lim_{l\to\infty}\left(z_{m\left(l\right)+1}^{\left(l\right)}-z_{1}^{\left(l\right)}\right)\sr ={\text{geschlossener}}{\text{Weg}}0
\end{align*}

\item Hierfür folgt:
\begin{align*}
S_{C} & =\lim_{l\to\infty}\sum_{i=1}^{m\left(l\right)}\left(z_{i+1}^{\left(l\right)}-z_{i}^{\left(l\right)}\right)\zeta_{i}^{\left(l\right)}=\\
 & \sr ={\zeta_{i}^{\left(l\right)}:=\frac{z_{i+1}^{\left(l\right)}+z_{i}^{\left(l\right)}}{2}}{}\frac{1}{2}\left(\lim_{l\to\infty}\sum_{i=1}^{m\left(l\right)}\left(z_{i+1}^{\left(l\right)}-z_{i}^{\left(l\right)}\right)z_{i+1}^{\left(l\right)}+\lim_{l\to\infty}\sum_{i=1}^{m\left(l\right)}\left(z_{i+1}^{\left(l\right)}-z_{i}^{\left(l\right)}\right)z_{i}^{\left(l\right)}\right)=\\
 & =\frac{1}{2}\lim_{l\to\infty}\sum_{i=1}^{m\left(l\right)}\left(\left(z_{i+1}^{\left(l\right)}\right)^{2}-\left(z_{i}^{\left(l\right)}\right)^{2}\right)=\frac{1}{2}\lim_{l\to\infty}\left(\left(z_{m\left(l\right)+1}^{\left(l\right)}\right)^{2}-\left(z_{1}^{\left(l\right)}\right)^{2}\right)=0
\end{align*}

\item Sei $D$ ein Dreieck mit Rand $\partial D$.\\
\begin{figure}[H]
\noindent \begin{centering}
\begin{tikzpicture}[scale=2.5]
  \draw (-1,0) -- (1,0) -- (0,1.732) -- cycle;
  \draw[thick,decoration={markings,mark=at position 0.5 with {\arrow[orange,scale=1.75]{>}};},postaction={decorate}] (-1,0) -- (0,0);
  \draw[thick,decoration={markings,mark=at position 0.5 with {\arrow[orange,scale=1.75]{>}};},postaction={decorate}] (0,0) -- (1,0);
  \draw[thick,decoration={markings,mark=at position 0.5 with {\arrow[orange,scale=1.75]{>}};},postaction={decorate}] (1,0) -- (0.5,0.866);
  \draw[thick,decoration={markings,mark=at position 0.5 with {\arrow[orange,scale=1.75]{>}};},postaction={decorate}] (0.5,0.866) -- (0,1.732);
  \draw[thick,decoration={markings,mark=at position 0.5 with {\arrow[orange,scale=1.75]{>}};},postaction={decorate}] (0,1.732) -- (-0.5,0.866);
  \draw[thick,decoration={markings,mark=at position 0.5 with {\arrow[orange,scale=1.75]{>}};},postaction={decorate}] (-0.5,0.866) -- (-1,0);
  \draw[thick,red] (0,0) -- (0.5,0.866) -- (-0.5,0.866) -- cycle;
  % Innere orange Pfeile
  \path[thick,decoration={markings,mark=at position 0.45 with {\arrow[orange,scale=1.75]{>}};},postaction={decorate}] (0,0) -- (-0.5,0.866);
  \path[thick,decoration={markings,mark=at position 0.45 with {\arrow[orange,scale=1.75]{>}};},postaction={decorate}] (0.5,0.866) -- (0,0);
  \path[thick,decoration={markings,mark=at position 0.45 with {\arrow[orange,scale=1.75]{>}};},postaction={decorate}] (-0.5,0.866) -- (0.5,0.866);
  % rote Pfeile
  \path[thick,decoration={markings,mark=at position 0.55 with {\arrow[red,scale=1.75]{<}};},postaction={decorate}] (0,0) -- (-0.5,0.866);
  \path[thick,decoration={markings,mark=at position 0.55 with {\arrow[red,scale=1.75]{<}};},postaction={decorate}] (0.5,0.866) -- (0,0);
  \path[thick,decoration={markings,mark=at position 0.55 with {\arrow[red,scale=1.75]{<}};},postaction={decorate}] (-0.5,0.866) -- (0.5,0.866);
  \node at (-0.5,0.3) {$D^{(1)}$};
  \node at (0.5,0.3) {$D^{(4)}$};
  \node at (0,0.6) {$D^{(2)}$};
  \node at (0,1.15) {$D^{(3)}$};
  \node at (0,-0.1) {};
\end{tikzpicture}
\par\end{centering}

\noindent \centering{}\caption{Zerlegung des Dreiecks}
\end{figure}
\begin{align*}
\int_{\partial D}f\left(z\right)\dd z & =\int_{\partial D^{\left(1\right)}}f\left(z\right)\dd z+\int_{\partial D^{\left(2\right)}}f\left(z\right)\dd z+\int_{\partial D^{\left(3\right)}}f\left(z\right)\dd z+\int_{\partial D^{\left(4\right)}}f\left(z\right)\dd z
\end{align*}
Daher gibt es unter den $D^{\left(i\right)}$ ein $D_{1}$, das folgende
Bedingung erfüllt. Iterativ wählt man so $D_{n}$ für $n\in\mathbb{N}$.
\begin{align*}
\abs{\int_{\partial D}f\left(z\right)\dd z} & \le4\abs{\int_{\partial D_{1}}f\left(z\right)\dd z}\le16\abs{\int_{\partial D_{2}}f\left(z\right)\dd z}\le4^{n}\abs{\int_{\partial D_{n}}f\left(z\right)\dd z}
\end{align*}
Sei nun $n\in\mathbb{N}$ so groß, dass für $z_{0}\in\bigcap_{k\in\mathbb{N}}D_{k}$
und alle $z\in D_{n}$ schon $\abs{z-z_{0}}<\delta$ gilt.\\
\begin{figure}[H]
\noindent \begin{centering}
\begin{tikzpicture}
  \draw (0,0) node[right]{$z_0$} +(0.1,0.1) -- +(-0.1,-0.1) +(0.1,-0.1) -- +(-0.1,0.1);
  \draw (0.3,-0.5) node[right]{$z$} +(0.1,0.1) -- +(-0.1,-0.1) +(0.1,-0.1) -- +(-0.1,0.1);
  \draw (0,0) circle(2);
  \draw[orange,thick] (-1,-0.7) -- (1,-0.7) --  node[right]{$D_n$} (0,1.032) -- cycle;
  \draw[thick,decoration={markings,mark=at position 0.5 with {\arrow[scale=1.75]{>}};},postaction={decorate}] (0,0) -- node[right]{$\delta$} (0,-2);
\end{tikzpicture}
\par\end{centering}

\noindent \centering{}\caption{$D_{n}$ wird für große $n$ sehr klein.}
\end{figure}
Da $f$ holomorph ist und $D_{n}$ beliebig klein wird, kann man für
ein beliebiges $\varepsilon\in\mathbb{R}_{>0}$ ein so großes $n\in\mathbb{N}$
wählen, dass gilt:
\begin{align*}
\abs{f\left(z\right)-f\left(z_{0}\right)-\left(z-z_{0}\right)f'\left(z_{0}\right)} & <\varepsilon\abs{z-z_{0}}
\end{align*}
Daher gibt es eine Funktion $\eta:D_{n}\to\mathbb{R}$ mit:
\begin{align*}
f\left(z\right) & =f\left(z_{0}\right)+\left(z-z_{0}\right)f'\left(z_{0}\right)+\eta\left(z\right)
\end{align*}
Damit folgt: 
\begin{align*}
\int_{\partial D_{n}}f\left(z\right)\dd z & =\int_{\partial D_{n}}f\left(z_{0}\right)\dd z+\int_{\partial D_{n}}f'\left(z_{0}\right)z\dd z-\int_{\partial D_{n}}f'\left(z_{0}\right)z_{0}\dd z+\int_{\partial D_{n}}\eta\left(z\right)\dd z=\\
 & =0+0+0+\int_{\partial D_{n}}\eta\left(z\right)\dd z
\end{align*}
Mit dem Umfang $s_{n}$ des Dreiecks $D_{n}$ folgt:
\begin{align*}
\abs{\int_{\partial D_{n}}f\left(z\right)\dd z} & =\abs{\int_{\partial D_{n}}\eta\left(z\right)\dd z}\stackrel{\abs{\eta\left(z\right)}<\varepsilon\abs{z-z_{0}}}{<}\varepsilon\frac{s_{n}}{2}\cdot s_{n}
\end{align*}
Nun erhält man:
\begin{align*}
\abs{\int_{\partial D}f\left(z\right)\dd z} & \le4^{n}\abs{\int_{\partial D_{n}}f\left(z\right)\dd z}<4^{n}\frac{\varepsilon}{2}s_{n}^{2}=4^{n}\frac{\varepsilon}{2}\left(\frac{s}{2^{n}}\right)^{2}=\frac{\varepsilon}{2}s^{2}
\end{align*}
Da dies für alle $\varepsilon\in\mathbb{R}_{>0}$ gilt, folgt:
\begin{align*}
\int_{D}f\left(z\right)\dd z & =0
\end{align*}
\qqed[A.2.2.]
\end{enumerate}

\subsection{Folgerungen}

Sei $f\left(z\right)$ in $G$ regulär bis auf den Punkt $z_{0}$.

\begin{figure}[H]
\noindent \begin{centering}
\begin{tikzpicture}
  \draw plot[smooth cycle,tension=.7] coordinates{(-2,-2) (0,-1.9) (1,-1.8) (2,0.4) (1,0.8) (0.5,1.7) (-1,1)};
  \draw (0,0) node[right]{$z_0$} +(0.1,0.1) -- +(-0.1,-0.1) +(0.1,-0.1) -- +(-0.1,0.1);
  \draw[thick,decoration={markings,mark=at position 0.5 with {\arrow[scale=1.75]{>}};},postaction={decorate}] (2,0.4) -- node[right]{$C$} (2,0.45);
  \draw[orange, dashed, line width=2pt] plot[smooth cycle,tension=.7] coordinates{(-2,-2) (0,-1.9) (1,-1.8) (2,0.4) (1,0.8) (0.5,1.7) (-1,1)};
  \draw[orange,thick,decoration={markings,mark=at position 0.5 with {\arrow[scale=1.75]{>}};},postaction={decorate}] (1,-1.8) -- node[below right]{$C_1$} (1.05,-1.75);
  \draw[orange,thick,decoration={markings,mark=at position 0.5 with {\arrow[scale=1.75]{>}};},postaction={decorate}] (-1,1) -- (-1.05,0.95);
  \draw[green] (0,0) circle (0.5);
  \node[green] at (0,-0.3) {$C_2$};
  \draw[green,thick,decoration={markings,mark=at position 0.5 with {\arrow[scale=1.75]{>}};},postaction={decorate}] (-0.5,0.05) -- (-0.5,-0.05);
  \draw[orange] (-0.05,1.7) -- (-0.07,0.75) arc (97:443:0.75) -- (0.12,1.7);
  \draw[orange,thick,decoration={markings,mark=at position 0.5 with {\arrow[scale=1.75]{>}};},postaction={decorate}] (-0.05,1.15) -- (-0.05,1.2);
  \draw[orange,thick,decoration={markings,mark=at position 0.5 with {\arrow[scale=1.75]{>}};},postaction={decorate}] (0.12,1.25) -- (0.12,1.2);
  \draw[orange,thick,decoration={markings,mark=at position 0.5 with {\arrow[scale=1.75]{>}};},postaction={decorate}] (0.75,0.025) -- (0.75,-0.025);
\end{tikzpicture}
\par\end{centering}

\noindent \centering{}\caption{Zurückführung auf Integration in kleiner Umgebung der Singularität}
\end{figure}
Dann gilt:
\begin{align*}
\int_{C}f\left(z\right)\dd z & =\underbrace{\int_{C_{1}}f\left(z\right)\dd z}_{=0}+\int_{C_{2}}f\left(z\right)\dd z
\end{align*}
Wir betrachten jetzt für $m\in\mathbb{Z}$:
\begin{align*}
I & :=\int_{C_{2}}\left(z-z_{0}\right)^{m}\dd z
\end{align*}
Wir parametrisieren den Weg $C_{2}$ mit $t\in\left[-\pi,\pi\right]$
wie folgt:
\begin{align*}
z & =z_{0}+r\left(\cos\left(t\right)+\ii\sin\left(t\right)\right)\\
\dd z & =\left(-r\sin\left(t\right)+\ii r\cos\left(t\right)\right)\dd t=\ii r\left(\cos\left(t\right)+\ii\sin\left(t\right)\right)\dd t=\ii re^{\ii t}\dd t
\end{align*}
Es folgt:
\begin{align*}
I & =\int_{-\pi}^{\pi}\ii re^{\ii t}r^{m}e^{\ii mt}\dd t=\ii r^{m+1}\int_{-\pi}^{\pi}\cos\left(\left(m+1\right)t\right)+\ii\sin\left(\left(m+1\right)t\right)\dd t=2\pi\ii\delta_{m,-1}
\end{align*}
Damit folgt die \emph{Cauchysche Integralformel}: Sei $f\left(z\right)$
im einfach zusammenhängenden Gebiet $G$ regulär, so gilt für jeden
geschlossenen doppelpunktfreien positiv orientierten Weg $C$ in $G$
und für jeden Punkt $z$ im Innengebiet von $C$:
\begin{align}
\frac{1}{2\pi\ii}\int_{C}\frac{f\left(\zeta\right)}{\zeta-z}\dd\zeta & =f\left(z\right)
\end{align}

\begin{description}
\item [{Behauptung:}] Es gilt:
\begin{align}
\lim_{\varepsilon\searrow0}\int_{-\infty}^{\infty}\dd x\frac{f\left(x\right)}{x\pm\ii\varepsilon} & =\int_{-\infty}^{\infty}\dd xf\left(x\right)\left(\mathcal{P}\left(\frac{1}{x}\right)\mp\ii\pi\delta\left(x\right)\right)
\end{align}
Dabei ist $\delta\left(x\right)$ die Delta-Distribution und $\mathcal{P}\left(\frac{1}{x}\right)$
die Hauptwertdistribution von $\frac{1}{x}$:
\begin{align*}
\mathcal{P}\left(\frac{1}{x}\right):\mathcal{S}\left(\mathbb{R}\right) & \to\mathbb{R}\\
f & \mapsto\lim_{\varepsilon\searrow0}\int_{\mathbb{R}\setminus\left[-\varepsilon,\varepsilon\right]}\frac{f\left(x\right)}{x}\dd x=\int_{0}^{\infty}\frac{f\left(x\right)-f\left(-x\right)}{x}\dd x
\end{align*}

\item [{Beweis:}] Zeige zunächst:
\begin{align*}
\int_{-\infty}^{\infty}\delta\left(x\right)f\left(x\right)\dd x & \stackrel{!}{=}\lim_{\varepsilon\to0}\frac{1}{\pi}\int_{-\infty}^{\infty}f\left(x\right)\frac{\varepsilon}{x^{2}+\varepsilon^{2}}\dd x
\end{align*}
Wähle dazu $\varepsilon=\delta\cdot\eta$ und schreibe:
\begin{align*}
\lim_{\varepsilon\to0}\frac{1}{\pi}\int_{-\infty}^{\infty}f\left(x\right)\frac{\varepsilon}{x^{2}+\varepsilon^{2}}\dd x & =\frac{1}{\pi}\lim_{\delta\to0}\lim_{\eta\to0}\int_{-\delta}^{\delta}f\left(x\right)\frac{\delta\eta}{x^{2}+\delta^{2}\eta^{2}}\dd x=\\
 & =\frac{f\left(0\right)}{\pi}\lim_{\delta\to0}\lim_{\eta\to0}\int_{-\infty}^{\infty}\frac{\delta\eta}{x^{2}+\delta^{2}\eta^{2}}\dd x=\\
 & =\frac{f\left(0\right)}{\pi}\lim_{\varepsilon\to0}\int_{-1}^{1}\frac{\varepsilon}{x^{2}+\varepsilon^{2}}\dd x=\lim_{\varepsilon\to0}\frac{f\left(0\right)}{\pi}\arctan\left(\frac{x}{\varepsilon}\right)\big|_{-1}^{1}=\\
 & =\frac{f\left(0\right)}{\pi}\left(\frac{\pi}{2}+\frac{\pi}{2}\right)=f\left(0\right)
\end{align*}
Nun zeigen für den Hauptwert:
\begin{align*}
\int_{-\infty}^{\infty}f\left(x\right)\mathcal{P}\left(\frac{1}{x}\right)\dd x & :=\lim_{\varepsilon\to0}\left(\int_{\varepsilon}^{\infty}\frac{f\left(x\right)}{x}\dd x+\int_{-\infty}^{-\varepsilon}\frac{f\left(x\right)}{x}\dd x\right)\stackrel{!}{=}\lim_{\varepsilon\to0}\int_{-\infty}^{\infty}f\left(x\right)\frac{x}{x^{2}+\varepsilon^{2}}\dd x
\end{align*}
Sei dazu $\varepsilon=\delta\eta$ und betrachte:
\begin{align*}
\text{r.S.}-\text{l.S.} & =\lim_{\delta\to0}\lim_{\eta\to0}\left(\int_{\delta\eta}^{\infty}\dd x+\int_{-\infty}^{-\delta\eta}\dd x\right)f\left(x\right)\left(\frac{x}{x^{2}+\delta^{2}\eta^{2}}-\frac{1}{x}\right)+\\
 & \qquad+\underbrace{\int_{-\delta\eta}^{\delta\eta}f\left(x\right)\frac{x}{x^{2}+\varepsilon^{2}}\dd x}_{\to0}=\\
 & \sr ={x=\delta y}{\dd x=\delta\dd y}\lim_{\delta\to0}\lim_{\eta\to0}\left(\int_{\eta}^{\infty}\dd y+\int_{-\infty}^{-\eta}\dd y\right)f\left(\delta y\right)\left(\frac{y}{y^{2}+\eta^{2}}-\frac{1}{y}\right)=\\
 & =f\left(0\right)\lim_{\eta\to0}\left(\int_{\eta}^{\infty}\dd y+\int_{-\infty}^{-\eta}\dd y\right)\left(\frac{y}{y^{2}+\eta^{2}}-\frac{1}{y}\right)=0
\end{align*}
Es folgt:
\begin{align*}
\lim_{\varepsilon\to0}\int_{-\infty}^{\infty}\dd x\frac{f\left(x\right)}{x\pm\ii\varepsilon} & =\lim_{\varepsilon\to0}\int_{-\infty}^{\infty}\dd xf\left(x\right)\frac{x\mp\ii\varepsilon}{x^{2}+\varepsilon^{2}}=\int_{-\infty}^{\infty}f\left(x\right)\left(\mathcal{P}\left(\frac{1}{x}\right)\mp\ii\pi\delta\left(x\right)\right)
\end{align*}
\qqed[Behauptung]
\end{description}

\chapter*{Danksagungen}

\addcontentsline{toc}{chapter}{\hspace*{1.5em}Danksagungen}

\fancyhead[R]{Danksagungen}

Mein besonderer Dank geht an Professor Schäfer, der diese Vorlesung
hielt und es mir gestattete, diese Vorlesungsmitschrift zu veröffentlichen.

Außerdem möchte ich mich ganz herzlich bei allen bedanken, die durch
aufmerksames Lesen Fehler gefunden und mir diese mitgeteilt haben.

\vspace{1cm}


\hfill{}Andreas Völklein

\label{END}
\end{document}
